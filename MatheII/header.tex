\documentclass[a4paper,12pt]{scrartcl}
\usepackage[ngerman]{babel}
\usepackage[utf8]{inputenc} %Umlaute & Co
\usepackage{amsmath,amssymb,amsthm,amsfonts}
\usepackage{graphicx}
\usepackage{ifthen}
\usepackage{tikz}
\usepackage{mathtools}
\usepackage{fancyhdr}
\usepackage{enumerate}
\usepackage{pgfplots}
\usepackage{color}
\usetikzlibrary{trees,automata,arrows,shapes}
\pagestyle{empty}
%counts the exercisenumber
\newcommand{\cd}{\cdot}
\newcommand{\C}{\mathbb{C}}
\newcommand{\Z}{\mathbb{Z}}
\renewcommand{\i}{\item}
\newcommand{\N}{\mathbb{N}}
\newcommand{\R}{\mathbb{R}}
\DeclarePairedDelimiter{\ceil}{\lceil}{\rceil}
\DeclarePairedDelimiter{\floor}{\lfloor}{\rfloor}
\newcounter{n}
\def\header#1#2#3#4#5#6#7{\pagestyle{empty}
\begin{minipage}[t]{0.47\textwidth}
\begin{flushleft}
{\bf #4}\\
#5\\
Tutor: #2
\end{flushleft}
\end{minipage}
\begin{minipage}[t]{0.5\textwidth}
\begin{flushright}
#6 \vspace{0.5cm}\\
\end{flushright}
\end{minipage}
\vspace{0.5cm}
\begin{center}
{\Large\bf Blatt #1}

{(Abgabe #3)}
\end{center}
}
%Kommando für Aufgaben
%\Aufgabe{AufgTitel}{Punktezahl}
\newcommand{\Aufgabe}[2]{\stepcounter{n}
\ifnum\value{n}>100
\\
\else
\fi
\indent\textbf{Aufgabe \arabic{n}: #1} (#2 Punkte)\\}
\newcommand{\heading}[2]{\header{#1}{Tutor}{#2}{Mathe II}{\textit{Flavia N\"ahrlich}\\ \textit{Finn Ickler}
    }{SS 15}{3}
    \vspace{1cm}}
