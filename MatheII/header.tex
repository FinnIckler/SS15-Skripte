\documentclass[a4paper,12pt,twoside]{article}
\usepackage{fourier}
\usepackage[ngerman]{babel}
\usepackage[utf8]{inputenc} %Umlaute & Co
\usepackage[leqno,tbtags,nointlimits]{amsmath}
\usepackage{amssymb,amsthm,amsfonts}
\usepackage{graphicx}
\usepackage{ifthen}
\usepackage{tikz}
\usepackage{mathtools}
\usepackage{fancyhdr,lastpage}
\usepackage{enumerate}
\usepackage{pgfplots}
\usepackage{color}
\usetikzlibrary{trees,automata,arrows,shapes}
\pagestyle{fancy}
\usepackage{hyperref}
\fancyhf{} %--Clear all fields
\renewcommand\sectionmark[1]{ \markboth{\thesection\ \textsc{#1}}{}}
\fancyhead[LO,RE]{\normalsize \leftmark}
\fancyhead[LE,RO]{ \rightmark}
\fancyfoot{} % clear all footer fields
\fancyfoot[LE,RO]{\thepage}
\newcommand{\cd}{\cdot}
\newcommand{\C}{\mathbb{C}}
\newcommand{\Z}{\mathbb{Z}}
\renewcommand{\i}{\item}
\newcommand{\N}{\mathbb{N}}
\newcommand{\R}{\mathbb{R}}
\DeclarePairedDelimiter{\ceil}{\lceil}{\rceil}
\DeclarePairedDelimiter{\floor}{\lfloor}{\rfloor}
\newcommand{\seriesalg}[3]{\sum\limits_{#3}^{#2} #1}
\newcommand{\seriesn}[1]{\sum\limits_{i=k}^{n} #1}
\newcommand{\seriesnplus}[1]{\sum\limits_{i=k}^{n+1} #1}
\newcommand{\series}[1]{\sum\limits_{i=k}^{\infty} #1}
\newcommand{\seriesnull}[1]{\sum\limits_{i=0}^{\infty} #1}
\usepackage[normalem]{ulem}
\usepackage{blkarray}
\usepackage{stmaryrd}
\usepackage{titletoc}
\usepackage%[margin=15mm]
{geometry}
\newcommand{\abs}[1]{\lvert #1 \rvert}
\renewcommand\headrule{{\color{gray}%
\hrule height 2pt width\headwidth
\vspace{1pt}%
\hrule height 1pt width\headwidth
\vspace{-4pt}}}
\makeatletter
\newcommand{\resetHeadWidth}{\fancy@setoffs}
\makeatother
\newcommand{\cucubr}[7]{%
%origin point, circle radius, start angle, end angle, distance c-b, brace radius, brace options
\pgfmathsetmacro{\helpangleedge}{acos(1-pow(#6,2)/2/pow(#2+#5,2))}%
\pgfmathsetmacro{\turnangleedge}{90+(\helpangleedge/2)}%
\pgfmathsetmacro{\helpanglemid}{acos(1-pow(#6,2)/2/pow(#2+#5+2*#6,2))}%
\pgfmathsetmacro{\turnanglemid}{90-(\helpanglemid/2)}%
\pgfmathsetmacro{\halfangle}{(#4-#3)/2+#3}%
\pgfmathsetmacro{\midradius}{#2+#5+#6}%
\pgfmathsetmacro{\outerradius}{#2+#5+1.88*#6}%
\pgfmathsetmacro{\firstmidanglestart}{mod(\halfangle-\helpanglemid+180,360)}%
\pgfmathsetmacro{\secondmidanglestart}{mod(\halfangle+\helpanglemid+180,360)}%
\pgfmathsetmacro{\firstmidanglestop}{mod(\halfangle-\helpanglemid/2+180,360)-\turnanglemid}%
\pgfmathsetmacro{\secondmidanglestop}{mod(\halfangle+\helpanglemid/2+180,360)++\turnanglemid}%
%
\draw[#7] (#1) ++ (\halfangle:\outerradius) arc (\firstmidanglestop:\firstmidanglestart:#6) arc (\halfangle-\helpanglemid:#3+\helpangleedge:\midradius) arc (#3+270+\turnangleedge+\helpangleedge/2:#3+270+\helpangleedge/2:#6) ;%
%
\draw[#7] (#1) ++ (\halfangle:\outerradius) arc (\secondmidanglestop:\secondmidanglestart:#6) arc (\halfangle+\helpanglemid:#4-\helpangleedge:\midradius) arc (#4+90-\turnangleedge-\helpangleedge/2:#4+90-\helpangleedge/2:#6);%
}
\newcommand{\limit}[1]{\displaystyle \lim_{#1}}
\pgfplotsset{compat=1.9}