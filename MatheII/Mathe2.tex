\documentclass[a4paper,12pt,twoside]{article}
\usepackage{fourier}
\usepackage[ngerman]{babel}
\usepackage[leqno,tbtags,nointlimits]{amsmath}
\usepackage{amssymb,amsthm,amsfonts}
\usepackage{graphicx}
\usepackage{ifthen}
\usepackage{tikz}
\usepackage{mathtools}
\usepackage{fancyhdr,lastpage}
\usepackage{enumerate}
\usepackage[onehalfspacing]{setspace}
\usepackage{mdsymbol}
\usepackage{pgfplots}
\usepackage{color}
\usepackage{bigints}
\usepackage{array}
\usepackage{mdframed}
\usepackage{marginnote}
\usetikzlibrary{trees,automata,arrows,shapes}
\pagestyle{fancy}
\usepackage{hyperref}
\fancyhf{} %--Clear all fields
\renewcommand\sectionmark[1]{ \markboth{\thesection\ \textsc{#1}}{}}
\fancyhead[LO,RE]{\normalsize \leftmark}
\fancyhead[LE,RO]{ \rightmark}
\fancyfoot{} % clear all footer fields
\fancyfoot[LE,RO]{\thepage}
\newcommand{\cd}{\cdot}
\newcommand{\C}{\mathbb{C}}
\newcommand{\Z}{\mathbb{Z}}
\renewcommand{\i}{\item}
\newcommand{\U}{\mathcal{U}}
\newcommand{\N}{\mathbb{N}}
\newcommand{\R}{\mathbb{R}}
\DeclarePairedDelimiter{\ceil}{\lceil}{\rceil}
\DeclarePairedDelimiter{\floor}{\lfloor}{\rfloor}
\newcommand{\seriesalg}[3]{\sum\limits_{#3}^{#2} #1}
\newcommand{\seriesn}[1]{\sum\limits_{i=k}^{n} #1}
\newcommand{\seriesnplus}[1]{\sum\limits_{i=k}^{n+1} #1}
\newcommand{\series}[1]{\sum\limits_{i=k}^{\infty} #1}
\newcommand{\seriesnull}[1]{\sum\limits_{i=0}^{\infty} #1}
\usepackage[normalem]{ulem}
\usepackage{blkarray}
\usepackage{stmaryrd}
\usepackage{titletoc}
\usepackage%[margin=15mm]
{geometry}
\newcommand{\abs}[1]{\lvert #1 \rvert}
\renewcommand\headrule{{\color{gray}%
\hrule height 2pt width\headwidth
\vspace{1pt}%
\hrule height 1pt width\headwidth
\vspace{-4pt}}}
\makeatletter
\newcommand{\resetHeadWidth}{\fancy@setoffs}
\makeatother
\newcommand{\cucubr}[7]{%
%origin point, circle radius, start angle, end angle, distance c-b, brace radius, brace options
\pgfmathsetmacro{\helpangleedge}{acos(1-pow(#6,2)/2/pow(#2+#5,2))}%
\pgfmathsetmacro{\turnangleedge}{90+(\helpangleedge/2)}%
\pgfmathsetmacro{\helpanglemid}{acos(1-pow(#6,2)/2/pow(#2+#5+2*#6,2))}%
\pgfmathsetmacro{\turnanglemid}{90-(\helpanglemid/2)}%
\pgfmathsetmacro{\halfangle}{(#4-#3)/2+#3}%
\pgfmathsetmacro{\midradius}{#2+#5+#6}%
\pgfmathsetmacro{\outerradius}{#2+#5+1.88*#6}%
\pgfmathsetmacro{\firstmidanglestart}{mod(\halfangle-\helpanglemid+180,360)}%
\pgfmathsetmacro{\secondmidanglestart}{mod(\halfangle+\helpanglemid+180,360)}%
\pgfmathsetmacro{\firstmidanglestop}{mod(\halfangle-\helpanglemid/2+180,360)-\turnanglemid}%
\pgfmathsetmacro{\secondmidanglestop}{mod(\halfangle+\helpanglemid/2+180,360)++\turnanglemid}%
%
\draw[#7] (#1) ++ (\halfangle:\outerradius) arc (\firstmidanglestop:\firstmidanglestart:#6) arc (\halfangle-\helpanglemid:#3+\helpangleedge:\midradius) arc (#3+270+\turnangleedge+\helpangleedge/2:#3+270+\helpangleedge/2:#6) ;%
%
\draw[#7] (#1) ++ (\halfangle:\outerradius) arc (\secondmidanglestop:\secondmidanglestart:#6) arc (\halfangle+\helpanglemid:#4-\helpangleedge:\midradius) arc (#4+90-\turnangleedge-\helpangleedge/2:#4+90-\helpangleedge/2:#6);%
}
\newcommand{\limit}[1]{\displaystyle \lim_{#1}}
\usepgfplotslibrary{fillbetween}
\pgfplotsset{compat=1.9}
\newcommand{\seriesalg}[3]{\sum\limits_{#1}^{#2} #1}
\newcommand{\seriesn}[1]{\sum\limits_{i=k}^{n} #1}
\newcommand{\seriesnplus}[1]{\sum\limits_{i=k}^{n+1} #1}
\newcommand{\series}[1]{\sum\limits_{i=k}^{\infty} #1}
\newcommand{\seriesnull}[1]{\sum\limits_{i=0}^{\infty} #1}
\usepackage[normalem]{ulem}
\usepackage{blkarray}
\usepackage{stmaryrd}
\usepackage[margin=15mm]{geometry}
\newcommand{\abs}[1]{\lvert #1 \rvert}
\newcommand{\cucubr}[7]{%
%origin point, circle radius, start angle, end angle, distance c-b, brace radius, brace options
\pgfmathsetmacro{\helpangleedge}{acos(1-pow(#6,2)/2/pow(#2+#5,2))}%
\pgfmathsetmacro{\turnangleedge}{90+(\helpangleedge/2)}%
\pgfmathsetmacro{\helpanglemid}{acos(1-pow(#6,2)/2/pow(#2+#5+2*#6,2))}%
\pgfmathsetmacro{\turnanglemid}{90-(\helpanglemid/2)}%
\pgfmathsetmacro{\halfangle}{(#4-#3)/2+#3}%
\pgfmathsetmacro{\midradius}{#2+#5+#6}%
\pgfmathsetmacro{\outerradius}{#2+#5+1.88*#6}%
\pgfmathsetmacro{\firstmidanglestart}{mod(\halfangle-\helpanglemid+180,360)}%
\pgfmathsetmacro{\secondmidanglestart}{mod(\halfangle+\helpanglemid+180,360)}%
\pgfmathsetmacro{\firstmidanglestop}{mod(\halfangle-\helpanglemid/2+180,360)-\turnanglemid}%
\pgfmathsetmacro{\secondmidanglestop}{mod(\halfangle+\helpanglemid/2+180,360)++\turnanglemid}%
%
\draw[#7] (#1) ++ (\halfangle:\outerradius) arc (\firstmidanglestop:\firstmidanglestart:#6) arc (\halfangle-\helpanglemid:#3+\helpangleedge:\midradius) arc (#3+270+\turnangleedge+\helpangleedge/2:#3+270+\helpangleedge/2:#6) ;%
%
\draw[#7] (#1) ++ (\halfangle:\outerradius) arc (\secondmidanglestop:\secondmidanglestart:#6) arc (\halfangle+\helpanglemid:#4-\helpangleedge:\midradius) arc (#4+90-\turnangleedge-\helpangleedge/2:#4+90-\helpangleedge/2:#6);%
}
\begin{document}
\tableofcontents

\section{Komplexe Zahlen}
\subsection{Definition}
Menge der komplexen Zahlen $\C =\{a+bi : a,b \in \R\}$\\
\underline{Addition:}$(a+bi) + (c+di) = (a+c)+(b+d)i$\\
\underline{Multiplikation:}$(a+bi) \cd (c+di) = (ac -bd) + (ad+bc)i$\\
\hspace*{2.75cm}(Ausmultiplizieren und $i^2 =-1$ beachten)\\
$\R \subset \C$\\
$a \in \R : a + 0 \cd i = a$\\
Rein imagin\"are Zahlen : $bi , b \in \R, (0 + bi)$\\
i \underline{imagin\"are Einheit}\\
$z = a+bi \in \C\\
a = \mathfrak{R}(z)$ Realteil von $z (Re(z))\\
b = \mathfrak{I}(z)$ Imagin\"arteil von $z (Im(z))\\
\bar{z} = a - bi (= a + (-b)i)$\\
Die zu $z$ \uline{konjugiert komplexe Zahl}\\
\subsection{Veranschaulichung}
\begin{tikzpicture}
    \begin{axis}[
    legend style={draw none},
    axis equal,
    xlabel={$Re(z)$},
    ylabel={$Im(z)$},
    ymin = -1,
    xmin = 0,
    ymax = 5.5,
    xmax = 5.5,
    xtick ={1,4},
    xticklabels={1,a},
    ytick ={1,3},
    yticklabels={i,b},
    extra x ticks={0},
    extra x tick label={0},
    extra y ticks={0},
    extra y tick labels={0},
    extra tick style = {grid = major}]
    \addplot[dashed,mark=none, black] coordinates {(4,0) (4,3)};
    \addplot[dashed,mark=none, black] coordinates {(0,3) (4,3)};
      \addplot [black, mark = *, nodes near coords=$a+bi$,every node near coord/.style={anchor=180}] coordinates {( 4, 3)};
    \end{axis}
  \end{tikzpicture}\\
Addition entspricht Vektoraddition\\
\begin{tikzpicture}
    \begin{axis}[
    legend style={draw none},
    axis equal,
    xlabel={$Re(z)$},
    ylabel={$Im(z)$},
    ymin = -1,
    xmin = -3,
    ymax = 7,
    xmax = 5.5,
    xtick ={1,4},
    xticklabels={1,a},
    ytick ={1,3},
    yticklabels={i,b},
    extra x ticks={0},
    extra x tick label={0},
    extra y ticks={0},
    extra y tick labels={0},
    extra tick style = {grid = major}]
    \addplot [black, mark = *, nodes near coords=$a+bi$,every node near coord/.style={anchor=180}] coordinates {( 4, 3)};
       \draw[->](axis cs:0,0)--(axis cs:3.88,2.88);
           \addplot [black, mark = *, nodes near coords=$c+di$,every node near coord/.style={anchor=0}] coordinates {( -1, 1)};
                  \draw[->](axis cs:0,0)--(axis cs:-0.88,0.88);
\addplot [black, mark = *, nodes near coords=$(a+c)+(b+d)i$,every node near coord/.style={anchor=0}] coordinates {( 3, 4)};
    \addplot[mark=none, black] coordinates {(-1,1) (3,4)};
    \addplot[mark=none, black] coordinates {(3,4) (4,3)};
    \draw[-> ,red](axis cs:0,0)--(axis cs:2.95,3.88);
    \end{axis}
  \end{tikzpicture}\\
\subsection{Rechenregeln in $\C$}
\begin{enumerate}
\item[a)]Es gelten alle Rechenregeln wie in $\R$. (z.B Kommutativit\"at bzgl. $+,\cd:z_1 + z_2 = z_2 + z_1$ und $z_1 \cd z_2 = z_2 \cd z_1$\\
\underline{Inversenbildung bzgl. $\cd$}:\\
$z = a+bi \not = 0,$ d.h $a \not = 0$ oder $b \not = 0\\
z^{-1} = \frac{1}{z} = \frac{a}{a^2 + b^2} - \frac{b}{a^2+b^2}i\\
z \cd z^{-1} = 1\\
\text{\underline{Beispiel:}}
\begin{array}{lcr}
\frac{5-7i}{3+2i} &=& (5-7i) \cd (3+2i)^{-1}\\
&=& (5-7i) \cd (\frac{3}{13} -\frac{2}{13}i)\\
&=& (\frac{15}{13}-\frac{14}{13}) +(-\frac{10}{13} -\frac{21}{13})i\\
&=&\frac{1}{13} - \frac{31}{13}i
\end{array}$\\
Speziell: $(bi)^{-1} = \frac{1}{bi} = -\frac{1}{b}i$ ; insbesondere: $\frac{1}{i} = -i$
\item[b)]$z,z_1,z_2 \in \C$:\\
\begin{center}
$\bar{\bar{z}}=z$\\
$\overline{z_1+z_2} = \bar{z_1} + \bar{z_2}$\\
$\overline{z_1\cd z_2} = \bar{z_1} \cd \bar{z_2}$\\
\end{center}
\end{enumerate}
\subsection{Definition Absolutbetrag}
\begin{enumerate}
\item[a)] \underline{Absolutbetrag} von $z =a+bi \C:\\
\mid z \mid = +\underbrace{\sqrt{a^2 +b^2}}_{\in \R, \geq 0}\\
$\fbox{$a^2 + b^2 = z \cd \bar{z}$} $\mid z \mid = +\sqrt{z \cd \bar{z}}\\
(a+bi) \cd (a-bi) = (a^2 +b^2) + 0i = a^2 + b^2$\\
    $
  \begin{array}{lcr}
  \lvert z \rvert &=& \text{Abstand von $z$ zu 0}\\
  &=& \text{L\"ange des Vektors, der $z$ entspricht}
  \end{array}$\\
\begin{tikzpicture}
    \begin{axis}[
    legend style={draw none},
    axis equal,
    xlabel={$Re(z)$},
    ylabel={$Im(z)$},
    ymin = -1,
    xmin = 0,
    ymax = 5.5,
    xmax = 5.5,
    xtick ={1,4},
    xticklabels={1,a},
    ytick ={1,3},
    yticklabels={i,b},
    extra x ticks={0},
    extra x tick label={0},
    extra y ticks={0},
    extra y tick labels={0},
    disabledatascaling,
    extra tick style = {grid = major}]
    \addplot[mark=none, black] coordinates {(4,0) (4,3)};
    \addplot[mark=none, black] coordinates {(0,0) (4,0)};
      \addplot [black, mark = *, nodes near coords=$a+bi$,every node near coord/.style={anchor=180}] coordinates {( 4, 3)};
   \draw[->] (axis cs:0,0)--(axis cs:3.95,2.88);
   \draw(axis cs:3.5,0) arc [radius=0.5,start angle=180,end angle=90];
   \addplot[mark=*, black] coordinates {(3.8,0.2)};
    \end{axis}   
  \end{tikzpicture}
  \item[b)]Abstand von $z_1,z_2 \in \C:\\
  d(z_1,z_2):= \mid z_1 - z_2 \mid$\\
\end{enumerate}
\subsection{Rechenreglen f\"ur den Absolutbetrag}
$z,z_1,z_2 \in \C$\\
\begin{enumerate}
\item[a)]$ \mid z \mid = 0 \Leftrightarrow z =0$
\item[b)]$ \mid z_1 \cd z_2 \mid = \abs{z_1} \cd \abs{z_2}$\\
\item[c)]$ \abs{z_1 + z_2} \leq \abs{z_1} + \abs{z_2}\\
\abs{\abs{z_1} -\abs{z_2}} \leq \abs{z_1 - z_2} \leq \abs{z_1} + \abs{z_2}\\
\abs{-z} = \abs{z}$
\end{enumerate}
\subsection{Darstellung durch Polarkoordinaten}
\begin{enumerate}
\item[a)]
Jeder Punkt $\not = (0,0)$ l\"asst sich durch seine Polarkoordinaten $(r,\varphi)$ beschreiben:\\
\begin{tikzpicture}
    \begin{axis}[
    legend style={draw none},
    axis equal,
    xlabel={$Re(z)$},
    ylabel={$Im(z)$},
    ymin = -1,
    xmin = 0,
    ymax = 5.5,
    xmax = 5.5,
    xtick ={1,4},
    xticklabels={1,a},
    ytick ={1,3},
    yticklabels={i,b},
    extra x ticks={0},
    extra x tick label={0},
    extra y ticks={0},
    extra y tick labels={0},
    disabledatascaling,
    extra tick style = {grid = major}]
      \addplot [black, mark = *, nodes near coords=$a+bi$,every node near coord/.style={anchor=180}] coordinates {( 4, 3)};
   \draw[decorate,decoration ={brace,amplitude =10pt},thick] (axis cs:0,0)--(axis cs:4,3);
      \draw[->] (axis cs:0,0)--(axis cs:3.95,2.88);
       \addplot [black, mark = none, nodes near coords=$\varphi$,every node near coord/.style={anchor=180}] coordinates {( 1.90, 0.8)};
       \addplot [black, mark = none, nodes near coords=$r$,every node near coord/.style={anchor=180}] coordinates {( 1.5, 2)};
   \draw[->](axis cs:2,0) arc [radius=1.95,start angle=0,end angle=37];
    \end{axis}   
  \end{tikzpicture}\\
  $-r \geq 0, r \in \R\\
  0 \leq \varphi \leq 2\pi$, wird gemessen von der positiven x-Achse entgegen des Uhrzeigersinnes\\
\begin{tikzpicture}
    \begin{axis}[
    legend style={draw none},
    axis equal,ymin = -2,xmin = -2,ymax = 2,
    xmax = 2,xtick ={},
    xticklabels={},
    ytick ={},
    yticklabels={},
    extra x ticks={0},
    extra x tick label={},
    extra y ticks={0},
    extra y tick labels={},
    disabledatascaling,
    extra tick style = {grid = major}]
    \addplot [black, mark = none, nodes near coords=$\pi$,every node near coord/.style={anchor=180}] coordinates {( -1.5, 0)};
    \addplot [black, mark = none, nodes near coords=2$\pi$,every node near coord/.style={anchor=180}] coordinates {( 1.25, 0)};
    \addplot [black, mark = none, nodes near coords=$\frac{\pi}{2}$,every node near coord/.style={anchor=90}] coordinates {(0, 1.5)};
    \addplot [black, mark = none, nodes near coords=$\frac{3\pi}{2}$,every node near coord/.style={anchor=-90}] coordinates {(0, -1.5)};
    \draw (axis cs:0,0) circle[radius=1];
    \draw(axis cs:0,0)--(axis cs:0.77,0.65);
    \draw(axis cs:0,0)--(axis cs:1,0);
    \draw[->](axis cs:0.3,0) arc [radius=0.3,start angle=0,end angle=37];
    %origin point, circle radius, start angle, end angle, distance c-b, brace radius, brace options
    \cucubr{0,2}{1}{0}{37}{0.1}{0.1}{red}
    \end{axis}   
  \end{tikzpicture}\\
  Umfang:$2\pi$\\
  $\varphi$ in Grad $\widehat{=} \frac{2\pi \cd \varphi}{360}$ im Bogenma\ss\\
   F\"ur Punkte mit kartesischen Koordinaten (0,0) werden als Polarkoordinate $(r,\varphi)$ verwendet.
\item[b)]komplexe Zahl $z = a +ib\\
\begin{array}{rcl}
r &=& \abs{z} = + \sqrt{a^2+b^2}\\
a &=& \abs{z} \cd \cos(\varphi)\\
b &=& \abs{z} \cd \sin(\varphi)\\
z &=& \abs{z} \cd \cos(\varphi) + i \cd \abs{z} \cd \sin(\varphi)\\
z &=& \abs{z}( \cd \cos(\varphi) + i \cd \sin(\varphi))\\
\end{array}\\
\text{Darstellung von $z$ durch Polarkoordinate}$
\item[\underline{Beispiel:}]\begin{enumerate}
\item[a)]$z_1 = 2 \cd (\cos(\frac{\pi}{4})+i \cd \sin(\frac{\pi}{4}))\\
\hspace*{2pt}= 2 \cd (0,5\sqrt{2}+i \cd 0.5\sqrt{2})$
\item[b)]$z_2 = 2+i\\
\abs{z_2} = \sqrt{5}\\
z_2 = \sqrt{5} \cd (\frac{2}{\sqrt{5} + \frac{1}{\sqrt{5}}}i)$
Suche $\varphi$ mit $0 \leq 2\pi$ mit $\cos(\varphi) =\frac{2}{\sqrt{5}}, \sin(\frac{1}{\sqrt{5}} z_2 \approx \sqrt{5} \cd (\cos(0,46) + i \cd \sin(0,46))$
\item[c)] Die komplexen Zahlen von Betrag 1 entsprechen den Punkten auf Einheitskreis:<
$\cos(\varphi) + i \sin(\varphi), 0  \leq \varphi \leq 2\pi$
\end{enumerate}
\end{enumerate}
\subsection{Additionstheoreme der Trigonometrie}
\begin{enumerate}
\item[a)]$\sin(\varphi + \psi) = \sin(\varphi) \cd \cos(\psi) + \cos(\varphi) \cd \sin(\psi)$\\
\item[b)]$\cos(\varphi + \psi) = \cos(\varphi) \cd \cos(\psi) - \sin(\varphi) \cd sin(\varphi) \cd \sin(\psi)$
\end{enumerate}
\subsection{geometrische Interpretation der Multiplikation}
\begin{enumerate}
\item[a)]$w = \abs{w} \cd (\cos(\varphi) + i \cd \sin(\varphi))\\
z= \abs{z} \cd (\cos(\psi) + i \cd \sin(\psi))\\
w \cd z = \abs{w} \cd \abs{z} \cd (\cos(\varphi) \cd \cos(\psi) - \sin(\varphi) \cd \sin(\psi)) + i (\sin(\varphi) \cd \cos(\psi) + \cos(\varphi) \cd \sin(\psi))\\
w \cd z =\abs{w \cd z}( \cos(\varphi + \psi) + i \cd \sin(\varphi + \psi))$\\
\begin{tikzpicture}
\begin{axis}[legend style={draw none},axis equal,xlabel={$Re(z)$},ylabel={$Im(z)$},ymin = -10,xmin = -10,ymax = 10,xmax = 10,xtick ={1,4},xticklabels={},ytick ={1,3},yticklabels={},extra x ticks={0},extra x tick label={},extra y ticks={0},extra y tick labels={},disabledatascaling,extra tick style = {grid = major}]
      \addplot [black, mark = *, nodes near coords=$w$,every node near coord/.style={anchor=180}] coordinates {( 4, 3)};
            \addplot [black, mark = *, nodes near coords=$z$,every node near coord/.style={anchor=0}] coordinates {( -1, 2)};
   \draw (axis cs:0,0)--(axis cs:-1,2);
   \draw (axis cs:0,0)--(axis cs:4,3);
   \draw[decorate,decoration ={brace,amplitude =10pt}] (axis cs:0,0)--(axis cs:-6.7,0);\\
   \draw(axis cs:0,0)--(axis cs:-6.7,0);
   \addplot [black, mark = *, nodes near coords=$w\cd z$,every node near coord/.style={anchor=0}] coordinates {( -6.70, 0)};
     \addplot [black, mark = none, nodes near coords=$\abs{w} \cd \abs{z}$,every node near coord/.style={anchor=0}] coordinates {( -0.8, -2)};
       \addplot [black, mark = none, nodes near coords=$\varphi$,every node near coord/.style={anchor=180}] coordinates {( 1.90, 0.8)};
   \draw[->](axis cs:2,0) arc [radius=1.95,start angle=0,end angle=37];
      \draw[->](axis cs:3,0) arc [radius=2,start angle=0,end angle=142];
    \addplot [black, mark = none, nodes near coords=$\psi$,every node near coord/.style={anchor=180}] coordinates {( 2.90, 0.8)};  
    \end{axis} 
  \end{tikzpicture}
  \item[b)]$z =i, w = a+ ib\\
  i\cd w = -b \cd ia$\\
  Multiplikation mit i $\widehat{=}$ Drehung um $90^{\circ}$\\
\begin{tikzpicture}
\begin{axis}[legend style={draw none},axis equal,xlabel={$Re(z)$},ylabel={$Im(z)$},ymin = 0,xmin = -5,ymax = 5,xmax = 7,xtick ={-3,3},xticklabels={-a,a},ytick ={1,4},yticklabels={i,-b},extra x ticks={0},extra x tick label={0},extra y ticks={0},extra y tick labels={},disabledatascaling,extra tick style = {grid = major}]
\draw [dashed] (axis cs:0,0)--(axis cs:3,4);
\draw [dashed] (axis cs:0,0)--(axis cs:-3,4);
   \addplot [black, mark = *, nodes near coords=$w$,every node near coord/.style={anchor=0}] coordinates {( 3, 4)};
      \addplot [black, mark = *, nodes near coords=$i \cd w$,every node near coord/.style={anchor=0}] coordinates {( -3, 4)};
      \draw[->](axis cs:3,4) arc [radius=3,start angle=0,end angle=180];      
    \end{axis} 
  \end{tikzpicture} 
\end{enumerate}
\subsection{Bemerkung und Definition}
Wir werden sp\"ater die komplexen Exponentialfunktion einf\"uhren.\\
$e^{z}$ f\"ur alle $z \in \C$ e = Euler`sche Zahl $\approx 2,718718\ldots\\
e^{z_1} =cd e^{z_2} = e^{z_1+z_2}, e^{-z} = \frac{1}{e^z}$\\
Es gilt: $t \in \R : e^{it} = \cos(t) + i \cd \sin(t)$\\
Jede komplexe Zahl l\"asst sich schreiben $z = r \cd e^{i \cd \varphi}, r =\abs{z}, \varphi$ Winkel\\
$r \cd (\cos(\varphi) + i \sin(\varphi))$ ist Polarform von $z.\\
z = a+bi$ ist kartesische Form von z.
$\bullet (r,\varphi)$ Polarkoordinaten\\
$\abs{e^{i\varphi}} = + \sqrt{cos^2(\varphi) + \sin^2(\varphi)}=1\\
e^{i\varphi},0 \leq \varphi \leq 2\pi,$ Punkte auf dem Einheitskreis.\\
$e^{i\pi} = -1$\\
\fbox{$e^{i\pi} +1 =0$} Eulersche Gleichung
\subsection{Satz}
Sei $w = \mid w \mid \cd ( \cos(\varphi) + i \cd \sin(\varphi)) \in \C$\\
\begin{enumerate}
\item[a)] Ist $m \in \Z$, so ist $w^m = \mid w \mid^m \cd ( \cos(m \cd \varphi) + i \cd \sin(m \cd \varphi))\\
( m < 0 : w^m = \frac{1}{w^{\mid m \mid}}), w \not = 0$\\
\item[b)] Quadratwurzeln
\item[c)] Ist $n \in \N, w \not = 0$, so gibt es genau n n-te Wurzeln von $w:\\
\sqrt[n]{w} = + \sqrt[n]{\mid w \mid} \cd (\cos(\frac{\varphi}{n} + \frac{2\pi \cd k}{n}) + i \sin(\frac{\varphi}{n}+ \frac{2\pi \cd k}{n})), n \in \N  ,k \in \{0, \ldots,n-1\}$
\end{enumerate}
\begin{proof}
a) richtig, wenn $m=0,1$\\
$m \geq 2.$ Folgt aus ($\star$)\\
$m=-a:\\
w^{-1} = \frac{1}{w} = \frac{1}{\mid w \mid^2 \cd ( \cos^2(\varphi) + i \cd \sin^2( \varphi))} \cd \mid w \mid \cd \cos(\varphi) - \sin(\varphi)\\
= \frac{1}{w} = \frac{1}{mid w \mid \cd \underbrace{( \cos^2(\varphi) + i \cd \sin^2( \varphi))}}_{=1} \cd \mid w \mid \cd \cos(\varphi) - \sin(\varphi)\\
= \frac{1}{\mid w \mid } \cd (\cos(-\varphi + i \cd \sin(-\varphi)) = \mid w \mid^{-1} \cd (\cos(-\varphi) + \sin(-\varphi))$\\
\end{proof}
\subsection{Beispiel}
Quadratwurzel aus $i:\\
\abs{i} =1\\
\begin{array}{llcl}
\text{Nach 1.10 b):} &\sqrt{i}&=& \pm (\cos(\frac{\pi}{4} + i \cd \sin(\frac{\pi}{4}))\\
& &=& \pm (\frac{1}{2}\sqrt{2} + \frac{1}{2}\sqrt{2}i)
\end{array}
$
\subsection{Bemerkung}
Nach 1.10 hat jedes Polynom\\
$ x^{n} -w$ ($w \in \C)$\\
eine Nullstelle in $\C$ (sogar n verschiedene wenn $w \not = 0$)\\
Es gilt sogar : \underline{Fundamentalsatz der Algebra}\\
\hspace*{7.5cm} (C. F. Gau\ss 1777-1855)\\
Jedes Polynom $a_n x^n + \ldots + a_0$\\
mit irgendwelchen Koeffizienten: $a_n \ldots a_0 \in \C$ hat Nullstelle in $\C$
\section{Folgen und Reihen}
\subsection{Definition}
Sei k $\in \Z$,
$A_k := \{ m \in \Z : m > k\}\\
(k = 0 A_0 \in \N_0, k = 1 , A_n \in \N)\\
Abbildung a : A \Rightarrow \R (oder \C)\\
\hspace*{2.5cm} m \Rightarrow a_n\\
$hei\ss t \underline{Folge} reeller Zahlen\\
\hspace*{2.5cm} ($a_k,a_{k-1} \ldots)$\\
Schreibweise:\\
$(a_m)_{m>k}$ oder einfach $(a_m)$\\
$a_m$ hei\ss t \underline{m-tes Glied} der Folge, m \underline{Index}\\
\subsection{Beispiel}
\begin{enumerate}
\item[a)] $a_n = 5$ f\"ur alle $n > 1$\\
$(5,5,5,5,5,5,5,5,5,5,5,5,5,\ldots$)\\
\item[b)] $a_n = n$ f\"ur alle $n>1$\\
 (1,2,3,4,5,6,7,8,9,10,$\ldots$)\\
\item[c)] $a_n = \frac{1}{n}\\
(\frac{1}{1}, \frac{1}{2} ,\frac{1}{3}, \frac{1}{4}, \ldots)$\\
\item[d)]$a_n \frac{(n+1)^2}{2^n}\\
(2, \frac{9}{4},2,\frac{25}{16}, \ldots)$\\
\item[e)]$a_n = (-1)^n\\
(-1,1,-1,1,-1,1,\ldots)\\ $
\item[f)]$a_n = \frac{1}{2}a_{n_1} = \frac{1}{a_{n-1}} $f\"ur $n\geq 2,
a_1 =1\\
(1,\frac{3}{2},\frac{17}{12}, \ldots)$
\item[g)]$a_n = \sum_{i=1}^{n} \frac{1}{i}\\
(1,\frac{3}{2}, \frac{11}{6},\ldots)$
\item[h)]$a_n = \sum_{i=1}^{n} (-1)^i\cd \frac{1}{i}\\
(-1,\frac{-1}{2},-\frac{-5}{6},\ldots)$
\end{enumerate}
\subsection{Definition}
Eine Folge $(a_n)_{n>k}$ hei\ss t \underline{beschr\"ankt}, wenn die Menge der Folgenglieder beschr\"ankt ist. \\
D.h. $\exists D > 0 : - D \leq a_n \leq D$ f\"ur alle $n > k$.\\
\begin{tikzpicture}
\begin{axis}[legend style={draw none},axis equal,ymin = -4,xmin = 0,ymax = 4,xmax = 7,xtick ={1,2,3,4,5},ytick ={3,-3},yticklabels={D,-D},extra x ticks={0},extra x tick label={0},extra y ticks={0},extra y tick labels={},extra tick style = {grid = major}]
\draw [dashed] (axis cs:-10,3)--(axis cs:10,3);
\draw [dashed] (axis cs:-10,-3)--(axis cs:10,-3);
\addplot [black, mark = +] coordinates {( 1, 0.3)};
\addplot [black, mark = +] coordinates {( 2, 0.1)};
\addplot [black, mark = +] coordinates {( 3, -2)};
\addplot [black, mark = +] coordinates {( 4, 1)};
\addplot [black, mark = +] coordinates {( 5, 0.5)};
               
\end{axis}
  \end{tikzpicture} 
\subsection{Definiton}
Eine Folge $(a_n)_{n \geq k}$ hei\ss t \underline{konvergent} gegen $\varepsilon \in \R$ (konvergent gegen $\varepsilon$), falls gilt:\\
$\forall \varepsilon > 0 \exists n(\varepsilon) \in \N \forall n \geq n(\varepsilon) : \mid a_n -c \mid < \varepsilon\\
c = \lim_{n \Rightarrow \infty} a_n $ (oder einfach $c = \lim a_n$)\\
c hei\ss t \underline{Grenzwert} (oder Limes) der Folge ($a_n$)\\
(Grenzwert h\"angt nicht von endlich vielen Anfangsgliedern ab (der Folge))\\
Eine Folge die gegen 0 konvertiert, hei\ss t \underline{Nullfolge}\\
\subsection{Beispiele}
\begin{enumerate}
\item[a)]$r \in \R : a_n = r$ f\"ur alle $n \geq 1$\\
$(r,r,\ldots)\\
\lim_{n \Rightarrow \infty} = r\\
\mid a_n - r \mid = 0$ f\"ur alle $n$\\
F\"ur jedes $\varepsilon >0$ kann man $n(\varepsilon) = 1$ w\"ahlen
\item[b)]$a_n = n$ f\"ur alle $ n\ \geq 1$\\
Folgte ist nicht beschr\"ankt, konvergiert nicht.
\item[c)] $a_n = \frac{1}{n}$ f\"ur alle $n \geq 1\\
(a_n)$ ist Nullfolge.\\
Sei $ \varepsilon > 0$ beliebig. Suche Index $n(\varepsilon)$ mit $\abs{a_n - o} < \varepsilon$ f\"ur alle $n \geq n(\varepsilon)$\\
D.s. es muss gelten.\\
$\frac{1}{n} < \varepsilon$ f\"ur alle $n \geq n(\varepsilon)$\\
Ich brauche : $\frac{1}{n(\varepsilon)} < \varepsilon$\\
Ich brauche $n(\varepsilon) > \frac{1}{\varepsilon}$\\
Aus Mathe I folgt, dass solch ein $n(\varepsilon)$ existiert.\\
z.B $n(\varepsilon) - \ceil{\frac{1}{2}} + 1 > \frac{1}{\varepsilon}$\\
Dann:\\
$\abs{a_n -0} < \frac{1}{n} < \varepsilon$ f\"ur alle $n \geq n(\varepsilon)$
\item[d)]$a_n = \frac{3n^2+1}{n^2+n+1}$ f\"ur lle $n \geq 1$\\
Behauptung: $\lim_{n\Rightarrow\infty}a_n = 3\\
\begin{array}{rlll}
\abs{a-3} &=& \abs{\frac{3n^2+1}{n^2+n+1}-3}&= \abs{\frac{3n^2+1-3(n^2+n+1)}{n^2+n+1}}\\
&=& \abs{\frac{-3n-2}{n^2+n+1}} &= \frac{3n+2}{n^2+n+1}
\end{array}$\\
Sei $\varepsilon > 0$. Ben\"otigt wird $n(\varepsilon) \in \N$ mit $\frac{3n+2}{n^2+n+1} < \varepsilon$ f\"ur alle $n > n(\varepsilon).\\
\frac{3n+2}{n^2+n+1} \leq \frac{5n}{n^2 } = \frac{5}{n}$\\
W\"ahle $n(\varepsilon)$ so, dass $n(\varepsilon) > \frac{5}{\varepsilon}$\\
Dann gilt f\"ur alle $n\geq n(\varepsilon).\\
\abs{a_n -3}=\frac{3n+2}{n^2+n+1} \leq \frac{5}{n} \leq  \frac{5}{n(\varepsilon)} < \frac{5\varepsilon}{5} = \varepsilon$\\
F\"ur alle $ n \geq n(\varepsilon)$
\item[e)]$a_n = (-1)^n$ beschr\"ankte Folge $-1 \leq a_ \leq 1$ konvergiert nicht.\\
Sei $c \in \mathbb{R}$ beliebig, W\"ahle $\varepsilon = \frac{1}{2}$\\
\begin{tikzpicture}
\begin{axis}[legend style={draw none},axis equal,ymin = -2,xmin = 0,ymax = 2,xmax = 6,xtick ={1,2,3,4,5},ytick ={3,-3},yticklabels={D,-D},extra x ticks={0},extra x tick label={0},extra y ticks={0},extra y tick labels={},extra tick style = {grid = major}]
\addplot [black, mark = +] coordinates {( 1, -1)};
\addplot [black, mark = +] coordinates {( 2, 1)};
\addplot [black, mark = +] coordinates {( 3, -1)};
\addplot [black, mark = +] coordinates {( 4, 1)};
\addplot [black, mark = +] coordinates {( 5, -1)};
               
\end{axis}
  \end{tikzpicture}\\
$2 = \abs{a_n-a_{n+1}} \leq \abs{a_n - c} + \abs{c - a_{n+1}} < \frac{1}{2}+ \frac{1}{2} =$\underline{$1$} $\lightning$
\end{enumerate}
\subsection{Satz}
Jede konvergente Folge ist beschr\"ankt. (Umkehrung nicht: $2.5_{e)}$)\\
\begin{proof}
Sei $c= \lim a_n$, w\"ahle $\varepsilon =1$,\\
Es existiert $n(1) \in \N$ mit $\abs{a_n -c} < 1$ f\"ur alle $n \geq n(1)$\\
Dann ist\\
$\abs{a_n} = \abs{a_n-c+c} \leq \abs{a_n -c} + \abs{c} < 1 + \abs{c}$ f\"ur alle $n \geq n(1)\\
M = max \{ \abs{a_k},\abs{a_{k+1}},\ldots,\abs{a_{n(1)-1)}, 1 + \abs{c} \}\\
$ Dann: $\abs{a_n} \leq M$ f\"ur alle $n \geq k\\
-M \leq a_n \leq M$
\end{proof}
\subsection{Bemerkung}
\begin{enumerate}
\item[a)]$(a_n)_{n \geq 1}$ Nullfolge $\Leftrightarrow (\abs{a_n})_{n \geq 1}$ Nullfolge ($\abs{a_n - 0} = \abs{a_n} - \abs{\abs{a_n}-0}}$
\item[b)]$\lim_{n\Rightarrow\infty} a_n = c \Leftrightarrow (a_n -c)_{n \geq k}$ ist Nullfolge $\Leftrightarrow (\abs{a_n -c})_{n \geq k}$ ist Nullfolge
\end{enumerate}
\subsection{Satz (Rechenregeln f\"ur konvergente Folgen)}
Seien $(a_n)_{n \geq k}$ und $(b_n)_{n \geq k}$ konvergente Folgen, $\lim a_n = c, \lim b_n = d.$\\
\begin{enumerate}
\item[a)]$\lim \abs{a_n} = \abs{c}$
\item[b)]$\lim (a_n \pm b_n) = c \pm d$
\item[c)]$\lim (a_n \cd b_n) = c \cd d$\\
insbesondere $\lim(r \cd b_N) = r \cd \lim b_n = r \cd d$ f\"ur jedes $r \in \R.$
\item[d)]Ist $b_n \not = 0$ f\"ur alle $n \geq k$ und ist $ d \not = 0$, so $\lim (\frac{a_n}{k_n}) = \frac{c}{d}$
\item[e)]Ist $(b_n)$ Nullfolge, $b_n \not = 0$ f\"ur alle $n \geq k$, so konvergiert $(\frac{1}{b_n}$ \underline{nicht}!.
\item[f)]Existiert $m \geq k$ mit $a_n \leq b_n$ f\"ur alle $n \geq m$, so ist $c \leq d$.
\item[g)] Ist $(c_n)_{n \geq k}$ Folge und existiert $ m \geq k$ mit $0 \leq c_n \leq a_n$ f\"ur alle $n \geq m$ und ist ($a_n$) eine Nullfolge, so ist auch $(c_n)$ eine Nullfolge.
\item[h)]Ist $(c_n)_{n \geq l}$ beschr\"ankte Folge und ist $(a_n)_{n \geq k}$ Nullfolge, so ist auch $(c_n \cd a_n)_{n \geq k}$ Nullfolge.\\
\fbox{$c_n$ muss nicht konvergieren!}
\end{enumerate}
\begin{proof}
Exemplarisch:\\
\begin{enumerate}
\item[b)] Sei $\varepsilon > 0$. Dann existiert $n_1(\frac{\varepsilon}{2})$ und $n_2(\frac{\varepsilon}{2})$ und $\abs{a_n -c} < \frac{\varepsilon}{2}$ f\"ur alle $n \geq n_1(\frac{\varepsilon}{2})\\
\abs{b_n-d} < \frac{\varepsilon}{2}$ f\"ur alle $n \geq n_2(\frac{\varepsilon}{2})$\\
Suche $n(\varepsilon) = max (n_1(\frac{\varepsilon}{2},n_2(\frac{\varepsilon}{2}))\\
$Dann gilt f\"ur alle $n > n(\varepsilon):\\
\abs{a_n+b_n - (c+d)} = \abs{(a_n -c) + (b_n -d)} \leq \abs{a_n -c} + \abs{b_n -d} < \frac{\varepsilon}{2} + \frac{\varepsilon}{2} = \varepsilon$
\item[f)]Angenommen $c > d$. Setze $\delta = c -d >0$\\
Es existiert \~m $ \geq m$ mit $\abs{c -a_n} < \frac{\delta}{2}$\\
und $\abs{b_n -d} < \frac{\delta}{2}$ f\"ur alle $n \geq$\~m.\\
F\"ur diese n gilt:\\
$0 < \delta \leq \delta + b_n - a_n = c -d+b_n -a_n \geq 0$ nach Voraussetzung\\
$= \abs{c -a_n - d +b_n} \leq \abs{c-a_n} + \abs{d -b_n}\\
\leq \frac{\delta}{2} + \frac{\delta}{2} = \delta \lightning$
\end{enumerate}
\end{proof}
\subsection{Satz}
\begin{enumerate}
\item[a)] $0 \leq q \leq 1$ Dann ist $(q^n)_{n \geq 1}$ Nullfolge
\item[b)]Ist $m \in \N$, so ist $((\frac{1}{n^m})_{n \geq 1}$ Nullfolge.
\item[c)]Sei $0 \leq q < 1, m \in \N$\\
Dann ist ($n^m \cd q^n)_{n \geq 1}$ Nullfolge
\item[d)] Ist $r > 1, m \in \N$, so ist $(\frac{n^m}{r^n}_{r \geq 1}$ eine Nullfolge
\item[e)] $P(x) = a_m \cd x^m + \ldots a_0, a_i \in \R, a_m \not = 0\\
Q(x) = b_e \cd x^e + \ldots b_0 , b_i \in \R , b_e \not = 0$\\
Sei $Q(n) \not = 0 $ f\"ur alle $n \geq k.$\\
\begin{enumerate}
\item[-]Ist $m>e$, so ist $\frac{P(n)}{Q(n)}$ nicht konvergent
\item[-]Ist $m = e$, so ist $\lim_{n\Rightarrow \infty} \frac{P(n)}{Q(n)}= \frac{a_m}{b_e} = \frac{a_m}{b_m}$\\
\item[-]Ist $m < l$, so ist$ (\frac{P(n)}{Q(n)})$ ein Nullfolge
\end{enumerate}
\end{enumerate}
\begin{enumerate}[a)]
\i Sei $0 \leq q \leq 1$ Dann ist $(q^n)_{n \geq}$ eine Nullfolge\\
\begin{proof}
a) Richtig für $q > 0$. Sei jetzt $q > 0$.\\
Sei $\varepsilon > 0$. Mathe I: Es gibt ein $n(\varepsilon) \in \N$ mit $q^{n(\varepsilon} < \varepsilon.$\\
Für alle $n \geq n(\varepsilon)$ gilt: $\abs{q^{n} -o} = q^n < q^{n(\varepsilon)} < \varepsilon.$\\
\end{proof}
\i 2.5.c): $\frac{1}{n}_{n \geq 1}$ Nullfolge Beh. folgt mit 2.8.c)\\
\i Richtig für $q=0$. Sei jetzt $q > 0$.\\
\underline{1.Fall}: m= 1\\
$\frac{1}{q} = 1+t, t > 0.\\
(t+1)^n \underbrace{=}_{Binomialsatz} 1 + nt + \frac{n(n+1)}{2}t^2 > \frac{n(n-1)}{2}t^2$ für alle $n \geq 2\\
q^n = \frac{1}{(1+t)^n} < \frac{2}{n(n-1)t^2}\\
0 \leq n \cd q^{n} < \frac{2}{(n-1)t^2} \Leftarrow$ Nullfolge 2.5e),2.8e)\\
Nach 2.9g) ist $(n \cd q^n)_{n \geq q}$ Nullfolge, also auch $(n \cd q^n)_{n \geq 1}.$\\
\underline{2.Fall}: $m > 1$.\\
Setze $0 < q' = \sqrt[m]{q} \in \R\\
\begin{array}{rcl}
n^m \cd q^n &=& n^m \cd (q')^n)^m)^n\\
&=& (n \cd (q')^n)^m)^n) m = 1 \text{anwenden}
\end{array}\\
0 < q' < 1\\
(n^m + q^n)_{n \geq 1}$ Nullfolge noch Fall $m=1$ und 2.8e)
\i Folgt aus c) und $q = \frac{1}{r}$
\i Ist $ m \leq l$ , so ist  $\frac{P(n(}{Q(n)} =\frac{n^m(a_m + a_{m-1}\cd\frac{1}{n} + \ldots + a_1 \cd \frac{1}{n^{m-1}} + a_0 \cd \frac{1}{n^m})}{n^l(b_l + b_{l-1}\cd\frac{1}{n} + \ldots + b_1 \cd \frac{1}{n^{l-1}} + b_0 \cd \frac{1}{n^l})}
= \frac{1}{n^{l-m}} \cd \frac{I}{II}\\
(I) \longrightarrow a_m , (II) \longrightarrow b_l$
$\frac{(I)}{(II)} \Rightarrow \frac{a_m}{b_l}\\
n < l, \frac{1}{n^{l-m}}$ Nullfolge\\
$\frac{P(n)}{Q(n)} \Rightarrow 0 \cd \frac{a_m}{b_l}\\
m > l:$\\
Beh. folgt aus Fall $m < l$ und 2.8e).
\end{enumerate}
\subsection{Bemerkung}
Betrachte Bijektionsverfahren, der Zahl $x \in \R$ bestimmt.\\
$a_0 \leq a_1 \leq a_2 \leq \ldots\\
b_0 \geq b_1 \geq b_2 \geq \ldots\\
a_n \leq x \leq b_n\\
0 < b_n - a_n = \frac{b_0 - a_0}{2^n}\\
0 \leq \abs{x-a_n} \leq b_n - a_n = \frac{b_0-a_n}{2} \Leftarrow$ Nullfolge (2.9b)\\
$2.8e) (\abs{x-a_n})$ Nullfolge.\\
2.7e): $\lim_{n \rightarrow \infty} a_n = x$\\
Analog: $\lim_{n \rightarrow \infty} b_n = x\\
$2.9 d) e) sind Beispiele für asymptotischen Vergleich von Folgen
\subsection{Definition}
\begin{enumerate}[a)]
\i Eine Folge $(a_n)_{ n \geq k}$ hei\ss t \underline{strikt positiv}, falls $a_n > 0$ für alle $n \geq k$.\\
Sei im Folgenden $(a_n)_{n \geq k}$ eine strikt positive Folge.\\
\i $\begin{array}{rcl}
\mathbb{O}(a_n) &=& \{ (b_n)_{n \geq k} : \text{ist beschränkt} \}\\
&=& \{ (b_n)_{n \geq k} \exists C > 0 \text{ mit } \abs{b_n} \leq C \cd a_n \}
\end{array}$
\i $O(a_n) = \{(b_n)_{n \geq k}: (\frac{b_n}{a_n} \text{ist Nullfolge} \}\\
(b_n) \in o(a_n)$ hei\ss t Folge $(a_n)$ wächst wesentlich schneller als die Folge $(b_n)$.
Klar: $o(a_n) \subset O(a_n)\\
O,o ($ \glqq gro\ss \ Oh\grqq, \glqq klein Oh\grqq)\\
\underline{Landau-Symbole}\\
$\begin{array}{lcrll}
\text{z.B}& (n^2) &\in& o(n^3)\\
& (n^2+n+1) &\in& O(n^2) & n^2 + n + 1 \leq 3n^2\\
& (n^2) &\in&O(n^2 + n +1) &n^2 \leq n^2 + n + 1
\end{array}\\
O(1) =$ Menge der beschränkten Folgen\\
$o(1)=$ Menge aller Nullfolgen\\
Häufig gewählte schreibweise:\\
$n^2 \underbrace{=}_{\text{eig. falsch!}} o(n^2)$ statt $(n^2) \in o(n^3)\\
n^2+n+1 =O(n^2)$ statt $(n^2+n+1)$
\end{enumerate}
\subsection{Satz}
Sei $P(x) = a_m \cd x^m + \ldots + a_1 \cd x + a_o, m \geq 0, a_m \not = 0$.
\begin{enumerate}[a)]
\i $(P(n)) \in o(n!)$ für alle $l>m$ und\\
$(P(n)) \in O(n')$ für alle $l \geq m$.
\i ist $r > 1$, so ist $(P(n)) \in o(r^n).\\
\lbrack(r^n)$ wächst deutlich schneller als $(P(n))\rbrack$\\
\begin{proof}
a) folgt aus 2.9e).\\
$m = l$ (2.6)\\
b) folgt aus 2.9d) und 2.8 b)c)
\end{proof}
\end{enumerate}
\subsection{Bemerkung}
Algorihmus:\\
Sei $t_n =$ maximale Anzahl von Reihenschritten des Algorithmus' bei Input der Länge n (binär codiert)\\
Worst-Case-Komplexität:\\
Algorithmus hat polynomielle Zeitkomplexität, falls ein $l \in \N$ existiert mit $(t_n) \in O(n^l)$. (\underline{gutartig})\\
Algorithmus hat polynomielle Zeitkomplexität, falls ein $l \in \N$ existert mind. exponentielle Zeitkomplexität, falls 
$ r > 1$ exestiert mit $(r^n) \in O(b_n)$ (\underline{bösartig)}
\subsection{Definition}
\begin{enumerate}[a)]
\i Eine Folge $(a_n)_{n \geq k}$ hei\ss t \underline{monoton wachsend (steigend)}, wenn $a_n \leq a_{n+1}$ f\"ur alle $n \geq k$. Sei hei\ss t \underline{steng monoton wachsend (steigend)}, wenn $a_n < a_{n+1}$ f\"ur alle $n \geq k$
\i $(a_n)_{n \geq k}$ hei\ss t \underline{monoton fallend}, falls $a \geq a_{n+1}$ f\"ur alle $n \geq k$
\end{enumerate}
\subsection{Beispiel}
\begin{enumerate}[a)]
\i $a_n = 1$ f\"ur alle $n > 1
(a_n)$ ist monoton steigend und monoton fallend.
\i $a_n = \frac{1}{n}$ f\"ur alle $n \geq 1$.\\
$(a_n)$ streng monoton fallend.
\i $a_n = \sqrt{n}$ (positive Wuzel)\\
$(a_n){n \geq 1}$ streng monoton steigend.
\i $a_n = 1 - \frac{1}{n}, n \geq 1\\
(a_n)_{n \geq 1}$ streng monoton steigend.
\i $a_n = (-1)^n, n \geq 1\\
(a_n)$ ist weder monoton steigend noch monoton fallend.
\end{enumerate}
\subsection{Satz}
\begin{enumerate}[a)]
\i Ist $(a_n)_{n \geq k}$ monoton steigend und nach oben beschränkt (d.h es exestiert $D \in \R$ mit $a_n \leq D$ für alle $n \geq k$), so konvergiert $(a_n)'$ und $\lim\limits_{n \rightarrow \infty}a_n = \sup\{a_n: n\geq k \}$
\i $(a_n)_{n \geq k}$ monoton fallend und nach unten beschränkt, so konvergiert $(a_n)_{n \geq k}$ und $\lim\limits_{n \rightarrow \infty} a_n = \inf \{a_n: n\geq k \}.$
\begin{proof}
a)\\
$ c \sup \{a_n : n \geq k\}.$ existiert (Mathe I).
Zeige: $\lim\limits_{a_n} = c$.\\
Sei $\varepsilon > 0$. Dann existiert $n(\varepsilon)$ mit $c-\varepsilon < a_{n(\varepsilon)} \leq c$\\
Denn sonst $a_n \leq c - \varepsilon$ für alle $n \geq k$ und $c - \varepsilon$ wäre obere Schranke für $\{a_n : n \geq k \}$ Widerspruch dazu, dass c kleinste obere Schranke. Für alle $n \geq n(\varepsilon)\\
c- \varepsilon \leq a_{n(\varepsilon)} \leq a_n \leq c\\
\abs{a_n -c} < \varepsilon$ für alle $n \geq n(\varepsilon).$\\
b) analog
\end{proof}
\end{enumerate}
\subsection{Satz (Cauchy'sches Konvergenzkriterium)}
\hspace*{6.8cm} (Cauchy, 1789 - 1859)\\
Sei $(a_n)_{n \geq k}$ eine Folge. Dann sind äquivalent:\\
\begin{enumerate}[(1)]
\i $(a_n)_{n \geq k}$ konvergent
\i $\forall \varepsilon > 0 \exists \ N -M(\varepsilon) \forall n,m \geq N: \abs{a_n - a_m} < \varepsilon$ (Cauchyfolge)\\
Grenzwert muss nicht bekannt sein!
\end{enumerate}
\begin{tikzpicture}
\begin{axis}[legend style={draw none},axis equal,ymin = -4,xmin = 0,ymax = 4,xmax = 7,xtick ={1, 6}, xticklabels={N(1),N($\varepsilon$)},ytick ={3,-3},yticklabels={D,-D},extra x ticks={0},extra x tick label={0},extra y ticks={0},extra y tick labels={},extra tick style = {grid = major}]
\draw [dashed] (axis cs:-10,3)--(axis cs:10,3);
\draw [dashed] (axis cs:-10,-3)--(axis cs:10,-3);
\draw [red] (axis cs:-10,0.1)--(axis cs:10,0.1);
\draw [red] (axis cs:-10,1)--(axis cs:10,1);
\addplot [black, mark = +] coordinates {( 1, 0.3)};
\addplot [black, mark = +] coordinates {( 1, 0.3)};
\addplot [black, mark = +] coordinates {( 2, 0.1)};
\addplot [black, mark = +] coordinates {( 3, -2)};
\addplot [black, mark = +] coordinates {( 4, 1)};
\addplot [black, mark = +] coordinates {( 5, 0.5)};
\addplot [black, mark = +] coordinates {( 8, 0.2)};
\addplot [black, mark = +] coordinates {( 6, 0.5)};
\addplot [black, mark = +] coordinates {( 7, 0.3)};
\addplot [black, mark = none, nodes near coords=$\varepsilon$,every node near coord/.style={anchor=180}] coordinates {( 5, -2)};
\draw [-> , red] (axis cs: 5,-2)--(axis cs: 5, 0.25);              
\end{axis}
  \end{tikzpicture}
\subsection{Definition}
\begin{enumerate}[a)]
\i Sei $(a_i)_{i \geq k}$ eine Folge, $s_n \sum\limits_{i = k}^{n} a_i , n \geq k$ (Partiealsummen der Folge)\\
Dann hei\ss t $(s_n)_{n \geq k}$ eine \underline{unendliche Reihe}\\
$(k-1: a_1, a_1+a_2, a_1 + a_2 + a_2,\ldots)$\\
Schreibweise : $\sum\limits_{i = k}^{\infty} a_i$\\
\i Ist die Folge $(s_n)_{n \geq k}$ konvergent mit $\lim\limits_{n \rightarrow \infty} s_n = c$,\\
so schreibt man $\sum\limits_{i = k}^{\infty} a_i =c.$ Reihe \underline{konvergiert}.\\
Wenn $(s_n)$ nicht konvergiert, so hei\ss t die Reihe $\sum\limits_{i =k}^{\infty} a_i$ \underline{divergent.}\\
(Zwei Bedeutungen von $\sum\limits_{i = k}^{\infty} a_i:$\\
\begin{enumerate}[-]
\i Folge der Partialsummen\\
\i Grenzwert von $(s_n)$, falls dieser existiert
\end{enumerate}
$\sum\limits_{i=k}^{\infty} a_i = \sum\limits_{n=k}^{\infty} a_n = (s_m)_{m \geq k}$
\end{enumerate}
\subsection{Satz}
\begin{enumerate}[a)]
\i Ist die Reihe $\series{a_1}$ konvergent, so ist $(a_1)_{i \geq k}$ eine Nullfolge.
\i Ist die Folge der Partialsummen $s_n = \series{a_i}$ beschränkt und ist $a_i \geq 0$ für alle i, so ist $\series{a_i}$ konvergent.
\begin{proof}
a)\\
Sei $\series{a_i} = c$.\\
Sei $\varepsilon > 0$ Dann existiert $n(\frac{\varepsilon}{2}) \geq k$ mit $\abs{\series2{a_i - c}} < \frac{\varepsilon	}{2}$ für alle $ n \geq n(\frac{\varepsilon}{2})$\\
Dann gilt $\abs{a_{n+1} -o} = \abs{a{n+1}} = \abs{\seriesnplus{a_i} + \seriesn{a_i}} = \\
\abs{\seriesnplus{a_i+c}-\seriesn{a_i +c}} \leq \abs{\seriesnplus{a_i+c}}+\abs{\seriesn{a_i -c }} <  \frac{\varepsilon}{2} + \frac{\varepsilon}{2} = \varepsilon.\\
(a_n)$ ist Nullfolge\\
b) folgt aus 2.16a), denn $(s_n)$ ist monoton steigend
\end{proof}
\end{enumerate}
\subsection{Beispiele}
\begin{enumerate}[a)]
\i Sei $q \in \R$.\\
Ist $q \not = 1$, so ist $\seriesn{q^i}=\frac{q^{n+1}-1}{q-1}\\
\bigl \lbrack (\seriesn{q^i}) \cd (q-1)\bigr\rbrack$\\
Sei $\abs{q} < 1$, d.h $ -1 < q <1$.\\
Dann ist $\series{q^i} = \frac{1}{1-q}$ (konvergiert)\\
$s_n = \seriesn{q^1} = \frac{q^{n+1}-1}{q-1}\\
\lim\limits_{n \rightarrow \infty} s_n = \lim\limits_{n \rightarrow} \frac{q^{n+1} =1}{q-1}\\
(q^n)$ Nullfolge ($2.9_{a)}$  für $q \geq 0 , 2.8_{e}) + 2.9_{a)}$ für $q < 0 , q = -\abs{q}$)\\
\underline{Geomtetrische Reihe}\\
Sei $\abs{q} \geq 1$. Dann ist $\series{q^i}$  divergent, da dann $(q^i)$ keine Nullfolge ($2.18_{a)} $)
\i $\series{\frac{1}{i}}$ divergiert\\
\underline{harmonische Reihe}\\
$\seriesn{\frac{1}{n}}\\
\begin{array}{rlll}
n &= 2^0 &=1 &: s_1 = 1\\
n &= 2^1 &=2 &: s_2 =  1 + \frac{1}{2}\\
\ldots\\
n &= 2^3 &=8 &: s_8 = 1 + \frac{1}{2} + \frac{1}{3} + \frac{1}{4}+ \frac{1}{5} + \frac{1}{6} + \frac{1}{7} + \frac{1}{8} > s_7 > s_6 \ldots 
\end{array}$\\
Per Induktion zu beweisen!
\i $\seriesnull{\frac{1}{n^2}}$ konvergiert.\\
Folge der Partialsummen ist monoton steigend.\\
2.16a) Zeige, dass die Folge der Partialsummen nach aber beschränkt ist.\\
$\begin{array}{lcl}
s_n \leq s_{2^n -1} &=& 1 + (\frac{1}{2} + \frac{1}{3})+ (\frac{1}{4^2}+ \frac{1}{5^2} + \frac{1}{6^2} + \frac{1}{7^2}) + \ldots + (\frac{1}{(2^{n-1})^2} + \ldots \frac{1}{(2^n -1)^2})\\
&\leq& 1 + 2 \cd \frac{2}{2^2} + 4 \cd \frac{1}{4^4}+ \ldots + 2^{n-1} \cd  \frac{1}{(2^{n-1})^2}\\
&\leq& \seriesnull{\frac{1}{2^i}} = \frac{1}{1-\frac{1}{2}} = 2
\end{array}$\\
2.16a) $\seriesnull{\frac{1}{2^i}}$ Kgt., Grenzwert$\leq 2$.
(später: Grenzwert ist $\frac{\pi^2}{6})$\\
Es gilt allgemeiner:\\
$s \in \N, s \geq 2 \Rightarrow \seriesnull{\frac{1}{i^s}}$ konvergiert.\\
Allgemeiner: $s\in \R , s > 1 \Rightarrow \seriesnull{\frac{i}{i^2}}$ konvergiert
\i $\seriesnull{(-1)^i \cd \frac{1}{i}}$ konvergiert:\\
$s_{2n} = \underbrace{(-1 + \frac{1}{2})}_{<0} + \underbrace{(-\frac{1}{3} +\frac{1}{4})}_{<0} + \ldots \underbrace{(- \frac{1}{2n-1}+ \frac{1}{2n})}_{<0}\\
s_{2n} \leq s{2(n+1)}$ für alle $n \in \N\\
(s_{2n})$ ist monoton fallend.
$s_{2n-1} = -1 + \underbrace{(\frac{1}{2} - \frac{1}{3})}_{>0}+ \ldots + \underbrace{(\frac{1}{2n-2} - \frac{1}{2n-1})}_{>0}\\
(s_{2n-1})$ ist monoton wachsend\\
Ist $k$ ungerade, so ist $s_k < s_l:$ Wähle n so, dass $2n-a \geq k, 2n \geq l$\\
$s_k \underset{(2)}{\leq} s_{2n-1} \underset{\uparrow}{<} s_{2n} \underset{(1)}{\leq} s_l$\\
\hspace*{0.75cm} $s_{2n} = s_{2n-1} + \frac{1}{2n}$\\
\begin{tikzpicture}
\begin{axis}[legend style={draw none},axis equal,ymin = -0.1,xmin = 0,ymax = 0,xmax = 7,xtick ={1,2,3,4,5,6}, xticklabels={$s_1$,$s_3$,$s_5$,$s_6$,$s_4$,$s_2$},ytick ={},yticklabels={},extra x ticks={0},extra x tick label={0},extra y ticks={},extra y tick labels={},extra tick style = {grid = major}]             
\end{axis}
\end{tikzpicture}\\
Abstand $s_{2n} - s_{2n-1} = \frac{1}{2n}$ geht gegen 0.\\
$\sup \{s_{2n-1} : n \geq 1\} \\
\inf\{s_{2n} : n \geq 1\} \\
= \lim\limits_{i \leftarrow \infty} (-1^i) \frac{1}{i} \in ]-1, -\frac{1}{2}[$ (Es gilt $limes = -\ln 2)$
\end{enumerate}
\subsection*{Bemerkung}
Was bedeutet $0.\bar{8} = 0.88888888\ldots$? (Dezimalsystem)\\
$0.\bar{8} = \frac{8}{10} + \frac{8}{100} + \frac{8}{1000}+ \ldots = 8 \cd \seriesnull{\frac{1}{10^i}} = 8 \cd (\frac{10}{9} - 1) = \frac{8}{9}\\
\seriesnull{\frac{1}{10^i}} = \seriesnull{(\frac{1}{10})^i} = \frac{1}{1 - \frac{1}{10}} = \frac{10}{9}$
\subsection{Satz (Leibniz-Kriterium)}
Ist $(a_i)_{i \geq k}$ eine mononton fallende Nullfolge (ins besondere $a_i \geq 0$ falls $i \geq k$), so ist $\series{(-1)^i a_i}$ konvergent.
\subsection{Satz (Majoranten-Kriterium)}
Seien $(a_i)_{i \geq k}, \ (b_i)_{i \geq k}$ Folgen, wobei $b_i \geq 0$ für alle $i \geq k$ und $\abs{a_i} \leq b_i$ für alle $i \geq k$. Dann gilt\\
Ist $\series{b_i}$ konvergent, so auch $\series{a_i}$ und $\series{\abs{a_i}}$. Für die Grenzwerte gilt:\\
$\abs{\series{a_i}} \leq \series{\abs{a_i}} \leq \series{b_i}$
\begin{proof}
Konvergenz\\
von $\series{\abs{a_i}}$ folgt aus 2.16 a).\\
$\series{\abs{a_i}} \leq \series{b_i}$ folgt aus 2.8 f).\\
Sei $m > n$:\\
$\abs{\sum\limits_{i = k}^{m} a_i - \sum\limits_{i=k}^{n} b_i} = \sum\limits_{i = n+ 1}^{m} a_i \leq \sum\limits_{i = n+1}^{m} \abs{a_i} = \abs{\sum\limits_{i=k}^{m} \abs{a_i}-\sum\limits_{i=k}^{n} \abs{a_i}}$\\
Mit Cauchy-Kriterium 2.17 folgt daher aus der Konvergenz von $\sum\limits_{i=k}^{m} \abs{a_i}$ auch die von $\sum\limits_{i=k}^{\infty} a_i$.\\
\end{proof}
\subsection{Beispiel}
$\sum\limits_{i=1}^{\infty} \frac{1}{+\sqrt{i}}
\\\sqrt{i} \leq i$ für alle $i \in \N\\
\frac{1}{\sqrt{i}} \geq \frac{1}{i}$ für alle $i \in \N\\
$Ang. $\sum\limits_{i=1}^{\infty} \frac{1}{+\sqrt{i}}$ konvergiert.
$\Rightarrow \sum\limits_{i=1}^{\infty} \frac{1}{i}$ konvergiert. $\lightning$\\
Widerspruch zu 2.20 b)
\bigskip\\
$a_i = (-1)^i \frac{1}{i}$\\
2.20d): $\sum\limits_{i=1}^{\infty} a_i$ konvergiert,
aber $\sum\limits_{i=1}^{\infty} \abs{a_i}$ konvergiert nicht. $(\star)$
\subsection{Definition}
$\series{a_i}$ hei\ss t \underline{absolut konvergent}, falls $\series{\abs{a_i}}$ konvergiert. \\
(Falls alle $a_i \geq 0:$ Konvergent = absolut Konvergent)
\subsection{Korollar}
Ist $\series{a_i}$ absolut konvergent, sp ist auch konvergiert. Die Umkehrung gilt im Allgemeinen nicht.\\
\underline{Beweis}: 1.Behauptung 2.22 mit $b_i = \abs{a_i}$\\
Umkehrung siehe $(\star)$
\subsection*{Bermerkung}
Was bedeutet $0,a_1,a_2,a_3,a_4 \ldots\\
a_i \in \{0 \ldots 9\}$ (Dezimalsystem)\\
$a_1 \cd \frac{1}{10} a_2 \cd \frac{1}{100} \ldots a_n \cd \frac{1}{10^n} \leq 9 \cd \frac{1}{10} 9 \cd \frac{1}{100} \ldots 9 \cd \cd \frac{1}{10^n}\\
a_i \frac{1}{10} \leq 9 \frac{1}{10}\\
\series{9 \frac{1}{10}} = 9 \cd (\frac{1}{1-\frac{1}{10}}-1) = 1 \Rightarrow \series{a_i \frac{1}{10}}$ konvergiert
\subsection{Satz}
Sei $\series{a_i}$ eine Reihe.\\
\begin{enumerate}[a)]
\i \underline{Wurzelkriterium}\\
Existiert $q < 1$ und ein Index $i_0$, so dass $\sqrt[i]{\abs{a_i}} \leq q$ für alle $i \geq i_0$.\\
so konvergiert die Reihe $\series{a_i}$ absolut.\\
Ist $\sqrt[i]{\abs{a_i}} \geq 1$ für unendlich viele im so divergiert $\series{a_i}$.\\
\i \underline{Quotientenkriterium}\\
Existiert $q > 1$ und ein Index $i_0$, so dass $\abs{\frac{a_{i+1}}{a_i } } \leq$ für alle $i \geq i_0$,\\
so konvergiert $\series{a_i}$ absolut. 
\end{enumerate}
\begin{proof} \ \\
\begin{enumerate}[a)]
\i $\abs{a_i} \leq q^i$ für alle $i \geq i_0$
\i[] $\sum\limits_{i=i_0}^{\infty} q^i$ konvergiert (2.20 a)) 
\i[$\underset{2.22}{\Rightarrow}$] $\sum\limits_{i=i_0}^{\infty} \abs{a_i}$ konvergiert
\i[$\Rightarrow$] $\series{\abs{a_i}}$ konvergiert.
\i[] $\sqrt[i]{\abs{a_i}} \geq 1$ für unendlich viele i
\i[$\Rightarrow$] $\abs{a_i} \geq 1$ für unendlich viele i
\i[$\Rightarrow$] $(a_i)$ sind keine Nullfolge
\i[$\Rightarrow$] $\series{a_i}$ divergiert.\\
\i Sei $i \geq i_0$.\\
$\abs{\frac{a_i}{a_{i0}}} = \abs{\frac{a_i}{a_{i-1}}} \cd \abs{\frac{a_i}{a_{i-2}}} \cd \ldots \cd \abs{\frac{a_{io+1}}{a_{i0} } } \leq q \cd q \cd \ldots \leq = q^{i-i0} = \frac{q^i}{q^{i0}}$\\
\hspace*{5.55cm} $\uparrow$ Voraussetzung: \\
\hspace*{5cm} jeder dieser Quotienten ist $\leq q$\\
$\abs{a_i} \leq \underbrace{\frac{\abs{a_i0}}{q^{i0}}}_{=:c} \cd q^i \hspace*{2cm} \sum\limits_{i=i_0}^{\infty} c \cd q^i$ konvergent\\
\i[$\underset{2.22}{\Rightarrow}$] $\sum\limits_{i=i_0}^{\infty} \abs{a_i}$ konvergiert.\\
\i[$\Rightarrow$] $\series{\abs{a_i}}$ konvergiert
\end{enumerate}
\end{proof}
\subsection{Bemerkung}
\begin{enumerate}[a)]
\i Es reicht \underline{nicht} in 2.26 nur vorauszusetzen, dass $\sqrt[i]{\abs{a_i}} > 1$ für alle $i \geq i_o$\\
bzw.  $\frac{a_{i+1}}{a_i} < 1$ für alle $i \geq i_0$.\\
z.B. harmonische Reihen : $\sum\limits_{i=1 }^{\infty} \frac{1}{i}$ divergiert.\\
\underline{Aber}: $\sqrt[i]{\frac{1}{i}} > 1$ für alle i.\\
\hspace*{30pt} $\frac{i}{i+1} < 1$ für alle i
\i Es gibt Beispiele von absolut konvergenten Reihen mit $\abs{\frac{a_{i+1}}{a_i}}$ für unendlich viele i.
\end{enumerate}
\subsection{Beispiel}
Sei $x \in \R$. Dann konvergiert $\seriesnull{\frac{x^i}{i!}}$ absolut $(0^0=1, 0! = 1):$\\
Quotientenkriterium:\\
$\abs{\frac{x^{i+1} \cd i!}{(i+1)! \cd x^i}} = \abs{frac{x}{i+1}} = \frac{\abs{x}}{i+1}$ Wähle $i_o$, so dass $i_0 + 1 > 2 \cd \abs{x}$\\
Für alle $i \geq i_0$:\\
$\frac{\abs{x}}{(i+1)} \leq \frac{\abs{x}}{(i_0+1)} < \frac{\abs{x}}{2 \cd \abs{x}} = \frac{1}{2} = q$.
\subsection{Bemerkung}
Gegeben seien zwei endliche Summen\\
$\seriesalg{n=0}{k}{a_n}, \seriesalg{n=0}{l}{b_n}$.\\
$(\seriesalg{n=0}{k}{a_n})(\seriesalg{n=0}{l}{b_n})$
\begin{huge}
$(\star)$
\end{huge}\\
Distributivgesetz: Multipliziere $a_i$ mit jedem $b_i$ und addiere diese Produkte.\\
\begin{huge}
$(\star)$
\end{huge}
$= \underbrace{a_0 b_0}_{\text{Indexsumme 0}} + \underbrace{(a_0b_1+ a_1b_0)}_{\text{Indexsumme 2}} + \ldots + \underbrace{a_kb_l}_{\text{Indexsumme k+l}}$
\subsection{Definition}
Seien $ \seriesnull{a_n}, \seriesnull{b_n}$ unendliche Reihen.\\
Das \underline{Cauchy-Produkt}(\underline{Faltungsprodukt}) der beiden Reihen ist die Reihe $\seriesnull{c_n}$, wobei $c_n = \seriesnull{a_i \cd b_{n-1}} = a_0b_n + ab_{n-1} + \ldots a_nb_0$
\subsection{Satz}
Sind $\seriesnull{a_n}, \seriesnull{b_n}$ absolut konvergent Reihen mit Grenzwert $c ,d$, so ist das Cauchy Produkt auch absolut konvergent mit Grenzwert $c \cd d$. \begin{flushright}
Beweis: Kreu\ss ler, Phister Satz 33.16
\end{flushright}
\section{Potenzreihen}
\subsection{Definition}
Seit $(b_n)$ eine reelle Zahlenfolge, $a \in \Re$\\
Dann hei\ss t $\sum\limits_{n=0}^{\infty} b_n \cd (x-a)^n$ eine \underline{Potenzreihe} (mit \underline{Entwicklungspunkt} a))
Speziell: $a=0\\
\sum\limits_{n = 0}^{\infty} b_n \cd x^n$\\
(Potenzreihe im Engeren Sinne)\\
\underline{Hauptfolge}: Für welche $x \in \R$ konv. die Potenzreihe (absolut)?\\
Suche für $x=a$\\
Dann Grenzwert $b_ 0$ (da $0^0 = 1$)\\
Ob Potenzreihe für andere x konvergiert, hängt von $b_n$ ab!
\subsection{Beispiel} 
\begin{enumerate}[a)]
\i $\seriesnull{x^n} (b_n =1$ für alle $n$)\\
geometrische Reihe, konvergiert für alle $x \in \  ]-1,1[$
\i $\seriesnull{2^n \cd x^n} (b_n = 2^n) = \seriesnull{(2 \cd x)^n}$
konvergiert genau dann nach a), wenn $\abs{2x} < 1$, d.h $\abs{x} < \frac{1}{2}$ d.h. $x \in ]-0.5, 0.5[$\\
\i $\seriesnull{\frac{x^n}{n!}} (b_n = \frac{1}{n})$\\
konvergiert für alle $x, \ x \in ]-\infty,\infty[ = \R $\\
\end{enumerate}
\subsection{Satz}
Sei $ \seriesnull{b_n \cd x^n}$ eine Potenzreihe (um 0). Dawnn gibt es $R \in \R \cup \{\infty\}, R \geq 0$, so dass gilt.
\begin{enumerate}
\i Für alle $x \in \R$ und $\abs{x} < R$ konvergiert Potenzreihe absolut (d.h. $\seriesnull{b_n \cd x^n}$
konvergiert, dann auch $\seriesnull{b_n \cd x^n}$)\\
Falls $R = \infty$, so hei\ss t das, dass Potenzreihe für alle $x \in \R$ absolut konvergiert.\\
\i Für alle $x \in \R$ mit $\abs{x} > R$ divergiert $\seriesnull{b_n \cd x^n}$\\
\begin{tikzpicture}
\begin{axis}[legend style={draw none},axis equal,ymin = 0,xmin = -2,ymax = 0,xmax = 2,xtick ={-1, 0 , 1}, xticklabels={-R, 0 ,R},ytick ={},yticklabels={},extra x ticks={0},extra x tick label={0},extra y ticks={0},extra y tick labels={},extra tick style = {dashed, grid = major}]   
\addplot [black, mark = *, nodes near coords=div.?,every node near coord/.style={anchor=-90}] coordinates {( -1, 0)};
\addplot [black, mark = *, nodes near coords=konvgt.,every node near coord/.style={anchor=-90}] coordinates {( 0, 0)};  
\addplot [black, mark = *, nodes near coords=div.,every node near coord/.style={anchor=-90}] coordinates {( 1, 0)};     
\end{axis}
\end{tikzpicture}\\
$(\lim\limits_{n \rightarrow \infty} \sqrt[n]{\abs{b_n}} = 0 \Rightarrow R = \infty)$
(Für $\abs{x} = R$ lassen sich keine allgeine Aussagen treffen).\\
R hei\ss t der \underline{Konvergenzradius} der Potenzreihe $\seriesnull{b_n \cd x^n}$\\
Konvergenzintervall $<-R,R>$\\
besteht aus allen $x$ für die $\seriesnull{b_n \cd x^n}$ konvergiert.\\
$<$ kann [ oder ] bedeuten.\\
$>$ kann ] oder [ bedeuten.
\end{enumerate}
\subsection{Bemerkung}
Konvergenz von Potenzreihen der Form $\seriesnull{b_n \cd (x-a)^n}$:\\
gleichen Konvergenzradius $R$ wie $\seriesnull{b_n \cd x^n}$\\
konvergiert absolut für $\abs{x-a} < R$, d.h $x \in \ ]a-R,  a+R[$
Divergiert für $\abs{x-a} > R$.\\
Keine Aussage für $\abs{x-a} = R$, d.h $ x = a-R$ oder $x = a+R$\\
Konvergenzintervall $<a-R,a+R>$
\end{document}