\documentclass[a4paper,12pt,twoside]{article}
\usepackage{fourier}
\usepackage[ngerman]{babel}
\usepackage[leqno,tbtags,nointlimits]{amsmath}
\usepackage{amssymb,amsthm,amsfonts}
\usepackage{graphicx}
\usepackage{ifthen}
\usepackage{tikz}
\usepackage{mathtools}
\usepackage{fancyhdr,lastpage}
\usepackage{enumerate}
\usepackage[onehalfspacing]{setspace}
\usepackage{mdsymbol}
\usepackage{pgfplots}
\usepackage{color}
\usepackage{bigints}
\usepackage{array}
\usepackage{mdframed}
\usepackage{marginnote}
\usetikzlibrary{trees,automata,arrows,shapes}
\pagestyle{fancy}
\usepackage{hyperref}
\fancyhf{} %--Clear all fields
\renewcommand\sectionmark[1]{ \markboth{\thesection\ \textsc{#1}}{}}
\fancyhead[LO,RE]{\normalsize \leftmark}
\fancyhead[LE,RO]{ \rightmark}
\fancyfoot{} % clear all footer fields
\fancyfoot[LE,RO]{\thepage}
\newcommand{\cd}{\cdot}
\newcommand{\C}{\mathbb{C}}
\newcommand{\Z}{\mathbb{Z}}
\renewcommand{\i}{\item}
\newcommand{\U}{\mathcal{U}}
\newcommand{\N}{\mathbb{N}}
\newcommand{\R}{\mathbb{R}}
\DeclarePairedDelimiter{\ceil}{\lceil}{\rceil}
\DeclarePairedDelimiter{\floor}{\lfloor}{\rfloor}
\newcommand{\seriesalg}[3]{\sum\limits_{#3}^{#2} #1}
\newcommand{\seriesn}[1]{\sum\limits_{i=k}^{n} #1}
\newcommand{\seriesnplus}[1]{\sum\limits_{i=k}^{n+1} #1}
\newcommand{\series}[1]{\sum\limits_{i=k}^{\infty} #1}
\newcommand{\seriesnull}[1]{\sum\limits_{i=0}^{\infty} #1}
\usepackage[normalem]{ulem}
\usepackage{blkarray}
\usepackage{stmaryrd}
\usepackage{titletoc}
\usepackage%[margin=15mm]
{geometry}
\newcommand{\abs}[1]{\lvert #1 \rvert}
\renewcommand\headrule{{\color{gray}%
\hrule height 2pt width\headwidth
\vspace{1pt}%
\hrule height 1pt width\headwidth
\vspace{-4pt}}}
\makeatletter
\newcommand{\resetHeadWidth}{\fancy@setoffs}
\makeatother
\newcommand{\cucubr}[7]{%
%origin point, circle radius, start angle, end angle, distance c-b, brace radius, brace options
\pgfmathsetmacro{\helpangleedge}{acos(1-pow(#6,2)/2/pow(#2+#5,2))}%
\pgfmathsetmacro{\turnangleedge}{90+(\helpangleedge/2)}%
\pgfmathsetmacro{\helpanglemid}{acos(1-pow(#6,2)/2/pow(#2+#5+2*#6,2))}%
\pgfmathsetmacro{\turnanglemid}{90-(\helpanglemid/2)}%
\pgfmathsetmacro{\halfangle}{(#4-#3)/2+#3}%
\pgfmathsetmacro{\midradius}{#2+#5+#6}%
\pgfmathsetmacro{\outerradius}{#2+#5+1.88*#6}%
\pgfmathsetmacro{\firstmidanglestart}{mod(\halfangle-\helpanglemid+180,360)}%
\pgfmathsetmacro{\secondmidanglestart}{mod(\halfangle+\helpanglemid+180,360)}%
\pgfmathsetmacro{\firstmidanglestop}{mod(\halfangle-\helpanglemid/2+180,360)-\turnanglemid}%
\pgfmathsetmacro{\secondmidanglestop}{mod(\halfangle+\helpanglemid/2+180,360)++\turnanglemid}%
%
\draw[#7] (#1) ++ (\halfangle:\outerradius) arc (\firstmidanglestop:\firstmidanglestart:#6) arc (\halfangle-\helpanglemid:#3+\helpangleedge:\midradius) arc (#3+270+\turnangleedge+\helpangleedge/2:#3+270+\helpangleedge/2:#6) ;%
%
\draw[#7] (#1) ++ (\halfangle:\outerradius) arc (\secondmidanglestop:\secondmidanglestart:#6) arc (\halfangle+\helpanglemid:#4-\helpangleedge:\midradius) arc (#4+90-\turnangleedge-\helpangleedge/2:#4+90-\helpangleedge/2:#6);%
}
\newcommand{\limit}[1]{\displaystyle \lim_{#1}}
\usepgfplotslibrary{fillbetween}
\pgfplotsset{compat=1.9}
\hypersetup{%
pdfborder = {0 0 0}
}
\begin{document}
\resetHeadWidth
\tableofcontents
\listoffigures
\clearpage
\section{Komplexe Zahlen}
\renewcommand{\uline}[1]{\emph{#1}}
\renewcommand{\underline}[1]{\emph{#1}}
\renewcommand{\footnoterule}{}
\subsection{Definition}
Menge der komplexen Zahlen $\C =\{a+bi : a,b \in \R\}$
\begin{table*}[h!]
\centering
$\begin{array}{rl}
\text{\quad\uline{Addition:}}& (a+bi) + (c+di) = (a+c)+(b+d)i\\
\text{\uline{Multiplikation:}}& (a+bi) \cd  (c+di) = (ac -bd) + (ad+bc)i\footnotemark
\end{array}$\end{table*}
\footnotetext{Ausmultiplizieren und $i^2 =-1$ beachten}\\
$\R \subset \C ,\ a \in \R : a + 0 \cd i = a$.
Rein imagin\"are Zahlen: $b \cd i , b \in \R,\ (0 + bi)$\\
$i$ \underline{imagin\"are Einheit}.
$z = a+bi \in \C.\\
a = \mathfrak{R}(z)$ Realteil von $z\ (Re(z)).\\
b = \mathfrak{I}(z)$ Imagin\"arteil von $z\ (Im(z)).\\
\bar{z} = a - bi\ (= a + (-b)i)$ Die zu $z$ \uline{konjugiert komplexe Zahl}.
\subsection{Veranschaulichung}
\begin{figure}[h!]
\begin{minipage}[h]{0.5\textwidth}
\begin{tikzpicture}
\begin{axis}[
axis x line=center,
axis y line=center,
axis equal,
xlabel={$Re(z)$},
ylabel={$Im(z)$},
ymin = -1,
xmin = 0,
ymax = 5.5,
xmax = 5.5,
xtick ={1,4},
xticklabels={1,a},
ytick ={1,3},
yticklabels={i,b}]
\addplot[dashed,mark=none, black] coordinates {(4,0) (4,3)};
\addplot[dashed,mark=none, black] coordinates {(0,3) (4,3)};
  \addplot [black, mark = *, nodes near coords=$a+bi$,every node near coord/.style={anchor=180}] coordinates {( 4, 3)};
\end{axis}
\end{tikzpicture}
\end{minipage}%
\begin{minipage}[h]{0.5\textwidth}
\begin{tikzpicture}
\begin{axis}[
axis x line=center,
axis y line=center,
axis equal,
xlabel={$Re(z)$},
ylabel={$Im(z)$},
ymin = -1,
xmin = -3,
ymax = 7,
xmax = 5.5,
xtick ={1,4},
xticklabels={1,a},
ytick ={1,3},
yticklabels={i,b}]
\addplot [black, mark = *, nodes near coords=$a+bi$,every node near coord/.style={anchor=180}] coordinates {( 4, 3)};
   \draw[->](axis cs:0,0)--(axis cs:3.88,2.88);
       \addplot [black, mark = *, nodes near coords=$c+di$,every node near coord/.style={anchor=0}] coordinates {( -1, 1)};
              \draw[->](axis cs:0,0)--(axis cs:-0.88,0.88);
\addplot [black, mark = *, nodes near coords=$(a+c)+(b+d)i$,every node near coord/.style={anchor=0}] coordinates {( 3, 4)};
\addplot[mark=none, black] coordinates {(-1,1) (3,4)};
\addplot[mark=none, black] coordinates {(3,4) (4,3)};
\draw[-> ,red](axis cs:0,0)--(axis cs:2.95,3.88);
\end{axis}
\end{tikzpicture}
\end{minipage}
\caption[Veranschaulichung Komplexe Zahlen]{Addition entspricht Vektoraddition}
\end{figure}
\subsection{Rechenregeln in $\C$}
\begin{enumerate}
\item[a)]Es gelten alle Rechenregeln wie in $\R$.\\
(z.B Kommutativit\"at bzgl. $+,\cd:z_1 + z_2 = z_2 + z_1$ und $z_1 \cd z_2 = z_2 \cd z_1$)\\
\underline{Inversenbildung bzgl. $\cd$}:\\
$z = a+bi \not = 0,$ d.h $a \not = 0$ oder $b \not = 0:$\\
\begin{align*}
z^{-1} &= \frac{1}{z} = \frac{a}{a^2 + b^2} - \frac{b}{a^2+b^2}i\\
z \cd z^{-1} &= 1\bigskip
\end{align*}
\begin{minipage}[t]{\textwidth}
$\begin{array}{lcl}
\text{\underline{Beispiel: }}\frac{5-7i}{3+2i} &=& (5-7i) \cd (3+2i)^{-1}\\
&=& (5-7i) \cd (\frac{3}{13} -\frac{2}{13}i)\\
&=& (\frac{15}{13}-\frac{14}{13}) +(-\frac{10}{13} -\frac{21}{13})i\\
&=&\frac{1}{13} - \frac{31}{13}i
\end{array}$\bigskip\\
\end{minipage}
Speziell: $(bi)^{-1} = \frac{1}{bi} = -\frac{1}{b}i$ \\
insbesondere:$\frac{1}{i} = -i$
\item[b)]
$\begin{array}{llcl}
z,z_1,z_2 \in \C:&
\bar{\bar{z}}&=&z\\
&\overline{z_1+z_2} &=& \bar{z_1} + \bar{z_2}\\
&\overline{z_1\cd z_2} &=& \bar{z_1} \cd \bar{z_2}\\
\end{array}$
\end{enumerate}
\subsection{Definition Absolutbetrag}
\begin{enumerate}
\item[a)] \underline{Absolutbetrag} von $z =a+bi \in \C:
\abs{z} = +\underbrace{\sqrt{a^2 +b^2}}_{\in \R, \geq 0}$\\
\begin{equation*}
\boxed{a^2 + b^2 = z \cd \bar{z}}
\end{equation*}

$\abs{z}= +\sqrt{z \cd \bar{z}}$\\
$$(a+bi) \cd (a-bi) = (a^2 +b^2) + 0i = a^2 + b^2$$\\
$
\begin{array}{lcl}
\lvert z \rvert &=& \text{Abstand von $z$ zu 0}\\
&=& \text{L\"ange des Vektors, der $z$ entspricht}
\end{array}$\\
\begin{figure}[h!]
\centering
\begin{tikzpicture}
\begin{axis}[
axis x line=center,
axis y line=center,
axis equal,
xlabel={$Re(z)$},
ylabel={$Im(z)$},
ymin = -1,
xmin = 0,
ymax = 5.5,
xmax = 5.5,
xtick ={1,4},
xticklabels={1,a},
ytick ={1,3},
yticklabels={i,b}]
\addplot[mark=none, black] coordinates {(4,0) (4,3)};
\addplot[mark=none, black] coordinates {(0,0) (4,0)};
  \addplot [black, mark = *, nodes near coords=$a+bi$,every node near coord/.style={anchor=180}] coordinates {( 4, 3)};
\draw[->] (axis cs:0,0)--(axis cs:3.95,2.88);
\draw(axis cs:3.5,0) arc [radius=0.5,start angle=180,end angle=90];
\end{axis}   
\end{tikzpicture}
\caption[Absolutbetrag]{Graphische Definition des Absolutbetrages}
\end{figure}
\item[b)]Abstand von $z_1,z_2 \in \C:$\\
$$ d(z_1,z_2):= \abs{z_1 - z_2}$$\\
\end{enumerate}
\subsection{Rechenreglen f\"ur den Absolutbetrag}
\begin{align}
\abs{z} = 0 &\Leftrightarrow z =0 \tag{a}\\\notag \\
\mid z_1 \cd z_2 \mid &= \abs{z_1} \cd \abs{z_2} \tag{b}\\\notag \\
\begin{split}
\abs{z_1 + z_2} &\leq \abs{z_1} + \abs{z_2}\\
\abs{\abs{z_1} -\abs{z_2}} \leq \abs{z_1 - z_2} &\leq \abs{z_1} + \abs{z_2}\\
\abs{-z} &= \abs{z}
\end{split}\tag{c}
\end{align}
\newpage
\subsection{Darstellung durch Polarkoordinaten}
\begin{enumerate}
\item[a)]
Jeder Punkt $\neq (0,0)$ l\"asst sich durch seine Polarkoordinaten $(r,\varphi)$ beschreiben:\\
\begin{figure}[h!]
\begin{tikzpicture}
\begin{axis}[
axis x line=center,
axis y line=center,
axis equal,
xlabel={$Re(z)$},
ylabel={$Im(z)$},
ymin = -1,
xmin = 0,
ymax = 5.5,
xmax = 5.5,
xtick ={1,4},
xticklabels={1,a},
ytick ={1,3},
yticklabels={i,b}]
  \addplot [black, mark = *, nodes near coords=$a+bi$,every node near coord/.style={anchor=180}] coordinates {( 4, 3)};
\draw[decorate,decoration ={brace,amplitude =10pt},thick] (axis cs:0,0)--(axis cs:4,3);
  \draw[->] (axis cs:0,0)--(axis cs:3.95,2.88);
   \addplot [black, mark = none, nodes near coords=$\varphi$,every node near coord/.style={anchor=180}] coordinates {( 1.90, 0.8)};
   \addplot [black, mark = none, nodes near coords=$r$,every node near coord/.style={anchor=180}] coordinates {( 1.5, 2)};
\draw[->](axis cs:2,0) arc [radius=1.95,start angle=0,end angle=37];
\end{axis}   
\end{tikzpicture}
\centering
\caption[Imaginäre Zahlen im Koordinatensystem durch Polarkoordinaten]{Polarkoordinaten}
\end{figure}
$-r \geq 0, r \in \R\\
0 \leq \varphi \leq 2\pi$, wird gemessen von der positiven x-Achse entgegen des Uhrzeigersinnes
\begin{figure}[h!]
\centering \caption[Winkel im Bogenma\ss]{Umrechnung Grad zu Bogenma\ss}
\begin{tikzpicture}
\begin{axis}[
axis x line=center,
axis y line=center,
axis equal,ymin = -2,xmin = -2,ymax = 2,
xmax = 2,xtick ={},
xticklabels={},
ytick ={},
yticklabels={},
disabledatascaling]
\addplot [black, mark = none, nodes near coords=$\pi$,every node near coord/.style={anchor=180}] coordinates {( -1.5, 0)};
\addplot [black, mark = none, nodes near coords=2$\pi$,every node near coord/.style={anchor=180}] coordinates {( 1.25, 0)};
\addplot [black, mark = none, nodes near coords=$\frac{\pi}{2}$,every node near coord/.style={anchor=90}] coordinates {(0, 1.5)};
\addplot [black, mark = none, nodes near coords=$\frac{3\pi}{2}$,every node near coord/.style={anchor=-90}] coordinates {(0, -1.5)};
\draw (axis cs:0,0) circle[radius=1];
\draw(axis cs:0,0)--(axis cs:0.77,0.65);
\draw(axis cs:0,0)--(axis cs:1,0);
\draw[->](axis cs:0.3,0) arc [radius=0.3,start angle=0,end angle=37];
%origin point, circle radius, start angle, end angle, distance c-b, brace radius, brace options
\cucubr{0,2}{1}{0}{37}{0.1}{0.1}{red}
\end{axis}   
\end{tikzpicture}
\end{figure}
\\
Umfang:$2\pi$\\
$\varphi$ in Grad $\hat{=} \frac{2\pi \cd \varphi}{360}$ im Bogenma\ss\\
F\"ur Punkte mit kartesischen Koordinaten $\neq$ (0,0) werden als Polarkoordinate $(r,\varphi)$ verwendet.
\item[b)]komplexe Zahl $z = a +ib\\
\begin{array}{rcl}
r &=& \abs{z} = + \sqrt{a^2+b^2}\\
a &=& \abs{z} \cd \cos(\varphi)\\
b &=& \abs{z} \cd \sin(\varphi)\\
z &=& \abs{z} \cd \cos(\varphi) + i \cd \abs{z} \cd \sin(\varphi)\\
z &=& \abs{z} \cd ( \cos(\varphi) + i \cd \sin(\varphi))\\
\end{array}\\
\text{Darstellung von $z$ durch Polarkoordinate}$
\item[\underline{Beispiel:}]\begin{enumerate}
\item[a)]$z_1 = 2 \cd (\cos(\frac{\pi}{4})+i \cd \sin(\frac{\pi}{4}))\\
\hspace*{2pt}= 2 \cd (0,5\sqrt{2}+i \cd 0.5\sqrt{2})$
\item[b)]$z_2 = 2+i\\
\abs{z_2} = \sqrt{5}\\
z_2 = \sqrt{5} \cd (\frac{2}{\sqrt{5} + \frac{1}{\sqrt{5}}}i)$
Suche $\varphi$ mit $0 \leq 2\pi$ mit $\cos(\varphi) =\frac{2}{\sqrt{5}}, \sin(\frac{1}{\sqrt{5}} z_2 \approx \sqrt{5} \cd (\cos(0,46) + i \cd \sin(0,46))$
\item[c)] Die komplexen Zahlen von Betrag 1 entsprechen den Punkten auf Einheitskreis: \\$\cos(\varphi) + i \sin(\varphi), 0  \leq \varphi \leq 2\pi$
\end{enumerate}
\end{enumerate}
\subsection{Additionstheoreme der Trigonometrie}\label{sec:1.7}
\begin{align}
\sin(\varphi + \psi) &= \sin(\varphi) \cd \cos(\psi) + \cos(\varphi) \cd \sin(\psi) \tag{a}\\
\cos(\varphi + \psi) &= \cos(\varphi) \cd \cos(\psi) - \sin(\varphi) \ \cd \sin(\psi) \tag{b}
\end{align}
\subsection{geometrische Interpretation der Multiplikation}\label{sec:1.8}
\begin{enumerate}
\item[a)]$w = \abs{w} \cd (\cos(\varphi) + i \cd \sin(\varphi))\\
z= \abs{z} \cd (\cos(\psi) + i \cd \sin(\psi))\\
w \cd z = \abs{w} \cd \abs{z} \cd (\cos(\varphi) \cd \cos(\psi) - \sin(\varphi) \cd \sin(\psi)) + i (\sin(\varphi) \cd \cos(\psi) + \cos(\varphi) \cd \sin(\psi))\\
w \cd z =\abs{w \cd z}( \cos(\varphi + \psi) + i \cd \sin(\varphi + \psi))$
\begin{figure}[h!]
\begin{tikzpicture}
\begin{axis}[axis x line=center,
axis y line=center,,axis equal,xlabel={$Re(z)$},ylabel={$Im(z)$},ymin = -10,xmin = -10,ymax = 10,xmax = 10,xtick ={1,4},xticklabels={},ytick ={1,3},yticklabels={},disabledatascaling]
  \addplot [black, mark = *, nodes near coords=$w$,every node near coord/.style={anchor=180}] coordinates {( 4, 3)};
        \addplot [black, mark = *, nodes near coords=$z$,every node near coord/.style={anchor=0}] coordinates {( -1, 2)};
\draw (axis cs:0,0)--(axis cs:-1,2);
\draw (axis cs:0,0)--(axis cs:4,3);
\draw[decorate,decoration ={brace,amplitude =10pt}] (axis cs:0,0)--(axis cs:-6.7,0);\\
\draw(axis cs:0,0)--(axis cs:-6.7,0);
\addplot [black, mark = *, nodes near coords=$w\cd z$,every node near coord/.style={anchor=0}] coordinates {( -6.70, 0)};
 \addplot [black, mark = none, nodes near coords=$\abs{w} \cd \abs{z}$,every node near coord/.style={anchor=0}] coordinates {( -0.8, -2)};
   \addplot [black, mark = none, nodes near coords=$\varphi$,every node near coord/.style={anchor=180}] coordinates {( 1.90, 0.8)};
\draw[->](axis cs:2,0) arc [radius=1.95,start angle=0,end angle=37];
  \draw[->](axis cs:3,0) arc [radius=2,start angle=0,end angle=142];
\addplot [black, mark = none, nodes near coords=$\psi$,every node near coord/.style={anchor=180}] coordinates {( 2.90, 0.8)};  
\end{axis} 
\end{tikzpicture}
\centering
\caption{Multiplizieren komplexer Zahlen}
\end{figure}
\item[b)]$z =i, w = a+ ib\\
i\cd w = -b \cd ia$\\
Multiplikation mit i $\hat{=}$ Drehung um $90^{\circ}$\\
\begin{figure}[h!]
\begin{tikzpicture}
\begin{axis}[axis x line=center,
axis y line=center,,axis equal,xlabel={$Re(z)$},ylabel={$Im(z)$},ymin = 0,xmin = -5,ymax = 5,xmax = 7,xtick ={-3,3},xticklabels={-a,a},ytick ={1,4},yticklabels={i,-b},disabledatascaling]
\draw [dashed] (axis cs:0,0)--(axis cs:3,4);
\draw [dashed] (axis cs:0,0)--(axis cs:-3,4);
\addplot [black, mark = *, nodes near coords=$w$,every node near coord/.style={anchor=0}] coordinates {( 3, 4)};
  \addplot [black, mark = *, nodes near coords=$i \cd w$,every node near coord/.style={anchor=0}] coordinates {( -3, 4)};
  \draw[->](axis cs:3,4) arc [radius=3,start angle=0,end angle=180];      
\end{axis} 
\end{tikzpicture}
\centering
\caption{Multiplikation mit i}
\end{figure}

\end{enumerate}
\subsection{Bemerkung und Definition}
Wir werden sp\"ater die komplexe Exponentialfunktion einf\"uhren.\\
$e^{z}$ f\"ur alle $z \in \C$\quad e = Euler`sche Zahl $\approx 2,718718\ldots\\
e^{z_1} \cd e^{z_2} = e^{z_1+z_2}\quad e^{-z} = \frac{1}{e^z}$\\
Es gilt: $t \in \R : e^{it} = \cos(t) + i \cd \sin(t)$\\
Jede komplexe Zahl l\"asst sich schreiben $z = r \cd e^{i \cd \varphi}, r =\abs{z}, \varphi$ Winkel\\
$r \cd (\cos(\varphi) + i \sin(\varphi))$ ist Polarform von $z.\\
z = a+bi$ ist kartesische Form von z.
$\bullet (r,\varphi)$ Polarkoordinaten\\
$\abs{e^{i\varphi}} = + \sqrt{cos^2(\varphi) + \sin^2(\varphi)}=1\\
e^{i\varphi},0 \leq \varphi \leq 2\pi,$ Punkte auf dem Einheitskreis.\\
$e^{i\pi} = -1$\\
\begin{equation}
\boxed{e^{i\pi} +1 =0} \tag{\text{Euler'sche Gleichung}}
\end{equation}
\subsection[Satz: Komplexe Wurzeln]{Satz}\label{sec:1.10}
Sei $w = \abs{w} \cd (\cos(\varphi) + i \cd \sin(\varphi)) \in \C$\\
\begin{enumerate}
\item[a)] Ist $m \in \Z$, so ist $w^m = \abs{w}^m \cd ( \cos(m \cd \varphi) + i \cd \sin(m \cd \varphi))\\
( m < 0 : w^m = \frac{1}{w^{\mid m \mid}}), w \ne 0$
\item[b)] Quadratwurzeln
\item[c)] Ist $n \in \N, w \not = 0$, so gibt es genau n n-te Wurzeln von $w:\\
\sqrt[n]{w} = + \sqrt[n]{\abs{w}} \cd (\cos(\frac{\varphi}{n} + \frac{2\pi \cd k}{n}) + i \sin(\frac{\varphi}{n}+ \frac{2\pi \cd k}{n})), n \in \N  ,k \in \{0, \ldots,n-1\}$
\end{enumerate}
\begin{proof}
a) richtig, wenn $m=0,1$\\
$m \geq 2.$ Folgt aus \ref{sec:1.8}\\
$m=-a:\\
w^{-1} = \frac{1}{w} = \frac{1}{\abs{w}^2 \cd ( \cos^2(\varphi) + i \cd \sin^2( \varphi))} \cd \abs{w}\cd \cos(\varphi) - \sin(\varphi)\\
= \frac{1}{w} = \frac{1}{\abs{w} \cd \underbrace{( \cos^2(\varphi) + i \cd \sin^2( \varphi))}}_{=1} \cd \abs{w} \cd \cos(\varphi) - \sin(\varphi)\\
= \frac{1}{\abs{w}} \cd (\cos(-\varphi + i \cd \sin(-\varphi)) = \abs{w}^{-1} \cd (\cos(-\varphi) + \sin(-\varphi))$\\
\end{proof}
\subsection{Beispiel}
Quadratwurzel aus $i:\\
\abs{i} =1\\
\begin{array}{llcl}
\text{Nach \ref{sec:1.10} b):} &\sqrt{i}&=& \pm (\cos(\frac{\pi}{4} + i \cd \sin(\frac{\pi}{4}))\\
& &=& \pm (\frac{1}{2}\sqrt{2} + \frac{1}{2}\sqrt{2}i)
\end{array}
$
\subsection{Bemerkung}
Nach \ref{sec:1.10} hat jedes Polynom\\
$ x^{n} -w$ ($w \in \C)$\\
eine Nullstelle in $\C$ (sogar n verschiedene wenn $w \not = 0$)\\
Es gilt sogar : \underline{Fundamentalsatz der Algebra}\\
\hfill (C. F. Gau\ss\ 1777-1855)\\
Jedes Polynom $a_n x^n + \ldots + a_0$\\
mit irgendwelchen Koeffizienten: $a_n \ldots a_0 \in \C$ hat Nullstelle in $\C$
\section{Folgen und Reihen}
\subsection{Definition}
Sei k $\in \Z$,
$A_k := \{ m \in \Z : m > k\}\\
(k = 0,\, A_0 \in \N_0,\, k = 1,\, A_n \in \N)$\\
Abbildung a $: A \Rightarrow \R $(oder $\C)\\
\hspace*{2.5cm} m \Rightarrow a_n\\
$hei\ss t \underline{Folge} reeller Zahlen\\
\hspace*{2.5cm} ($a_k,a_{k-1} \ldots)$\\
Schreibweise:\\
$(a_m)_{m>k}$ oder einfach $(a_m)$\\
$a_m$ hei\ss t \underline{m-tes Glied} der Folge, m \underline{Index}\\
\subsection{Beispiel}
\begin{enumerate}
\item[a)] $a_n = 5$ f\"ur alle $n > 1$\\
$(5,5,5,5,5,5,5,5,5,5,5,5,5,\ldots$)\\
\item[b)] $a_n = n$ f\"ur alle $n>1$\\
(1,2,3,4,5,6,7,8,9,10,$\ldots$)\\
\item[c)] $a_n = \frac{1}{n}\\
(\frac{1}{1}, \frac{1}{2} ,\frac{1}{3}, \frac{1}{4}, \ldots)$\\
\item[d)]$a_n \frac{(n+1)^2}{2^n}\\
(2, \frac{9}{4},2,\frac{25}{16}, \ldots)$\\
\item[e)]$a_n = (-1)^n\\
(-1,1,-1,1,-1,1,\ldots)\\ $
\item[f)]$a_n = \frac{1}{2}a_{n_1} = \frac{1}{a_{n-1}} $f\"ur $n\geq 2,
a_1 =1\\
(1,\frac{3}{2},\frac{17}{12}, \ldots)$
\item[g)]$a_n = \sum_{i=1}^{n} \frac{1}{i}\\
(1,\frac{3}{2}, \frac{11}{6},\ldots)$
\item[h)]$a_n = \sum_{i=1}^{n} (-1)^i\cd \frac{1}{i}\\
(-1,\frac{-1}{2},-\frac{-5}{6},\ldots)$
\end{enumerate}
\subsection{Definition}
Eine Folge $(a_n)_{n>k}$ hei\ss t \underline{beschr\"ankt}, wenn die Menge der Folgenglieder beschr\"ankt ist. \\
D.h. $\exists D > 0 : - D \leq a_n \leq D$ f\"ur alle $n > k$.\\

\begin{figure}[h!]
\centering \caption{Beschränktheit von Folgen}
\begin{tikzpicture}
\begin{axis}[axis x line=center,
axis y line=center,axis equal,ymin = -4,xmin = 0,ymax = 4,xmax = 7,xtick ={1,2,3,4,5},ytick ={3,-3},yticklabels={D,-D}]
\draw [dashed] (axis cs:-10,3)--(axis cs:10,3);
\draw [dashed] (axis cs:-10,-3)--(axis cs:10,-3);
\addplot [black, mark = +] coordinates {( 1, 0.3)};
\addplot [black, mark = +] coordinates {( 2, 0.1)};
\addplot [black, mark = +] coordinates {( 3, -2)};
\addplot [black, mark = +] coordinates {( 4, 1)};
\addplot [black, mark = +] coordinates {( 5, 0.5)};   
\end{axis}
\end{tikzpicture} 
\end{figure}
\subsection{Definition}
Eine Folge $(a_n)_{n \geq k}$ hei\ss t \uline{konvergent} gegen $\varepsilon \in \R$ (konvergent gegen $\varepsilon$), falls gilt:\\
$\forall \varepsilon > 0 \exists n(\varepsilon) \in \N \forall n \geq n(\varepsilon) : \abs{a_n -c} < \varepsilon\\
c = \lim_{n \Rightarrow \infty} a_n $ (oder einfach $c = \lim a_n$)\\
c hei\ss t \uline{Grenzwert} (oder Limes) der Folge ($a_n$)\\
(Grenzwert h\"angt nicht von endlich vielen Anfangsgliedern ab (der Folge))\\
Eine Folge die gegen 0 konvertiert, hei\ss t \uline{Nullfolge}\\
\subsection{Beispiele}\label{sec:2.5}
\begin{enumerate}
\item[a)]$r \in \R : a_n = r$ f\"ur alle $n \geq 1$\\
$(r,r,\ldots)\\
\lim\limits_{n \Rightarrow \infty} = r\\
\abs{a_n - r} = 0$ f\"ur alle $n$\\
F\"ur jedes $\varepsilon >0$ kann man $n(\varepsilon) = 1$ w\"ahlen
\item[b)]$a_n = n$ f\"ur alle $ n\ \geq 1$\\
Folge ist nicht beschr\"ankt, konvergiert nicht.
\item[c)] $a_n = \frac{1}{n}$ f\"ur alle $n \geq 1\\
(a_n)$ ist Nullfolge.\\
Sei $ \varepsilon > 0$ beliebig. Suche Index $n(\varepsilon)$ mit $\abs{a_n - o} < \varepsilon$ f\"ur alle $n \geq n(\varepsilon)$\\
D.s. es muss gelten.\\
$\frac{1}{n} < \varepsilon$ f\"ur alle $n \geq n(\varepsilon)$\\
Ich brauche : $\frac{1}{n(\varepsilon)} < \varepsilon$\\
Ich brauche $n(\varepsilon) > \frac{1}{\varepsilon}$\\
Aus Mathe I folgt, dass solch ein $n(\varepsilon)$ existiert.\\
z.B $n(\varepsilon) - \ceil{\frac{1}{2}} + 1 > \frac{1}{\varepsilon}$\\
Dann:\\
$\abs{a_n -0} < \frac{1}{n} < \varepsilon$ f\"ur alle $n \geq n(\varepsilon)$
\item[d)]$a_n = \frac{3n^2+1}{n^2+n+1}$ f\"ur alle $n \geq 1$\\
Behauptung: $\lim\limits_{n\Rightarrow\infty}a_n = 3\\
\begin{array}{rlll}
\abs{a-3} &=& \abs{\frac{3n^2+1}{n^2+n+1}-3}&= \abs{\frac{3n^2+1-3(n^2+n+1)}{n^2+n+1}}\\
&=& \abs{\frac{-3n-2}{n^2+n+1}} &= \frac{3n+2}{n^2+n+1}
\end{array}$\\
Sei $\varepsilon > 0$. Ben\"otigt wird $n(\varepsilon) \in \N$ mit $\frac{3n+2}{n^2+n+1} < \varepsilon$ f\"ur alle $n > n(\varepsilon).\\
\frac{3n+2}{n^2+n+1} \leq \frac{5n}{n^2 } = \frac{5}{n}$\\
W\"ahle $n(\varepsilon)$ so, dass $n(\varepsilon) > \frac{5}{\varepsilon}$\\
Dann gilt f\"ur alle $n\geq n(\varepsilon).\\
\abs{a_n -3}=\frac{3n+2}{n^2+n+1} \leq \frac{5}{n} \leq  \frac{5}{n(\varepsilon)} < \frac{5\varepsilon}{5} = \varepsilon$\\
F\"ur alle $ n \geq n(\varepsilon)$
\item[e)]$a_n = (-1)^n$ beschr\"ankte Folge $-1 \leq a_ \leq 1$ konvergiert nicht.\\
Sei $c \in \mathbb{R}$ beliebig, W\"ahle $\varepsilon = \frac{1}{2}$\\
\begin{figure}[h!]
\centering
\begin{tikzpicture}
\begin{axis}[axis x line=center,
axis y line=center,axis equal,ymin = -2,xmin = 0,ymax = 2,xmax = 6,xtick ={1,2,3,4,5},ytick ={3,-3},yticklabels={D,-D},extra x ticks={0},extra x tick label={0},extra y ticks={0},extra y tick labels={},extra tick style = {grid = major}]
\addplot [black, mark = +] coordinates {( 1, -1)};
\addplot [black, mark = +] coordinates {( 2, 1)};
\addplot [black, mark = +] coordinates {( 3, -1)};
\addplot [black, mark = +] coordinates {( 4, 1)};
\addplot [black, mark = +] coordinates {( 5, -1)};
           
\end{axis}
\end{tikzpicture}
\caption[Beschränkte aber nicht konvergente Folge]{$(-1)^n$ ist beschränkt aber konvergiert nicht}
\end{figure}

$2 = \abs{a_n-a_{n+1}} \leq \abs{a_n - c} + \abs{c - a_{n+1}} < \frac{1}{2}+ \frac{1}{2} =$\underline{$1$} $\lightning$
\end{enumerate}
\subsection[Satz: Beschränktheit und Konvergenz]{Satz}\label{sec:2.6}
Jede konvergente Folge ist beschr\"ankt. (Umkehrung nicht: \ref{sec:2.5}e))
\begin{proof}
Sei $c= \lim a_n$, w\"ahle $\varepsilon =1$,\\
Es existiert $n(1) \in \N$ mit $\abs{a_n -c} < 1$ f\"ur alle $n \geq n(1)$\\
Dann ist\\
$\abs{a_n} = \abs{a_n-c+c} \leq \abs{a_n -c} + \abs{c} < 1 + \abs{c}$ f\"ur alle $n \geq n(1)\\
M = \max \{ \abs{a_k},\abs{a_{k+1}},\ldots,\abs{a_{n(1)-1)}}, 1 + \abs{c} \}\\
$ Dann: $\abs{a_n} \leq M$ f\"ur alle $n \geq k\\
-M \leq a_n \leq M$
\end{proof}
\subsection{Bemerkung}\label{sec:2.7}
\begin{enumerate}
\item[a)]$(a_n)_{n \geq 1}$ Nullfolge $\Leftrightarrow (\abs{a_n})_{n \geq 1}$ Nullfolge ($\abs{a_n - 0} = \abs{a_n} - \abs{\abs{a_n}-0}$
\item[b)]$\lim\limits_{n\Rightarrow\infty} a_n = c \Leftrightarrow (a_n -c)_{n \geq k}$ ist Nullfolge $\Leftrightarrow (\abs{a_n -c})_{n \geq k}$ ist Nullfolge
\end{enumerate}
\subsection{Satz (Rechenregeln f\"ur konvergente Folgen)}\label{sec:2.8}
Seien $(a_n)_{n \geq k}$ und $(b_n)_{n \geq k}$ konvergente Folgen, $\lim a_n = c, \lim b_n = d.$\\
\begin{enumerate}
\item[a)]$\lim \abs{a_n} = \abs{c}$
\item[b)]$\lim (a_n \pm b_n) = c \pm d$
\item[c)]$\lim (a_n \cd b_n) = c \cd d$\\
insbesondere $\lim(r \cd b_N) = r \cd \lim b_n = r \cd d$ f\"ur jedes $r \in \R.$
\item[d)]Ist $b_n \not = 0$ f\"ur alle $n \geq k$ und ist $ d \not = 0$, so $\lim (\frac{a_n}{k_n}) = \frac{c}{d}$
\item[e)]Ist $(b_n)$ Nullfolge, $b_n \not = 0$ f\"ur alle $n \geq k$, so konvergiert $(\frac{1}{b_n}$ \underline{nicht}!).
\item[f)]Existiert $m \geq k$ mit $a_n \leq b_n$ f\"ur alle $n \geq m$, so ist $c \leq d$.
\item[g)] Ist $(c_n)_{n \geq k}$ Folge und existiert $ m \geq k$ mit $0 \leq c_n \leq a_n$ f\"ur alle $n \geq m$ und ist ($a_n$) eine Nullfolge, so ist auch $(c_n)$ eine Nullfolge.
\item[h)]Ist $(c_n)_{n \geq l}$ beschr\"ankte Folge und ist $(a_n)_{n \geq k}$ Nullfolge, so ist auch $(c_n \cd a_n)_{n \geq k}$ Nullfolge.\\
\fbox{$c_n$ muss nicht konvergieren!}
\end{enumerate}
\begin{proof}
Exemplarisch:
\begin{enumerate}
\item[b)] Sei $\varepsilon > 0$. Dann existiert $n_1(\frac{\varepsilon}{2})$ und $n_2(\frac{\varepsilon}{2})$ und $\abs{a_n -c} < \frac{\varepsilon}{2}$ f\"ur alle $n \geq n_1(\frac{\varepsilon}{2})\\
\abs{b_n-d} < \frac{\varepsilon}{2}$ f\"ur alle $n \geq n_2(\frac{\varepsilon}{2})$\\
Suche $n(\varepsilon) = max (n_1(\frac{\varepsilon}{2},n_2(\frac{\varepsilon}{2}))\\
$Dann gilt f\"ur alle $n > n(\varepsilon):\\
\abs{a_n+b_n - (c+d)} = \abs{(a_n -c) + (b_n -d)} \leq \abs{a_n -c} + \abs{b_n -d} < \frac{\varepsilon}{2} + \frac{\varepsilon}{2} = \varepsilon$
\item[f)]Angenommen $c > d$. Setze $\delta = c -d >0$\\
Es existiert \~m $ \geq m$ mit $\abs{c -a_n} < \frac{\delta}{2}$\\
und $\abs{b_n -d} < \frac{\delta}{2}$ f\"ur alle $n \geq$\~m.\\
F\"ur diese n gilt:\\
$0 < \delta \leq \delta + b_n - a_n = c -d+b_n -a_n \geq 0$ nach Voraussetzung\\
$= \abs{c -a_n - d +b_n} \leq \abs{c-a_n} + \abs{d -b_n}\\
\leq \frac{\delta}{2} + \frac{\delta}{2} = \delta \lightning$
\end{enumerate}
\end{proof}
\subsection[Satz: Kriterien für Nullfolgen]{Satz}\label{sec:2.9}
\begin{enumerate}
\item[a)] $0 \leq q \leq 1$ Dann ist $(q^n)_{n \geq 1}$ Nullfolge
\item[b)]Ist $m \in \N$, so ist $(\frac{1}{n^m})_{n \geq 1}$ Nullfolge.
\item[c)]Sei $0 \leq q < 1, m \in \N$\\
Dann ist ($n^m \cd q^n)_{n \geq 1}$ Nullfolge
\item[d)] Ist $r > 1, m \in \N$, so ist $(\frac{n^m}{r^n}_{r \geq 1}$ eine Nullfolge)
\item[e)] $P(x) = a_m \cd x^m + \ldots a_0, a_i \in \R, a_m \not = 0\\
Q(x) = b_e \cd x^e + \ldots b_0 , b_i \in \R , b_e \not = 0$\\
Sei $Q(n) \not = 0 $ f\"ur alle $n \geq k.$\\
\begin{enumerate}
\item[-]Ist $m>e$, so ist $\frac{P(n)}{Q(n)}$ nicht konvergent
\item[-]Ist $m = e$, so ist $\lim_{n\Rightarrow \infty} \frac{P(n)}{Q(n)}= \frac{a_m}{b_e} = \frac{a_m}{b_m}$
\item[-]Ist $m < l$, so ist$ (\frac{P(n)}{Q(n)})$ ein Nullfolge
\end{enumerate}
\end{enumerate}
\begin{enumerate}[a)]
\i Sei $0 \leq q \leq 1$ Dann ist $(q^n)_{n \geq}$ eine Nullfolge
\begin{proof}
a) Richtig für $q > 0$. Sei jetzt $q > 0$.\\
Sei $\varepsilon > 0$. Mathe I: Es gibt ein $n(\varepsilon) \in \N$ mit $q^{n(\varepsilon} < \varepsilon.$\\
Für alle $n \geq n(\varepsilon)$ gilt: $\abs{q^{n} -o} = q^n < q^{n(\varepsilon)} < \varepsilon.$\\
\end{proof}
\i \ref{sec:2.5}c): $\frac{1}{n}_{n \geq 1}$ Nullfolge Beh. folgt mit 2.8.c)
\i Richtig für $q=0$. Sei jetzt $q > 0$.\\
\underline{1.Fall}: m= 1\\
$\frac{1}{q} = 1+t, t > 0.\\
(t+1)^n \underbrace{=}_{Binomialsatz} 1 + nt + \frac{n(n+1)}{2}t^2 > \frac{n(n-1)}{2}t^2$ für alle $n \geq 2\\
q^n = \frac{1}{(1+t)^n} < \frac{2}{n(n-1)t^2}\\
0 \leq n \cd q^{n} < \frac{2}{(n-1)t^2} \Leftarrow$ Nullfolge \ref{sec:2.5}e),\ref{sec:2.8}e)\\
Nach \ref{sec:2.8}g) ist $(n \cd q^n)_{n \geq q}$ Nullfolge, also auch $(n \cd q^n)_{n \geq 1}.$\\
\underline{2.Fall}: $m > 1$.\\
Setze $0 < q' = \sqrt[m]{q} \in \R\\
\begin{array}{rcl}
n^m \cd q^n &=& n^m \cd (q')^n)^m)^n\\
&=& (n \cd (q')^n)^m)^n) m = 1 \text{anwenden}
\end{array}\\
0 < q' < 1\\
(n^m + q^n)_{n \geq 1}$ Nullfolge noch Fall $m=1$ und \ref{sec:2.8}e)
\i Folgt aus c) und $q = \frac{1}{r}$
\i Ist $ m \leq l$ , so ist  $\frac{P(n(}{Q(n)} =\frac{n^m(a_m + a_{m-1}\cd\frac{1}{n} + \ldots + a_1 \cd \frac{1}{n^{m-1}} + a_0 \cd \frac{1}{n^m})}{n^l(b_l + b_{l-1}\cd\frac{1}{n} + \ldots + b_1 \cd \frac{1}{n^{l-1}} + b_0 \cd \frac{1}{n^l})}
= \frac{1}{n^{l-m}} \cd \frac{I}{II}\\
(I) \longrightarrow a_m , (II) \longrightarrow b_l$
$\frac{(I)}{(II)} \Rightarrow \frac{a_m}{b_l}\\
n < l, \frac{1}{n^{l-m}}$ Nullfolge\\
$\frac{P(n)}{Q(n)} \Rightarrow 0 \cd \frac{a_m}{b_l}\\
m > l:$\\
Beh. folgt aus Fall $m < l$ und \ref{sec:2.8}e).
\end{enumerate}
\subsection{Bemerkung}
Betrachte Bijektionsverfahren, die Zahl $x \in \R$ bestimmt.\\
$a_0 \leq a_1 \leq a_2 \leq \ldots\\
b_0 \geq b_1 \geq b_2 \geq \ldots\\
a_n \leq x \leq b_n\\
0 < b_n - a_n = \frac{b_0 - a_0}{2^n}\\
0 \leq \abs{x-a_n} \leq b_n - a_n = \frac{b_0-a_n}{2} \Leftarrow$ Nullfolge (\ref{sec:2.9}b)\\
\ref{sec:2.8}e)$ (\abs{x-a_n})$ Nullfolge.\\
\ref{sec:2.7}e): $\lim\limits_{n \rightarrow \infty} a_n = x$\\
Analog: $\lim\limits_{n \rightarrow \infty} b_n = x\\
$\ref{sec:2.9} d) e) sind Beispiele für asymptotischen Vergleich von Folgen
\subsection{Definition}
\begin{enumerate}[a)]
\i Eine Folge $(a_n)_{ n \geq k}$ hei\ss t \underline{strikt positiv}, falls $a_n > 0$ für alle $n \geq k$.\\
Sei im Folgenden $(a_n)_{n \geq k}$ eine strikt positive Folge.
\i $\begin{array}{rcl}
\mathbb{O}(a_n) &=& \{ (b_n)_{n \geq k} : \text{ist beschränkt} \}\\
&=& \{ (b_n)_{n \geq k} \exists C > 0 \text{ mit } \abs{b_n} \leq C \cd a_n \}
\end{array}$
\i $O(a_n) = \{(b_n)_{n \geq k}: (\frac{b_n}{a_n}) \text{ist Nullfolge} \}\\
(b_n) \in o(a_n)$ hei\ss t Folge $(a_n)$ wächst wesentlich schneller als die Folge $(b_n)$.
Klar: $o(a_n) \subset O(a_n)\\
O,o ($ \glqq gro\ss \ Oh\grqq, \glqq klein Oh\grqq)\\
\underline{Landau-Symbole}\\
$\begin{array}{lrrll}
\text{z.B}& (n^2) &\in& o(n^3)\\
& (n^2+n+1) &\in& O(n^2) & n^2 + n + 1 \leq 3n^2\\
& (n^2) &\in&O(n^2 + n +1) &n^2 \leq n^2 + n + 1
\end{array}\\
O(1) =$ Menge der beschränkten Folgen\\
$o(1)=$ Menge aller Nullfolgen\\
Häufig gewählte Schreibweise:\\
$n^2 \underbrace{=}_{\text{eig. falsch!}} o(n^2)$ statt $(n^2) \in o(n^3)\\
n^2+n+1 =O(n^2)$ statt $(n^2+n+1)$
\end{enumerate}
\subsection[Satz: Landausymbole bei Polynomen]{Satz}
Sei $P(x) = a_m \cd x^m + \ldots + a_1 \cd x + a_o, m \geq 0, a_m \not = 0$.
\begin{enumerate}[a)]
\i $(P(n)) \in o(n!)$ für alle $l>m$ und\\
$(P(n)) \in O(n')$ für alle $l \geq m$.
\i ist $r > 1$, so ist $(P(n)) \in o(r^n).\\
\lbrack(r^n)$ wächst deutlich schneller als $(P(n))\rbrack$
\begin{proof}
a) folgt aus \ref{sec:2.9}e).\\
$m = l$ (\ref{sec:2.6})\\
b) folgt aus \ref{sec:2.9}d) und \ref{sec:2.8} b)c)
\end{proof}
\end{enumerate}
\subsection{Bemerkung}
Algorithmus:\\
Sei $t_n =$ maximale Anzahl von Reihenschritten des Algorithmus' bei Input der Länge n (binär codiert).\\
Worst-Case-Komplexität:\\
Algorithmus hat polynomielle Zeitkomplexität, falls ein $l \in \N$ existiert mit $(t_n) \in O(n^l)$. (\underline{gutartig})\\
Algorithmus hat polynomielle Zeitkomplexität, falls ein $l \in \N$ existiert mindestens exponentielle Zeitkomplexität, falls 
$ r > 1$ exestiert mit $(r^n) \in O(b_n)$ (\underline{bösartig)}
\subsection{Definition}\label{sec:2.18}
\begin{enumerate}[a)]
\i Eine Folge $(a_n)_{n \geq k}$ hei\ss t \underline{monoton wachsend (steigend)}, wenn $a_n \leq a_{n+1}$ f\"ur alle $n \geq k$. Sie hei\ss t \underline{steng monoton wachsend (steigend)}, wenn $a_n < a_{n+1}$ f\"ur alle $n \geq k$
\i $(a_n)_{n \geq k}$ hei\ss t \underline{monoton fallend}, falls $a \geq a_{n+1}$ f\"ur alle $n \geq k$
\end{enumerate}
\subsection{Beispiel}
\begin{enumerate}[a)]
\i $a_n = 1$ f\"ur alle $n > 1
(a_n)$ ist monoton steigend und monoton fallend.
\i $a_n = \frac{1}{n}$ f\"ur alle $n \geq 1$.\\
$(a_n)$ streng monoton fallend.
\i $a_n = \sqrt{n}$ (positive Wuzel)\\
$(a_n){n \geq 1}$ streng monoton steigend.
\i $a_n = 1 - \frac{1}{n}, n \geq 1\\
(a_n)_{n \geq 1}$ streng monoton steigend.
\i $a_n = (-1)^n, n \geq 1\\
(a_n)$ ist weder monoton steigend noch monoton fallend.
\end{enumerate}
\subsection[Satz: Monotonie und Konvergenz]{Satz}\label{sec:2.16}
\begin{enumerate}[a)]
\i Ist $(a_n)_{n \geq k}$ monoton steigend und nach oben beschränkt (d.h es existiert $D \in \R$ mit $a_n \leq D$ für alle $n \geq k$), so konvergiert $(a_n)'$ und $\lim\limits_{n \rightarrow \infty}a_n = \sup\{a_n: n\geq k \}$
\i $(a_n)_{n \geq k}$ monoton fallend und nach unten beschränkt, so konvergiert $(a_n)_{n \geq k}$ und $\lim\limits_{n \rightarrow \infty} a_n = \inf \{a_n: n\geq k \}.$
\begin{proof}
a)\\
$ c = \sup \{a_n : n \geq k\}.$ existiert (Mathe I).
Zeige: $\lim\limits_{n \to \infty}a_n = c$.\\
Sei $\varepsilon > 0$. Dann existiert $n(\varepsilon)$ mit $c-\varepsilon < a_{n(\varepsilon)} \leq c$\\
Denn sonst $a_n \leq c - \varepsilon$ für alle $n \geq k$ und $c - \varepsilon$ wäre obere Schranke für $\{a_n : n \geq k \}$ Widerspruch dazu, dass c kleinste obere Schranke. Für alle $n \geq n(\varepsilon)\\
c- \varepsilon \leq a_{n(\varepsilon)} \leq a_n \leq c\\
\abs{a_n -c} < \varepsilon$ für alle $n \geq n(\varepsilon).$\\
b) analog
\end{proof}
\end{enumerate}
\subsection{Satz (Cauchy'sches Konvergenzkriterium)}
\hfill (Cauchy, 1789 - 1859)\\
Sei $(a_n)_{n \geq k}$ eine Folge. Dann sind äquivalent:
\begin{enumerate}[(1)]
\i $(a_n)_{n \geq k}$ konvergent
\i $\forall \varepsilon > 0 \exists \ N -M(\varepsilon) \forall n,m \geq N: \abs{a_n - a_m} < \varepsilon$ (Cauchyfolge)\\
Grenzwert muss nicht bekannt sein!
\end{enumerate}
\begin{figure}[h!]
\centering
\begin{tikzpicture}
\begin{axis}[axis x line=center,
axis y line=center,axis equal,ymin = -4,xmin = 0,ymax = 4,xmax = 7,xtick ={1, 6}, xticklabels={N(1),N($\varepsilon$)},ytick ={3,-3},yticklabels={D,-D},extra x ticks={0},extra x tick label={0},extra y ticks={0},extra y tick labels={},extra tick style = {grid = major}]
\draw [dashed] (axis cs:-10,3)--(axis cs:10,3);
\draw [dashed] (axis cs:-10,-3)--(axis cs:10,-3);
\draw [red] (axis cs:-10,0.1)--(axis cs:10,0.1);
\draw [red] (axis cs:-10,1)--(axis cs:10,1);
\addplot [black, mark = +] coordinates {( 1, 0.3)};
\addplot [black, mark = +] coordinates {( 1, 0.3)};
\addplot [black, mark = +] coordinates {( 2, 0.1)};
\addplot [black, mark = +] coordinates {( 3, -2)};
\addplot [black, mark = +] coordinates {( 4, 1)};
\addplot [black, mark = +] coordinates {( 5, 0.5)};
\addplot [black, mark = +] coordinates {( 8, 0.2)};
\addplot [black, mark = +] coordinates {( 6, 0.5)};
\addplot [black, mark = +] coordinates {( 7, 0.3)};
\addplot [black, mark = none, nodes near coords=$\varepsilon$,every node near coord/.style={anchor=180}] coordinates {( 5, -2)};
\draw [-> , red] (axis cs: 5,-2)--(axis cs: 5, 0.25);              
\end{axis}
\end{tikzpicture}
\caption{Cauchy'sches Konvergenzkriterium}
\end{figure}
\subsection{Definition}
\begin{enumerate}[a)]
\i Sei $(a_i)_{i \geq k}$ eine Folge, $s_n = \sum\limits_{i = k}^{n} a_i , n \geq k$ (Partialsummen der Folge)\\
Dann hei\ss t $(s_n)_{n \geq k}$ eine \underline{unendliche Reihe}\\
$(k-1: a_1, a_1+a_2, a_1 + a_2 + a_2,\ldots)$\\
Schreibweise : $\sum\limits_{i = k}^{\infty} a_i$\\
\i Ist die Folge $(s_n)_{n \geq k}$ konvergent mit $\lim\limits_{n \rightarrow \infty} s_n = c$,\\
so schreibt man $\sum\limits_{i = k}^{\infty} a_i =c.$ Reihe \underline{konvergiert}.\\
Wenn $(s_n)$ nicht konvergiert, so hei\ss t die Reihe $\sum\limits_{i =k}^{\infty} a_i$ \underline{divergent.}\\
(Zwei Bedeutungen von $\sum\limits_{i = k}^{\infty} a_i:$\\
\begin{enumerate}[-]
\i Folge der Partialsummen
\i Grenzwert von $(s_n)$, falls dieser existiert
\end{enumerate}
$\sum\limits_{i=k}^{\infty} a_i = \sum\limits_{n=k}^{\infty} a_n = (s_m)_{m \geq k}$
\end{enumerate}
\subsection[Satz: Reihenkonvergenz]{Satz}
\begin{enumerate}[a)]
\i Ist die Reihe $\series{a_1}$ konvergent, so ist $(a_1)_{i \geq k}$ eine Nullfolge.
\i Ist die Folge der Partialsummen $s_n = \series{a_i}$ beschränkt und ist $a_i \geq 0$ für alle i, so ist $\series{a_i}$ konvergent.
\begin{proof}
a)\\
Sei $\series{a_i} = c$.\\
Sei $\varepsilon > 0$ Dann existiert $n(\frac{\varepsilon}{2}) \geq k$ mit $\abs{\series2{a_i - c}} < \frac{\varepsilon	}{2}$ für alle $ n \geq n(\frac{\varepsilon}{2})$\\
Dann gilt $\abs{a_{n+1} -o} = \abs{a{n+1}} = \abs{\seriesnplus{a_i} + \seriesn{a_i}} = \\
\abs{\seriesnplus{a_i+c}-\seriesn{a_i +c}} \leq \abs{\seriesnplus{a_i+c}}+\abs{\seriesn{a_i -c }} <  \frac{\varepsilon}{2} + \frac{\varepsilon}{2} = \varepsilon.\\
(a_n)$ ist Nullfolge\\
b) folgt aus \ref{sec:2.16} a), denn $(s_n)$ ist monoton steigend
\end{proof}
\end{enumerate}
\subsection{Beispiele}
\begin{enumerate}[a)]
\i Sei $q \in \R$.\\
Ist $q \not = 1$, so ist $\seriesn{q^i}=\frac{q^{n+1}-1}{q-1}\\
\bigl \lbrack (\seriesn{q^i}) \cd (q-1)\bigr\rbrack$\\
Sei $\abs{q} < 1$, d.h $ -1 < q <1$.\\
Dann ist $\series{q^i} = \frac{1}{1-q}$ (konvergiert)\\
$s_n = \seriesn{q^1} = \frac{q^{n+1}-1}{q-1}\\
\lim\limits_{n \rightarrow \infty} s_n = \lim\limits_{n \rightarrow} \frac{q^{n+1} =1}{q-1}\\
(q^n)$ Nullfolge ($2.9_{a)}$  für $q \geq 0 , 2.8_{e}) + 2.9_{a)}$ für $q < 0 , q = -\abs{q}$)\\
\underline{Geometrische Reihe}\\
Sei $\abs{q} \geq 1$. Dann ist $\series{q^i}$  divergent, da dann $(q^i)$ keine Nullfolge ($2.18_{a)} $)
\i $\series{\frac{1}{i}}$ divergiert\\
\underline{harmonische Reihe}\\
$\seriesn{\frac{1}{n}}\\
\begin{array}{rlll}
n &= 2^0 &=1 &: s_1 = 1\\
n &= 2^1 &=2 &: s_2 =  1 + \frac{1}{2}\\
\ldots\\
n &= 2^3 &=8 &: s_8 = 1 + \frac{1}{2} + \frac{1}{3} + \frac{1}{4}+ \frac{1}{5} + \frac{1}{6} + \frac{1}{7} + \frac{1}{8} > s_7 > s_6 \ldots 
\end{array}$\\
Per Induktion zu beweisen!
\i $\seriesnull{\frac{1}{i^2}}$ konvergiert.\\
Folge der Partialsummen ist monoton steigend.\\
2.16a) Zeige, dass die Folge der Partialsummen nach aber beschränkt ist.\\
$\begin{array}{lcl}
s_n \leq s_{2^n -1} &=& 1 + (\frac{1}{2} + \frac{1}{3})+ (\frac{1}{4^2}+ \frac{1}{5^2} + \frac{1}{6^2} + \frac{1}{7^2}) + \ldots + (\frac{1}{(2^{n-1})^2} + \ldots \frac{1}{(2^n -1)^2})\\
&\leq& 1 + 2 \cd \frac{2}{2^2} + 4 \cd \frac{1}{4^4}+ \ldots + 2^{n-1} \cd  \frac{1}{(2^{n-1})^2}\\
&\leq& \seriesnull{\frac{1}{2^i}} = \frac{1}{1-\frac{1}{2}} = 2
\end{array}$\\
2.16a) $\seriesnull{\frac{1}{2^i}}$ Kgt., Grenzwert$\leq 2$.
(später: Grenzwert ist $\frac{\pi^2}{6})$\\
Es gilt allgemeiner:\\
$s \in \N, s \geq 2 \Rightarrow \seriesnull{\frac{1}{i^s}}$ konvergiert.\\
Allgemeiner: $s\in \R , s > 1 \Rightarrow \seriesnull{\frac{i}{i^2}}$ konvergiert
\i $\seriesnull{(-1)^i \cd \frac{1}{i}}$ konvergiert:\\
$s_{2n} = \underbrace{(-1 + \frac{1}{2})}_{<0} + \underbrace{(-\frac{1}{3} +\frac{1}{4})}_{<0} + \ldots \underbrace{(- \frac{1}{2n-1}+ \frac{1}{2n})}_{<0}\\
s_{2n} \leq s{2(n+1)}$ für alle $n \in \N\\
(s_{2n})$ ist monoton fallend.
$s_{2n-1} = -1 + \underbrace{(\frac{1}{2} - \frac{1}{3})}_{>0}+ \ldots + \underbrace{(\frac{1}{2n-2} - \frac{1}{2n-1})}_{>0}\\
(s_{2n-1})$ ist monoton wachsend\\
Ist $k$ ungerade, so ist $s_k < s_l:$ Wähle n so, dass $2n-a \geq k, 2n \geq l$\\
$s_k \underset{(2)}{\leq} s_{2n-1} \underset{\uparrow}{<} s_{2n} \underset{(1)}{\leq} s_l$\\
\hspace*{0.75cm} $s_{2n} = s_{2n-1} + \frac{1}{2n}$\\
\begin{figure}[h!]
\centering \caption{Monotonie}
\begin{tikzpicture}
\begin{axis}[height=3cm,width=9cm,axis x line=center,
axis y line=none,axis equal,ymin = 0,xmin = 0,ymax = 0,xmax = 7,xtick ={1,2,3,4,5,6}, xticklabels={$s_1$,$s_3$,$s_5$,$s_6$,$s_4$,$s_2$},ytick ={},yticklabels={}]             
\end{axis}
\end{tikzpicture}
\end{figure}
Abstand $s_{2n} - s_{2n-1} = \frac{1}{2n}$ geht gegen 0.\\
$\sup \{s_{2n-1} : n \geq 1\} \\
\inf\{s_{2n} : n \geq 1\} \\
= \lim\limits_{i \to \infty} (-1^i) \frac{1}{i} \in ]-1, -\frac{1}{2}[$ (Es gilt $\lim = -\ln 2)$
\end{enumerate}
\subsection*{Bemerkung}
Was bedeutet $0.\bar{8} = 0.88888888\ldots$? (Dezimalsystem)\\
$0.\bar{8} = \frac{8}{10} + \frac{8}{100} + \frac{8}{1000}+ \ldots = 8 \cd \seriesnull{\frac{1}{10^i}} = 8 \cd (\frac{10}{9} - 1) = \frac{8}{9}\\
\seriesnull{\frac{1}{10^i}} = \seriesnull{(\frac{1}{10})^i} = \frac{1}{1 - \frac{1}{10}} = \frac{10}{9}$
\subsection{Satz (Leibniz-Kriterium)}
Ist $(a_i)_{i \geq k}$ eine monoton fallende Nullfolge (insbesondere $a_i \geq 0$ falls $i \geq k$), so ist $\series{(-1)^i a_i}$ konvergent.
\subsection{Satz (Majoranten-Kriterium)}
Seien $(a_i)_{i \geq k}, \ (b_i)_{i \geq k}$ Folgen, wobei $b_i \geq 0$ für alle $i \geq k$ und $\abs{a_i} \leq b_i$ für alle $i \geq k$. Dann gilt\\
Ist $\series{b_i}$ konvergent, so auch $\series{a_i}$ und $\series{\abs{a_i}}$. Für die Grenzwerte gilt:\\
$\abs{\series{a_i}} \leq \series{\abs{a_i}} \leq \series{b_i}$
\begin{proof}
Konvergenz\\
von $\series{\abs{a_i}}$ folgt aus 2.16 a).\\
$\series{\abs{a_i}} \leq \series{b_i}$ folgt aus 2.8 f).\\
Sei $m > n$:\\
$\abs{\sum\limits_{i = k}^{m} a_i - \sum\limits_{i=k}^{n} b_i} = \sum\limits_{i = n+ 1}^{m} a_i \leq \sum\limits_{i = n+1}^{m} \abs{a_i} = \abs{\sum\limits_{i=k}^{m} \abs{a_i}-\sum\limits_{i=k}^{n} \abs{a_i}}$\\
Mit Cauchy-Kriterium 2.17 folgt daher aus der Konvergenz von $\sum\limits_{i=k}^{m} \abs{a_i}$ auch die von $\sum\limits_{i=k}^{\infty} a_i$.\\
\end{proof}
\subsection{Beispiel}
$\sum\limits_{i=1}^{\infty} \frac{1}{+\sqrt{i}}
\\\sqrt{i} \leq i$ für alle $i \in \N\\
\frac{1}{\sqrt{i}} \geq \frac{1}{i}$ für alle $i \in \N\\
$Ang. $\sum\limits_{i=1}^{\infty} \frac{1}{+\sqrt{i}}$ konvergiert.
$\Rightarrow \sum\limits_{i=1}^{\infty} \frac{1}{i}$ konvergiert. $\lightning$\\
Widerspruch zu 2.20 b)
\bigskip\\
$a_i = (-1)^i \frac{1}{i}$\\
2.20d): $\sum\limits_{i=1}^{\infty} a_i$ konvergiert,
aber $\sum\limits_{i=1}^{\infty} \abs{a_i}$ konvergiert nicht. $(\star)$
\subsection{Definition}
$\series{a_i}$ hei\ss t \underline{absolut konvergent}, falls $\series{\abs{a_i}}$ konvergiert. \\
(Falls alle $a_i \geq 0:$ Konvergent = absolut Konvergent)
\subsection{Korollar}
Ist $\series{a_i}$ absolut konvergent, so auch konvergent. Die Umkehrung gilt im Allgemeinen nicht.\\
\underline{Beweis}: 1.Behauptung 2.22 mit $b_i = \abs{a_i}$\\
Umkehrung siehe $(\star)$
\subsection*{Bemerkung}
Was bedeutet $0,a_1,a_2,a_3,a_4 \ldots\\
a_i \in \{0 \ldots 9\}$ (Dezimalsystem)\\
$a_1 \cd \frac{1}{10} a_2 \cd \frac{1}{100} \ldots a_n \cd \frac{1}{10^n} \leq 9 \cd \frac{1}{10} 9 \cd \frac{1}{100} \ldots 9 \cd \cd \frac{1}{10^n}\\
a_i \frac{1}{10} \leq 9 \frac{1}{10}\\
\series{9 \frac{1}{10}} = 9 \cd (\frac{1}{1-\frac{1}{10}}-1) = 1 \Rightarrow \series{a_i \frac{1}{10}}$ konvergiert
\subsection[Satz: Wurzel- und Quotientenkriterium]{Satz}
Sei $\series{a_i}$ eine Reihe.\\
\begin{enumerate}[a)]
\i \underline{Wurzelkriterium}\\
Existiert $q < 1$ und ein Index $i_0$, so dass $\sqrt[i]{\abs{a_i}} \leq q$ für alle $i \geq i_0$.\\
so konvergiert die Reihe $\series{a_i}$ absolut.\\
Ist $\sqrt[i]{\abs{a_i}} \geq 1$ für unendlich viele i so divergiert $\series{a_i}$.\\
\i \underline{Quotientenkriterium}\\
Existiert $q > 1$ und ein Index $i_0$, so dass $\abs{\frac{a_{i+1}}{a_i } } \leq$ für alle $i \geq i_0$,\\
so konvergiert $\series{a_i}$ absolut. 
\end{enumerate}
\begin{proof}
\begin{enumerate}[a)]
\i $\abs{a_i} \leq q^i$ für alle $i \geq i_0$
\i[] $\sum\limits_{i=i_0}^{\infty} q^i$ konvergiert (2.20 a)) 
\i[$\underset{2.22}{\Rightarrow}$] $\sum\limits_{i=i_0}^{\infty} \abs{a_i}$ konvergiert
\i[$\Rightarrow$] $\series{\abs{a_i}}$ konvergiert.
\i[] $\sqrt[i]{\abs{a_i}} \geq 1$ für unendlich viele i
\i[$\Rightarrow$] $\abs{a_i} \geq 1$ für unendlich viele i
\i[$\Rightarrow$] $(a_i)$ sind keine Nullfolge
\i[$\Rightarrow$] $\series{a_i}$ divergiert.\\
\i Sei $i \geq i_0$.\\
$\abs{\frac{a_i}{a_{i0}}} = \abs{\frac{a_i}{a_{i-1}}} \cd \abs{\frac{a_i}{a_{i-2}}} \cd \ldots \cd \abs{\frac{a_{io+1}}{a_{i0} } } \leq q \cd q \cd \ldots \leq = q^{i-i0} = \frac{q^i}{q^{i0}}$\\
\hspace*{5.55cm} $\uparrow$ Voraussetzung: \\
\hspace*{5cm} jeder dieser Quotienten ist $\leq q$\\
$\abs{a_i} \leq \underbrace{\frac{\abs{a_i0}}{q^{i0}}}_{=:c} \cd q^i \hspace*{2cm} \sum\limits_{i=i_0}^{\infty} c \cd q^i$ konvergent\\
\i[$\underset{2.22}{\Rightarrow}$] $\sum\limits_{i=i_0}^{\infty} \abs{a_i}$ konvergiert.\\
\i[$\Rightarrow$] $\series{\abs{a_i}}$ konvergiert
\end{enumerate}
\end{proof}
\subsection{Bemerkung}
\begin{enumerate}[a)]
\i Es reicht \underline{nicht} in 2.26 nur vorauszusetzen, dass $\sqrt[i]{\abs{a_i}} > 1$ für alle $i \geq i_o$\\
bzw.  $\frac{a_{i+1}}{a_i} < 1$ für alle $i \geq i_0$.\\
z.B. harmonische Reihen : $\sum\limits_{i=1 }^{\infty} \frac{1}{i}$ divergiert.\\
\underline{Aber}: $\sqrt[i]{\frac{1}{i}} > 1$ für alle i.\\
\hspace*{30pt} $\frac{i}{i+1} < 1$ für alle i
\i Es gibt Beispiele von absolut konvergenten Reihen mit $\abs{\frac{a_{i+1}}{a_i}}$ für unendlich viele i.
\end{enumerate}
\subsection{Beispiel}
Sei $x \in \R$. Dann konvergiert $\seriesnull{\frac{x^i}{i!}}$ absolut $(0^0=1, 0! = 1):$\\
Quotientenkriterium:\\
$\abs{\frac{x^{i+1} \cd i!}{(i+1)! \cd x^i}} = \abs{\frac{x}{i+1}} = \frac{\abs{x}}{i+1}$ Wähle $i_o$, so dass $i_0 + 1 > 2 \cd \abs{x}$\\
Für alle $i \geq i_0$:\\
$\frac{\abs{x}}{(i+1)} \leq \frac{\abs{x}}{(i_0+1)} < \frac{\abs{x}}{2 \cd \abs{x}} = \frac{1}{2} = q$.
\subsection{Bemerkung}
Gegeben seien zwei endliche Summen\\
$\seriesalg{a_n}{k}{n=0}, \seriesalg{b_n}{l}{n=0}$.\\
$(\seriesalg{a_n}{k}{n=0})(\seriesalg{b_n}{l}{n=0})$
\begin{huge}
$(\star)$
\end{huge}\\
Distributivgesetz: Multipliziere $a_i$ mit jedem $b_i$ und addiere diese Produkte.\\
\begin{huge}
$(\star)$
\end{huge}
$= \underbrace{a_0 b_0}_{\text{Indexsumme 0}} + \underbrace{(a_0b_1+ a_1b_0)}_{\text{Indexsumme 2}} + \ldots + \underbrace{a_kb_l}_{\text{Indexsumme k+l}}$
\subsection{Definition}
Seien $ \seriesnull{a_n}, \seriesnnull{b_n}$ unendliche Reihen.\\
Das \underline{Cauchy-Produkt}(\underline{Faltungsprodukt}) der beiden Reihen ist die Reihe $\seriesnnull{c_n}$, wobei $c_n = \seriesnnull{a_i \cd b_{n-1}} = a_0b_n + ab_{n-1} + \ldots a_nb_0$
\subsection[Satz: Konvergenz im Cauchy Produkt]{Satz}
Sind $\seriesnnull{a_n}, \seriesnnull{b_n}$ absolut konvergent Reihen mit Grenzwert $c ,d$, so ist das Cauchy Produkt auch absolut konvergent mit Grenzwert $c \cd d$. \hfill
Beweis: \cite{k1}

\section{Potenzreihen}
\subsection{Definition}
Sei $(b_n)$ eine reelle Zahlenfolge, $a \in \R$\\
Dann hei\ss t $\sum\limits_{n=0}^{\infty} b_n \cd (x-a)^n$ eine \underline{Potenzreihe} (mit \underline{Entwicklungspunkt} a))
Speziell: $a=0\\
\sum\limits_{n = 0}^{\infty} b_n \cd x^n$\\
(Potenzreihe im engeren Sinne)\\
\underline{Hauptfolge}: Für welche $x \in \R$ konv. die Potenzreihe (absolut)?\\
Suche für $x=a$\\
Dann Grenzwert $b_ 0$ (da $0^0 = 1$)\\
Ob Potenzreihe für andere x konvergiert, hängt von $b_n$ ab!
\subsection{Beispiel} 
\begin{enumerate}[a)]
\i $\seriesnnull{x^n} (b_n =1$ für alle $n$)\\
geometrische Reihe, konvergiert für alle $x \in \  ]-1,1[$
\i $\seriesnnull{2^n \cd x^n} (b_n = 2^n) = \seriesnnull{(2 \cd x)^n}$
konvergiert genau dann nach a), wenn $\abs{2x} < 1$, d.h $\abs{x} < \frac{1}{2}$ d.h. $x \in ]-0.5, 0.5[$\\
\i $\seriesnnull{\frac{x^n}{n!}} (b_n = \frac{1}{n})$\\
konvergiert für alle $x, \ x \in ]-\infty,\infty[ = \R $\\
\end{enumerate}
\subsection{Satz}
Sei $ \seriesnull{b_n \cd x^n}$ eine Potenzreihe (um 0). Dann gibt es $R \in \R \cup \{\infty\}, R \geq 0$, so dass gilt.
\begin{enumerate}
\i Für alle $x \in \R$ und $\abs{x} < R$ konvergiert Potenzreihe absolut (d.h. $\seriesnull{b_n \cd x^n}$
konvergiert, dann auch $\seriesnull{b_n \cd x^n}$)\\
Falls $R = \infty$, so hei\ss t das, dass Potenzreihe für alle $x \in \R$ absolut konvergiert.
\i Für alle $x \in \R$ mit $\abs{x} > R$ divergiert $\seriesnnull{b_n \cd x^n}$
\begin{figure}[h!]
\centering \caption[Konvergenzradien]{Konvergenzradien und ihre Aussagen}
\begin{tikzpicture}
\begin{axis}[width =8cm,height=5cm,axis x line=center,
axis y line=none,axis equal,ymin = 0,xmin = -2,ymax = 0,xmax = 2,xtick ={-1, 0 , 1}, xticklabels={-R,  ,R},ytick ={},yticklabels={}]   
\addplot [black, mark = *, nodes near coords=div.,every node near coord/.style={anchor=90}] coordinates {( -1.5, 0)};
\addplot [black, mark = *, nodes near coords=konvgt.,every node near coord/.style={anchor=90}] coordinates {( 0, 0)};
\addplot [black, mark = *, nodes near coords=keine Aus.,every node near coord/.style={anchor=-90}] coordinates {( 1, 0)};
\addplot [black, mark = *, nodes near coords=keine Aus.,every node near coord/.style={anchor=-90}] coordinates {( -1, 0)};
\addplot [black, mark = *, nodes near coords=div.,every node near coord/.style={anchor=90}] coordinates {( 1.5, 0)};     
\end{axis}
\end{tikzpicture}
\end{figure}
$(\lim\limits_{n \rightarrow \infty} \sqrt[n]{\abs{b_n}} = 0 \Rightarrow R = \infty)$
(Für $\abs{x} = R$ lassen sich keine allgemeine Aussagen treffen).\\
R hei\ss t der \underline{Konvergenzradius} der Potenzreihe $\seriesnnull{b_n \cd x^n}$\\
Konvergenzintervall $<-R,R>$\\
besteht aus allen $x$ für die $\seriesnnull{b_n \cd x^n}$ konvergiert.\\
$<$ kann [ oder ] bedeuten.\\
$>$ kann ] oder [ bedeuten.
\end{enumerate}
\begin{proof}
$\abs{x_1,x_2} \R, \abs{x_1} \leq \abs{x_2}$\\
Dann: Falls $\seriesnnull{\abs{b_n} \cd \abs{x_n}^n}$ konvergiert, so auch$ \seriesnnull{\abs{b_n} \cd \abs{x_n}^n}$ (2.22) \begin{huge}
$(\star)$
\end{huge}\\
Falls $\sum b_n \cd x_n$ für alle x absolut konvergiert, so setze $R= \infty$\\
Wenn nicht, so setze $R = \sup \{\abs{x}:x\in \R, \seriesnnull{\abs{b_n}\cd \abs{x_n}} $ konvergiert$ \} < \infty$
Nach $(\star)$ gilt: $\abs{x} < R \Rightarrow \sum b_nx^n$ konvergiert absolut.\\
Für $\abs{x} > R$ konvergiert $\sum b_nx^n$ nicht absolut.\\
Sie konvergiert sogar selbst nicht. \\
\begin{enumerate}[$\Leftrightarrow$]
\i[]$\sqrt[n]{\abs{b_n} \cd \abs{x}^n} \leq q < 1 $ für alle $n \geq n_0$
\i $\abs{x} \cd \sqrt[n]{\abs{b_n}} \leq 1 < 1$ für alle $n \geq n_0$
\i $\lim\limits_{n \rightarrow \infty} \abs{x_n} \cd \sqrt[n]{\abs{b_n}} < 1$
\i[$\uparrow$](setze $\varepsilon = 1 - \lim\limits_{n \rightarrow \infty} \abs{x} \cd \sqrt[n]{\abs{b_n}} > 0)$
\i $\abs{x} < \frac{1}{\lim\limits_{x \rightarrow \infty} \sqrt[n]{\abs{b_n}}}$
\i[] $\exists n_0 \forall n \geq n_0 : s - \frac{\varepsilon}{2} < \abs{x} \cd \sqrt[n]{b_n} \leq s + \frac{\varepsilon}{2} =: q < 1$
\end{enumerate} 
\end{proof}
\subsection{Bemerkung}
Konvergenz von Potenzreihen der Form $\seriesnnull{b_n \cd (x-a)^n}$:\\
gleichen Konvergenzradius $R$ wie $\seriesnnull{b_n \cd x^n}$\\
konvergiert absolut für $\abs{x-a} < R$, d.h $x \in \ ]a-R,  a+R[$
Divergiert für $\abs{x-a} > R$.\\
Keine Aussage für $\abs{x-a} = R$, d.h $ x = a-R$ oder $x = a+R$\\
Konvergenzintervall $<a-R,a+R>$
\subsection{Die Exponentialreihe}\label{sec:3.5}
\begin{enumerate}[a)]
\i Exponentialreihe
\i[] $\seriesnnull{\frac{x^n}{n!}} (b_n = \frac{1}{n!})$
\i[]\hspace*{1pt} 2.28 Reihe konvergiert für alle $x \in \R.$
\end{enumerate}
Setze für $x \in \R : \exp(x):= \seriesnnull{\frac{x^n}{n!}}$\\
\underline{Exponentialfunktion}
$\exp(0) = \frac{0^n}{0!} = 1$\\
\begin{enumerate}
\i[b)] Serien $x,y \in \R\\
\exp(x) \cd \exp(y) \underset{2.31}{=}$ Limes des Cauchy Produkts der beiden Reihen. \\
\i[=] $\seriesalg{(\seriesnull{\frac{x^i}{i!}\cd \frac{y^{n-i}}{(n-i)!}})}{\infty}{n=0}$\\
\i[=]$\seriesalg{(\seriesnull{\frac{1}{n!}\cd \frac{n!}{i! \cd (n-i)!} \cd x^i \cd y^{n-i})}}{\infty}{n=0}$
\i[=]$\seriesalg{(\seriesnull{{n\choose i} \cd x \cd y^{n-i})}}{\infty}{n=0}$
\i[=] $\seriesnnull{\frac{1}{n}! \cd (x+y)^n}$ = exp(x+y)\\
\fbox{$\exp(x+y) = \exp(x) \cd \exp(y)$ für alle x,y $\in \R$}\\
Daraus folgt: 1 = $\exp(0) = \exp(x + (-x)) = \exp(x) \cd \exp(-x)$\\
\fbox{$\exp(-x) = \frac{1}{\exp(x)}$ für alle x $\in \R$}$(\star)$\\
Für alle x $\geq 0: \exp(x) > 0.$ Dann auch wegen $(\star)$\\
\fbox{$\exp(x) > 0$ für alle $x \in \R$}
\begin{figure}[h!]
\centering
\begin{tikzpicture}
\begin{axis}[axis x line=center,
axis y line=center,axis equal,ymin = -0.1,xmin = -2,ymax = 5,xmax = 2]   
\addplot gnuplot[id=e]{exp(x)};
\end{axis}
\end{tikzpicture}
\caption{Die Exponentialreihe}
\end{figure}
\i[c)] $\exp(1) = \seriesnull{\frac{1}{n!}} = e$\\
\uline{Euler'sche Zahl}\\
Approximation $e$ durch
$\seriesnull{\frac{1}{n!}}
\begin{array}{lll}
m = 2 & 1 + 1 + \frac{1}{2} &= 2,5\\
m = 3 & 2,5 + \frac{1}{6} &= 2,\bar{6}\\
...
m = 6 & \frac{326}{126} + \frac{1}{720} &= 2,7180\bar{5}
\end{array}$\\
Es ist: $e \approx 2,71828\ldots$ (irrationale Zahl)\\
$\seriesnull{\frac{1}{n!}}$ konvergiert schnell
\bigskip\\
$m \in \N\\
\exp(m) = \underset{\leftarrow m \rightarrow}{\exp(1 + \ldots + 1)}\\
\exp(1)^m = e^m\\
e^0 = 1$
$\exp(-m) = \frac{1}{\exp(m)} = e^{-m}\\
n \not = 0, n \in \N:\\
e = \exp(1) = \exp(\frac{n}{n}) = \exp(\frac{1}{n}^n)\\
\exp(\frac{1}{n}) = + \sqrt[n]{e} = e^{\frac{1}{n}}\\
\exp(\frac{m}{n}) = e^{\frac{m}{n}}.$\\
Für alle $x \in \mathbb{Q}$ stimmt $\exp(x)$ mit der 'normalen' Potenz $e^x$ überein.\\
Dann definiert man für beliebige $x \in R$:\\
\fbox{$e^x := \exp(x) = \seriesnull{\frac{x^n}{n!}}$}\\
In kürze: Definition $a^x$ für $a>0, x \in \R$\\
\i[d)] Bei komplexen Zahlen kam $e^{it}$ ($i^2 = -1, t \in \R)$ vor als Abkürzung für $\cos(t) +i \sin(t)$\\
Tatsächlich kann auch für jedes $z \in \C$ definieren $e^z = \seriesnull{\frac{z^n}{n!}}$\\
Dabei: Konvergenz von Folgen/Reihen in $\C$ wie in $\R$ mit komplexem Absolutbetrag.\\
Man kann dann zeigen:\\
$\seriesnull{\frac{z^n}{n!}}$ konvergiert für alle $z \in \C$.\\
Dass tatsächlich dann gilt:\\
$e^{it} = \seriesnull{\frac{(it)^n}{n!}} = \cos(t) + \sin(t).$ zeigen wir später\\
\i[2.718${\tiny
...})$] Man kann zeigen.\\
\underline{$e = \lim\limits_{n \rightarrow \infty }(1 + (\frac{1}{n} )^n)$}\\
Bedeutung:
\i[-] Angelegtes Guthaben $G$ wird in einem Jahr mit 100\% verzinst. Guthaben am Ende eines Jahres $2G (= G(1+1)$
\i[-] Angelegtes Geld wird jedes halbe Jahr mit 50\% verzinst. Am Ende eines Jahres (mit Zinsenzinsen)\\
$G (1 + \frac{1}{2}) (1 + \frac{1}{2}) = 2,25 G$\\
n- mal pro Jahr mit $\frac{100}{n}\%$ verzinsen. Am Ende desx Jahres $G (1 + \frac{1}{n})^n.\\
\lim\limits_{n \rightarrow \infty} G (1 + \frac{1}{n})^n = e \cd G \approx 2.718\ldots \cd G$
(stetige Verzinsung)\\
$a\%$ statt $100\% \cd G e^\frac{a}{100}$ 
\end{enumerate}
\section[Funktionen und Grenzwerte]{Reelle Funktionen und Grenzwerte von Funktionen}
\subsection{Definition}
\underline{Reelle Funktionen f in einer Variable} ist Abbildung\\
$ f : D \rightarrow \R$, wobei $D \subseteq \R $ ($D$ = Definitionsbereich).\\
Typisch: $ D = \R$, Intervall, Verschachtlung von Intervallen
\subsection{Beispiel}
\begin{enumerate}[a)]
\i \underline{Polynomfunktionen} (ganzrationale Funktion, Polynome)\\$
\begin{cases}
\R \rightarrow \R &\\
x \rightarrow a_n \cd x^n + \ldots a_1 x + a_0& 
\end{cases}\\
f(x) = a_n \cd x^n + \ldots a_1 \cd x + q\\
a_n \not = 0 : n = $ Grad ($f$)
f = 0 (Nullfunktion), Grad($f$) = $\infty$\\
Grad 0: konstante Funktionen $\not = 0$\\
Graph von $f$:\\
\\
\begin{figure}[h!]
\centering
\begin{tikzpicture}
\begin{axis}[axis x line=center,
axis y line=center]   
\addplot[samples=50] gnuplot[id=4punkt2a]{x**3 - 2*x**2 - x+2};
\end{axis}
\end{tikzpicture}
\caption{$f(x) = x^3 - 2x^2 - x+2$}
\end{figure}
\i $f,g: D \rightarrow R$
$(f \pm g)(x) := f(x) \pm g(x)$ für alle $x \in D$\\
\underline{Summe}: Differenz, Produkt von f und g.\\
Ist $g(x) \not = 0$ für $x \in D$, so \underline{Quotient}.
$\frac{f}{g}(x):= \frac{f(x)}{g(x)}$ für alle $x \in D$,\\
Quotient von Polynomen = (gebrochen-)rationalen Funktionen\\
$\abs{f}(x):= \abs{f(x)}$ Betrag von $f$. 
\i Potenzreihe definiert Funktion auf ihrem Konvergenzintervall.\\
z.B : $e^x = \sum\limits_{n=0}^{\infty} \frac{x^n}{n!}$\\
Fkt. $\R \rightarrow \R$
\begin{figure}[h!]
\centering
\caption{$e^x$}
\begin{tikzpicture}
\begin{axis}[axis x line=center,
axis y line=center]   
\addplot[samples=50] gnuplot[id=4punkt2c]{exp(x)};
\end{axis}
\end{tikzpicture}
\end{figure}
\i Hintereinanderausführung von Funktionen:\\
$f: D_1 \rightarrow \R, g: D_2 \rightarrow \R  f(D_1) \subseteq f(D_2)$, dann $ g \circ f : \\\begin{cases}
D_1 \Rightarrow & \R \\
x \rightarrow & g(f(x)) 
\end{cases} $\\
\i $f(x) = e^x , g(x) = x^2 +1\\
f,g : \R \rightarrow \R \\
(g \circ f)(x) = g(e^x) = (e^x)^2 + 1 = e^2x +1\\
(f \circ g)(x) = f(x^2 +1) = e^{x^2 +1}$
\i \underline{Trigonometrische Funktionen: Sinus- und Cosinusfunktion (vgl. $\C$)}
\begin{figure*}[h!]
\centering
\begin{tikzpicture}
\begin{axis}[
axis x line=center,
axis y line=center,
axis equal,ymin = -2,xmin = -2,ymax = 2,
xmax = 2,xtick ={},
xticklabels={},
ytick ={0.65},
yticklabels={s},
disabledatascaling,
extra tick style = {grid = major}]
\addplot [black, mark = none, nodes near coords=$\pi$,every node near coord/.style={anchor=180}] coordinates {( -1.5, 0)};
\addplot [black, mark = none, nodes near coords=2$\pi$,every node near coord/.style={anchor=180}] coordinates {( 1.25, 0)};
\addplot [black, mark = none, nodes near coords=$\frac{\pi}{2}$,every node near coord/.style={anchor=90}] coordinates {(0, 1.5)};
\addplot [black, mark = none, nodes near coords=$\frac{3\pi}{2}$,every node near coord/.style={anchor=-90}] coordinates {(0, -1.5)};
\addplot [black, mark = none, nodes near coords=$x$,every node near coord/.style={anchor=0}] coordinates {(1.65, 0.4)};
\draw (axis cs:0.77,0.65)--(axis cs:0.77,0);
\draw (axis cs:0,0) circle[radius=1];
\draw(axis cs:0,0)--(axis cs:0.77,0.65);
\draw(axis cs:0,0)--(axis cs:1,0);
\draw[->](axis cs:0.3,0) arc [radius=0.3,start angle=0,end angle=37];
%origin point, circle radius, start angle, end angle, distance c-b, brace radius, brace options
\cucubr{0,2}{1}{0}{37}{0.1}{0.1}{red}
\end{axis}   
\end{tikzpicture}
\caption{Bogenma\ss}
\end{figure*}
\\$0 \geq x \geq 2\pi$ x = Bogenma\ss \ von $\varphi$ in Grad, so $x = \frac{\varphi}{360} \cd \pi\\
\sin(x) = s , \cos(x) = c$
Für beliebig $x \in \R$: \\
Periodische Fortsetzung, d.h. $x \in \R . x = x' + k \cd 2\pi , k \in \Z , x' \in [0,2\pi[\\
\sin(x) := \sin(x')\\
\cos(x) := \cos(x')\\$
\begin{figure}[h!]
\centering
\begin{tikzpicture}
\begin{axis}[ymin=-1,ymax=1,xmin=-5,xmax=5,axis x line=center,
axis y line=center]   
\addplot[blue,samples=500] gnuplot[id=sin3]{sin(x)};
\addplot[red,samples=500] gnuplot[id=cos]{cos(x)};
\end{axis}
\end{tikzpicture}
\caption[Sinus und Cosinus]{sin(x) und cos(x)}
\end{figure}
$\abs{\cos(x)}, \abs{\sin(x)} \leq 1\\
\cos^2(x) + \sin^2(x) = 1\\
\cos(x) = \sin(x + \frac{\pi}{2})\\
\sin(x) = 0 \Leftrightarrow x = k \pi, k \in \Z\\
\cos(x) = 0 \Leftrightarrow x = \frac{\pi}{2} + k \pi, k \in \Z$\\
\underline{Tangens und Cotangensfunktion}\\
$\tan(x) = \frac{\sin(x)}{\cos(x)}$ für alle $x \in \R$ mit $\cos(x) \not = 0$\\
$\cot(x) = \frac{\cos(x)}{\sin(x)}$ für alle $x \in \R$ mit $\sin(x) \not = 0$\\
\begin{figure}[h!]
\centering
\begin{tikzpicture}
\begin{axis}[axis x line=center,
axis y line=center,xmin=-3,xmax=3,ymin=-3,ymax=3] 
\addplot[blue,samples=100] gnuplot[id=tan]{tan(x)};
\addplot[red,samples=100] gnuplot[id=cotan]{1/tan(x)};
\end{axis}
\end{tikzpicture}
\caption[Tangens und Kotangens]{tan(x) and cot(x)}
\end{figure}
\end{enumerate}
\subsection{Definition}
Sei $D \subseteq \R , c \in \R$ hei\ss t \underline{Adharenzpunkt} von D, falls es eine Folge $(a_n)_n , a_n \in D$, mit $\lim\limits_{n \rightarrow \infty} a_n = c$ gibt.\\
$\bar{D} =$ Menge der Adharenzpunkte von D\\
\phantom{D} = \underline{Abschluss} von D\\
klar: $D \subset \bar{D}.\\
d \in D.$ konstante Folge $(a_n)_{n \geq 1}$ mit $a_n = d \ . \lim\limits_{n \rightarrow \infty} a_n = \lim\limits_{n \rightarrow \infty} d = d.$\\
Also: $d \in \bar{D}$.
\subsection[Beispiel]{Beispiel:}
\begin{enumerate}[a)]
\i $a,b \in \R , a > b , D = ]a,b[\\
$\begin{tikzpicture}
\draw[black, solid](0,0)--(7,0);
\node at (1,-0.3){c};
\draw[black,solid](1,-0.14)--(1,0.14);
\node at (2,-0.3){a};
\draw[black,solid](2,-0.14)--(2,0.14);
\node at (5,-0.3){b};
\draw[black,solid](5,-0.14)--(5,0.14);
\end{tikzpicture}\\$
\bar{D} = [a,b] D \in \bar{D}\\
a \in \bar{D}\\
a_n = a + \frac{b-a}{n} \in D, n \geq 2\\
\lim\limits_{n \rightarrow \infty} a_n = a$\\
Also $[a,b] \subset \bar{D}$.\\
Ist $c \ni [a,b]$ ,etwa $ c< a$, dann ist $\abs{a_n -c} \geq a- c > 0$ für alle $a_n \in ]a,b[$ Also: $\lim\limits_{a_n} \not = c$\\
\i $\mathcal{I}$ Intervall in $\R , x_1 , \ldots , x_r \in \mathcal{I},\\
D = \mathcal{I} \ \{x_1,\ldots,x_r\}\\
$\begin{tikzpicture}
\draw[black, solid](0,0)--(7,0);
\node at (1,-0.3){a};
\draw[black,solid](1,-0.14)--(1,0.14);
\node at (1.5,-0.3){$x_1$};
\draw[black,solid](1.5,-0.14)--(1.5,0.14);
\node at (2,-0.3){$x_2$};
\draw[black,solid](2,-0.14)--(2,0.14);
\node at (2.5,-0.3){$x_3$};
\draw[black,solid](2.5,-0.14)--(2.5,0.14);
\node at (3,-0.3){$x_4$};
\draw[black,solid](3,-0.14)--(3,0.14);
\node at (3.5,-0.3){$x_5$};
\draw[black,solid](3.5,-0.14)--(3.5,0.14);
\node at (4,-0.3){$\ldots$};
\draw[black,solid](4,-0.14)--(4,0.14);
\node at (4.5,-0.3){$x_r$};
\draw[black,solid](4.5,-0.14)--(4.5,0.14);
\node at (5,-0.3){b};
\draw[black,solid](5,-0.14)--(5,0.14);
\end{tikzpicture}\\$
\bar{D} = \bar{\mathcal{I}} = [a,b],$\\
falls $\mathcal{I} = <a,b>.$\\
\i $\mathbb{Q} \subset \R \\
\bar{\mathbb{Q}} = \R $
\end{enumerate}
\subsection{Definition}
$f: D \rightarrow , c \in \bar{D}.\\
d \in \R$ hei\ss t \underline{Grenzwert von f(x) für x gegen c},$ d = \lim\limits_{x \rightarrow c},$ wenn für \underline{jede} Folge $(a_n) \in D$, die gegen c konvergiert, die Bildfolge $(f(a_n))_n$ gegen d konvergiert.

\subsection*{Bemerkung}
$\lim\limits_{x \rightarrow c} f(x) = d \Leftrightarrow$ Für alle Folgen $(a_n), a_n \in D$, mit $\lim\limits_{n \rightarrow \infty} a_n =c $ gilt $\lim\limits_{n \rightarrow \infty} f(a_n) =e $
Wenn man zeigen will, dass $\lim\limits_{x \rightarrow c} f(x)$ nicht existiert, gibt es 2 Möglichkeiten:
\begin{enumerate}[-]
\i Suche \underline{eine} \underline{bestimmte} Folge $(a_n), \lim\limits_{n \Rightarrow \infty} a_n = c,$ so dass $\lim\limits_{x \rightarrow \infty} f(a_n)$ nicht existiert.
\i Suche zwei Folgen $(a_n), (b_n), \lim\limits_{x \rightarrow \infty} a_n = c, \lim\limits_{x \rightarrow \infty} b_n = c$ und $\lim\limits_{x \rightarrow \infty} f(a_n) \not = \lim\limits_{x \rightarrow \infty} f(b_n)$
\end{enumerate}
\begin{figure*}[h!]
\centering
\begin{tikzpicture}
\begin{axis}[legend style={draw none},axis equal,ymin = -4,xmin = -3,ymax = 4,xmax = 3,axis x line=center,
axis y line=center]   
\draw [red] (axis cs:0,1)--(axis cs:10,1);
\draw [red] (axis cs:-10,0)--(axis cs:0,0);
\end{axis}
\end{tikzpicture}
\caption{Abschnittsweise definierte Funktion}
\end{figure*}
$a_n = (-1)^n \cd \frac{1}{n}\\
\lim\limits_{n \rightarrow \infty} a_n = 0\\
f(a_n) = (1 0 1 0 1 0 \ldots)\\
\lim\limits_{n \rightarrow \infty} f(a_n)$ existiert nicht.\\
Oder:\\
\textcolor{red}{$a_n = \frac{1}{n}
\lim\limits_{n \rightarrow \infty} a_n = 0$}\\
\textcolor{red}{$b_n = -\frac{1}{n}
\lim\limits_{n \rightarrow \infty} b_n = 0$}\\
Aber: $\lim\limits_{x \rightarrow \infty} f(a_n) \not = \lim\limits_{x \rightarrow \infty} f(b_n)$
\subsection[Beispiel]{Beispiel:}
\begin{enumerate}[a)]
\i Sei $f(x) = b_k x^k + \ldots + b_1x + b_0$, eine Polynomfunktion, $ c \in \R $.
Sei $(a_n)$ Folge mit $\lim\limits_{n \rightarrow \infty} a_n = c\\
\begin{array}{rcll}
\lim\limits_{n \rightarrow \infty} f(a_n) &=& \lim\limits_{n \rightarrow \infty} b_k x^k + \ldots + b_1x + b_0 \\
&=& b_k (\lim\limits_{n \rightarrow \infty} a_n)^k + b_{k-1} \cd (\lim\limits_{n \rightarrow \infty} a_n)^{k-1} + \ldots + b_0 & \text{Rechenregeln f\"ur Folgen, 2.8}\\
&=& b_k \cd c^k + b_{k-1} \cd c^{k-1} + \ldots + b_1 \cd c + b_0 = f(c).
\end{array}$
\begin{figure}[h!]
\centering
\begin{tikzpicture}
\begin{axis}[axis x line=center,
axis y line=center]   
\addplot[samples=25] gnuplot[id=4punkt6a]{x**2};
\end{axis}
\end{tikzpicture}
\caption{$x^2$}
\end{figure}
\i Sei $f(x) = \frac{x^2-1}{x-1},\\
D = R \backslash \{1\}$\\
Auf $D$ ist $f(x) = \frac{(x+1)(x-1)}{(x-1)} = (x+1) $
\begin{figure}[h!]
\centering
\begin{tikzpicture}
\begin{axis}[axis x line=center,
axis y line=center]   
\addplot[samples=10] gnuplot[id=4punkt6b]{x+1};
\end{axis}
\end{tikzpicture}
\caption{x+1}
\end{figure}
$\bar{D} = \R \\
\lim\limits_{x \rightarrow 1} f(x) = $ ?\\
Sei $(a_n)$ Folge mit $D = \R \backslash \{ 1\}$ mit $\lim\limits_{n \rightarrow \infty} a_n = 1\\
f(a_n) = a_n + 1\\
\lim\limits_{n \rightarrow \infty} f(a_n) = \lim\limits_{n \rightarrow \infty} (a_n +1) = 1+ 1 +2.
\lim\limits_{x \rightarrow 1} = 2.$\\
\i $f(x) = \begin{cases}
1 & \text{für } x > 0\\
0 & \text{für } x < 0
\end{cases}
D = \R \\$
\begin{figure}[h!]
\centering
\begin{tikzpicture}
\begin{axis}[legend style={draw none},axis equal,ymin = -4,xmin = -3,ymax = 4,xmax = 3,axis x line=center,
axis y line=center]   
\draw [red] (axis cs:0,1)--(axis cs:10,1);
\draw [red] (axis cs:-10,0)--(axis cs:0,0);
\end{axis}
\end{tikzpicture}
\caption{Abschnittsweise definierte Funktion}
\end{figure}
$\lim\limits_{x \rightarrow 0} f(x)$ ?\\
$ a_n = \frac{1}{n}. \lim\limits_{} a_n =0.\\
\lim\limits_{x \rightarrow \infty} f(a_n) = \lim\limits_{n \rightarrow \infty} 1 =\underline{1}\\
a_n  =- \frac{1}{n}, \lim a_n = 0\\
\lim\limits_{n \rightarrow \infty} f(a_n) = \lim\limits_{n \rightarrow \infty} 0 = \underline{0}.\\
\lim\limits_{x \rightarrow 0}$ existiert nicht.
\i $f(x) = \sin(\frac{1}{x}), D = \R \backslash \{0\}$
\begin{figure}[h!]
\centering
\begin{tikzpicture}
\begin{axis}[axis x line=center,
axis y line=center]   
\addplot[samples=5000] gnuplot[id=sin2]{sin(1/x)};
\end{axis}
\end{tikzpicture}
\caption{$\sin(\frac{1}{x})$}
\end{figure}
$a_n = \frac{1}{n \pi} , f(a_n) = \sin(n \pi) = 0 \\
a'_n = \frac{1}{(2n + \frac{1}{2}\pi)} \rightarrow 0 , f(a'n) = \sin(2\pi n + \frac{\pi}{2}) = 1 \\
\lim (a_n) = 0 \\
\lim(f(a_n)) = \lim 0 = 0
\lim(f(a'_n)) = \lim 1 = 1\\
\lim(f(x))_{x \rightarrow 0}$ existiert nicht
\i $f(x) = x \cd \sin(\frac{1}{x}), D = \R \backslash \{0\}$
\begin{figure}[h!]
\centering
\begin{tikzpicture}
\begin{axis}[ymin = -1,ymax = 1 , xmax = 1,xmin=-1,axis x line=center,
axis y line=center]   
\addplot[samples=5000] gnuplot[id=sin1]{x* sin(1/x)};
\end{axis}
\end{tikzpicture}
\caption{$x \cd \sin(\frac{1}{x})$}
\end{figure}
$\lim\limits_{x \rightarrow 0} f(x) = 0$ dann:\\
$(a_n) \rightarrow 0 , a_n \in \R \backslash \{0\}\\
\lim\limits_{n \rightarrow \infty} f(a_n) = \lim\limits_{n \rightarrow \infty} a_n \cd \sin(\frac{1}{a_n}) \underset{2.8g)}{=} 0$
\end{enumerate}
\subsection{Satz ($\varepsilon-\delta$)-Kriterium}
$f: D \rightarrow \R, c \in \bar{D}.$ Dann gilt: $\lim\limits_{x \rightarrow c} f(x) = d \Leftrightarrow \forall \varepsilon > 0 \exists \delta \forall x \in D : \abs{x-c} \leq \delta \rightarrow \abs{f(x)-d} \leq \varepsilon$
\begin{figure}[h!]
\centering
\caption{geometrische Darstellung des $\varepsilon-\delta$ Kriteriums}
\begin{tikzpicture}
\begin{axis} [
ytick={1,1.5,2},
yticklabels={$d-\varepsilon$,$d$,$d+\varepsilon$},
xtick={3,3.5,4},
xticklabels={$c-d$,$c$,$c+d$},
axis x line=center,
axis y line=center,
xmin=-1,xmax=5,ymin=-1,ymax=4]
\draw[red] (axis cs:-1,1)--(axis cs: 5,1);
\draw[red] (axis cs:-1,2)--(axis cs: 5,2);
\draw[black,dashed] (axis cs:4,-5)--(axis cs: 4,8);
\draw[black,dashed] (axis cs:3,-5)--(axis cs: 3,8);    
\addplot[samples=500] gnuplot{(2/(x+3))*sin(2*x)+(x/3.141592653)};
:    \end{axis}
\end{tikzpicture}
\end{figure}
\begin{proof}
$\rightarrow:$ Angenommen falsch.\\
Dass hei\ss t $\exists \varepsilon > 0$, so dass für alle $\delta >0$ (z.B $\delta = \frac{1}{n}$) ein $x_n \in D$ existiert mit $\abs{x_n -c} \leq \frac{1}{n}$ und $\abs{f(x_n) -d} > \varepsilon\\
\lim\limits_{n \rightarrow \infty} x_n =c.$ Aber:\\
$\lim\limits_{n \rightarrow \infty} f(x_n) \not = d \lightning\\
\Leftarrow:$ Sei $(a_n)$ Folge, $a_n \in D\\
\lim\limits_{n \rightarrow \infty} a_n =c.\\
$Zu zeigen : $ \lim\limits_{n \rightarrow \infty} f(a_n) =d$, d.h $\forall \varepsilon > 0 \exists n(\varepsilon) \forall n \geq n(\varepsilon): \abs{f(a_n)-d} < \varepsilon.\\
$ Sei $\varepsilon >0$ beliebig, ex. $ d > 0$:
\begin{flushright}$(\star)$\end{flushright}
Für alle $x \in D$ mit \underline{$\abs{x-c} \leq \delta$} gilt $\abs{f(x) -d} < \varepsilon$.\\
Da $\lim\limits_{n \rightarrow \infty} a_n =c$, existiert $n_0$ mit $\abs{a_n -c} \geq \delta$ für alle $n \geq n_0$\\
Nach $(\star)$ gilt: $\abs{f(a_n) -d} < \varepsilon \forall n \geq n_0. \checkmark$
\end{proof}
\subsection{Satz (Rechenregeln für Grenzwerte)}
$f,g, D \rightarrow \R, c \in \bar{D}$, Existieren die Grenzwerte auf der rechten Seite der folgenden Gleichungen, so auch die auf der linken (und es gilt Gleichheit)\\
\begin{enumerate}[a)]
\i$ \lim\limits_{x \rightarrow c}(f \pm/\cd g) = \lim\limits_{x \rightarrow c} f(x) \pm/\cd \lim\limits_{x \rightarrow c} g(x).$\\
\i Ist $g(x) \not = 0$ für alle $x \in D$ und $\lim\limits_{x \rightarrow c} g(x) \not = 0$, so\\
\begin{align*}
\lim\limits_{x \rightarrow c}\biggl(\frac{f(x)}{g(x)}\biggr) = \frac{\lim\limits_{x \rightarrow c }f(x)}{\lim\limits_{x \rightarrow c} g(x)}
\end{align*}
\i $\lim\limits_{x \rightarrow c} \abs{f(x)} = \abs{\lim\limits_{x \rightarrow c} f(x)}$\\
\begin{proof}
Folgt aus den entsprechenden Regeln für Folgen.
\end{proof}
\end{enumerate}
\subsection[Beispiel]{Beispiel:}
$f(x) = \frac{x^3 + 3x +1}{2x^2 + 1}, D = \R\\
\lim\limits_{x \rightarrow 2} \underset{4.2b)}{=} \frac{\lim\limits_{x \rightarrow 2}(x^3 + 3x +1) }{\lim\limits_{x \rightarrow 2}(2x^2 + 1)}$\\
$\underset{4.6a)}{=} \frac{4+6+1}{8+1} =\frac{11}{9}$\\
\subsection{Bemerkung}
Rechts- und linksseitige Grenzwerte:\\
Rechtsseitiger Grenzwert:\\
$\lim\limits_{x \rightarrow c^+} f(x) = d \Rightarrow \forall (a_n)_n , a_n \in D, a_n \geq c$ und $\lim\limits_{n \rightarrow \infty} a_n = c $ gilt: $\lim\limits_{n \rightarrow \infty} f(a_n) =d.$
Analog: linksseitiger Grenzwert: $\lim\limits_{x \rightarrow c^-} f(x) = d\\
(a_n \leq c).$\\
\subsection[Beispiel]{Beispiel:}
$f(x) = \begin{cases}
1 & x>0\\
0 & x<0
\end{cases}
D = \R \setminus \{0\}, c = 0 \in \bar{D}\\
\lim\limits_{x \rightarrow 0^+} f(x) = 1, \lim\limits_{x \rightarrow 0^-} f(x) = 0.\\
\lim\limits_{x \rightarrow 0} f(x)$ existiert nicht.\\
Falls $\lim\limits_{x \rightarrow c^+}$ und$ \lim\limits_{x \rightarrow c^-}$ existieren\\
\underline{und} $\lim\limits_{x \rightarrow c^+} f(x) = \lim\limits_{x \rightarrow c^-} = d$\\
so exisitiert $\lim\limits_{x \rightarrow c} f(x) = d$.
\begin{figure}[h!]
\begin{minipage}[t]{\textwidth}
\begin{minipage}[t]{.5\textwidth}
\begin{tikzpicture}
\begin{axis}[ymin = -1,ymax = 1 , xmax = 5,xmin=-5,axis x line=center,
axis y line=center]   
\addplot[samples=5000] gnuplot[id=abgefahrenersinus]{sin(2*x**2) * sin(x) * sin(0.25*x)};
\end{axis}
\end{tikzpicture}
\end{minipage}%
\begin{minipage}[t]{.5\textwidth}
\begin{tikzpicture}
\begin{axis}[domain= -1:100,ymin = -0.5,ymax = 3, extra x ticks={0},xmin =0,xmax=100,axis x line=center,
axis y line=center]   
\addplot[samples=1000] gnuplot[id=geradeasymptote]{(2*x**2+2)/(x**2 + x + 3)};
\end{axis}
\end{tikzpicture}
\end{minipage}
\end{minipage}
\caption{Grenzwerte gegen einen Festen Wert}
\end{figure}
Grenzwert: $d \in \R$
\subsection{Definition}
$D = \langle b, \infty[ , f : D \rightarrow \R$
\hspace*{3cm} (z.B $D=\R$)\\
\underline{$f$ konvergiert gegen $d \in \R$ für x gegen unendlich},\\
$\lim\limits_{f(x)} = d, $falls gilt:\\
$\forall \varepsilon > 0 \exists M = M(\varepsilon) \forall x \geq M: \abs{f(x) - d} < \varepsilon.$\\
(Analog : $\lim\limits_{x \rightarrow - \infty}f(x) = d$)
\subsection[Beispiel]{Beispiel}
\begin{enumerate}[a)]
\i $\lim\limits_{x \rightarrow \infty} \frac{1}{x} = 0$\qquad \begin{minipage}[c]{\textwidth}
\begin{tikzpicture}
\begin{axis}[width =5cm,height=3cm, ymin = -5,ymax = 5 , xmax = 5,xmin=-5,axis x line=center,
axis y line=center]   
\addplot[samples=100] gnuplot[id=1durchx]{x**-1};
\end{axis}
\end{tikzpicture}
\end{minipage}
Sei $\varepsilon > 0.$ Wähle $M = \frac{1}{\varepsilon}$. Dann gilt für alle $x \geq M:\\
\abs{f(x) - 0} = \abs{\frac{1}{x}} \leq \frac{1}{m} = \varepsilon.$\\
\i Allgemein gilt: \\
$P,Q$ Polynome vom Grad k bzw. l $l \geq k\\
P(x) = a_k \cd x^k + \ldots , Q(x) = b_i \cd x^i + \ldots, a_k \not = 0, b_i \not = 0
\lim\limits_{x \rightarrow \infty}\frac{P(x)}{Q(x)} = \begin{cases}
0 & \text{für }l \geq k\\
\frac{a_k}{b_k} &\text{für } l = k
\end{cases}\\$
(Beweis wie für Folgen $\lim\limits_{x \rightarrow \infty}\frac{P(n)}{Q(n)})\\
\lim\limits_{x \rightarrow \infty} \frac{7x^5 + 205x^3+x^2 + 17}{14x^5 + 0,5} = \frac{1}{2}$
\end{enumerate}
\subsection{Bemerkung}
Die Rechenregeln aus 4.8 gelten auch für $x \rightarrow \infty / -\infty$
\subsection{Definition}
\begin{enumerate}[a)]
\i $f: D \rightarrow \R , c \in \bar{D}$\\
\underline{f geht gegen $\infty$ für x gegen c},\\
$\lim\limits_{x \rightarrow c} f(x) = \infty$, falls gilt:\\
$\forall L > 0 \exists \underset{= \delta (L)}{\delta > 0} \forall x \in D : \abs{x -c} \leq \delta \Rightarrow f(x) \geq L.$\\
\i $<b, \infty[ \supset D, f : D \rightarrow \R, $ \underline{f geht gegen $\infty$, für x gegen $\infty$}: $\lim\limits_{x \rightarrow \infty} f(x) = \infty$,\\
falls gilt: \\
$\forall L > 0 \exists M > 0 \forall x \in D, x \geq M, f(x) \geq L$.
\begin{figure}[h!]
\begin{minipage}[t]{\textwidth}
\begin{minipage}[t]{.5\textwidth}
\begin{tikzpicture}
\begin{axis}[ymin = -2,ymax = 2 , xmax = 2,xmin=-2,axis x line=center,
axis y line=center]   
\addplot[samples=10] gnuplot[id=gerade413]{(1.33333333333)*x - 1};
\draw[dashed] (axis cs: 0,1)--(axis cs: 1.5,1);
\draw[dashed] (axis cs: 1.5,0)--(axis cs: 1.5,1);
\draw[->] (axis cs: 1.6,0)--(axis cs: 1.6,1);
\draw[->] (axis cs: 0.4,-0.5)--(axis cs: 1.2,-0.5);
\addplot [black, mark = none, nodes near coords=$x$,every node near coord/.style={anchor=0}] coordinates {(1.85, 0.4)};
\addplot [black, mark = none, nodes near coords=$\infty$,every node near coord/.style={anchor=0}] coordinates {(0.4, -0.5)};
\addplot [black, mark = none, nodes near coords=$L$,every node near coord/.style={anchor=0}] coordinates {(-0.2, 1)};
\addplot [black, mark = none, nodes near coords=$M$,every node near coord/.style={anchor=0}] coordinates {(1.85, -0.1)};
\end{axis}
\end{tikzpicture}
\end{minipage}%
\begin{minipage}[t]{.5\textwidth}
\begin{tikzpicture}
\begin{axis}[ymin = -5,ymax = 5 , xmax = 5,xmin=-5,axis x line=center,
axis y line=center]   
\addplot[samples=500] gnuplot[id=1durchxabs]{abs(x)**-1};
\end{axis}
\end{tikzpicture}
\end{minipage}
\end{minipage}
\caption{Funktionen $\lim\limits_{x \rightarrow \infty} = \infty$}
\end{figure}
\i[](Entsprechend:
$\lim\limits_{x \rightarrow c} f(x) = - \infty \begin{minipage}[c]{\textwidth}
\begin{tikzpicture}
\begin{axis}[width =5cm,height=3cm, ymin = -5,ymax = 5 , xmax = 5,xmin=-5,axis x line=center,
axis y line=center]   
\addplot[samples=100] gnuplot[id=1durchx]{-1 * abs(x)**-1};
\end{axis}
\end{tikzpicture}
\end{minipage}\\
\lim\limits_{x \rightarrow \infty} f(x) = - \infty \begin{minipage}[c]{\textwidth}
\begin{tikzpicture}
\begin{axis}[width =5cm,height=3cm, ymin = -5,ymax = 5 , xmax = 5,xmin=-5,axis x line=center,
axis y line=center]   
\addplot[samples=100] gnuplot[id=1durchx]{-1 *x**3};
\end{axis}
\end{tikzpicture}
\end{minipage}\\
\lim\limits_{x \rightarrow -\infty} f(x) =  \infty \begin{minipage}[c]{\textwidth}
\begin{tikzpicture}
\begin{axis}[width =5cm,height=3cm, ymin = -1,ymax = 5 , xmax = 5,xmin=-5,axis x line=center,
axis y line=center]   
\addplot[samples=100] gnuplot[id=1durchx]{x**2};
\end{axis}
\end{tikzpicture}
\end{minipage}\\
\lim\limits_{x \rightarrow -\infty} f(x) = - \infty$ \begin{minipage}[c]{\textwidth}
\begin{tikzpicture}
\begin{axis}[width =5cm,height=3cm, ymin = -5,ymax = 1 , xmax = 5,xmin=-5,axis x line=center,
axis y line=center]   
\addplot[samples=100] gnuplot[id=1durchx]{-1 *x**2};
\end{axis}
\end{tikzpicture}
\end{minipage}\\
\end{enumerate}
\subsection[Satz: Grenzwerte gegen unendlich]{Satz}\label{sec:4.16}
\begin{enumerate}[a)]
\i[]$f : D \rightarrow \R$.
\i Sei $c \in \bar{D}$, oder c = $\infty, - \infty $\\
falls $\lim\limits_{x \rightarrow c} f(x) = \infty$ oder $-\infty$, so ist $\lim\limits_{x \rightarrow c} \frac{1}{f(x)} = 0.$\\
\i $c \in \bar{D} \supset \R$.\\
Falls $\lim\limits_{x \rightarrow c} f(x) = 0$ und falls $s > 0$\\
existiert mit $f(x) > 0$ für alle $x \in [c-s,c+s], (f(x) < 0)$ \\
dann ist $\lim\limits_{x \rightarrow c} \frac{1}{f(x)} = \infty (-\infty)$
\begin{figure*}[h!]
\centering
\begin{tikzpicture}
\begin{axis}[ymin = -1,ymax = 1 , xmax = 1,xmin=-1,axis x line=center,
axis y line=center]   
\addplot[samples=5000] gnuplot[id=sin1]{x* sin(1/x)};
\end{axis}
\end{tikzpicture}
\caption{$sin(\frac{1}{x}$)}
\end{figure*}
\i Falls $\lim\limits_{x \rightarrow \infty} =0$ und falls $T >0$ existiert mit $f(x) > 0 $ für alle $x \geq T ,$ so $(f(x) < 0)$\\
ist $\lim\limits_{x \rightarrow \infty} \frac{1}{f(x)} = \infty (- \infty)$\\
(Entsprechend für $\lim\limits_{x \rightarrow - \infty})$
\end{enumerate}
\subsection{Beispiel}\label{sec:4.17}
\begin{enumerate}[a)]
\i 
\begin{itemize}
\i $f(x) = \frac{1}{x} , D = ]0,\infty[\\
\lim\limits_{x \rightarrow 0} f(x) = \infty$
\i $f(x) = \frac{1}{x}, D = ]-\infty,0[\\
\lim\limits_{x \rightarrow 0} f(x) = -\infty$
\i $f(x) = \frac{1}{x} , D = ]0,\infty[\\
\lim\limits_{x \rightarrow 0} f(x) = \infty$
\end{itemize} 
\i $\lim\limits_{x \rightarrow \infty} \sin(x)$ existiert nicht
\begin{minipage}[c]{\textwidth}
\begin{tikzpicture}
\begin{axis}[ymin = -1,ymax = 1 , xmax = 5,xmin=-5,ytick ={-1,0,1},axis x line=center,
axis y line=center, width = 5cm, height = 3cm]   
\addplot[samples=100] gnuplot[id=sin417]{sin(x)};
\end{axis}
\end{tikzpicture}
\end{minipage}
\\
\i $P(x) = ak_x^k + \ldots + a_0.\\
\lim\limits_{x \rightarrow \infty} P(x) = \begin{cases}
\infty , \text{falls }& a_k > 0\\
-\infty , \text{falls }& a_k < 0
\end{cases}\\
\lim\limits_{x \rightarrow -\infty} P(x) = 
\begin{cases}
\infty , \text{falls }& a_k > 0 \text{k gerade oder } a_k < 0 \text{ k ungerade}\\
-\infty , \text{falls }& a_k < 0 \text{k gerade oder } a_k > 0 \text{ k ungerade}
\end{cases}$
\i $P(x)$ wie in c)\\
$Q(x) = b_l^l + \ldots + b_0$\\
$\lim\limits_{x \rightarrow \infty} \frac{P(x)}{Q(x)} = 
\begin{cases}
\infty, &\text{falls $a_k$ und $b_k$ gleiche Vorzeichen}\\
-\infty, &\text{falls $a_k$ und $b_k$ verschiedene Vorzeichen}\\
\end{cases}$
\i $\lim\limits_{x \rightarrow \infty} \frac{e^x}{x^n} = \infty$
\begin{figure}[h!tbp]
\centering
\begin{tikzpicture}
\begin{axis}[axis x line=center,
axis y line=center]   
\addplot[samples=100] gnuplot[id=4punkt17e]{exp(x)*(x**5)};
\end{axis}
\end{tikzpicture}
\caption{$\frac{e^x}{x^n}$}
\end{figure}
$\forall L > 0 \exists\ M \forall\ x \ge M : f(x) \ge L$\\
Sei L $ \geq 0, x > 0$. \\
$e^x = \sum\limits_{k =0}^{\infty} \frac{x^k}{k!} > \frac{x^{n+1}}{(n+1)!}\\
\frac{e^x}{x^n} > \frac{x}{(n+1)!}$\\
Ist $x \geq (n+1)! L =: M,$ so ist $\frac{e^x}{x^n} > L$.
\i $\lim\limits_{x \rightarrow \infty} \frac{x^n}{e^x} = 0$.
Folgt aus e) und \ref{sec:4.16}a)
\end{enumerate}
\section{Stetigkeit}
\subsection{Definition}
$f : D \rightarrow \R.$
\begin{enumerate}[a)]
\i $f$ ist \uline{stetig} an $c \in D$, falls $\lim\limits_{x \rightarrow c} f(x) = f(c)$.
\i $f$ hei\ss t (absolut) stetig, falls $f$ an allen $c \in D$ stetig ist.
\end{enumerate}
\subsection{Satz}\label{sec:5.2}
$f: D \rightarrow \R , c \in D.$\\
Existiert Konstante $\mathbf{K} > 0$ mit $\abs{f(x) - f(c)} \leq \mathbf{K} \cd \abs{x-c}$ für alle $x \in D$, dann ist $f$ stetig in c.
\begin{proof}\ 
\\ Sei $\varepsilon > 0$.\\
Wähle $\delta = \frac{\varepsilon}{\mathbf{K}}$. Ist $\abs{x-c} \leq \delta,$ so ist $\abs{f(x) - f(c)} \leq \mathbf{K} \cd \abs{x-c} \leq \mathbf{K} \cd \delta -\varepsilon.\\
$\reversemarginpar{4.7} $\lim\limits_{x \rightarrow c} f(x) = f(c).$
\end{proof}
\subsection{Beispiel}
\begin{enumerate}[a)]
\i Polynome sind auf ganz $\R$ stetig
\i $f(x) = \begin{cases}
0 , \text{falls }&, x \not = 0\\
1, \text{falls }&, x = 0\\
\end{cases} $ \\
$f$ ist nicht steig in 0.\\
$a_n = \frac{1}{n}, a_n \rightarrow 0\\
f(a_n) = 0\\
(f(f(a_n)) \rightarrow 0 \neq f(0)$
\i $f(x) = \begin{cases}
0 , \text{falls }&, x > 0\\
1, \text{falls }&, x < 0\\
\end{cases}$\\
f ist nicht stetig in 0.
\begin{minipage}[c]{0.3\textwidth}
\begin{tikzpicture}
\begin{axis}[legend style={draw none},axis equal,ymin = -4,xmin = -3,ymax = 4,xmax = 3,axis x line=center,
axis y line=center,width=5cm,height=3cm]   
\draw [red] (axis cs:0,1)--(axis cs:10,1);
\draw [red] (axis cs:-10,0)--(axis cs:0,0);
\end{axis}
\end{tikzpicture}
\end{minipage} 
\i $f(x) = \begin{cases}
\sin(\frac{1}{x}) , \text{falls }&, x \not = 0\\
0, \text{falls }&, x = 0\\
\end{cases}\\
\lim\limits_x \rightarrow 0 \sin(\frac{1}{x})$ ex. nicht.\\
$f$ ist nicht stetig in 0.
\begin{minipage}[c]{0.3\textwidth}
\begin{tikzpicture}
\begin{axis}[ymin = -1,ymax = 1 , xmax = 1,xmin=-1,axis x line=center,
axis y line=center,width = 5cm,height =3cm]   
\addplot[samples=5000] gnuplot[id=sin1]{sin(1/x)};
\end{axis}
\end{tikzpicture}
\end{minipage}
\i $f(x) = \begin{cases}
x \cd \sin(\frac{1}{x}) , \text{falls }&, x \not = 0\\
0, \text{falls }&, x = 0\\
\end{cases}\\
\lim\limits_{x \rightarrow 0} x \cd \sin(\frac{1}{x}) = 0 = f(0)\\
f$ ist stetig in 0.
\begin{minipage}[c]{0.3\textwidth}
\begin{tikzpicture}
\begin{axis}[ymin = -1,ymax = 1 , xmax = 1,xmin=-1,axis x line=center,
axis y line=center,width = 5cm,height =3cm]   
\addplot[samples=5000] gnuplot[id=sin1]{x * sin(1/x)};
\end{axis}
\end{tikzpicture}
\end{minipage}
\i $f(x) = \sin(x)\\
g(x) = \cos(x)
$
Sind stetig auf $\R$:
\begin{figure}[h!]
\centering
\caption{Sinus und Cosinus am Einheitskreis}
\begin{tikzpicture}[scale=3,cap=round]
% Local definitions
\def\costhirty{0.8660256}

% Colors
\colorlet{anglecolor}{green!50!black}
\colorlet{sincolor}{red}
\colorlet{tancolor}{orange!80!black}
\colorlet{coscolor}{blue}

% Styles
\tikzstyle{axes}=[]
\tikzstyle{important line}=[very thick]
\tikzstyle{information text}=[rounded corners,fill=red!10,inner sep=1ex]

% The graphic
\draw[style=help lines,step=0.5cm] (-1.4,-1.4) grid (1.4,1.4);

\draw (0,0) circle (1cm);

\begin{scope}[style=axes]
\draw[->] (-1.5,0) -- (1.5,0) node[right] {$x$};
\draw[->] (0,-1.5) -- (0,1.5) node[above] {$y$};

\foreach \x/\xtext in {-1, -.5/-\frac{1}{2}, 1}
  \draw[xshift=\x cm] (0pt,1pt) -- (0pt,-1pt) node[below,fill=white]
        {$\xtext$};

\foreach \y/\ytext in {-1, -.5/-\frac{1}{2}, .5/\frac{1}{2}, 1}
  \draw[yshift=\y cm] (1pt,0pt) -- (-1pt,0pt) node[left,fill=white]
        {$\ytext$};
\end{scope}

\filldraw[fill=green!20,draw=anglecolor] (0,0) -- (3mm,0pt) arc(0:30:3mm);
\draw (15:2mm) node[anglecolor] {$\alpha$};

\draw[style=important line,sincolor]
(30:1cm) -- node[left=1pt,fill=white] {$\sin \alpha$} +(0,-.5);

\draw[style=important line,coscolor]
(0,0) -- node[below=2pt,fill=white] {$\cos \alpha$} (\costhirty,0);

\draw[style=important line,tancolor] (1,0) --
node [right=1pt,fill=white]
{
  $\displaystyle \tan \alpha \color{black}=
  \frac{{\color{sincolor}\sin \alpha}}{\color{coscolor}\cos \alpha}$
} (intersection of 0,0--30:1cm and 1,0--1,1) coordinate (t);

\draw (0,0) -- (t);
\end{tikzpicture}
\end{figure}
Fúr alle $x,c \in \R$ gilt:\\
$\abs{\sin(x) - \sin(c)} \leq \abs{x-c}.\\
\sin(x)$ ist stetig auf $\R$ (5.2, $\mathbf{K}$ =1)
\end{enumerate}
\subsection{Satz (Rechenregeln für Stetigkeit)}
$f,g : D \rightarrow \R, c \in D$,\\
sind $f$ und $g$ stetig in c, dann auch $f \pm/\cdot$ und $\abs{f}$. Ist $g(x) \neq 0$ für alle $x \in D,$ so ist auch $\frac{f}{g}$ stetig in c.
\begin{proof}
Folgt aus 4.8
\end{proof}
\subsection[Satz: Hintereinanderausführung von stetigen Funktionen]{Satz}
$D,D' \subseteq \R , f : D \rightarrow \R,\\
g: D' \rightarrow \R , f(D) \subseteq D'.$\\
Ist $f$ stetig in $c \in D$ und ist g stetig in $f(c) \in D'$, so ist $g \circ f$ stetig in c,
\begin{proof}
$(a_n) \rightarrow c, a_n \in D$.\\
f stetig: $f(a_n) \rightarrow f(c)$\\
g stetig in f(c): $(g \circ f)(a_n)(a_n)\ (g \circ f)(c)$
\end{proof}
\subsection{Beispiel}
\begin{enumerate}[a)]
\i $f(x) = \sin(\frac{1}{\abs{x^2-1}}), D = \R \setminus \{1,-1 \}.
f$ ist stetig auf D.
Folgt aus $5.3_{\text{a),f) und 5.4,5.5}}$.
\i $f(x) = \begin{cases}
x \cd \sin(\frac{1}{x}) &\text{falls } x \neq 0\\
0 &\text{falls } x = 0 
\end{cases}$\\
stetig auf $\R$,
5.3e) für $c = 0$ für $c \neq 0$. 5.3,5.4,5.5
\i $f(x) = \tan(x) (= \frac{\sin(x)}{\cos(x)})\\
D = \R \setminus \{\frac{\pi}{2}+ k\pi: k \in \Z \}
f$ stetig auf D
\end{enumerate}
\subsection[Satz: Stetigkeit von Potenzreihen]{Satz}
Sei $f(x) = \sum\limits_{i=0}^{\infty} a_i (x-a)^i$\\
eine Potenzreihe mit Konvergenzradius R. Dann ist $f$ stetig $m]a-R[ =: D\\
c \in D
\lim\limits_{x \rightarrow c} f(x)$
\begin{enumerate}[=]
\i$ \lim\limits_{x \rightarrow c} f(x) \lim\limits_{n \rightarrow \infty} \sum\limits_{i=0}^{\infty} a_i (x-a)^i$
\i[$\underset{=}{?}$] $ \lim\limits_{n \rightarrow \infty} \lim\limits_{x \rightarrow c} \sum\limits_{i=0}^{\infty} a_i (x-a)^i$ \marginpar{\cite{k3}}
\i $ \lim\limits_{n \rightarrow \infty} \sum\limits_{i=0}^{\infty} a_i (x-a)^i = f(c)$
\end{enumerate}
\subsection{Korollar}\label{sec:5.8}
$f(x) = \exp(x) = e^x$ ist stetig auf $\R$
\subsection{Satz (Nullstellensatz für stetige Funktionen)}\label{sec:5.9}
$f: D \rightarrow$ stetig, $[u,v] \subseteq D, u < v$\\
Es gelte $f(v) \cd f(v) < 0$ \\
(d.h $f(u) > 0, f(v) > 0$, oder $f(u) > 0, f(v) < 0$)
Dann existiert $w \in ]u,v[$ mit $f(v)=0$\\
\begin{proof}
O.B.d.A., $f(n) < 0 < f(v).$\\
\begin{figure}[h!]
\centering
\begin{tikzpicture}
\begin{axis}[xmin=-5,xmax=2,axis x line=center,
axis y line=none,disabledatascaling, xtick={-2,0,2},xticklabels={$a=u$,$c$,$v=b$}]
\end{axis}
\end{tikzpicture}
\end{figure}
Bijektionsverfahren:\\
Falls $f(c) < 0$, so $a = c$, sonst $b= c$. Liefert Folgen $(a_n),(b_n)$ und eindeutig bestimmte $w \in [u,v]$ mit $a_n \leq a_{n+1} w \leq b_{n+1} \leq b_{n}$ für alle $n$\\
$f(a_n) < 0\\
f(b_n) \geq 0\\
$für alle $n$.
$\lim\limits_{n \rightarrow \infty} a_n = \lim\limits_{n \rightarrow \infty} b_n = w
f$ ist stetig in w $\Rightarrow \lim\limits_{n \rightarrow \infty} f(a_n) = \lim\limits_{n \rightarrow \infty}f(b_n) = f(b_n) = f(w).\\
f(a_n) < 0 \forall n \rightarrow \lim\limits_{n \rightarrow \infty} f(a_n) \leq 0.\\
f(b_n) \geq 0 \forall n \rightarrow \lim\limits_{n \rightarrow \infty} f(b_n) \geq 0.\\
\Rightarrow 0 = \lim(a_n) = \lim(b_n) = f(w).$
\end{proof}
\subsection{Korollar (Zwischenwertsatz)}\label{sec:5.10}
$f: D \rightarrow \R $ stetig, $[u,v] \subseteq D$\\
Dann nimmt $f$ in $[u,v]$ jeden Wert zwischen $f(u)$ und $f(v)$ an (und evtl. weitere)
\begin{figure}[h!]
\centering
\begin{tikzpicture}
\begin{axis}[xmin=-0.5,xmax=5,axis x line=center,axis y line=center, ymin = -1.5,ymax = 5,disabledatascaling,ytick = {2,1},yticklabels={$f(u)$,$f(v)$}, xtick={1,3},xticklabels={$u$,$v$}]
\addplot[samples=100,domain={1.5:5},red] gnuplot {cos(0.5*(x-2)**1.5)*sin(2.6*x)+1*(cos(1.5*x))+2};
\addplot[samples=100,domain={0:1.5}] gnuplot {cos(0.5*(x-2)**1.5)*sin(2.6*x)+1*(cos(1.5*x))+2};
\end{axis}
\end{tikzpicture}
\caption{Zwischenwerte}
\end{figure}
\begin{proof}
O.B.d.A $f(u) < f(v)$\\
Sei $f(u) < b < f(w)\, b$\quad beliebig, aber dann fest.\\
Definiere $g(x) = f(x) -b $ stetig\\
$g(u) = f(u) -b < 0\\
g(v) = f(v) -b > 0$\\
\ref{sec:5.9} (angewandt auf g): Ex. $w \in ]u,v[$ mit $g(w) = 0$,d.h $f(w) =b$.
\end{proof}
\subsection{Satz (Min-Max-Theorem)}\label{sec:5.11}
$a,b \in \R , a < b, f : [a,b] \to \R$ stetig\\
(Wichtig: \emph{abgeschlossenes} Intervall)\\
Dann hat $f$ ein Maximum und ein Minimum auf $[a,b]$, d.h es existieren\\ $x_{min},\ x_{max} \in [a,b]$ mit $f(x_{max}) \leq f(x) \leq f(x_max)$ für alle $x \in [a,b]$ (Beweis mit Bisektionsverfahren, \cite{k4})
\subsection*{Zur Erinnerung}
$f: D \to D'$ bijektiv, dann existiert Umkehrfunktion $f^{-1} D' \to D$ mit \\
$f \circ f^{-1} = id_{D'}$ und \\
$f^{-1} \circ f = id_D$\\
zum Beispiel $f(x) = x^2\\
f :[0,\infty[ \to [0,\infty[$\\
bijektiv\\
$f^{-1} :[0,\infty[ \to [0,\infty[\\
f^{-1}(x) = +\sqrt{x}$
\subsection{Definition}
$f: D \to \R$ hei\ss t {\em(streng) monoton wachsend (oder steigend)}, falls gilt:\\
Sind $x,y \in D, x < y$, so ist $f(x) \leq f(y) (f(x) < f(y))$\\
Entsprechend: {\em streng monoton fallend}. $f$ hei\ss t {\em (streng) monoton}, dalls sie entweder (streng) monoton wachsend oder (steng) monoton fallend ist.
\subsection[Satz: Injektive Funktionen nur bei Monotonie]{Satz}
$D$ Intervall (rechte/linke Grenze) $\infty,-\infty$ möglich), $f D \to \R$ stetig. Dann gilt: $f$ ist injektiv auf D $\Leftrightarrow f $ ist streng monoton auf D.
\begin{proof}
$\Leftarrow$ \checkmark \\
$\Rightarrow$: Angenommen $f$ ist nicht streng monoton auf D.\\
Dann existieren $x_1,x_2,x_3,x_4 \in D.$ mit $x_1 < x_2$ und $f(x_1) < f(x_2)$ und $x_3 < x_4$ und $f(x_3) > f(x_4)\\
(f(x_1) = f(x_2)$ bzw. $f(x_3) = f(x_4)$ nicht möglich, da $f$ injektiv)
Jetzt muss man Fallunterscheidungen machen.\\
z.B\\
$x_1 < x_2 < x_3 < x_4,\ f(x_1) < f(x_3) < f(x_2)$
\end{proof}
\subsection{Satz (Stetigkeit der Umkehrfunktion)}\label{sec:5.14}
$D$ Intervall, $f : D \to f(D) =: D'$\\
eine stetige, streng monotone (also bijektive) Funktion. Dann ist die Umkehrfunktion $f^{-1} D' \to D$ stetig.\\
{\em Beweis:} \cite{k5}
\begin{figure}[h!]
\centering
\begin{tikzpicture}
\begin{axis}[axis x line = center, axis y line = center,xmin =-3, xmax=15, ymax= 10, ymin = -10,axis equal]
\addplot[samples=100] gnuplot[id=5punkt14]{x**3};
\addplot[samples=100,color = red, domain = {-3:20}] gnuplot[id=5punkt14b]{x**(0.3333333333)};
\end{axis}
\end{tikzpicture}
\caption{Eine Funktion und ihre Umkehrfunktion}
\end{figure}
$f$ streng monoton wachsend (fallend) $\Rightarrow$ $f{-1}$ streng monoton wachsend (fallend)
\subsection{Korollar}
Ist $n \in \N \begin{cases}
\text{gerade}\\
\text{ungerade}
\end{cases}$, so \\
ist $f(x) = x^n$ stetig und bijektiv $\begin{cases}
[0,\infty[ \to [0, \infty[\\
\R \to \R
\end{cases}$\\
Die Umkehrfunktion $f^{-1} = \sqrt[n]{x}$ ist stetig und bijektiv $\begin{cases}
[0,\infty[ \to [0, \infty[\\
\R \to \R
\end{cases}$
Nach \ref{sec:5.8} ist $\exp(x)$ stetig auf $\R$. Nach \ref{sec:3.5}b) ist $\exp(x) > 0$ für alle $x \in \R$. Für $x > 0$, so ist $\exp(x) = 1 + x + \frac{x^2}{2} + \frac{x^2}{3!} + \ldots \geq 1$, Ist $ x > y$ so ist $\exp(y) = \exp(x + (y-x)) \underset{\ref{sec:3.5}b)}{=} \exp(x) \cd \underbrace{\exp(\underset{<0}{y-x)}}_{> 1} > \exp(x)$\\
\subsection[Satz: Exponentialfunktion und Logarithmus naturalis]{Satz}\label{sec:5.16}
$\exp : \R \to ]0,\infty[$ ist streng monoton wachsend und bijektiv. Die Umkehrfunktion hei\ss t $\ln(x) : ]0,\infty[ \to \R$ ist stetig und streng monoton wachsend und bijektiv.\\
Es gilt: $\ln(x \cd y) = \ln(x) + \ln(y)$ für alle $x,y > 0$, $\ln(1) = 0$
\begin{figure}[h!]
\centering
\begin{tikzpicture}
\begin{axis}[axis x line = center, axis y line = center,axis equal, xmax = 5,ymax=10, ymin= -4, xmin = -3]
\addplot[samples=100] gnuplot[id=5punkt16]{exp(x)};
\addplot[samples=100,color=red,domain ={0:10}] gnuplot[id=5punkt16]{log(x)};
\end{axis}
\end{tikzpicture}
\caption{$\exp(x)$ und $\ln(x)$}
\end{figure}
\begin{proof}
$\exp$ streng monoton steigen s.V,\\
$\lim\limits_{x \to \infty} \exp(x) = \infty$ \hfill(\ref{sec:4.17}e))\\
$\lim\limits_{x \to \infty} \exp(x) = \lim\limits_{x \to \infty} \exp(-x) = \lim\limits_{x \to \infty} \frac{1}{\exp(x)} \underset{\ref{sec:4.16}}{=} 0$
Also: exp: $\R \to ]0,\infty[$ bijektiv\\
$\ln:\ ]0,\infty[ \to \R$, streng monoton wachsend, stetig, bijektiv (\ref{sec:5.14}).\\
$x,y > 0. \exists a,b \in \R $ mit $x \in \exp(a), y =\exp(b).\\
\begin{array}{lcl}
\ln(xy) &=& \ln(\exp(a)  \cd \exp(b))\\
&=& \ln(\exp(a+b)) = a + b\\
&=& \ln(x) + \ln(y)\\
\end{array}$\\
\end{proof}
\subsection[Satz: Wachstum des natürlichen Logarithmus']{Satz}\label{sec:5.17}
$\lim\limits_{x \to \infty} \frac{\ln(x)}{x^n} = 0$ (für jedes $k \in \N)$\\
(D.h. ($\ln(n) \in o(n))$\\
\begin{proof}
$x = \exp(y), x \leq 1,$ d.h $y \leq 0.\\
\frac{\ln(x)}{x^k} = \frac{y}{(\exp(y)^k)} \leq \frac{y}{\exp(y)} \to 0 $\ref{sec:4.17}e)
\end{proof}
\subsection{Definition}\label{sec:5.18}
Für $a > 0$ setze $a^x = \exp(x \cd \ln(a)) \underbrace{(\mathit{\exp(\ln(a))})}_0$
$a \leq e : e^x = \exp(x), a^x$, falls $a > 0$
\begin{figure}[h!]
\centering
\begin{tikzpicture}
\begin{axis}[axis equal,axis y line =center,axis x line =center,ymin = -5, xmin = -5,xmax = 5,ymax= 5]
\addplot[smooth] gnuplot[id=expos]{2**x} node[above,pos=0.05] {$a^x > 1$};
\addplot[smooth,orange] gnuplot[id=expos2]{0.5**x} node[black,above,pos=0.95] {$0 < a^x < 1$};
\draw(axis cs: -6,1)--(axis cs: 6,1) node[black,above,pos=0.95] {$1^x$};
\end{axis}
\end{tikzpicture}
\caption{Verschiedene Arten Exponentialfunktionen}
\end{figure}
\subsection[Satz]{Satz}\label{sec:5.19}
Sei $a > 0$
\begin{enumerate}[a)]
\i $a^x: \R \to ]0,\infty[$ ist streng monoton wachsend für alle $a > 1$ und streng monoton fallend für $0 < a < 1$.
\i $a^x, a^y = a^{x+y}\\
(a^{x^{y}} = a^{xy})$ für alle $x,y \in \R$
\i Für $x = \frac{p}{q} \in Q (p \in \Z, q > 0)$ stimmt Def. von $a^x$ entsprechend. \ref{sec:5.18} mit der der üblichen Definition $a^{\frac{p}{q}} = \sqrt[q]{a^p}$ überein.
\begin{proof}
Folgt aus Definition mit \ref{sec:3.5}
\end{proof}
\end{enumerate}
\subsection{Bemerkung}
Ist $x \in \R$ und $(x_n)$ Folge mit $x_n \in \mathbb{Q},\ \lim\limits_{n \to \infty} x_n = x,$\\
so $\lim\limits_{n \to \infty} a^{x_n} = a^x$ \hfill(Stetigkeit)\\
D.h $a^x$ lässt sich durch $a^{x_n}, x_n \in \mathbb{Q}$, beliebig gut approximieren
\subsection{Definition}
Für $a > 0, a \neq 1$, hei\ss t die Umkehrfunktion von $a^x$ \emph{Logarithmus zur Basis a} \\
$\log_a (x) \hfill (a =2,a=e,a=10 \text{ wichtig})\\
\log_e (x) = \ln (x)\\
$
\begin{figure}[h!]
\centering
\begin{tikzpicture}
\begin{axis}[axis x line=center,
axis y line=center, xmin = -3, ymin = -3,xmax =5,ymax =5]   
\addplot[samples=100] gnuplot[id=5punktirgendwas]{log10(x)};
\addplot[samples=100,domain={-2:5}] gnuplot[id=5punktirgendwas2]{log(x)/(log(0.5))};
\end{axis}
\end{tikzpicture}
\caption{Logithmen mit Basen $>1$ und $<1$}
\end{figure}
\subsection[Satz]{Satz}
Seien $a,b > 0 , a \neq 1 \neq b , x , y > 0$
\begin{align}
\log_a(x\cd y) &= \log(x) + \log(y) \tag{a}\\
\log_a(x^y) &= y \cd \log(x) \tag{b}\\
\log_a(x) &= \log_a(b) \cd \log_b(x) \tag{c}\\
\text{Sind a,b > 1 , so } O(\log_a(n)) &= O(\log_b(n)) \tag{d}
\end{align}
\begin{proof}
$a)$ wie \ref{sec:5.10}\\
b) $a^{y \cd \log_a(a^y)} \underset{\ref{sec:5.19}b)}{=}(a^{\log_a(x)})^{y} = x^y\\
\Rightarrow \log_a(x^y) = \log_a(a^{y \cd \log_a(x)}) = y \cd \log_a(x)$\\
c) $\log_a(x) = \log_a(b^{\log_a(x)}) \overset{b)}{=} \log_b(x) \cd \log_a(b)$\\
d) Folgt aus c), da $\log_a(b) > 0$
\end{proof}
\section{Differenzierbare Funktionen}
\begin{figure}[h!]
\centering
\begin{tikzpicture}[
thick,
>=stealth',
dot/.style = {
  draw,
  fill = white,
  circle,
  inner sep = 0pt,
  minimum size = 4pt
}
]
\coordinate (O) at (0,0);
\draw[->] (-0.3,0) -- (8,0) coordinate[label = {below:$x$}] (xmax);
\draw[->] (0,-0.3) -- (0,5) coordinate[label = {right:$f(x)$}] (ymax);
\path[name path=x] (0.3,0.5) -- (6.7,4.7);
\path[name path=y] plot[smooth] coordinates {(-0.3,2) (2,1.5) (4,2.8) (6,5)};
\scope[name intersections = {of = x and y, name = i}]
\fill[gray!20] (i-1) -- (i-2 |- i-1) -- (i-2) -- cycle;
\draw      (0.3,0.5) -- (6.7,4.7) node[pos=0.8, below right] {Sekante};
\draw[red] plot[smooth] coordinates {(-0.3,2) (2,1.5) (4,2.8) (6,5)};
\draw (i-1) node[dot, label = {above:$P$}] (i-1) {} -- node[left]
  {$f(x_0)$} (i-1 |- O) node[dot, label = {below:$x_0$}] {};
\path (i-2) node[dot, label = {above:$Q$}] (i-2) {} -- (i-2 |- i-1)
  node[dot] (i-12) {};
\draw           (i-12) -- (i-12 |- O) node[dot,
                          label = {below:$x_0 + c$}] {};
\draw[blue, <->] (i-2) -- node[right] {$f(x_0 + c) - f(x_0)$}
                          (i-12);
\draw[blue, <->] (i-1) -- node[below] {$c$} (i-12);
\path       (i-1 |- O) -- node[below] {$c$} (i-2 |- O);
\draw[gray]      (i-2) -- (i-2 -| xmax);
\draw[gray, <->] ([xshift = -0.5cm]i-2 -| xmax) -- node[fill = white]
  {$f(x_0 + c)$}  ([xshift = -0.5cm]xmax);
\endscope
\end{tikzpicture}
\caption{Steigung am Steigungsdreieck}
\end{figure}
Sekante durch (c,f(c)), (x,f(x))\\
Steigung der Sekante:\\
$x \neq c: \underset{\text{Differenzenquotient}}{\frac{f(x) -f(c)}{x-c} =s(x)}$ definiert auf $\R \setminus \{c\}$\\
Falls $\limit{x \to c} \frac{f(x)-f(c)}{x-c}$ existiert: Steigung der Tangente an Graph von $f$ in $(c,f(c)))$\\
(Änderungsrate von $f$ in $(c,f(c))$
\subsection{Definition}
$\mathcal{I}$ Intervall, $f: \mathcal{I} \to \R ,\, c \in \mathcal{I}$
\begin{enumerate}[a)]
\item $f$ he\ss t {\em differenzierbar} (diffbar) an der Stelle $c$, falls $\limit{x \to c} \frac{f(x)-f(c)}{x-c}$ existiert.\\
Grenzwert hei\ss t \emph{Ableitung} oder \emph{Differentialquotient} von $f$ an der Stelle $c$.\\
$f'(c) = \biggl( \frac{df}{dx} \raisebox{1pt}{(c)} \biggr) \qquad
\biggl\lbrack f'(c) = {\displaystyle \lim_{n \to 0}} \frac{f(c+h) -f(c)}{h},\, h := x-c \biggr\rbrack$
\item f hei\ss t \emph{differenzierbar} auf $\mathcal{I}$, falls $f$ in jedem Punkt von $\mathcal{I}$ differenzierbar ist.\\
$f' :\begin{cases}
\mathcal{I} \to R\\
x \to f(x)
\end{cases}$
\end{enumerate}
\subsection{Beispiel}\label{sec:6.2}
\begin{enumerate}[a)]
\item $f(x) = a \cd x^n , n \in \N a \in \R.\\
x \neq c: \frac{ax^n-ac^n}{x-c} = \frac{a(x-c)(x^{n-1}\ldots)}{x-c}\\
\limit{x \to c} \frac{ax^n-ac^n}{x-c} = \limit{x \to c} = \frac{a(x-c)(x^{n-1}\ldots)}{x-c} = a \cd n \cd c^{n-1} = f(x).\\
f'(x) = a \cd n \cd x^{n-1}$ Gilt auch für $n=0$. ($f$ konstant auf $f'$ = 0)
\item $f(x) = \abs{x}$\\
\begin{tikzpicture}
\begin{axis}[axis x line=center,
axis y line=center, xmin = -3, ymin = 0,xmax =3,ymax =3, width=5cm,height=3cm]   
\addplot[samples=100,domain={-3:3}] gnuplot[id=6punkt3]{abs(x)};
\end{axis}
\end{tikzpicture}
\\
$f$ ist diffbar in 0?\\
Zu zeigen $\limit{x \to 0} \frac{\abs{x}-0}{x-0}$ existiert nicht. \\
Sei $(a_n)$ Folge, $a_n < 0 , \limit{n \to \infty} a_n = 0$ (z.B $a_n = -\frac{1}{n}$)\\
$\limit{n \to \infty} \frac{\abs{a_n}}{a_n} = -1\\
b_n > 0 , \lim b_n = 0 $(z.B $b_n = \frac{1}{n})\\
\limit{n \to \infty} \frac{\abs{b_n}}{b_n} = \limit{n \to \infty} \frac{b_n}{b_n} =1$\\
{\em $f'(0)$ existiert nicht!}
\end{enumerate}
\subsection[Satz]{Satz}\label{sec:6.3}
$f: \mathcal{I} \to \R $ in $c \in \mathcal{I}$ diffbar.
Dann gilt für alle $x \in \mathcal{I}:\\
f(x) = f(c) + f'(c) \cd (x-c) + \mathcal{R}(x) \cd (x-c), \,\\
$wobei $\mathcal{R}, \mathcal{I} \to \R$ stetig in $c$, $\limit{x \to c}\mathcal{R}(c) = 0$\\
\begin{figure}[h!]
\centering
\begin{tikzpicture}
\begin{axis}[axis x line=center,
axis y line=center, xmin = 0, ymin = 0,xmax =5,ymax =5,ytick={2.1,2.5},yticklabels={$f(x_0)$,$f(x_1)$},xtick={2.1,2.7},xticklabels={$x_0$,$x_1$}]   
\addplot[samples=2,domain={0:5}] gnuplot[id=sekante]{x};
\addplot[samples=50,domain={0.5:4.5}] gnuplot[id=sekante]{x**2-3*x+sin(x)+3.15};
\end{axis}
\end{tikzpicture}
\caption{Sekante an Funktion}
\end{figure}
D.h. : $f$ lässt sich in der Nähe von c sehr gut durch eine lineare Funktion (d.h Graph ist Gerade) approximieren.
\subsection{Korollar}\label{sec:6.4}
$f : \mathcal{I} \to \R$ diffbar in $c
\Rightarrow f$ ist steig in $c$.
Beweis folgt aus \ref{sec:6.3}\\
Beachte: Umkehrung von 6.4 gilt im Allgemeinen nicht. \ref{sec:6.2}b).\\
Diffbare Funktionen sind stetig, aber sie haben keine Knicke im Graphen.
\subsection{Satz (Ableitungsregeln)}\label{sec:6.5}
$\mathcal{I}$ Intervall, $c \in \mathcal{I}$. Für a)-c)\\
seien $f,g : \mathcal{I} \to \R $ diffbar in c\\
\begin{enumerate}[a)]
\item $\alpha, \beta \in \R$, so $\alpha f + \beta g $ diffbar in $c$,
$$(\alpha f + \beta g)'(c) = \alpha \cd f'(c) + \beta \cd g'(c)$$
\item (Produktregel) $f \cd g$ diffbar in $c$,\\
$$(f \cd g)'(c) = f(c) \cd g'(c) + f'(c) \cd g(c)$$
\item (Quotientenregel) Ist $g (x) \neq 0$ auf $\mathcal{I}$, so $$\frac{f}{g}'(c) = \frac{f'(c) \cd g(c) - f(c) \cd g'(c)}{g(c)^2}$$
\item (Kettenregel)
$\mathcal{I}_1$ Intervall, $f: \mathcal{I} \to \mathcal{I}_1,$ diffbar in $c\, ,g:\: \mathcal{I} \to \R$ diffbar in $f(c)$, so $g \circ f$ diffbar in $c$, und \\
\[(g \circ f)' = g'(f(c)) \cd f'(c) \]
\end{enumerate}
\begin{proof}
Nur b):\\
$\limit{x \to c} \frac{f(x) \cd g(x) - f(c) \cd g(c)}{x-c} = \limit{x \to c} \frac{f(x)(g(x) -g(c)) + g(c)(f(x) -f(c)}{x-c} = \limit{x \to c}f(x) \cd \limit{x \to c} \frac{g(x) -g(c)}{x-c} + g(c) \cd \limit{x \to c} \frac{f(x) -f(c)}{x-c} \underset{\ref{sec:6.4}}{=}f(c) g'(c) + g(c) f'(c).$\\
\end{proof}
\subsection{Beispiel}
\begin{enumerate}[a)]
\item $f(x) = a_n \cd x^n + a_{n-1} \cd x^{n-1} + \ldots + a_0\\
f'(x) \underset{\substack{\ref{sec:6.5}a)\\
\ref{sec:6.2}a)}}{=} a_n \cd n \cd x^{n-1} \cd x^{n-2} + \ldots + a_1$
\item $f(x) = \frac{1}{x^n} = x^{-n} \, (n \in \N)\\
\mathcal{I} = ]0,\infty[\\
f'(x) \underset{\substack{\ref{sec:6.2}a)\\
\ref{sec:6.5}c)}}{=} \frac{0 \cd x^n - 1 \cd x^{n-1}}{x^{2n}} = \frac{-n}{x^{n+1}} = (-n) \cd x^{-n-1}$ gilt auch auf $]-\infty,0[$
\item $h(x) = (x^2 + x + 1)^2$\\
(\ref{sec:6.5}d): $f(x) = x^2 + x + 1)\\
\phantom{(\ref{sec:6.5}d)}g(x)= x^2\\
h'(x) = 2 \cd (x^2 + x + 1) \cd (2x + 1)$
\end{enumerate}
\subsection[Satz]{Satz}\label{sec:6.7}
\begin{enumerate}[a)]
\i $\lim\limits_{x \to 0} \frac{\sin(x)}{1}$
\i $\lim\limits_{x \to 0} \frac{1-\cos(x)}{x} = 0$
\end{enumerate}
\begin{proof}\ \\
\begin{enumerate}[a)]
\i Elementargeometrisch + Additionstheoreme \ref{sec:1.7}
(Man zeigt: $\cos(x) < \frac{\sin(x)}{x} < 1$ für $0 < x < \frac{\pi}{2}$
\i $\frac{1-\cos(x)}{x} = \frac{(1-\cos(x))}{x(1+\cos(x))} = \frac{1-\cos(x)}{x(1 + \cos(x))} = \frac{sin^2(x)}{x} \cd \frac{x}{1 + \cos(x)} \to 0$
\end{enumerate}
\end{proof}
\subsection[Satz: Ableitungsregeln von cosinus und sinus]{Satz}
\begin{enumerate}[a)]
\i $f(x) = \sin(x)$, so $f'(x) = \cos(x)$
\i $f(x) = \cos(x),$ so $f'(x) = -\sin(x)$
\i $f(x) = \tan(x)$, so $f'(x) = 1 + \tan^2(x)$
\end{enumerate}
\begin{proof}
a), $c \in \R$\\
$\sin'(c) = \lim\limits_{h \to 0}\frac{\sin(h+c) + \sin(c)}{h}\\
\lim\limits_{h \to 0} \frac{\sin(c) \cd \cos(h) + \cos(c) \cd \sin(h) - \sin(c)}{h}\\
= \frac{\sin(c) \cd \cos(h) -1}{h} + \lim\limits_{h \to 0} \frac{\cos(c)\sin(h)}{h} = \sin(c) \cd 0 + \cos(c) \cd 1 = \cos(c)$
b) analog\\
c) $f(x) = \frac{\sin(x)}{\cos(x)}$ Quotientenregel + a)b) + $\sin^2(x) + \cos^2(x) =1$.\\
\end{proof}
\subsection{Beispiel}
\begin{enumerate}[a)]
\item $f(x) = \begin{cases}
1 & \text{für }x < 0\\
\cos(x) & \text{für } x > 0
\end{cases}$\\
\begin{figure}[h!]
\centering
\begin{tikzpicture}
\begin{axis}[axis x line=center,
axis y line=center, xmin = -3, ymin = -1,xmax =4,ymax =1,]   
\addplot[samples=2,domain={-3:0}] gnuplot[id=6punkt9]{1};
\addplot[samples=100,domain={0:3.1416}] gnuplot[id=6punkt92]{cos(x)};
\end{axis}
\end{tikzpicture}
\caption{Abschnittsweise definierte cosinus Funktion}
\end{figure}
$f$ ist diffbar für alle $ x \ne 0\\
\lim\limits_{x \to 0} \frac{f(x) - f(0)}{x}\\
\lim\limits_{x \to 0^+} \frac{\cos(x) -1}{x} \underset{\text{\ref{sec:6.7}b)}}{=} 0\\
\lim\limits_{x \to 0^-} \frac{1 -1 }{x} = 0\\
\lim\limits_{x \to 0} \frac{f(x) - f(0)}{x} = 0 = f'(0)$
\item $f(x) = \sin^2(x^3) = (\sin(x^3))^2\\
f'(x) = 2 \cd \sin(x^3) \cd (\sin(x^3))' = 6 \cd \sin(x^3) \cd \cos(x^3) \cd x^2$\\
\end{enumerate}
\subsection[Satz: Potenzreihen und diverenzierbarkeit]{Satz}
Im Inneren ihres Konvergenzintervalls definieren Potenzreihen eine Funktion\\
Sei $f(x) = \sum\limits_{k = 0}^{\infty} a_k (x-a)^k$\\
eine Potenzreihe um a mit Konvergenzradius $\mathbf{R} > 0.$\\
Dann ist $f$ in $]a-R, a+R[$ diffbar und es gilt : $\sum\limits_{k =1}^{\infty} k \cd a_k \cd (x-a)^{k-1} = f'(x).$\\
\hfil (gliedweise Ableitung)\newline
\hfil (Beweis \cite{k7})
\subsection{Korollar}\label{sec:6.11}
$(\exp(x))' = \exp(x)$
\begin{proof}
$\exp(x) = \sum\limits_{k = 0}^{\infty} \frac{x^k}{k!}$ für alle $x \in \R \\
(\frac{x^k}{k!}) = \frac{k \cd x^{k-1}}{k!} = \frac{x^{k-1}}{(k-1)!}\\
k = 1, ...$\\
Behauptung folgt.
\end{proof}
\subsection{Satz (Ableitung der Umkehrfunktion)}\label{sec:6.12}
$f: \mathcal{I} \to \mathcal{I}_1$ bijektiv, $\mathcal{I},\mathcal{I}_1$ Intervall (linke und rechte Grenze darf nicht $-\infty/\infty$ sein)
Sei $f$ in $c \in \mathcal{I}$ diffbar und $f'(c) \ne 0$.\\
Dann ist $f': \mathcal{I}_1 \to \mathcal{I}$ in $f(c) \in \mathcal{I}_1$ diffbar, und es gilt: $(f^{-1})' (f(c)) = \frac{1}{f(c)}$\\
Ist $f$ überall auf $\mathcal{I}$ diffbar und $f'(y) \ne 0$ für alle $y \in \mathcal{I}$, so ist $f^{-1}$ auf $\mathcal{I}_1$ diffbar und es gilt: \\
\[(f^{-1})'(x) = \frac{1}{f'(f^{-1}(x))} \] für alle $x \in \mathcal{I}$.\\
\emph{Beweisidee}: $f^{-1}$ diffbar an Stelle $f(c)$, falls $f'(c) \ne 0$. Grund: Graph von $f' =$ Graph von $f$ gespiegelt an Winkelhalbierende $s(x) = x.$\\
\begin{figure}[h!]
\centering
\begin{tikzpicture}
\begin{axis}[axis x line=center,
axis y line=center, xmin = -3, ymin = -1,xmax =4,ymax =1,]   
\addplot[samples=2,domain={-3:3},dashed] gnuplot[id=6punkt11]{x};
\addplot[samples=50,domain={-3:3}] gnuplot[id=6punkt112]{x**3};
\addplot[samples=100,domain={-3:3}] gnuplot[id=6punkt113]{sgn(x) * abs(x)**(1./3.)};
\end{axis}
\end{tikzpicture}
\caption{Zwei Funktionen an der Winkelhalbierenden}
\end{figure}
$(f^{-1} \circ f)(x) =x$\\
Ableiten mit Kettenregel.\\
$f^{-1}(f(x)) \cd f'(x) = 1.$ Behauptung folgt.
\subsection{Bemerkung}
Bedingung $f'(c) \ne 0$ in \ref{sec:6.12} ist notwendig.\\
$f : \begin{cases}
\R \to R\\
x \to x^3
\end{cases}$ bijektiv \qquad
\begin{tikzpicture}
\begin{axis}[axis x line=center,
axis y line=center, xmin = -3, ymin = -1,xmax =4,ymax =1,width=5cm, height=3cm]
\addplot[samples=50,domain={-3:3}] gnuplot[id=6punkt12]{x**3};
\end{axis}
\end{tikzpicture}\\
$f'(0) = 0.$ \hfill ($f'(x) =3x^2)$\\
$f^{-1}(x) = \sqrt[3]{x} = x^{\frac{1}{3}}\\
\lim\limits_{x \to 0} \frac{x^{\frac{1}{3}} - 0}{x - 0} = \lim\limits_{x \to 0} \frac{1}{x^{\frac{2}{3}}} = \infty\\
(f^{-1})'(0)$ existiert nicht. (jedenfalls nicht als reelle Zahl!)
\subsection[Satz]{Satz}\label{sec:6.14}
\begin{tabular}{cll|l}
& $f(x)$ & & $f'(x)$\\
\hline
a) & $a^x$ & $(a \in \R, a> 0), x\in \R$ & $\ln(a) \cd a^x$\\
b) & $\ln(x)$ & auf $]0,\infty[$ & $\frac{1}{x}$\\
c) & $\log_{10}(x)$ & (konst. $a > 0,\, a \ne 1$) auf $]0,\infty[$ & $\frac{1}{\ln(a) \cd x}$\\
d) & $x \cd (\ln(x) -1 )$ & auf $]0,\infty[$ & $\ln(x)$\\
e) & $x^b \cd (b \in \R)$& auf $]0,\infty[ $ & $b \cd x^{b-1}$ 
\end{tabular}
\begin{proof}
a)\\
$f(x) = \exp(x \cd \ln(a))\\
f'(x) \underset{\substack
	{\text{\ref{sec:6.12}}\\\text{Kettenregel}}}{=} \exp(x \cd \ln(a)) \cd \ln(a) = a^x \cd \ln(a)$\\
b) $\ln(x)'  \underset{\text{\ref{sec:6.12}}}{=} \frac{1}{\exp'(\ln(x))} \underset{\text{\ref{sec:6.11}}}{=} \frac{1}{x}$\\
c)$ \log'_a(x) \underset{\text{\ref{sec:6.12}}}{=} \frac{1}{\ln(a) \cd a^{\log_a(x)}} = \frac{1}{\ln(a) \cd x}$
\end{proof}
\subsection{Satz (logarithmische Abbildung)}
$f : \mathcal{I} \to ]0,\infty[$ diffbar.\\
$(\ln(f(x)))' = \frac{f'(x)}{f(x)}$\\
Beweis : Kettenregel und \ref{sec:6.14}b)\\
\subsection{Beispiel}
$f(x) = e^x \cd (\sin(x) +2) \cd x^6$ für $x \ne 0\\
\ln(f(x)) = x + \ln(\sin(x) +2) + 6 \cd \ln(x)\\
\ln(f(x))' = 1 + \frac{\cos(x)}{\sin(x)} + \frac{6}{x}\\
f'(x) = (1 + \frac{\cos(x)}{\sin(x) + 2} + \frac{6}{x}) \cd e^x \cd (\sin(x) + 2) \cd x^6$\\
\subsection{Definition}
$f : D \to \R $ hat \emph{lokales Maximum}
\subsection[Satz]{Satz}\label{sec:6.18}
$f\: : D \to \R $ diffbar.\\
Hat $f$ in $c \in D$ lokales Minimum/Maximum, so $f'(c) =0$
\begin{proof}
\ \\
c lokale Max.stelle.\\
$f'(c)$ existiert nach Voraussetzung.\\
$ f'(c) = \lim\limits_{x \to c^+} \frac{f(x) - f(c)}{x-c}\le 0.\\
f'(c) = \lim\limits_{x \to c^-} \frac{f(x) - f(c)}{x-c}\ge 0.\\
\Rightarrow f'(c) = 0.$
\end{proof}
\noindent\emph{Vorsicht}: $f'(c) = 0$ ist nicht hinreichend für lokale Maxima/Minima.\\
z.B $f(x) = x^3 \quad f'(x) = 3x^2\quad f'(0) = 0\\
f$ hat kein Maximum oder Minimum in 0\\
\begin{figure}[h!]
	\centering
	\begin{tikzpicture}
	\begin{axis}[axis x line=center,
	axis y line=center,axis equal,xmax = 5,xmin = -5, ymax =10,ymin=-10]
	\addplot[samples=50] gnuplot[id=sechspunkt18] {x**3};      
	\end{axis} 
	\end{tikzpicture}
	\caption{Ableitung keine Hinreichende Bedingung für Minima/Maxima}
\end{figure}
Globale Max/Min von $f$ auf $[a,b]$:
\begin{enumerate}[-]
	\item Bestimme $c \in ]a,b[$ mit $f'(c) = 0$ Teste, ob lokale Max/Min.
	\item Teste Intervallgrenzen a und b.
\end{enumerate}
\subsection{Satz (Mittelwertsatz)}\label{sec:6.19}
$\mathcal{I} = [a,b], a < b , a,b \in \R$\marginpar{Speziell: $f(a) =f(b) \Rightarrow \exists c \in ]a,b[$ mit $f'(c) = 0$ Satz von Rolle} $\\
f: \mathcal{I} \to \R$ stetig und diffbar auf $]a,b[$.\\
Dann existiert $c \in ]a,b[$ mit $f'(c) = \frac{f(b)-f(a)}{b-a}$
\begin{figure}[h!]
	\centering
	\begin{tikzpicture}
	\begin{axis}[axis x line=center,
	axis y line=center, xmin = 0, ymin = 0,xmax =5,ymax =5,xtick={2},xticklabels={$c$}]   
	\addplot[samples=2,domain={0:5},dashed] gnuplot[id=sekante]{2*x};
	\addplot[samples=50,domain={0:5}] gnuplot[id=sekante]{2*x**2 - 0.5 * x **3};
	\end{axis}
	\end{tikzpicture}
	\caption{Eine Funktion und ihrer Steigung an der Stelle c}
\end{figure}
\begin{proof}
	Setze $s(x) = f(a) + \frac{f(b) -f(a)}{b-a} \cd (x-a)$\\
	(Sekante durch $(a,f(a)),(b,f(b))$\\
	Def. $h(x) = f(x) - s(x)$. $h(a) = h(b) = 0$.\\
	Zeige: $\exists c \in ]a,b[ $ mit $h'(c) = 0$.\\
	Fertig, denn 
	\begin{align*}
		h'(x) &=f'(x) - s'(x)
		= f'(x) - \frac{f(b)-f(a)}{b-a}\\
		h'(c) &= 0 \Rightarrow f'(c) = \frac{f(b)-f(a)}{b-a} 
	\end{align*}
	Ist $h$ konstant, so kann man jedes $c \in ]a,b[$ wählen. Also sei $h$ nicht konstant. $h$ ist stetig auf $[a.b]$. \ref{sec:5.11}. h nimmt auf $[a,b]]$ globales Max. und Min. an: $x_{max}, x_{min}\, , x_{max} \ne x_{min}$, da $h$ nicht konstant $h(a) = h(b)$ O.B.d.A\\
	$x_{max} \in ]a,b[.$ \ref{sec:6.18} : $h'(x_{max}) = 0$
\end{proof}
\subsection{Korollar}\label{sec:6.20}
$\mathcal{I} = [a,b]. a<b, f: I \to \R$ stetig, diffbar in $]a,b[$. (auch $\mathcal{I} = \R$ oder $[a,\infty]\, ,]-\infty,b]$ erlaubt)
\begin{enumerate}[a)]
	\item Ist $f'(x) = 0$ für alle $x \in ]a,b[,$ so ist $f$ konstant auf $[a,b]$,
	\item Ist $f'(x) \ge 0$ für alle $ x\in ]a,b[,$ so ist $f$ (streng) monoton wachsend auf $\mathcal{I}$
	\item Ist $f(x) \le 0$ für alle $x \in ]a,b[$ so ist $f$ (streng) monoton fallend auf $\mathcal{I}$.
\end{enumerate}
\begin{minipage}[c]{0.5\textwidth}
\begin{proof}
	\ \\
	Wähle $u<v\, , u,v \in [a,b]$ beliebig.\\
	Wende \ref{sec:6.19} auf $[u,v]$ an. $\exists c \in ]u,v[$ mit $f'(c) = \frac{f(v)-f(u)}{v-u}$\\
	Daraus folgt im Fall\\
	a) $f(v) = f(u)$\\
	b) $f(v) \ge f(u)$\\
	c) $f(v) \le f(u)$
\end{proof}
\end{minipage}
\begin{minipage}[t]{0.5\textwidth}
Bedingung für strenge Montonie nur hinreichend, nicht notwendig $f(x)= x^3$ streng monoton steigend $f'(0)=0$
\end{minipage}
\subsection{Korollar}\label{sec:6.21}
$\mathcal{I} = [a,b], a<b$ wie in \ref{sec:6.20}.\\
$c \in ]a,b[. f: \mathcal{I} \to \R $ sei stetig in $\mathcal{I}.\\
f$ auf $\mathcal{I}_0 = ]a,b[ \setminus \{c\}$ diffbar\\
Existiert $\lim\limits_{x \to c} f'(x)$ auf $\mathcal{I}_0$, so existiert $f'(c)$ und $f'(c) = \lim\limits_{x \to c} f'(x).$
\subsection{Satz (Regeln von L'Hôpital)} \label{sec:6.22}
\begin{enumerate}[a)]
	\item $\mathcal{I}$ Intervall, c $\in \mathcal{I}$, $f,g: \mathcal{I} \setminus \{c\} \to \R $ diffbar.\\
	Es gelte $g'(x) > 0$ für alle $ x \in \mathcal{I} \setminus \{c\}$\\
	Oder\quad : $g'(x) > 0$ für alle $ x \in \mathcal{I} \setminus \{c\}$\\
	Es gelte $\lim\limits_{x \to c} f(x) = \lim\limits_{x \to c} g(x) = 0$ oder $ \infty$\\
	Existiert $\lim\limits_{x \to c} \frac{f'(x)}{g'(x)} = L$, so ist $\lim\limits_{x \to c} \frac{f(x)}{g(x)} = L$
	\item $f,g : [a,\infty[ \to \R $ diffbar.\\
	Es gelte $ g'(x) > 0$ für alle $x \in [a,\infty[ $\\
	oder \ \ \quad $g'(x) < 0$ für alle $x \in [a,\infty[ $
	und $\lim\limits_{x \to \infty} f(x) = \lim\limits_{x \to \infty} g(x) = 0 $ oder $\infty$ \\
	Existiert $\lim\limits_{x \to \infty} \frac{f'(x)}{g'(x)} = L$, so ist \\
	\phantom{Existiert} $\lim\limits_{x \to \infty} \frac{f(x)}{g(x)} = L.$
\end{enumerate}
\subsection{Beispiel} \label{sec:6.23}
\begin{enumerate}[a)]
	\item $\lim\limits_{x \to \infty} \frac{\ln(1 + ax)}{x} = ? (a \in \R)$ Zähler definiert für alle $x \in \R$ mit $1+ax >0$ \ref{sec:6.22}a):\\
	$\lim\limits_{x \to 0} \frac{\ln(1 +ax )}{x}\\
	\lim\limits_{x \to 0} \frac{\frac{a}{1+ ax}}{1} = a$\\
	\item $\lim\limits_{x \cd \ln(x)}\\
	\lim\limits_{x\to 0^+} - \frac{-ln(x)}{\frac{1}{x}} \underset{\text{\ref{sec:6.22}}}{=} \lim\limits_{x \to 0^0} \frac{\frac{1}{x}}{-\frac{1}{x^2}}=\\
	\lim\limits_{x \to 0}-\frac{-x^2}{x} = \lim\limits_{x \to 0^+} - x = 0 $
	\item $\lim\limits_{x \to 0} x^x = \lim\limits_{x \to 0} \exp(x \cd \ln(x))\\
	\underset{\text{exp stetig}}{=} \exp(\lim\limits_{x \to 0} x \cd \ln(x)) \underset{\text{b)}}{=} \exp(0) =1$.\\
	(Deshalb definiert man $0^0 = 1$)
	\item $\lim\limits_{x \to \infty} = \frac{\ln(x)}{x}  \underset{\text{\ref{sec:6.22}}}{=}\\
	\lim\limits_{x \to \infty} \frac{\frac{1}{x}}{1} =\\
	\lim\limits_{x \to \infty} = \frac{1}{x} = 0$ (schon in 5.17)
\end{enumerate}
\section{Das bestimmte Integral}
Ziel: Bestimmung des Flächeninhalts zwischen Graph einer Funktion und x-Achse zwischen zwei Grenzen a und b (sofern möglich).
\begin{figure}[h!t]
	\centering
	\begin{tikzpicture}
	\begin{axis}[axis x line=center,
	axis y line=center, xmin = 0, ymin = 0,xmax =5,ymax =5]  
	\path[name path=axis] (axis cs:0,0) -- (axis cs:4,0); 
	\addplot[name path=f,samples=50,domain={0:5}] gnuplot[id=flache]{2*x**2 - 0.5 * x **3};
	\addplot [
	thick,
	color=blue,
	fill=blue, 
	fill opacity=0.05
	]
	fill between[
	of=f and axis,
	soft clip={domain=0:4},
	];
	\end{axis}
	\end{tikzpicture}
	\caption{Flächeninhalt unter einer Funktion $f$}
\end{figure}
\subsection{Definition}
\begin{enumerate}[a)]
	\item $a,b \in \R, a<b. f:[a,b] \to \R$ hei\ss t \emph{Treppenfunktion}, falls es $ a = a_0 < a_1 < \ldots < a_n = b$ gibt, so dass $f$ auf jedem offenem Intervall $]a_i,a_{i+1}[/, , i= 0 \ldots, n-1 $, konstant ist. (Wert an den $a_i$ beliebig.)\\
\begin{figure}[h!]
	\centering
	\begin{tikzpicture}
	\begin{axis}[axis x line=center,
	axis y line=center,axis equal,xmax = 5,xmin = 0, ymax =10,ymin=0]
	\draw(axis cs: 0,4)--(axis cs: 1,4);
	\draw(axis cs: 1,1)--(axis cs: 3,1);
	\draw(axis cs: 3,2)--(axis cs: 5,2);
	\draw(axis cs: 5,2.2)--(axis cs: 8,2.2);        
	\end{axis} 
	\end{tikzpicture}
	\caption{Treppenfunktion}
\end{figure}
	\item $f$ wie in a).
	\[ \int_a^b fdx = \int_a^b f(x)dx := \sum_{i=0}^{n-1} c_i (a_{i+1} -a_i) \]
	wobei $f(x) = c_i$ auf $]a_i, a_{i+1}[.$\\
	\emph{Integral} von $f$ über $[a,b]$ (Integral kann negativ sein)
\end{enumerate}
\subsection{Definition}\label{sec:7.2}
$a,b \in \R . a < b.\\
f: [a,b] \to \R$ hei\ss t \emph{Regelfunktion} (oder integrierbare Funktion) $\Leftrightarrow\\
\forall \varepsilon > 0 \exists$ Treppenfunktion $g:\:[a,b] \to \R$ (abh. von $\varepsilon$): $\abs{f(x) - g(x)} \ge \epsilon$ für alle $x \in [a,b]$.\\
Bedeutung:\\
Gleichmä\ss ige Approximierbarkeit durch Treppenfunktion.
\subsection[Satz: Regelfunktionen]{Satz}\label{sec:7.3}
$\mathcal{I} = [a,b],\, a,b \in \R,\, a<b$.\\
\begin{enumerate}[a)]
\item Jede Regelfunktion $f$ auf $\mathcal{I}$ ist beschränkt d.h. $\exists m,M \in \R :\: m \leq f(x) \leq M$ für alle $x \in [a,b]$.\\
\item Summe, Produkt und Betrag von Regelfunktionen ist wieder eine Regelfunktion
\end{enumerate}
\emph{Beweisidee} für a),b):\\
Man beweist \ref{sec:7.3} zunächst für Treppenfunktionen. Für b): Bestimme gemeinsame Verfeinerung der Intervallunterteilung der beiden Treppenfunktionen
Dann auf Regelfunktionen übertragen.
\subsection[Satz: Regelfunktion und Stetigkeit]{Satz}\label{sec:7.4}
Jede stetige Funktion auf $[a,b]$ ist Regelfunktion
\emph{Beweis}: \cite{k8}\\
\begin{figure}[h!]
\centering
\begin{tikzpicture}
\begin{axis}[axis x line=center,
axis y line=center,axis equal,xmax = 5,xmin = 0, ymax =10,ymin=0, xticklabels={a,b},xtick={1,5}]
\addplot[samples=50,domain={1:5}] gnuplot[id=7punkt4]{ 1/12 * x**5 + sin(x+3.141) + 3};       
\end{axis} 
\end{tikzpicture}
\caption{Treppenfunktion}
\end{figure}
7.4 gilt auch für sogenannte \emph{stückweise stetige} Funktionen auf $[a,b]$.
\begin{figure}[h!]
\centering
\begin{tikzpicture}
\begin{axis}[axis x line=center,
axis y line=center,axis equal,xmax = 5,xmin = 0, ymax =10,ymin=-2]
\addplot[samples=50,domain={1:2}] gnuplot[id=7punkt41]{ 1/12 * x**5 + sin(x+3.141) + 3};
\addplot[samples=50,domain={2:3}] gnuplot[id=7punkt42]{ 1/12 * x**5 + sin(x+3.141) - 1};
\addplot[samples=50,domain={3:4}] gnuplot[id=7punkt43]{ 1/12 * x**5 + sin(x+3.141) + 8};           
\end{axis} 
\end{tikzpicture}
\caption{Abschnittsweise stetige Funktion}
\end{figure}
$[a,b]$ ist Vereinigung \emph{endlicher} Teilintervalle, auf denen Funktion stetig ist.
\subsection{Beispiel}\label{sec:7.5}
\begin{enumerate}[a)]
\item $f(x) =x^2,\, \mathcal{I} = [0,t]$\\
\begin{figure}[h!]
	\centering
	\begin{tikzpicture}
	\begin{axis}[axis x line=center,
	axis y line=center,axis equal,xmax = 5,xmin = 0, ymax =20,ymin=0,xtick={5},xticklabels={t}]
	\addplot[samples=50,domain={0:5}] gnuplot[id=7punkt5]{x**2};
	\draw(axis cs: 0,0)--(axis cs: 0,1);
	\draw(axis cs: 1,2)--(axis cs: 2,2);
	\draw(axis cs: 2,2)--(axis cs: 2,0);
	\draw(axis cs: 1,2)--(axis cs: 1,0);
	\draw(axis cs: 2,4)--(axis cs: 3,4);
	\draw(axis cs: 3,4)--(axis cs: 3,0);
	\draw(axis cs: 2,4)--(axis cs: 2,0);
	\draw(axis cs: 3,9)--(axis cs: 4,9);
	\draw(axis cs: 4,9)--(axis cs: 4,0);
	\draw(axis cs: 3,9)--(axis cs: 3,0);
	\draw(axis cs: 4,16)--(axis cs: 5,16);
	\draw(axis cs: 5,16)--(axis cs: 5,0);
	\draw(axis cs: 4,16)--(axis cs: 4,0);
	\end{axis} 
	\end{tikzpicture}
	\caption{Treppenfunktion (Untersumme) von $x^2$}
\end{figure}
Definition für $x \in \mathbb{N}$ Treppenfunktion.\\
$f_n: [0,t] \to \R \\
f_n(x)=\begin{cases}
(\frac{it^2}{n}) & \text{falls } x\in [\frac{it}{n}, \frac{(i+1)t}{n}] \text{für ein }i \in \{0,\ldots,n-1\}\\
t^2 & \text{falls } x = t
\end{cases}\\
x \in [0,t]:\: \abs{f(x) -f_n(x)} = ?\\
x = t:\: \abs{f(t)-f_n(x)} = 0.\\
0 \le x < t:$ Dann $ x \in [\frac{it}{n}, \frac{(i+1)t}{n}]$ für alle $i \in \{0,\ldots,n-1\}.\\
\abs{f(x)-f_n(x)} = \abs{x^2 - (\frac{it}{n})^2} \le (\frac{(i+1)t}{n})^2 - (\frac{it}{n})^2 = \frac{2it+t^2}{n^2}\le \frac{2t}{n} + \frac{t^2}{n^2} \xrightarrow[n \to \infty]{} 0$\\
\item $f(x) = \begin{cases}
0 & x \in \mathbb{Q}\\
1 & x \not \in \mathbb{Q}
\end{cases}\\
f: [0,1] \to \R$
\begin{figure}[h!]
	\centering
	\begin{tikzpicture}
	\begin{axis}[axis x line=center,
	axis y line=center, xmin = 0, ymin = 0,xmax =5,ymax =2]  
	\addplot[only marks,samples=55] gnuplot {0};
	\addplot[only marks,samples=50] gnuplot {1};
	\end{axis}
	\end{tikzpicture}
	\caption{Nicht integrierbare Funktion}
\end{figure}
\end{enumerate}
\subsection{Lemma}\label{sec:7.6}
$f$ Regelfunktion auf $[a,b]$\\
\begin{enumerate}[a)]
\item $(f_n)_n$ Folge von Treppenfunktion, die \emph{gleichmä\ss ig} gegen $f$ konvergiert, dass hei\ss t es existiert Nullfolge $(a_n)_n$, $a \geq 0$, und $\abs{f_n(x)-f(x)} \le a_n$ für alle $x \in [a,b]$.\\
Dann konvergiert die Folge \[\underbrace{\biggl(\int_{a}^{b}f_n(x)dx \biggr)_n}_{\in \R}\]
\item Sind $(f_n)_n$ und $(g_n)_n$ zwei Folgen von Treppenfunktionen die gegen $f$ gleichmä\ss ig konvergieren, so :\\
\begin{align}
\lim_{n \to \infty} \int_{a}^{b} f_n(x)dx = \lim_{n \to \infty} \int_{a}^{b} g_n(x)dx \tag{\text{WHK, 7.20}}
\end{align}
\end{enumerate}
\subsection{Definition}\label{sec:7.7}
$f : [a,b] \to \R $ Regelfunktion, $(f_n)_n$ Folge von Treppenfunktionen, die gleichmä\ss if gegen $f$ konvergiert (wie in \ref{sec:7.6} a).\\
Definition \emph{(bestimmtes) Integral}:
\[\int_{a}^{b} f(x) dx = \lim_{n \to \infty} \int_a^b f_n(x)dx \]\\
Treppenfunktion:\\
$\sum\limits_{i=0}^{n-1} c_i (x_m - x_i)$\\
\begin{tikzpicture}
\draw (0,0)--(5,0);
\draw (0,0.2)--(0,-0.2);
\node () at (0,0.4) {$a=x_0$};
\draw (1,0.2)--(1,-0.2);
\node () at (1,-0.4) {$a=x_i$};
\draw (2,0.2)--(2,-0.2);
\draw (3,0.2)--(3,-0.2);
\draw (4,0.2)--(4,-0.2);
\node () at (4,-0.4) {$a=x_{i+1}$};
\draw (5,0.2)--(5,-0.2);
\node () at (5,0.4) {$b=x_n$};
\end{tikzpicture}
\subsection{Beispiel}\label{sec:7.8}
$f(x) = x^2$ auf $[0,t]\\
f_n$ wie in \ref{sec:7.5}.\\
\[\int\limits_{a}^b f_n(x)dx = \sum\limits_{i=0}^{n-1}(\frac{it}{n})^2 \cd \frac{t}{n} =\sum\limits_{i=0}^{n+1} i^2 \cd \frac{t^2}{n^3} = \frac{t^3}{n^3}\cd \sum\limits_{i=0}^{n-1} i^2\]\\
Per Induktion nach n kann man zeigen : $\sum\limits_{i=0}^{n-1} i^2 = \frac{(n-1)n(2n-1)}{6}$\\
Also : $\int_{0}^{t} f_n(x)dx = \frac{t^3}{n^3} \cd \frac{(n-1)n(2n-1)}{6}\\
\lim\limits_{n \to \infty} \int_{0}^{t} f_n(x)dx = \frac{t^3}{6} \cd 2 = \frac{t^3}{3}$
Falls $ t > 0 -\frac{t^3}{3}$
\subsection{Satz (Rechenregeln für Integrale)}\label{sec:7.9}
$f,g$ Regelfunktionen auf $[a,b]$.
\begin{align}
\int_{a}^{b} (f+g)(x)dx &= \int_{a}^{b} f(x)dx + \int_{a}^{b} g(x)dx \notag\\
\int_{a}^{b} a \cd f(x)dx &= a \cd \int_{a}^{b} f(x)dx  \tag{a}\\
f(x) \le g(x) \text{ für alle } x \in [a,b] &\Rightarrow \int_{a}^{b} f(x)dx \le \int_{a}^{b} g(x)dx \tag{b}\\
\Bigl\lvert\int_{a}^{b} f(x)dx\Bigr\rvert &\le \int_{a}^{b} \abs{f(x)}dx \tag{c}\\
\text{Sei m $\le f(x)$} \le M\text{ für alle $x \in [a,b]$}: \notag \\
m(b-a) \le \int_{a}^{b} f(x)dx &\le M(b-a) \tag{d}\\
a < c < b,\, \text{ so } \int_{a}^{c} f(x)dx &= \int_{c}^{b} f(x)dx \tag{e}
\end{align}
\subsection{Beispiel}
$a < b. \int\limits_{a}^{b} x^2 dx = \frac{b^3}{3} - \frac{a^3}{3}\\
(o < a < b :)$ \ref{sec:7.9}e $\int\limits_{a}^{b} x^2 dx = \int\limits_{0}^{b} x^2 dx - \int\limits_{0}^{a} x^2 dx = \frac{b^3}{3} - \frac{a^3}{3}$)\\
Analog für die Fälle $a \le 0 < b$ und $a < b \le 0$
\subsection{Satz (Mittelwertsatz der Integralrechnung)}\label{sec:7.11}
$f : [a.b] \to \R$ stetig.\\
Dann existiert $c \in [a,b]$ mit $\int_{b}^{a} f(x)dx = f(c) \cd (b-a)$
\begin{figure}[h!]
\centering
\begin{tikzpicture}
\begin{axis}[axis x line=center,
axis y line=center, xmin = 0, ymin = 0,xmax =5,ymax =5,xtick={1,4.9},xticklabels={$a$,$b$},ytick =3,yticklabels={$f(c)$}]  
\path[name path=axis] (axis cs:0,0) -- (axis cs:4.9,0); \
\draw[red] (axis cs:1,3) -- (axis cs:4.9,3); 
\addplot[name path=f,samples=50,domain={1:5}] gnuplot[id=flache2]{ 1/12 * x**5 + sin(x+3.141) + 3};
\addplot [
thick,
color=blue,
fill=blue, 
fill opacity=0.05
]
fill between[
of=f and axis,
soft clip={domain=1:4.9},
];
\end{axis}
\end{tikzpicture}
\caption{Mittelwertsatz der Integralrechnung}
\end{figure}
\begin{proof}
$f$ ist stetig nimmt also das Maximum von m an Stelle $x_{min}$ und Maximum $M$ an der Stelle $x_{max}$ an. (\ref{sec:5.11})\\
\ref{sec:7.9}d) : $m(b-a) \le \int\limits_{a}^{b} f(x) dx\\
f(x_{min}) = m \le \frac{1}{b-a} \cd \int_{a}^{b} f(x)dx \le M = f(x_{max})$\\
Zwischenwertsatz für stetige Funktionen \ref{sec:5.10}: $\exists c$ zwischen $x_{min}$ und $x_{max}$ (d.h $c \in [a,b])$ mit $f(c) = \frac{1}{b-a} \int_{a}^{b} f(x)dx$\\
\end{proof}
\section{Der Hauptsatz der Differential- und Integralrechnung}
\subsection{Definition}\label{sec:8.1}
\begin{enumerate}[a)]
\item Sei $[a,b]$ abgeschlossenes, beschränktes (d.h $a,b \in \R , a < b$) Intervall.\\
$f : [a,b] \to \R$ integrierbar,\\
\[ \int_{a}^{b} f(t) dt = - \int_{b}^{a} f(t) dt \]
\item \[ \int_{a}^{a} f(t) dt = 0 \]
\end{enumerate} \
\begin{table}[h!]
\centering
\begin{minipage}[t]{0.7\textwidth}
	\ref{sec:7.8} $x >0$\\
	$ x > 0 \int_{0}^{x} t^2 dt = $\fbox{$\frac{x^3}{3}$}$ \\ \biggl(\frac{x^3}{3}\biggr)' = x^2$\\ Kein Zufall\\
	$x \le 0\\
	\int\limits_{0}^{x}t^2dt = - \int\limits_{x}^{0} = -(-\frac{x^3}{3}) = \frac{x^3}{3}\\
	\int\limits_{a}^{b}t^2 = \frac{b^3}{3} - \frac{a^3}{3}$ gilt für alle $a,b \in \R$
\end{minipage}
\end{table} 
\subsection{Definition}
Sei $\mathcal{I}$ beliebiges Intervall ($-\infty$ bzw. $\infty$ als linke/rechte Grenze erlaubt).
\begin{enumerate}[a)]
\item $f : \mathcal{I} \to \R$ hei\ss t \emph{lokal integrierbar},wenn $f$ auf jedem ageschlossenem beschränktem Teilintervall $[u,v]$ von $\mathcal{I}$ integrierbar ist.\\
\begin{figure}[h!]
\centering
\caption{Lokal Integrierbar von u bis v}
\begin{tikzpicture}
\begin{axis}[axis x line=center,
axis y line=center, xmin = -3.5, ymin = -2,xmax =5,ymax =5,xtick={-2,4},xticklabels={u,v}]
\addplot[smooth] gnuplot[id=sehrwellig]{0.2*(sin(x) * cos(3*x) - sin(-5*x) * sin(x)) + 1};
\draw node at (axis cs: 4.25,-0.1) {\Huge $]$};
\draw node at (axis cs: -2.25,-0.1) {\Huge $[$};
\end{axis}
\end{tikzpicture}
\end{figure}
(Ist $\mathcal{I}$ selbst abgeschlossen und beschränkt, so \glqq lokal integrierbar \glqq = \glqq integrierbar \glqq)
\item $F : \mathcal{I} \to \R $ hei\ss t \emph{Stammfunktion} der lokal Integrierbaren Funktion $f : \mathcal{I} \to \R $, wenn gilt \[ \int_{a}^{V} f(t)dt = F(v) - F(u) \] für alle $u,v \in \mathcal{I}$.\\
Eine Stammfunktion von $f$ wird auch als \emph{unbestimmtes Integral} von f bezeichnet $F = \int f(t)dt$
\subsection{Bemerkung}
Ist $f$ lokal integrierbar auf $\mathcal{I}$, so gilt \[ \int_{u}^{v}f(t)dt + \int_{v}^{w} f(t)dt = \int_{u}^{w} f(t)dt \marginnote{\text{Folgt aus \ref{sec:7.9} + \ref{sec:8.1}}} \] für alle $u,v,w \in \mathcal{I}$ (nicht notwendig $ u < v  <w$)
\end{enumerate}
\subsection{Beispiel}
\begin{enumerate}[a)]
\item $f(x) = x^2$ lokal integrierbar auf $\R$.\\
Stammfunktion von $f$.\\
$F(x) = \frac{x^3}{3}$\\
\[ \int_{a}^{b} F(b) - F(a) \]
\item $f(x) = \begin{cases}
0 &\text{ für }x < 0\\
1 &\text{ für }x \ge 0\\
\end{cases}$\\
Heaviside - Funktion\\
$f$ ist lokal integrierbar auf $\R$\quad\begin{minipage}[c]{0.3\textwidth}
\begin{tikzpicture}
\begin{axis}[legend style={draw none},axis equal,ymin = -4,xmin = -3,ymax = 4,xmax = 3,axis x line=center,
axis y line=center,width=5cm,height=3cm]   
\draw [red] (axis cs:0,1)--(axis cs:10,1);
\draw [red] (axis cs:-10,0)--(axis cs:0,0);
\end{axis}
\end{tikzpicture}
\end{minipage}\\
Stammfunktion von $f$:\\
$F(x) =  \begin{cases}
0 &\text{ für }x < 0\\
x &\text{ für }x \ge 0\\
\end{cases}$\quad\begin{minipage}[c]{0.3\textwidth}
\begin{tikzpicture}
\begin{axis}[legend style={draw none},axis equal,ymin = -4,xmin = -3,ymax = 4,xmax = 3,axis x line=center,
axis y line=center,width=5cm,height=3cm]   
\draw [red] (axis cs:0,0)--(axis cs:10,10);
\draw [red] (axis cs:-10,0)--(axis cs:0,0);
\end{axis}
\end{tikzpicture}
\end{minipage}\\
Zeige: $\forall u, v \in \R:$
\[\int_{u}^{v} f(t)dt= F(v)-F(u)\]
\begin{align}
\int_{u}^{v} f(t)dt &= 0 F(v) - F(u) \tag{u < v < 0}\\
\int_{u}^{v} f(t)dt &= 0 = \int_{0}^{v} f(t)dt = 1 \cd v = F(v) - F(u) \tag{u < 0 < v}\\
\int_{u}^{v} f(t)dt &= 1 \cd (v - u) F(v) - F(u) \tag{0 < u < v}\\
\int_{u}^{v} f(t)dt &= -(F(u) - F(v))= F(v) - F(u) \tag{u \ge 0}
\end{align}
\end{enumerate}
\subsection[Satz]{Satz}
Sei $\mathcal{I} \ne \emptyset$ Intervall, $f : \mathcal{I} \to \R$ lokal \glqq Integrierbar \grqq
\begin{enumerate}[a)]
\item Ist $F$ Stammfunktion von $f$, so auch $G(x) = F(x) +c$ für jedes $ c \in \R$.
\item Sind $F$ und $G$ Stammfunktionen von $f$, so ist $F(x) = G(x) + c$ für ein $c \in \R$
\item Sei $x_0 \in \mathcal{I}$ beliebig, aber fest gewählt.
Dann ist F(x) = $\int_{x_0}^{x} f(t)dt$ eine Stammfunktion von f.\\
(Beachte \[ \int_{x_0}^{x} f(t)dt = \int_{x_0'}^{x} f(t)dt + \int_{x_0'}^{x} f(t)dt  \])
\end{enumerate}
\begin{figure}[h!]
\centering
\begin{tikzpicture}
\draw circle (4);
\draw circle (3);
\draw circle (2);
\draw circle (1);
\node () at (0,3.65) {alle Fkt.};
\node () at (0,2.55) {lokal integr.};
\node () at (0,1.45) {stetige Fkt.};
\node () at (0,0.0) {diffbare Fkt.};
\end{tikzpicture}
\caption{Die Welt der Funktionen}
\end{figure}
\begin{proof}
a),b)\\
$F$ Stammfunktion, $c\in \R G(x) = F(x)+c$ ist Stammfunktion von $f$:\\
$G(v)-G(u) = F(v) = F(u) = \int_{u}^{v} f(t)dt$\\
Umgekehrt: Seien $F,G$ zwei Stammfunktionen von $f$.
Sei $x_0 \in \mathcal{J}$ halte es fest.\\
$G(x)-G(x_0) = \int\limits_{x_0}^{x}f(t)dt = F(x)-F(x_0)$ für alle $x \in \mathcal{J}\\
G(x) = F(x) + \underbrace{G(x_0)-F(x_0)}_{=: c}$ für alle $x \in \mathcal{J}$
c) $u,v \in \mathcal{J}\\
F(v) - F(u) = \int\limits_{x_0}^{v}f(t)dt = \int\limits_{x_0}^{u} f(t)dt = + \int\limits_{u}^{x_0} f(t)dt + \int\limits_{x_0}^{v} f(t)dt = \int\limits_{u}^{x_0} f(t)dt$
\end{proof}
\subsection[Satz]{Satz}
Jede Stammfunktion einer lokal integrierbaren Funktion ist stetig.
\begin{figure}[h!]
\centering
\begin{tikzpicture}
\draw circle (4);
\draw circle (3);
\draw circle (2);
\draw circle (1);
\node () at (0,3.65) {alle Fkt.};
\node () at (0,2.55) {lokal integr.};
\node () at (0,1.45) {stetige Fkt.};
\node () at (0,0.0) {diffbare Fkt.};
\draw[->,red] (-1.5,1.55)--(-1,1);
\draw[->,red] (-1.8,1.55)--(0,-4.5);
\end{tikzpicture}
\caption{Stammfunktionbildung}
\end{figure}
\begin{proof}
$f: \mathcal{J} \to \R$ lokal integrierbar\\
$x_0 \in \mathcal{J}$. Zeige: $F$ ist stetig in $x_0$ (Stammfunktion von $f$).\\
Betrachte $f$ auf $[x_0-1,x_0+1]\cap \mathcal{J}$\\
Sei $\abs{f(x)} \le M$ für alle $ \in \mathcal{J}_0. (\ref{sec:7.3}a)\\
\abs{F(x) - F(x_0)} = \abs{\int\limits_{x_0}^{x} f(t)dt} \underset{\text{\ref{sec:7.9}c)}}{\ge} \int\limits_{x_0}^{x} \abs{f(t)dt} \underset{\text{\ref{sec:7.9}d)}}{\ge} M \cd \abs{x-x_0}$ für alle $x\in \mathcal{J}_0.\\ F$ stetig in $x_0$ nach \ref{sec:5.2}
\end{proof}
\subsection{Definiton}
$f: \mathcal{J} \to \R$ hei\ss t \emph{stetig differenzierbar}, falls $f$ differenzierbar ist und die Ableitung $f'$ stetig ist.\\
$[$Beachte : Nicht jede differenzierbare Funktion ist stetig diffbar\\
Bsp: $f(x) = \begin{cases}
x^2 \cd \sin(\frac{1}{x}) & \text{ für } x \ne 0\\
0 & \text{ für } x = 0
\end{cases}\\
f'$ nicht stetig in 0
\subsection{Satz (Hauptsatz der Differential- und Integralrechnung)}\label{sec:8.8}
$\mathcal{J}$ beliebiges Intervall, $f : \mathcal{J}\to \R.$
\begin{enumerate}[a)]
\item Ist $f$ stetig, so ist jede Stammfunktion $F$ von $f$ differejzierbar auf $\mathcal{J}$ und es gilt $F' = f$.
\begin{equation}
\Bigl(\int f(t)dt\Bigr)' = f \tag{a}
\end{equation}
Dass hei\ss t $F(x) = \int\limits_{x_0}^{x} f(t)dt \Rightarrow F'(x) = f(x)$
\item Ist $f$ stetig diffbar auf $\mathcal{J}$, so ist $f$ Stammfunktion von $f'$, dass hei\ss t.
\begin{equation}
\int_{x_0}^{x} f'(dt) = f + c \tag{b}
\end{equation}
$\forall u,v \in \mathcal{J}:$
\[ \int_{u}^{v} f'(t)dt = f(v) - f(u) =f(x) \Big\vert_u^v \]
\end{enumerate}$
\begin{array}{ccc}
\textcircled{f}&\xrightarrow{\hspace*{3.cm}}& F = \int f(t)dt\\
F' = f &\xleftarrow{\hspace*{3.cm}} \\
&\\
f'&\xleftarrow{\hspace*{3.cm}} &\textcircled{f} \text{stetig diffbar}\\
& \xrightarrow{\hspace*{3.cm}}& F = \int f'(t)dt = f + c
\end{array}$
\begin{proof}
a) Sei $c \in \mathcal{J}$.\\
Zu zeigen: $\lim\limits_{x \to c} \frac{F(x)-F(c)}{x-c} =f(c), x \ne c,\,x\in\mathcal{J}:\\
\frac{F(x)-F(c)}{x-c} \underset{\text{F. St.fkt. von f}}{=} \frac{1}{x-c} \int^x_cf(t)dt$\\
Mittelwertsatz der Integralrechnung (\ref{sec:7.11}). Es existiert $\Theta(x)$ zwischen $x$ und $c$ mit 
\[ \int_{x}^{c} f(t)dt = f(\Theta(x)) \cd (x-c) \]\\
$\begin{array}{ll}
\frac{F(x)-F(c)}{x-c} &= \frac{1}{x-c} \cd f(\Theta(x)) \cd (x-c)\\
&= f(\Theta(x)) \xrightarrow[\substack{x \to c,\,\text{so}\\\Theta(x) \to c}]{} f(c),\, \text{da $f$ stetig}
\end{array}$
b) $f'$ ist stetig.\\
Sei $F$ eine Stammfunktion von $f'$. Nach a): $F'=f'$.\\
$(F-f)' = 0.$\\
\ref{sec:6.20}a) $F -f =c $ konstant, dass hei\ss t $F =f+c$ 
\end{proof}
\subsection{Beispiele}
Zu. \ref{sec:8.8}a):
\begin{enumerate}[a)]
\item $g(x) = \int_{1}^{x} \underbrace{e^{t^2} \cd (\sin(t) + \cos(\frac{t}{2})}_{\text{stetig}} dt$
\ref{sec:8.8}a) : $g'(x) = e^{x^2} \cd (\sin(x) + \cos(\frac{x}{2})$
\item $g(x) = \int_{0}^{x^2} e^t \cd \sin(t)dt\\
F(x) = \int\limits_{0}^{x} e^t \cd \sin(t)dt\\
h(x) = x^2\\
g = F(h(x)) = (F \circ h)(x)\\
g'(x) = F'(h(x)) \cd h'(x) = e^{x^2} \cd \sin(x^2) \cd 2x$\\
\end{enumerate}
\subsection{Beispiel}\label{sec:8.10}
zu \ref{sec:8.8}b)
\begin{enumerate}[a)]
\item $n \in \mathbb{N}_0$:\\
$ \int a x^n dx = a \cd \frac{x^{n+1}}{n+1} + c \text{ denn } \Bigl(a\frac{x^{n+1}}{n+1}\Bigr)' = ax^n$\\
d.h : $\int\limits_{u}^{v} a \cd x^n dx = \frac{a}{n+1} (v^{n+1} - u^{n+1})$\\
Damit : $\int \sum\limits_{i=0}^{n} a_i \cd x^i = \sum\limits_{i=0}^{n} a_i \frac{x{i+1}}{(i+1)}$
\item $n \ge -2$,so \[ \int \frac{1}{x^n} dx = \frac{1}{n+1} \cd \frac{1}{x^{1-n}} +c \]
\item Für $ x > 0$ ist $\ln(x)' \frac{1}{x}$ \hfill (\ref{sec:6.14}b)\\
Also $\int \frac{1}{x} dx = \ln(x) + c$ auf $]0,\infty[$\\
Auf $]-\infty,0[$ gilt $\int \frac{1}{x}dx = \ln(\abs{x}) +c$
\item $\int \ln(x)dx = x \cd \ln(x) - x + c$ auf $]0,\infty[$ \hfill (\ref{sec:6.14}d)
\item $\int e^x dx = e^x+c$
\item $\int \sin(x)dx = -\cos(x) +c$
$\int \cos(x) dx = \sin(x) +c$
\end{enumerate}
\subsection{Satz (Partielle Integration)}\label{sec:8.11}
Seien $f$ und $g$ stetig diffbare Funktionen auf Intervall $\mathcal{J}$. Dann:\[ fg'dx = f \cd g - \int f'\cd g dx \]
Für bestimmte Integrale hei\ss t das:\\
\[ \int\limits_{u}^{v} f(x)\cd g'(x)dx = \underbrace{f \cd g}_{f(v)\cd g(v) - f(u)\cd g(u)}\Bigg\vert - \int\limits_{y}^{u}f'(x)g(x)dx \]
Für alle $u,v \in \mathcal{J}$\\
\begin{proof}
\ref{sec:8.8}b)\\
$(f\cd g)' = f \cd g' + f' \cd g\\
\bigintsss (f \cd g' + f' \cd g)dx = f\cd g +c\\
\bigintsss f \cd g' + \int f'g \cd g = f \cd g$
\end{proof}
\subsection{Beispiele}
\begin{enumerate}[a)]
\item $\bigintssss \underbrace{x}_{f} \cd \underbrace{\cos(x)}_{g'}dx \underset{\text{\ref{sec:8.11}}}{=} x \cd \sin(x) - \bigintssss \sin(x)dx = x \cd \sin(x) + \cos(x) +c$
\begin{figure}[h!]
\centering
\begin{tikzpicture}
\begin{axis}[axis x line=center,
axis y line=center, xmin = 0, ymin = -5,xmax =5,ymax =5,xtick={3.141},xticklabels={$\pi$},ytick ={-3.141},yticklabels={$-\pi$}]  
\path[name path=axis] (axis cs:0,0) -- (axis cs:4.9,0); \
\addplot[name path=f,samples=100,domain={0:5}] gnuplot[id=flache3]{ x * cos(x)};
\addplot [
thick,
color=blue,
fill=blue, 
fill opacity=0.05
]
fill between[
of=f and axis,
soft clip={domain=0:4.9},
];
\draw[red,dashed] (axis cs: 3.141,0)--(axis cs: 3.141,-3.141);
\end{axis}
\end{tikzpicture}
\caption[Integral Berechnung $x \cd \cos(x)$]{${\displaystyle \int_{0}^{\pi} x \cd \cos(x)dx = 2} $}
\end{figure}
\item $\bigintssss \ln(x) dx = \bigintsss \underset{\substack{\uparrow\\f'}}{1} \cd \underset{\substack{\uparrow\\g}}{\ln(x)}dx = x \cd \ln(x) - \bigintsss x \cd \frac{1}{x}dx = x \ln(x) - x +c$ \hfill (vgl. \ref{sec:8.10}d)
\item $\bigintsss \cos^2(x)dx \underset{\text{\ref{sec:8.11}}}{=} \underset{\substack{\uparrow\\f'}}{\cos(x)} \cd \underset{\substack{\uparrow\\g}}{\sin(x)} + \bigintsss \sin(x)dx$ \hfill (${\Huge \star}$)\\
$\begin{array}{ll}
\bigintsss \cos^2(x)dx &= \cos(x)\cd\sin(x) + \bigintsss 1-\cos^2(x)dx\\
&= \cos(x) \cd \sin(x) + x - \bigintsss \cos^x(x)dx\\
&\Rightarrow 2 \cd \bigintsss \cos^2(x)dx = \cos^2(x)dx = \cos(x) \cd \sin(x) + x + c
\end{array}$
\end{enumerate}
\subsection[\tiny Satz (Integration durch Substitution)]{Satz (Integration durch Substitution)}\label{sec:8.13}
$\mathcal{I},\mathcal{J}$ Intervalle $f:\mathcal{I} \to \mathcal{J}$ stetig diffbar, $g: \mathcal{J} \to \R$ stetig mit Stammfunktion $G$. Dann ist:
\[ \int g(f(x)) \cd f'(x)dx = G(f(x))+c \]
Für das bestimmte Integral hei\ss t das:\\
\[ \int_{u}^{v} g(f(x)) \cd f'(x) dx = G(f(v)) - G(f(u)) = \int_{f(u)}^{f(v)} g(t)dt \]
für alle $u,v \in \mathcal{I}$
\begin{proof}
$G \circ f$ diffbar: \ref{sec:8.8}a)\\
Kettenregel:\\
$\begin{array}{ll}
(G \circ f)'(x) &= G'(f(x)) - f'(x)\\
&\underset{\text{\ref{sec:8.8}a}}{=} \underbrace{g(f(x)) \cd f'(x)}_{\text{stetig}}
\end{array}\\
$Hauptsatz $G \circ f$ ist Stammfunktion von $g(f(x)) \cd f'(x)$
\end{proof}
\emph{Bemerkung}:
\ref{sec:8.13} kann in 2 Arten angewandt werden:\\
\emph{1.Art}: Mann hat ein Integral der Form $$\int g(f(x)) \cd f'(x)dx$$\\
Berechne $$\int g(t)dt = \int_{x_0}^{x}g(t)dt = \underset{\substack{\uparrow\\\text{und ersetze}\\\text{x durch f(x)}}}{G(x)}$$
\emph{2.Art}:\\
Man will $\int g(t)dt$ berechnen\\
Ersetze $t$ durch $f(x)$ (Substitution)\\
$\bigl\lbrack \frac{dt}{dx} = f'(x) \to dt = f'(x)dx \bigr\rbrack$\\
und $dt$ durch $f'(x)dx$ ersetzen.\\
$\to \int g(f(x)) \cd f'(x) dy$\\
Hoffnung: $\uparrow$ ist einfacher zu berechnen.
\subsection[Satz]{Satz}\label{sec:8.14}
$f$ ist stetig diffbar auf $\mathcal{I}$, f stetig auf $\mathcal{I}$.\\
\begin{enumerate}[a)]
\item Ist $f(x) \ne 0$ auf $\mathcal{I}$,\\
so $\int \frac{f'(x)}{f(x)}dx = \ln(\abs{f(x)}) + c$\\
dass hei\ss t $\int_{a}^{b} \frac{f'(x)}{f(x)} = \ln(\abs{f(b)}) - \ln(\abs{f(a)})$\\
für alle $a,b \in \mathcal{I}$
\item ${\displaystyle \int_{a}^{b} g(x + c)dx = \int_{a+c}^{b+c} g(x)dx}$\\
für alle c mit $a+c, b+c \in \mathcal{I}$.\\
\item ${\displaystyle \int_{a}^{b} g(c \cd x)dx = \frac{1}{c}\int_{a \cd c}^{b \cd c}g(x)dx}$ für alle $c \ne 0 $ mit $a\cd c, b\cd c\in \mathcal{I}$\\
\end{enumerate}
\begin{proof}
a) Setze $g(x)= \frac{1}{x}$.\\
Also : $G(x) = \ln(\abs{x}) + c$ für alle x $\ne 0$.\\
\ref{sec:8.13} ${\displaystyle \int \frac{f'(x)}{f(x)}dx} = \ln(\abs{f(x)}) + c$
b) Setze $f(x) = x + c$, \ref{sec:8.13}\\
c) Setze $f(x) = x \cd c$, \ref{sec:8.13}\footnote{Beachte $f'(x)=c$}\\
\end{proof}
\subsection{Beispiel}
\begin{enumerate}[a)]
\item ${\displaystyle \int \tan(x)dx = \int \frac{\sin(x)}{\cos(x)}} = - \int \frac{-(\cos(x))'}{\cos(x)}dx \underset{\text{\ref{sec:8.14}a)}}{=} - \ln(\abs{cos(x)}) +c$\footnote{Gilt auf jedem Intervall $]k\pi + \frac{\pi}{2}, (k+1)\pi+\frac{\pi}{2}[$}
\item ${\displaystyle x \cd \sin(x^2)dx = \frac{1}{2} \int 2x \sin(x^2)dx \underset{\text{\ref{sec:8.13}}}{=} - \frac{1}{2}\cos(x^2) +c}$
\item ${\displaystyle \int \frac{x}{x^2+a^2}dx = \frac{1}{2} \int \frac{2x}{x^2 + a^2}dx \underset{\text{\ref{sec:8.14}a)}}{=} \ln(x^2+a^2) + c}$auf $\R$ $(a \ne 0)$
\item ${\displaystyle \int_{-1}^{1} \sqrt{1-t^2}dt}$\\
Fläche des Halbkreises vom Radius 1.\\
Substitution $t = \sin(x)$\\
$\frac{dt}{dx} = \cos(x), dt = \cos(x)dx\\
{\displaystyle \int_{-1=\sin(-\frac{\pi}{2})}^{1= \sin(\frac{\pi}{2})}\sqrt{1-t^2} \overset{\text{\ref{sec:8.13}}}{=}\int_{-\frac{\pi}{2}}^{\frac{\pi}{2}}
\sqrt{1-\sin(x)^2} \cd \cos(x)dx = \int\cos^2(x)}dx \overset{\text{\ref{sec:8.11}c}}{=} \frac{\cos(x) \cd \sin(x)}{2} = \frac{\frac{\pi}{2}}{2} - (\frac{\frac{\pi}{2}}{2}) - (\frac{\pi}{4} + \frac{\pi}{4} = \frac{\pi}{2})$ 
\end{enumerate}
\begin{figure}[h!]
\centering
\begin{tikzpicture}
\node () at (0,0) {\Huge Lineare Algebra};
\draw[->] (-2,-2)--(-1,-0.4);
\node () at (-2,-2.4) {lineare Gleichungssysteme};
\draw[->] (2,-2)--(0,-0.4);
\node () at (2,-2.4) {Geometrie};
\end{tikzpicture}
\end{figure}
\section{Matrizen und lineare Gleichungssysteme}
\subsection{Definition}
$K = \mathbb{Q},\,\R,\,\C$\\
\begin{enumerate}[a)]
\item Eine \emph{$m \times n$ Matrix} A über k ist rechteckiges Schema .\\
Spalten$\Bigg\downarrow A = \begin{Bmatrix}
a_{11} & a_{12} & \cdots & a_{1n}\\
\vdots & \vdots & \vdots & \vdots\\
a_{m1} & a_{m2} & \cdots & a_{mn}\\
\end{Bmatrix}\xrightarrow{\text{Zeilen}}$\\
mit m \emph{Zeilen} und n \emph{Spalten}, $a_{ij} \in K$.\\
(Bezeichnung der Indizes ist Standard! 1.Index : Zeile, 2.Index : Spalte)\\ (m,n) hei\ss t \emph{Typ} der Matrix.\\
Schreibweise:\\
$A = (a_{ij})_{\substack{i=1\ldots m\\j=1\ldots n}}$, $A = (a_{ij})$
\item $\mathcal{M}_{m,n}(K)=$Menge aller $m \times n$-Matrizen über K (\emph{quadratische Matrizen})
\item $A = (a_{ij}) \in \mathcal{M}_{n,m}(K).$.\\
Definiere $A^t = (b_{ij}) \in M_{n,m}(k)$ mit $b_{ij} = a_{ij}$ für $i=1\ldots m,\, j=1\ldots n\\
A^t$ ist die zu A \emph{transponierte} Matrix.\\
$\begin{pmatrix}
a_{11}&a_{12}&\ldots&a_{1n}\\
a_{21}&a_{22}&\ldots&a_{2n}\\
a_{31}&a_{32}&\ldots&a_{3n}\\
\vdots&\ddots&\ddots\ldots\\
a_{m1}&a_{m2}&\ldots&a_{mn}
\end{pmatrix} \to \begin{pmatrix}
a_{11}&a_{21}&\ldots&a_{m1}\\
a_{12}&a_{22}&\ldots&a_{m2}\\
a_{13}&a_{23}&\ldots&a_{m3}\\
\vdots&\ddots&\ddots&\ldots\\
a_{1n}&a_{2n}&\ldots&a_{mn}
\end{pmatrix}$\\
(Manchmal $A^T$ statt $A^t$ oder andere Schreibweise)\\
$1 \times n -$Matrix ($a_11,\ldots 1_1n)$ \emph{Zeilenvektor}\\
$m \times 1-$-Matrix \emph{Spaltenvektor}$\begin{pmatrix}
a_{11}\\
a_{21}\\
\vdots\\
a_{m1}
\end{pmatrix}$\\
Alle $a_{ij} = 0: A = \begin{Bsmallmatrix}
0 &\cdots& 0\\
\vdots & \vdots &\vdots\\
0 & \cdots & 0
\end{Bsmallmatrix}=: 0$\emph{Nullmatrix} (vom Typ (m,n))\\
$n \times n$-Matrix $E_n = \begin{Bsmallmatrix}
1 &\cdots& 0\\
\vdots & 1 &\vdots\\
0 & \cdots & 1
\end{Bsmallmatrix} n \times n - $Einheitsmatrix\\
$E_n = (\delta_{ij})_{i,j = 1\ldots n}\\
\delta_{ij} = \begin{cases}
1 &,\text{ falls } i= j,\\
0 &,\text{ falls } i \ne j
\end{cases}$
\emph{Kronecker Symbol}
\end{enumerate}
\subsection{Definition}
$A = (a_{ij}), B (b_{ij}) \in \mathcal{M}_{m,n}(K)$ (beide vom gleichen Typ!) $a\in K$.
\begin{enumerate}[a)]
\item $A + B := (a_{ij} + b_{ij})_{\substack{i= 1\ldots m\\j=1\ldots n}}$\\
\emph{Summe von Matrizen}
\item $a\cd A = (a\cd a_{ij})_{\substack{i=1\ldots m\\j=1\ldots n}}$\\
\emph{(skalares) Vielfaches von $A$}.\\
Für Matrizen unterschiedlichen Typs ist keine Summe definiert.
\end{enumerate}
\emph{Beispiel}: $A = \begin{pmatrix}
1 & 2 & 3\\
4 & 5 & 6
\end{pmatrix} B = \begin{pmatrix}
-2 & 5 & 3\\
\frac{1}{2} & 0 & 0
\end{pmatrix}\\
A + B = \begin{pmatrix}
-1 & 7 & 6\\
\frac{9}{2} & 5 & 6
\end{pmatrix}\\
A+ B^t$ nicht definiert\\
$B^t = \begin{pmatrix}
-2 & \frac{1}{2}\\
5 & 0\\
3 & 0 
\end{pmatrix}\\
A + A = 2A = \begin{pmatrix}
2 & 4 & 6\\
8 & 10 & 12
\end{pmatrix}$
Produkt:\\
\begin{enumerate}[-]
\item Produkt von Matrizen gleichen Typs durch komponentenweise Multiplikation (kaum Anwendungen)
\item wichtig ist Produkt von $m \times n$-Matrizen mit $n \times p$ Matrizen:
\end{enumerate}
\subsection{Definition}
$m,n,p \in \mathbb{N}.\\
A = (a_{ij}) \in \mathcal{M}_{m,n}(K)\\
B = (b_{ij}) \in \mathcal{M}_{n,p}(K)\\$\\
Das \emph{Produkt} $A \cd B$ von $A$ und $B$ = $(d_{ik})_{\substack{i=\ldots m\\k=1\ldots p}}$\\
mit $d_{ik} = a_{i1} \cd b_{1k} +  a_{i2} \cd b_{2k} + \ldots +  a_in \cd b_{nk} = \sum\limits_{j=1}^{n} a_{ij} \cd b_{jk}$\\
($m \times n$ multiplizieren mit $n \times p\to m \times p$)
\\
\newcommand{\myunit}{1 cm}
\tikzset{
node style sp/.style={draw,circle,minimum size=\myunit},
node style ge/.style={circle,minimum size=\myunit},
arrow style mul/.style={draw,sloped,midway,fill=white},
arrow style plus/.style={midway,sloped,fill=white},
}
\begin{figure}[h!]
\centering
\caption{Matrix Multiplikation}
\begin{tikzpicture}[>=latex]
% les matrices
\matrix (A) [matrix of math nodes,%
         nodes = {node style ge},%
         left delimiter  = (,%
         right delimiter = )] at (0,0)
{%
a_{11} & a_{12} & \ldots & a_{1p}  \\
|[node style sp]| a_{21}%
     & |[node style sp]| a_{22}%
              & \ldots%
                       & |[node style sp]| {a_{2p}}; \\
\vdots & \vdots & \ddots & \vdots  \\
a_{n1} & a_{n2} & \ldots & a_{np}  \\
};
\node [draw,below=10pt] at (A.south) 
{ $A$ : \textcolor{red}{$n$ Reihen} $p$ Spalten};

\matrix (B) [matrix of math nodes,%
         nodes = {node style ge},%
         left delimiter  = (,%
         right delimiter =)] at (6*\myunit,6*\myunit)
{%
b_{11} & |[node style sp]| b_{12}%
              & \ldots & b_{1q}  \\
b_{21} & |[node style sp]| b_{22}%
              & \ldots & b_{2q}  \\
\vdots & \vdots & \ddots & \vdots  \\
b_{p1} & |[node style sp]| b_{p2}%
              & \ldots & b_{pq}  \\
};
\node [draw,above=10pt] at (B.north) 
{ $B$ : $p$ Reihen \textcolor{red}{$q$ Spalten}};
% matrice résultat
\matrix (C) [matrix of math nodes,%
         nodes = {node style ge},%
         left delimiter  = (,%
         right delimiter = )] at (6*\myunit,0)
{%
c_{11} & c_{12} & \ldots & c_{1q} \\
c_{21} & |[node style sp,red]| c_{22}%
              & \ldots & c_{2q} \\
\vdots & \vdots & \ddots & \vdots \\
c_{n1} & c_{n2} & \ldots & c_{nq} \\
};
% les fleches
\draw[blue] (A-2-1.north) -- (C-2-2.north);
\draw[blue] (A-2-1.south) -- (C-2-2.south);
\draw[blue] (B-1-2.west)  -- (C-2-2.west);
\draw[blue] (B-1-2.east)  -- (C-2-2.east);
\draw[<->,red](A-2-1) to[in=180,out=90]
node[arrow style mul] (x) {$a_{21}\times b_{12}$} (B-1-2);
\draw[<->,red](A-2-2) to[in=180,out=90]
node[arrow style mul] (y) {$a_{22}\times b_{22}$} (B-2-2);
\draw[<->,red](A-2-4) to[in=180,out=90]
node[arrow style mul] (z) {$a_{2p}\times b_{p2}$} (B-4-2);
\draw[red,->] (x) to node[arrow style plus] {$+$} (y)%
              to node[arrow style plus] {$+\raisebox{.5ex}{\ldots}+$} (z)%
              to (C-2-2.north west);


\node [draw,below=10pt] at (C.south) 
{$ C=A\times B$ : \textcolor{red}{$n$ Zeilen}  \textcolor{red}{$q$ Spalten}};

\end{tikzpicture}
\end{figure}
Beachte : Produkt von $m\times n$- mit $r \times p$-Matrix ist nicht definiert falls $n \ne r$\\
\subsection{Beispiel}
\begin{enumerate}[a)]
\item$A = \begin{pmatrix}
1& 2& 3\\
-1& 0& 2
\end{pmatrix}\in \mathcal{M}_{3}(\R),\,B = \begin{pmatrix}
2& 3& 0\\
0& 0& 0\\
1 & 1& 0
\end{pmatrix}\in \mathcal{M}_{2,3}(\R)\\
A \cd B = \begin{pmatrix}
5& 6& 0\\
0& -1& 0\\
\end{pmatrix}\in \mathcal{M}_{2,3}(\R)$
\item $A = \begin{pmatrix}
1& 2\\
3& 4
\end{pmatrix}\in \mathcal{M}_{2}(\R),\,B = \begin{pmatrix}
2& 3\\
4& 5
\end{pmatrix}\in \mathcal{M}_{2}(\R)\\
A \cd B = \begin{pmatrix}
10& 13\\
22& 29
\end{pmatrix}, B \cd A = \begin{pmatrix}
11& 16\\
19& 28
\end{pmatrix}, AB \ne BA!,\\
A^2 = A \cd A = \begin{pmatrix}
7& 10\\
15& 22
\end{pmatrix}$
\item $A = \begin{pmatrix}
1& 2& 3
\end{pmatrix}\in \mathcal{M}{1,3}(\R),\, B\in \mathcal{M}_{1,2}(\R)$ wie in a)\\
$A \cd B = \begin{pmatrix}
5& 6& 0
\end{pmatrix}\\
B \cd A^t = B \cd \begin{pmatrix}
1\\2\\3
\end{pmatrix} = \begin{pmatrix}
8\\0\\3
\end{pmatrix}$
\item $A = \begin{pmatrix}
1& 2& 3
\end{pmatrix},\,B = \begin{pmatrix}
2\\0\\1
\end{pmatrix}\in \mathcal{M_{1,3}}(\R)\\
A \cd B = (5) \in \mathcal{M}_1(\R) (= \R)\\
B \cd A = \begin{pmatrix}
2 & 4 & 5\\ 0 & 0 & 0\\ 1 & 2 & 3
\end{pmatrix}$
\item $A \in \mathcal{M}_{m,n}(K)\\
E_m \cd A = A\\
A \cd E_n = A.
\begin{pmatrix}
1 & 0\\
0 & 1
\end{pmatrix} \cd \begin{pmatrix}
2 & 3 & 1\\
-1 & 2 & 5
\end{pmatrix} = \begin{pmatrix}
2 & 3 & 1\\
-1 & 2 & 5
\end{pmatrix}\\
A \in \mathcal{M}_n(K)\\
E_n \cd A = A \cd E_n = A.\\
a \cd A = \begin{pmatrix}
a & \cdots & 0 \\
\vdots & a & \vdots\\
0 &\cdots & a
\end{pmatrix} \cd A$
\end{enumerate}
\subsection{Satz (Rechenregeln von Matrizen)}
$A,A_1,A_2 \in \mathcal{M}_{m,n}(K),\\
B,B_1,B_2 \in \mathcal{M}_{n,r}(K),\\
C \in \mathcal{M}_{r,s}(K), a \in K.$
\begin{align}
(A_1 + A_2) \cd B &= A_1 \cd B + A_2 \cd B \tag{a}\\
A \cd (B_1 + B_2) &= A \cd B_1 + A \cd B_2 \tag{b}\\
(a \cd A) \cd B &= A \cd (aB) = a(A \cd B) \tag{c}\\
\underbracket{\underbrace{(A \cd B)}_{m \times r} \cd \underset{\substack{\downarrow\\r\times s}}{C}}_{m \times s} &= \underbracket{\underset{\substack{\downarrow\\m \times n}}{A} \cd \underbrace{(B \cd C)}_{n \times s}}_{m \times s} \tag{d}\\
(A \cd B)^t &= B^t \cd A^t \tag{e}
\end{align}
\begin{proof}
Nur d)
$A = (a_ij)_{\substack{i=1\ldots m\\j = 1\ldots n}}\\
B = (b_ij)_{\substack{i=1\ldots n\\j = 1\ldots r}}\\
C = (c_ij)_{\substack{k=1\ldots r\\j = 1\ldots s}}\\
A \cd B = (d_{ik})_{\substack{i=1\ldots m\\k = 1\ldots r}}\\
d_{ik} = \sum\limits_{j=1}^{n} a_{ij}b_{jk}\\
B \cd C = (e_{jl})_{\substack{j=1\ldots n\\l = 1\ldots s}}\\
e_{jl} = \sum\limits_{k=1}^{r} b_{ij}c_{jk}\\
(A\cd B)\cd C\\$
Eintrag an der Stelle $(i,l):\\
\sum\limits_{k=1}^{r} d_{ik} \cd c_{kl} = \sum\limits_{k=1}^{r} (\sum\limits_{j=1}^{n} a_{ij} \cd b_{jk}) \cd c_{kl} = \sum\limits_{k=1}^{r} \sum\limits_{j=1}^{n} a_{ij} \cd b{jk} \cd c_{kl}\\
A \cd (B \cd C)$ Eintrag an Stelle $(i,l)$:\\
$\sum\limits_{j=1}^{n} a_{ij} \cd e_{jl} = \sum\limits_{j=1}^{n} a_{ij} \cd (\sum\limits_{k=1}^{r} b_{ij} \cd c_{kl}) =  \sum\limits_{j=1}^{n} (\sum\limits_{k=1}^{r} a_{ij} b_{ij} \cd c_{kl}) = \sum\limits_{k=1}^{r} \sum\limits_{j=1}^{n} a_{ij} b_{ij} \cd c_{kl}$
\end{proof}
\subsection{Definition}\label{sec:9.6}
Allgemeine Form eines \emph{lineares Gleichungssystem} (LGS) über K:\\
$\begin{matrix}
a_{11}x_1 +& \ldots &+ a_{1n}x_n& = b_1\\
\marginnote{(*)}\vdots&\vdots & \vdots &\vdots\\
a_{m1} +& \ldots &+  a_{mn}x_n &= b_m
\end{matrix}$\\
m Gleichungen, n unbekannte $x_1,\ldots,x_n$\\
$(n = 2\|3 : x,y\|z)$\\
Koeffizienten $a_{ij} \in K,$ rechte Seite $b_1 \ldots n_m \in K$ (fest).\\
$x = \begin{pmatrix}
x_1\\
\vdots\\
x_n
\end{pmatrix}
\in M_n1(K) = K^n$\\
(Elemente der $K^n = K \times K \ldots \times K$ werden als Spalten geschrieben) hei\ss t Lösung von $(*)$ wenn $x_1\ldots x_n$ sämtliche Gleichungen erfüllen.\\
Ist $b_1 = \ldots b_m = 0$, so hei\ss t $(*)$ \emph{homogenes} LGS, sonst \emph{inhomogenes} LGS.\\
$A = \begin{pmatrix}
a_11 &\ldots& a_{1n}\\
\vdots&\vdots&\vdots\\
a_{m1}&\ldots& a_{mn}
\end{pmatrix} \in \mathcal{M}_{m,n}(K)$\\
{\em Koeffizientenmatrix} des LGS $b=\begin{pmatrix}
b_1\\
\vdots\\
b_m
\end{pmatrix} \in \mathcal{M}_{m,1} = K^m$ \hfill (rechte Seite)\\
$(*)$, lässt sich schreiben in \emph{Matrizenform}:
$A \cd x = b:\\
\begin{pmatrix}
a_{11}x_1 + \ldots + a_{1n}x_n\\
\vdots\\
a_{m1}x_n + \ldots + a_{mn}x_n
\end{pmatrix} = \begin{pmatrix}
b_1\\
\vdots\\
b_m
\end{pmatrix}$\\
Sins $s_1,\ldots,s_n$ die Spaltenvektoren von A,
d.h. $s_i = \begin{pmatrix}
a_{1i}\\
a_{2i}\\
\vdots\\
a_{mi}
\end{pmatrix}$, so lässt sich $(*)$ schreiben als $x_1 \cd s_1 + \ldots x_n \cd s_n =b\\
\begin{pmatrix}
x_1a_{11}\\
\vdots\\
x_1a_{m1}
\end{pmatrix} + \ldots + \begin{pmatrix}
x_1a_{mn}\\
\vdots\\
x_na_{mn}
\end{pmatrix} = \begin{pmatrix}
b_1\\
\vdots\\
b_n
\end{pmatrix}$\\
\emph{Spaltenform} des LGS
Beachte: Homogenes LGS hat immer \emph{mindestens} eine Lösung $x =\begin{pmatrix}
0\\
\vdots\\
0
\end{pmatrix}$ \emph{Null-Lösung}\marginpar{(triviale Lösung)}\\
\emph{Fragen}:
\begin{enumerate}[(1)]
\item Wann hat LGS mindestens eine Lösung?
\item Wenn es Lösungen gibt, wie bestimmt man alle?
\end{enumerate}
Antwort: Gau\ss\ Algorithmus (C.F.Gau\ss\ 1777-1855)\bigskip\\
Lösungsmenge eines LGS ändert sich nicht bei :
\begin{enumerate}[-]
\item Addition der Vielfachen einer Gleichung zu einer anderen Gleichung
\item Multiplikation einer Gleichung mit Zahl $\ne$ 0.
\item Vertauschen zweier Gleichungen
\end{enumerate}
Was passiert dann:\\
Aus $Ax = b$ wird\par$A'x = b'$\\
Die Menge der $x$, die $ax = b$ erfüllen, stimmt mit der überein, die $A'x=b'$ erfüllen.\\
Ziel des Gau\ss\ Algorithmus: Mit obigen Umformungen einfache Form A' zu finden.
\emph{Beispiel:}\label{bsp1}\\
$\begin{matrix}
2x &+& 3y &=& 4\\
x &+& y &=& -2
\end{matrix}\qquad \rightarrow\qquad \begin{matrix}
x &+& y &=& -2\\
2x &+& 3y &=& 4\\
\end{matrix}$ $(-2)$fache 1.Gl. zu 2.Gl. addieren\bigskip\\
$\begin{matrix}
x &+& y &=& -2\\
& & y &=& 8
\end{matrix}\qquad y = 8,\qquad x+8 = -2\qquad x=-10$, eindeutige Lösung
\subsection{Definition}\label{sec:9.7}
Unter \emph{elementaren Zeilenumformungen} an einer Matrix versteht man folgende Operationen:
\begin{enumerate}[-]
\item Addition des skalaren vielfachem einer Zeile zu anderen
\item Multiplikation einer Zeile mit Zahl $\ne 0$
\item Vertauschen von zwei Zeilen
\end{enumerate}
Analog: \emph{Elementare Spaltenumformungen}\\ (wird nicht für :GS benötigt --- au\ss er ggf. Spaltenvertauschung)
\subsection{Definition}\label{sec:9.8}
Ist $Ax =b$ ein LGS, so nennt man $(A,b) \in 
\mathcal{M}_{m,n+1}(K)$ die \emph{erweiterte Koeffizientenmatrix}. b als letzte Spalte an $A$ anhängen.
\subsection{Bemerkung}\label{sec:9.9}
Führt man an $(A,b)$ elementare Zeilenumformungen durch und erhält man dabei Matrix $(A',b')$, so ist $x \in k^n$ ist Lösung von $A \cd x = b$ genau dann wenn x Lösung von $A'x=b'$.\\
{\em Beispiel \ref{bsp1}}:\\
$(A,b) = \begin{pmatrix}
2& 3& 4\\
1& 1&-2
\end{pmatrix}\longrightsquigarrow (A'b') = \begin{pmatrix}
1& 1&-2\\
0& 1& 8
\end{pmatrix}\\$
Kernstück des Gau\ss-Algorithmus: \\
Jede Matrix $B$ lässt sich durch elementare Zeilenumformungen auf \emph{Zeilenstufenform (Treppenform) bringen} \qquad
$\left(\begin{tikzpicture}
\node () at (0.2,-0.2) {0};
\node () at (0.7,-0.2) {0};
\node () at (1.1,-0.5) {0};
\node () at (1.35,-0.7) {0};
\node () at (1.55,-0.7) {0};
\node () at (1.85,-1.2) {0};
\node () at (1.15,0) {1};
\draw (0,0)--(1,0)--(1,-0.25)--(1.25,-0.25)--(1.25,-0.5)--(1.5,-0.5)--(1.75,-0.5)--(1.75,-1)--(2,-1);
\end{tikzpicture}\right)$\\
Beispiel : $\begin{pmatrix}
\underbar{1 2}& 3\  4 & 5\\
0\ 0 &\vert\underbar{1 0}& 0\\
0\ 0 & 0\ 0 &\vert\underbar{1}
\end{pmatrix}$
\subsection[Algorithmus zur Transformation einer Matrix auf Zeilenstufenform mit elementaren Zeilenumformungen]{Gau\ss\ Algorithmus}\label{sec:9.10}
Sei $B$ eine $(m,n)$-Matrix über $K$, Ist $B$ Nullmatrix, so fertig.\\
Sei also im Folgenden $B$ nicht Nullmatrix.
\begin{enumerate}[(1)]
\item Suche erste Spalte $j_1$, die nicht nur Nullen enthält.
\item Eventuell durch Zeilenvertauschung:\\
Eintrag $a$ am der Stelle $(i,j_1)$ ist $\ne 0$
\item Multipliziere die 1.Zeile mit $\frac{1}{a}$, Jetzt: Eintrag 1 an der Stelle $(i,j_1)$.
\item Durch Addition geeigneter Vielfacher der ersten Zeilen zu den übrigen Zeilen alle Einträge in der Spalte $j_1$ unterhalb der ersten Zeile gleich 0 sind.
Ab jetzt wird Zeile 1 nicht mehr benutzt. Sie bleibt unverändert.
\item Suche erste Spalte $j_2 (> j_1)$,der unterhalb der ersten Zeile einen Eintrag $\ne 0$ enthält.
\item Wie in (2) imd (3) Eintrag 1 an Stelle $(2,j_2)$
\item Wie in (4) alle Einträge in Spalte $j_2$ unterhalb der 2. Zeile zu Null machen.\\
Ab jetzt wird Zeile 2 nie mehr benutzt.
\end{enumerate}
So fortfahrend erhält man Zeilenstufenform.
\subsection{Beispiel}
$ B = \begin{pmatrix}
0&0&6&1\\
0&2&2&3\\
0&3&1&0\\
0&0&2&1/3
\end{pmatrix}\quad \xrightarrow{\text{Vert. 1Z./2.Z.}} \quad \begin{pmatrix}
0&2&-2&3\\
0&0&6&1\\
0&3&1&0\\
0&0&2&1/3
\end{pmatrix}\quad \xrightarrow{\text{1Z.}x\cd1/2} \quad \begin{pmatrix}
0&1&-1&3/2\\
0&0&6&1\\
0&3&1&0\\
0&0&2&1/3
\end{pmatrix}
\xrightarrow{\text{3Z.}+(-3)\cd\text{1.Z.}} \quad B=\begin{pmatrix}
0&1&-1&3/2\\
0&0&6&1\\
0&0&4&-9/2\\
0&0&2&1/3
\end{pmatrix}\quad \xrightarrow{1/6 \cd \text{2. Z.}} \quad \begin{pmatrix}
0&1&-1&3/2\\
0&0&1&1/6\\
0&0&4&-9/2\\
0&0&2&1/3
\end{pmatrix}\quad \xrightarrow{1/6 \cd \text{2. Z.}} \quad \begin{pmatrix}
0&1&-1&3/2\\
0&0&1&1/6\\
0&0&0&-1/6\\
0&0&0&0
\end{pmatrix}$
\subsection{Gau\ss-Algorithmus}\label{sec:9.12}
Gegeben sei LGS $\underset{m\times n}{A}\underset{n\times 1}{x} = \underset{m\times1}{b}$\\
Wende Algorithmus \ref{sec:9.10} auf (A,b) an, bis man Matrix $(\tilde{A},b')$ erhält, so dem $\tilde{A}$ Zeilenstufenform hat (letzte Spalte b' muss nicht mehr bearbeitet werden).\\
Man kann noch Spaltenvertauschung am $\tilde{A}$ vornehmen,\\ Beachte: Vertauschung von Spalte $i$ und $k$ bedeutet vertauschung von $x_i$ und $x_k$ (Buch führen).
Dann kann Matrix $(A',b')$ erhalten werden wobei\\ $(A',b') = 
\begin{pmatrix}
1 & a'_{12}&\ldots&\ldots&\ldots& a'_{1n}& b_1'\\
0 & 1 & a'_{23}&\ldots&\ldots& a'_{2n}& b_2'\\
0 & 0 & 1 & \ldots & \ldots & a'_{3n} & b_3'\\
\vdots & \vdots & \vdots & \vdots & \vdots & \vdots & \vdots\\
0 & 0 & 0 & 0 & 0 & 1 & b_m'
\end{pmatrix}$\\
Neues LGS $A'x'=b'\\
x = \begin{pmatrix}
x_1'\\
\vdots\\
x_n'
\end{pmatrix}$ entsteht aus $x$. durch Permutation der Einträge entsprechend der durchgeführten Spaltenvertauschungen.\\
Lösungsmenge des LGS $A'x'=b'$ ist leicht zu ermitteln:
\begin{enumerate}[(1)]
\item Ist $r < m$ und einer der Einträge $b'_r+a,b'_m$ ungleich 0 ist, so ist LGS nicht lösbar.
\item Ist $r=m \begin{pmatrix}
1 & * & * & * & b_1'\\
0 & 1 & * & * & \vdots\\
0 & 0 & 1 & * & \vdots\\
0 & 0 & 0 & 1 & b_m'
\end{pmatrix}$\\
oder $ r < m$ und $b'_{r+1} = \ldots = b'_m = 0\\
\begin{pmatrix}
1 & * & * & * & b_1'\\
0 & 1 & * & * & \vdots\\
0 & 0 & 1 & * & \vdots\\
0 & 0 & 0 & 0 & b_m'
\end{pmatrix}$\\
(betrachte dann nur die ersten $r$ Gleichungen),so gibt es mindestens eine Lösung:
\item[(2a)] $r<n$:\\
Wähle $x_{r+1}, \ldots,x_n'$ beliebig uas $K$. Dann:\\ $x'_r = b_r' - \sum\limits_{j=r+1}^{n}a'_{rj}x'_j\\
\vdots\\
x'_1 = b'_1 - \sum\limits_{j=2}^{n}a'_{1j}x'_j$\\
(rekursive Bestimmung der Lösungsmenge).\\
\item[(2b)] $r = n\\
\begin{pmatrix}
1&*&\cdots&\cdots&*&\vdots\\
0&1&\cdots&\cdots&*&\vdots\\
\vdots&\vdots&\vdots&\vdots&\vdots&\vdots\\
0&0&0&0&0&0
\end{pmatrix}$\\
Dann sind $x'_1,\ldots,x'_r$ eindeutig bestimmt.\\
$\begin{array}{ll}
x_n' &=b'_n\\
x'_{n-1}&= b'_{n-1}-a_{n-1}x'_n\\
\vdots &= \vdots\\
x'_1 &=b'_1 - \sum\limits_{j=2}^{n} a'_{ij}\cd x_j'
\end{array}
$\end{enumerate}
\subsection{Beispiel}
\begin{enumerate}[a)]
\item
$\begin{matrix}
x_1&+2x_2&-3x_3&=5\\
2x_1&-x_2&+4x_3&=0\\
x_1&+x_2&+2x_3&=1
\end{matrix}\\
\begin{pmatrix}
1&2&-3&&5\\
2&-1&4&&0\\
1&1&2&&1
\end{pmatrix}\quad \xrightarrow[\substack{2Z. + (-2)\cd 1.Z\\3Z.+(-1)\cd1.Z}]{}\quad\begin{pmatrix}
1&2&-3&&5\\
0&-5&10&&-10\\
0&-1&5&&-4
\end{pmatrix}\quad \xrightarrow[-\frac{1}{5}\cd2.Z]{}\quad\begin{pmatrix}
1&2&-3&&5\\
0&1&-2&&2\\
0&-1&5&&-4
\end{pmatrix}\\
\xrightarrow[3.Z+2.Z]{}\quad\begin{pmatrix}
1&2&-3&&5\\
0&1&-2&&2\\
0&0&3&&-2
\end{pmatrix}\quad \xrightarrow[\frac{1}{3}\cd 3.Z]{}\quad\begin{pmatrix}
1&2&-3&&5\\
0&1&-2&&2\\
0&0&1&&-\frac{2}{3}
\end{pmatrix}\\
x_3=-\frac{2}{3}\\
x_2=2+2x_3=2-\frac{4}{3}=\frac{2}{3}\\
x_1=5-\frac{4}{3}-2=\frac{5}{3}$
eindeutige Lösung: Lösungsmenge $\mathbb{L} = \left\{\begin{pmatrix}
-\frac{2}{3}\\
\frac{2}{3}\\
\frac{5}{3}
\end{pmatrix}\right\}$
\item $\begin{matrix}
x_1&+2x_2&x_3&+x_4&=0\\
x_1&-x_2&+2x_3&-x_4&=6\\
\end{matrix}\\
\begin{pmatrix}
1&+2&1&+1&&0\\
1&-1&+2&-1&&6\\
\end{pmatrix}\quad\longrightarrow\quad\begin{pmatrix}
1&+2&1&+1&&0\\
0&1&-\frac{1}{3}&+\frac{2}{3}&&-2\\
\end{pmatrix}$\\
(Fall 2a)\\
$x_4,x_3$ frei wählbar\\
$\begin{array}{ll}
x_2 &= -2 + \frac{1}{3}x_3 - \frac{2}{3}x_4\\
x_1 &= -2x_2 - x_3 -x_4\\
&=-2\cd (-2 + \frac{1}{3}x_3 - \frac{2}{3}x_4)-x_3-x_4\\
&=4-\frac{5}{3}+\frac{1}{3}x_4
\end{array}\\
\mathbb{L}= \left\{\begin{pmatrix}
4-\frac{5}{3}x_3+\frac{1}{3}x_4\\
-2+\frac{1}{3}x_3-\frac{2}{3}x_4\\
x_3\\
x_4
\end{pmatrix}: x_3,x_4\in K \right\}$
\item $\begin{matrix}
x_1&+x_2&=1\\
2x_1&+x_2&=2\\
x_1 &-x_2=-1
\end{matrix}\\
\begin{pmatrix}
1&+1&&1\\
2&1&&2\\
1 &-1&&-1
\end{pmatrix}\quad\longrightarrow\quad\begin{pmatrix}
1&+1&&1\\
0&-1&&0\\
0&-2&&-2
\end{pmatrix}\quad\longrightarrow\quad\begin{pmatrix}
1&+1&&1\\
0&-&&0\\
0&0&&-2
\end{pmatrix}$\\
LGS nicht lösbar (Fall 1)
\item $\begin{matrix}
x_1&+x_2&=1\\
2x_1&+x_2&=2\\
x_1 &-x_2=1
\end{matrix}\\
\begin{pmatrix}
1&+1&&1\\
2&1&&2\\
1 &-1&&1
\end{pmatrix}\quad\longrightarrow\quad\begin{pmatrix}
1&+1&&1\\
0&-1&&0\\
0&-2&&-0
\end{pmatrix}\quad\longrightarrow\quad\begin{pmatrix}
1&+1&&1\\
0&1&&0\\
0&0&&-0
\end{pmatrix}$\\
LGS eindeutig lösbar\\
$x_2=0\\
x_1=1-x_2=1\\
\mathbb{L}=\left\{ \begin{pmatrix}
1\\
0
\end{pmatrix} \right\}$ (Fall 2b)
\item $\begin{matrix}
x_1&-x_2&+x_3&=1\\
-2x_1&+2x_2&-2x_3&=3
\end{matrix}\\
\begin{pmatrix}
1&-1&1&=1\\
-2&2&-2&=3
\end{pmatrix}\quad\longrightarrow\quad\begin{pmatrix}
1&-1&1&=1\\
0&0&0&=5
\end{pmatrix}$\\
LGS ist nicht lösbar
\item$\begin{matrix}
x_1&+x_2&+x_3&=0\\
x_1&+x_2&-2x_3&=1
\end{matrix}\\
\begin{pmatrix}
1&1&1&&0\\
1&1&-2&&1
\end{pmatrix}\quad\longrightarrow\quad
\begin{pmatrix}
1&1&1&&0\\
0&0&-3&&1
\end{pmatrix}\quad\longrightarrow\quad
\begin{pmatrix}
1&1&1&&0\\
0&0&1&&-\frac{1}{3}
\end{pmatrix}$\quad Vert. 2./3.Sp.\\
$\begin{pmatrix}
1&1&1&&0\\
0&1&0&&-\frac{1}{3}
\end{pmatrix}\\
x'3$ frei wählbar\\
$x_2' = -\frac{1}{3}- 0 \cd x'_3 = -\frac{1}{3}\\
x_1' = 9 - x_2' - x_3' = \frac{1}{3} - x_3'\\
x_2$ frei wählbar\\
$x_3 = -\frac{1}{3}\\
x_1=\frac{1}{3}-x_2\\
\mathbb{L}=\left\{ \begin{pmatrix}
\frac{1}{3}-x_2\\
x_2\\
-\frac{1}{3}
\end{pmatrix} : x_2 \in K \right\}$
\item LGS über $\mathbb{C}\\
\begin{matrix}
(1+i)x_1 &+2x_2 &=3-i\\
i x_1 &+(-2+2i)x_2&=4
\end{matrix}\\
\begin{pmatrix}
(1+i)&2 &&3-i\\
i &(-2+2i)&&4
\end{pmatrix}\quad\longrightarrow\quad\begin{pmatrix}
1&1-i &&1-2i\\
i &(-2+2i)&&4
\end{pmatrix}\quad\longrightarrow\quad\begin{pmatrix}
1&1-i &&1-2i\\
0 &-3+i&&2-i
\end{pmatrix}\\
\longrightarrow\quad\begin{pmatrix}
1&1-i &&1-2i\\
0 &1&&-\frac{7}{10}+\frac{1}{10}i
\end{pmatrix}$\\
$x_2 = -\frac{7}{10}+\frac{1}{10}i\\
x_1 = 1-2i - (1-i)x_2 = \frac{8}{5}-\frac{14}{5}i\\
\mathbb{L} = \left\{ \begin{pmatrix}
\frac{8}{5}-\frac{14}{5}i\\
-\frac{7}{10}+ \frac{1}{10}i
\end{pmatrix}  \right\}$
\end{enumerate}
\newpage
\begin{center}
\Huge Nicht mehr Klausurrelevant
\end{center}
\section{Der Vektorraum $\R^+$ (Nicht mehr Klausurrelevant)}
$n \in \N\quad \R^n = \left\{ \begin{pmatrix}
a_1\\
\vdots\\
a_n
\end{pmatrix} : a_1 \in \R \right\}$\\
\emph{Spaltenvektoren} der Länge $n:\: \begin{pmatrix}
a_1\\\vdots\\a_n
\end{pmatrix} = (a_1,\ldots,a_n)^t\\
a_1,\ldots,a_n$ \emph{Komponente} der Spaltenvektoren.\\
Wie bei Matrizen:\\
\begin{minipage}{.5\textwidth}\[ \begin{pmatrix}
a_1\\\vdots\\a_n
\end{pmatrix}+ \begin{pmatrix}
b_1\\\vdots\\b_n
\end{pmatrix} = \begin{pmatrix}
a_1 + b_1\\\vdots\\a_n + b_n
\end{pmatrix}\]
\end{minipage}%
\begin{minipage}{.5\textwidth}
(Multiplikation entspricht der Matrixmultiplikation und ist nicht möglich falls $n > 1$)
\end{minipage}\\
Multiplikation eines Spaltenvektors mit einer Zahl (\emph{Skalar})
\[ a \cd \begin{pmatrix}
a_1\\\vdots\\a_n
\end{pmatrix} = \begin{pmatrix}
aa_1\\\vdots\\aa_n
\end{pmatrix} \]
Addition+Abbildung : $\R^n \times \R^n \to \R^n$\\
$\R^n$ mit Addition und Multiplikation mit Skalaren : \emph{$\R$-Vektorraum}\\
Die Vektoren im $\R^1 (= \R),\R^2$ und $\R^3$ entsprechen Punkten auf der Zahlengerade, Ebene, dreidimensionalen Raums.
\begin{figure}[h!]
\centering
\caption{Ein Vektor dargestellt durch seinen Ortsvektor}
\begin{tikzpicture}
\begin{axis}[axis equal, axis y line = center, axis x line = center, xmax = 3 , ymax = 3, ymin = 0 , xmin = -0.5,xtick={0.1,2},ytick={2},xticklabels={$\begin{pmatrix}
0\\0
\end{pmatrix}$,$a_1$},yticklabels={$a_2$}]
\draw[->] (axis cs:0,0)--(axis cs: 2,2);
\end{axis}
\end{tikzpicture}
\end{figure}
Punkte des $\R^2,\R^3$ lassen sich identifizieren mit, {\em Ortsvektoren} Pfeile mit Beginn in 0 (Komp = 0) und Ende im entsprechenden Punkt\\
Addition von Spaltenvektoren entspricht der Addition von Ortsvektoren entsprechend der Parallelogrammregel.
\begin{figure}[h!]
\centering
\caption{Vektoraddition durch Parallelogrammbildung}
\begin{tikzpicture}
\begin{axis}[
axis x line=center,
axis y line=center,
axis equal,
ymin = -1,
xmin = -3,
ymax = 7,
xmax = 5.5,
xtick ={4},
xticklabels={a},
ytick ={1},
yticklabels={b}]
\addplot [black, mark = *, nodes near coords=,every node near coord/.style={anchor=180}] coordinates {( 4, 3)};
   \draw[->](axis cs:0,0)--(axis cs:3.88,2.88);
       \addplot [black, mark = *, nodes near coords=,every node near coord/.style={anchor=0}] coordinates {( -1, 1)};
              \draw[->](axis cs:0,0)--(axis cs:-0.88,0.88);
\addplot [black, mark = *, nodes near coords=,every node near coord/.style={anchor=0}] coordinates {( 3, 4)};
\addplot[mark=none, black] coordinates {(-1,1) (3,4)};
\addplot[mark=none, black] coordinates {(3,4) (4,3)};
\draw[-> ,red](axis cs:0,0)--(axis cs:2.95,3.88);
\end{axis}
\end{tikzpicture}\end{figure}
Multiplikation mit Skalaren a :\\ Streckung (falls $\abs{a} > 1$)\\ Stauchung (falls $0 \ge \abs{a} \ge 1$)\\
Richtungspunkt, falls $a < 0$
TODO: Steckung und Stauchung\\
\subsection{Satz (Rechenregeln in $\R^n$)}
Seien $u,v,w \in \R^n,a,b\in\R$ Dann gilt:\\
\begin{enumerate}[a)]
\item \begin{align}
u + (v + w) &= (u + v) + w \tag{1.1}\\
v + 0 = 0 + v &= v, \text{wobei } 0\ \textit{Nullvektor} \tag{1.2}\\
v + -v &= 0 \tag{1.3}\marginnote{$\R^n$ kommutative Gruppe}\\
u + v &= v + u \tag{1.4}\\
(a + b)v &= av + bv \tag{2.1}\\
a(u + v) &= au + av \tag{2.2}\\
(a \cd b)v &=a(bv)\tag{2.3}\\
1v &= v \tag{2.4}
\end{align}
\item $0 \cdot v = 0$ und $ a \cdot 0 = 0$\\
Beweis folgt aus entsprechenden Rechenregeln in 0
\end{enumerate}
\subsection{Definition}\label{sec:10.2}
Eine nicht-leere Teilmenge $\mathcal{U} \supset \R^n$ hei\ss t \emph{Unterraum} (oder \emph{Teilraum} von $\R^n$), falls gilt:
\begin{enumerate}[(1)]
\item $\forall u_1,\,u_2 \in \mathcal{U}:\: u_1 + u_2 \in \mathcal{U}$ (Abgeschlossenheit bezüglich +)
\item $\forall u \in \mathcal{U} \forall a \in \R:\: au \in \mathcal{U}
$(Abgeschlossenheit bezüglich Mult. mit Skalaren)
\end{enumerate}
$\mathcal{U}$ enthält Nullvektor \{0\} Unterraum von $\R^n$ (Nullraum)\\
$\R^n $ ist Unterraum von $\R$
\subsection{Beispiele}
\begin{enumerate}[a)]
\item $0 \ne v \in \R^2\quad G = \{ av: a \in \R \}$ ist Unterraum von $\R^n$\begin{minipage}{.3\textwidth}
($a_1v,a_2v \in G, (a_1 + a_2)v \in G$\quad2.1 in \ref{sec:10.2}\\
$av \in G, b \in \R (ba)v \in G$)
\end{minipage}\\
G = Ursprungsgerade durch $\begin{pmatrix}
0\\0
\end{pmatrix}$ und v = $\begin{pmatrix}
a_1\\a_2
\end{pmatrix}
n = 2$:
\begin{figure}[h!]
\centering
\caption{Gerade dargestellt durch Vektoren}
\begin{tikzpicture}
\begin{axis}[axis equal, axis y line = center, axis x line = center, xmax = 3 , ymax = 2, ymin = -2, xmin = -0.5,xtick={0.1,2},ytick={2},xticklabels={$\begin{pmatrix}
0\\0
\end{pmatrix}$,$a_1$},yticklabels={$a_2$}]
\draw[blue,thick] (axis cs:-2,-2)--(axis cs: 2,2);
\end{axis}
\end{tikzpicture}
\end{figure}
\item $v,w \in \R^n\\
E = \{ av + bw : a,b \in \R \}$ ist Unterraum von $\R^n\\
v = o, w = o :\: E = \{ o \}\\
v \ne o\quad w \not\in \{ av:\: a \in \R \}\\
E = \R^2 $
$n = 3:\:$ Ebene durch $\begin{pmatrix}
0\\0\\0
\end{pmatrix}$ und durch $v,w$\\
Ist $w \in \{ av : a \in \R \},$ so ist $E = G$ (aus a))
\item $v,w \ne o\\
G' = \{ w + av : a \in \R\}$\\
$[v \in G' \Leftrightarrow \exists a \in \R: w+ av \in o \Leftrightarrow \exists a \in \R :\: w = (-a)v \in G]$
\end{enumerate}
\subsection[Satz]{Satz}
Seien $\mathcal{U}_1,\mathcal{U}_2$ Unterräume von $\R^n$\\
\begin{enumerate}[a)]
\item $\mathcal{U}_1 \cap \mathcal{U}_2$ ist Unterraum von $\R^n$
\item $\mathcal{U}_1 \cup \mathcal{U}_2$ ist im Allgemeinen \textsc{kein} Unterraum von $\R^n$
\item $\mathcal{U}_1 + \mathcal{U}_2 := \{u_1 + u_2 : u_1:\: \mathcal{U}_1, u_2:\: \mathcal{U}_2\}$
(Summe von $\mathcal{U}_1$ und $\mathcal{U}_2$) ist Unterraum von $\R^n$.
\item $\U_1 \subseteq \U_1 + \U_2$\quad$\U_2 \subseteq \U_1+\U_2$ und $\U_1 + \U_2$ ist der kleinste Unterraum von $\R^n$, der $\U_1$ und $\U_2$ enthält. (d.h ist $w$ Unterraum von $\R^n$ mit $\U_1,\U_2 \in w,$ so $\U_1 + \U_2 \subseteq W$)
\end{enumerate}
\begin{proof}
a) $\checkmark$\\
b) TODO\\
c) TODO
\end{proof}
\subsection{Beispiel}
\begin{enumerate}[a)]
\item \ref{sec:10.3}b)
$G_1 = \{ av :\: a \in \R \}\\
G_2 = \{ aw :\: a \}\\
G_1 + G_2 = E$
\item $\R^3\\
E_1 = \left\{ r \cd \begin{pmatrix}
1\\0\\0
\end{pmatrix} + s \cd \begin{pmatrix}
0\\0\\1
\end{pmatrix}:\: r,s\in\R \right\}\\
\phantom{E_1} = \left\{\begin{pmatrix}
r\\0\\s
\end{pmatrix}:\: r,s\in\R \right\}\\
E_2= \left\{ t \cd \begin{pmatrix}
0\\1\\0
\end{pmatrix} + u \cd \begin{pmatrix}
1\\1\\1
\end{pmatrix} \right\}\\
\phantom{E_2}= \left\{\begin{pmatrix}
u\\t+u\\u
\end{pmatrix} \right\}$\\
$E_1 + E_2$ Unterräume von $\R^3$ (10.3.b)\\
$E_1 \cap E_2 = ?\\
v \in E_1 \cap E_2 \Leftrightarrow v = \begin{pmatrix}
r\\0\\s
\end{pmatrix} = \begin{pmatrix}
u\\t+u\\u
\end{pmatrix} \Leftrightarrow r = u, t + u = 0 , s = u\\
E_1 \cap E_2 = \left\{ \begin{pmatrix}
u\\0\\u 
\end{pmatrix}: u \in \R\right\}\\
\phantom{E_1 \cap E_2} = \left\{ u \cd \begin{pmatrix}
1\\0\\1
\end{pmatrix}: u \in \R \right\}\\
E_1 + E_2 = ?\\
E_1 + E_2 = \R^3$, denn :\\
Es gilt sogar:\\
$\R^3 = E_1 + G_2$, wobei\\
$G_2 = \left\{ t \cd \begin{pmatrix}
0\\1\\0
\end{pmatrix}: t \in \R \right\} \subseteq E_@\\
\begin{pmatrix}
x\\y\\z
\end{pmatrix} = x \cd \begin{pmatrix}
1\\0\\1
\end{pmatrix} z\cd \begin{pmatrix}
0\\0\\1
\end{pmatrix} + y \cd \begin{pmatrix}
0\\1\\0
\end{pmatrix} = \begin{pmatrix}
x\\0\\z
\end{pmatrix} + \begin{pmatrix}
0\\y\\0
\end{pmatrix}\\
\begin{pmatrix}
x\\y\\z
\end{pmatrix} =  (x-y) \begin{pmatrix}
1\\0\\0
\end{pmatrix} + (z-y)\begin{pmatrix}
0\\0\\1
\end{pmatrix} + y \begin{pmatrix}
1\\1\\1
\end{pmatrix}$\\
$\phantom{\begin{pmatrix}
0\\0\\1
\end{pmatrix}}=\begin{pmatrix}
x-y\\0\\z-y
\end{pmatrix} + \begin{pmatrix}
y\\y\\y
\end{pmatrix}$
\end{enumerate}
\subsection{Definition}
\begin{enumerate}[a)]
\item$v_1,\ldots , v_m \in \R^n, a_1,\ldots a_m \in \R$\\
Dann hei\ss t $a_1v_1 + \ldots + a_m v_m = \sum_{i = 1}^{m} a_iv_i\\
$\emph{Linearkombination} von $v_1,\ldots ,v_m$ (mit Koeffizienten $a_1,\ldots,a_m$).\\
$[ $Zwei formal verschiedene Linearkombinationen der gleichen $v_1,\ldots, v_m$ können den gleichen Vektor darstellen \\
$1 \cd \begin{pmatrix}
1\\0
\end{pmatrix} + 2 \cd \begin{pmatrix}
0\\1
\end{pmatrix} + 3 \cd \begin{pmatrix}
1\\1
\end{pmatrix} = 2 \cd \begin{pmatrix}
1\\0
\end{pmatrix} + 3 \cd \begin{pmatrix}
0\\1
\end{pmatrix} + 2 \cd \begin{pmatrix}
1\\1
\end{pmatrix} = \begin{pmatrix}
4\\5
\end{pmatrix}]$
\item Ist $M \subseteq R^n$, so ist der von M \emph{erzeugte} (oder \emph{aufgespannte}) Unterraum $\langle M \rangle_\R$ (oder $\langle M \rangle$) die Menge aller (endlichen) Linearkombinationen, die man mit Vektoren aus M bilden kann.\\
$\langle M \rangle_\R = \left\{ \sum\limits_{i=1}^{n}a_iv_i : n\in\N , a_i \in\R ,v_i\in M \right\}$ falls $M \ne \varnothing\\
\langle\varnothing\rangle_\R := \{ \varnothing \}\\
M = \{ v_1,\ldots v_m \}$, so TODO...
\end{enumerate}
\subsection{Beispiel}
\begin{enumerate}[a)]
\item $e_i = \begin{pmatrix}
0\\0\\1\\0\\0
\end{pmatrix}\in\R^n\\
\langle e_1,\ldots e_n\rangle = \R^n\\
\begin{pmatrix}
x_1\\\vdots\\x_n
\end{pmatrix} = x_1e_1+x_2e_2+\ldots+x_ne_n$
\item $\U = \langle \begin{pmatrix}
1\\2\\3
\end{pmatrix},\begin{pmatrix}
3\\2\\1
\end{pmatrix},\begin{pmatrix}
2\\3\\4
\end{pmatrix}\rangle_\R$\\
Ist $\U = \R^3$?\\
Für welche $\vektor{x}{y}{z} \in \R^3$ gibt es geeignete Skalare $a,b,c\in\R$ mit $a\vektor{1}{2}{3}+b\vektor{3}{2}{1}+c\vektor{2}{3}{4}?\\$
\[ \begin{matrix}
a &+3b&+2c&=x\\
2a&+2b&+3c&=y\\
3a&+b &+4c&=z
\end{matrix} \]
LGS für die Unbekannten $a,b,c$ mit variabler rechter Seite : Gau\ss\\
\[ \begin{pmatrix}
1&3&2&&x\\
2&2&3&&y\\
3&1&4&&z
\end{pmatrix}\quad\to\quad\begin{pmatrix}
1&3&2&&x\\
2&-4&-1&&y-2x\\
0&-8&-2&&z-3x
\end{pmatrix}   \]\\
\[\to\quad\begin{pmatrix}
1&3&2&&x\\
0&1&\frac14&&\frac{2x-y}4\\
0&0&0&&x-2y+z
\end{pmatrix} \]
LGS ist lösbar $\Leftrightarrow x-2y+z=0$.\\
Dass hei\ss t $\vektor{x}{y}{z} \in \U \Leftrightarrow x -2y = z =0\\
\U = \left\{ \vektor{x}{y}{z} : x-2y+z=0,x,y,z\in\R \right\}\\
\phantom{\U}= \left\{ \vektor{x}{y}{-x+2y} : x,y\in\R \right\}\\
\vektor{2}{3}{4}\in\U$
\end{enumerate}
Lösungen des LGS: $c$ frei wählen, b, a ergeben sich, (falls $x-2y + z = 0)$ z.B $c = 0, b = \frac12x-\frac14y, a = x-3b = -\frac12x + \frac34y$\\
Ist $\mathit{x-2y+z=0}$, so ist\\ $\vektor{x}{y}{z} = (-\frac12x+\frac34y)\vektor{x}{y}{z}+ (\frac12x - \frac14y)\vektor{3}{2}{1}\\
\vektor{2}{3}{4}\frac54\vektor123+\frac14\vektor321\\
\U= \left\langle\vektor123, \vektor321 \right\rangle_\R$
\[\cancel{\begin{matrix}
6x^2&- 3xy&+y^3 &=5\\
7x^3&+ 3x^2y^2&-xy&=7
\end{matrix}} \]
\begin{thebibliography}{9}
\bibitem{k1} Kreu\ss ler, Phister Satz 33.16
\bibitem{k2} WHK 5.37
\bibitem{k3} WHK 6.21
\bibitem{k4} WHK 6.24
\bibitem{k5} WHK 6.25
\bibitem{k6} WHK 6.25
\bibitem{k7} WHK 7.32
\bibitem{k8} WHK 7.19
\end{thebibliography}
\end{document}