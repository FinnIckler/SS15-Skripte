\lstinputlisting[firstline=152,lastline=155]{HigherOrderProcedures.rkt}
Zwei Arten von \emph{Higher Order Prozeduren} (H.O.P)\\
\begin{enumerate}[(1)]
\i akzeptieren, Prozeduren als Parameter oder/und
\i liefern Prozeduren als Ergebnis
\end{enumerate}
\lstinline|filter| ist vom Typ (1).\\
H.O.P vermeiden Duplizierung von Code und führen zu kompakteren Programmen, verbesserte Lesbarkeit und verbesserte Wartbarkeit.\\
\emph{Beispiel}: \lstinline|(map f x)|\\

\begin{minipage}{0.45\textwidth}
	\begin{tikzpicture}
	\draw[black,solid] (0,0)--(0,1);
	\draw[black,solid] (0,0)--(1,0);
	\draw[black,solid] (0,1)--(1,1);
	\draw[black,solid] (1,1)--(1,0);
	\draw[black,solid] (0.5,1)--(0.5,0);
	\node () at (0.25,0.5) {$x_1$};
	\draw[black,solid] (0.75,0.5)--(1.5,-0.75);
	\draw[black,solid] (1.5,-0.75)--(1.5,-1.75);
	\draw[black,solid] (1.5,-0.75)--(2.5,-0.75);
	\draw[black,solid] (1.5,-1.75)--(2.5,-1.75);
	\draw[black,solid] (2.5,-1.75)--(2.5,-0.75);
	\draw[black,solid] (2.0,-1.75)--(2.0,-0.75);
	\node () at (1.75,-1.25) {$x_2$};
	\draw[black,dashed] (2.25,-1.25)--(3,-2.5);
	\draw (3,-2.5)rectangle(4,-3.5);
	\draw[black,solid] (3.5,-2.5)--(3.5,-3.5);
	\node () at (3.85,-3.5) {\begin{rotate}{90}
		\lstinline!empty!
		\end{rotate}};
	\node () at (3.25,-3) {$x_n$};
	\end{tikzpicture}
\end{minipage}%
\begin{minipage}{0.1\textwidth}
	\lstinline|(map f xs)|\eval 
\end{minipage}%
\begin{minipage}{0.45\textwidth}
	\begin{tikzpicture}
	\draw[black,solid] (0,0)--(0,1);
	\draw[black,solid] (0,0)--(1,0);
	\draw[black,solid] (0,1)--(1,1);
	\draw[black,solid] (1,1)--(1,0);
	\draw[black,solid] (0.5,1)--(0.5,0);
	\node () at (0,-0.4) {\lstinline[mathescape]|(f $x_1$)|};
	\draw[black,solid] (0.75,0.5)--(1.5,-0.75);
	\draw[black,solid] (1.5,-0.75)--(1.5,-1.75);
	\draw[black,solid] (1.5,-0.75)--(2.5,-0.75);
	\draw[black,solid] (1.5,-1.75)--(2.5,-1.75);
	\draw[black,solid] (2.5,-1.75)--(2.5,-0.75);
	\draw[black,solid] (2.0,-1.75)--(2.0,-0.75);
	\node () at (1.75,-2.4) {\lstinline[mathescape]|(f $x_2$)|};
	\draw[black,dashed] (2.25,-1.25)--(3,-2.5);
	\draw (3,-2.5)rectangle(4,-3.5);
	\draw[black,solid] (3.5,-2.5)--(3.5,-3.5);
	\node () at (3.85,-3.5) {\begin{rotate}{90}
		\lstinline!empty!
		\end{rotate}};
	\node () at (3.25,-4) {\lstinline[mathescape]|(f $x_n$)|};
	\end{tikzpicture}
\end{minipage}
\begin{lstlisting}
;Wende f auf Elemente von Liste xs an
(: map ((%a -> %b) (list-of %a) -> (list-of %a)))
(define map
	(lambda (f xs)
	  (cond
	    ((empty? xs) empty)
	    ((pair? xs) (make-pair (f (first xs) 
			(map f (rest xs)))))))
\end{lstlisting}
Allgemeine Transformation von Listen \emph{Listenfaltung} (list folding)\\
Idee: Ersetze die Listenkonstruktoren \lstinline|make-pair| und \lstinline|empty| systematisch.\\
\begin{minipage}{0.45\textwidth}
	\begin{tikzpicture}
	\draw[black,solid] (0,0)--(0,1);
	\draw[black,solid] (0,0)--(1,0);
	\draw[black,solid] (0,1)--(1,1);
	\draw[black,solid] (1,1)--(1,0);
	\draw[black,solid] (0.5,1)--(0.5,0);
	\node () at (0.25,0.5) {$x_1$};
	\draw[black,solid] (0.75,0.5)--(1.5,-0.75);
	\draw[black,solid] (1.5,-0.75)--(1.5,-1.75);
	\draw[black,solid] (1.5,-0.75)--(2.5,-0.75);
	\draw[black,solid] (1.5,-1.75)--(2.5,-1.75);
	\draw[black,solid] (2.5,-1.75)--(2.5,-0.75);
	\draw[black,solid] (2.0,-1.75)--(2.0,-0.75);
	\node () at (1.75,-1.25) {$x_2$};
	\draw[black,dashed] (2.25,-1.25)--(3,-2.5);
	\draw (3,-2.5)rectangle(4,-3.5);
	\draw[black,solid] (3.5,-2.5)--(3.5,-3.5);
	\node () at (3.85,-3.5) {\begin{rotate}{90}
		\lstinline!empty!
		\end{rotate}};
	\node () at (3.25,-3) {$x_n$};
	\end{tikzpicture}
\end{minipage}%
\begin{minipage}{0.1\textwidth}
	\lstinline|(foldr z c xs)|\eval 
\end{minipage}%
\begin{minipage}{0.45\textwidth}
	\begin{tikzpicture}
	\node () at (0,-0.4) {$x_1$};
	\draw[black,solid] (0.75,0.5)--(1.5,-0.75);
	\draw[black,solid] (0.75,0.5)--(0.1,-0.4);
	\node () at (0.75,0.6) {$c$};
	\node () at (1.5,-0.8) {$c$};
	\draw[black,solid] (1.55,-1)--(1.25,-2.4);
	\node () at (1,-2.4) {$x_2$};
	\draw[black,dashed] (1.55,-1)--(2.55,-2.4);
	\node () at (2.55,-2.4) {$c$};
	\draw[black,solid] (2.55,-2.5)--(3.85,-3.4);
	\draw[black,solid] (2.55,-2.5)--(2.5,-3.4);
	\node () at (3.85,-3.5) {$z$};
	\node () at (2.5,-3.5) {$x_n$};
	\end{tikzpicture}
\end{minipage}
\lstinline|(foldr z c xs)| wirkt als Spinetransformer
\begin{itemize}
	\i[-] \lstinline|empty| \eval z
	\i[-] \lstinline|make-pair| \eval c
	\i[-] Eingabe : Liste \lstinline|(list-of %a)|
	\i[-] Ausgabe : im Allgemeinen \emph{keine} Liste mehr: \lstinline|%b|
\end{itemize}
\begin{lstlisting}
;Falte Liste xs bzgl. c und z
(: foldr (%b (%a %b -> %b) (list-of %a) -> %b))
(define foldr
	(lambda (z c xs)
	  (cond 
	    ((empty? xs) z)
	    ((pair? xs)
	      (c (first xs)
	         (foldr z c (rest xs)))))))
\end{lstlisting}
Beispiele: Listenreduktion mit \lstinline|foldr|\\
TODO: Gro\ss es Bild von foldr Funktionen
\begin{lstlisting}
(: sum ((list-of number) -> number))
(define sum(lambda (xs)(foldr 0 + xs)))
\end{lstlisting}
Beispiel: Länge einer Liste durch Listenreduktion
TODO: Bild Plotten
\begin{lstlisting}
; Listenreduktion via foldr: Länge der Liste xs
(: my-length ((list-of %a) -> natural))
(define my-length
	(lambda (xs)
		(foldr 0 (lambda (x l) (+ 1 l)) xs)))
\end{lstlisting}
\rackett{Anwendungsbeispiele foldr}{Fold und seine Anwendungen}{HigherOrderProcedures}{91}{151}