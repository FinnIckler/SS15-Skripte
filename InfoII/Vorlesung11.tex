\lstinputlisting[frame=single,caption={[Bildmanipulation mit Listen aus Pixeln]Ausflug: Bluescreen Berechnung wie in Starwars mit Listen:}]{bluescreen.rkt}
Generierung aller natürlichen Zahlen (vgl. gemischte Daten)\\
Eine natürliche Zahl (natural) ist entweder 
\begin{enumerate}[-]
\i die 0 (zero)
\i der Nachfolge (succ) einer natürlichen Zahl
\end{enumerate}

$\mathbb{N} = \{0, (succ(0)), (succ(succ(0))), \ldots \} $\\
\uline{Konstruktoren}\\
\begin{lstlisting}
(: zero natural)
(define zero 0)
(: succ (natural -> natural))
(define succ (lambda (n)(+ n 1)))
\end{lstlisting}
Vorgänger (pred), definiert für $n > 0$
\begin{lstlisting}
(: pred (natural -> natural))
(define pred
	(lambda (n) (- n 1)))
\end{lstlisting}
Bedingte algebraische Eigenschaft (für check-property):\\
\lstinline!(==> <p> <t>)!\\
Nur wenn \auf p \zu \eval \# t ist, wird Ausdruck \argt{} ausgwertet und getestet \argt{} \eval \# t
\racket{Check-property mit Einschränkungen}{$==>$ als Einschränkungsoperator}{notquiteall}
Beispiel für Rekursion auf natürlichen Zahlen: Fakultät\\$
\begin{array}{rcl}
0! &=& 1\\
n! &=& n \cdot (n -1)!\\
&\\
3! &=& 3 \cdot 2!\\
&=& 3 \cdot 2 \cdot 1!\\
&=& 3 \cdot 2 \cdot 1 \cdot 0!\\
&=& 3 \cdot 2 \cdot 1 \cdot 1\\
&=& 6\\
&\\
10 &=& 3628800
\end{array}$\bigskip \\
\racket{Rekursion auf natürlichen Zahlen: Fakultät}{Fakultät rekursiv}{Fakultaet}
Konstruktionsanleitung für Prozeduren über natürlichen Zahlen:\\
\begin{lstlisting}
(:<f> (natural -> <t>))
	(define <f>
		(lambda (n)
			(cond ((= n 0)...)
				((> n 0) ... (<f> (- n 1))...))))
\end{lstlisting}
Beobachtung:\\
\begin{enumerate}[-]
\i Im letzten Zweig ist $n > 0 \rightarrow$ pred angewandt
\i \lstinline!(<f> (- n 1))! hat die Signatur \argt{}
\end{enumerate}
Satz:\\
Eine Prozedur, die nach der Konstruktionsanleitung für Listen oder natürliche Zahlen konstruiert wurde \uline{terminiert immer} (= liefert immer ein Ergebnis).\\
(Beweis in Kürze)\\
\racket{Fehlerhafte Rekursionen}{Fehlerhafte Rekursionen}{FehlerRekursiv}
$\overbrace{(3 \cdot (2 \cdot (1}^{\t{merken}} \cdot 0!)))$\\
Die Grö\ss e eines Ausdrucks ist proportional zum Platzverbrauch des Reduktionsprozesses im Rechner\\
$\Rightarrow$ Wenn möglich Reduktionsprozesse, die \uline{konstanten} Platzverbrauch - unabhängig von Eingabeparametern - benötigen