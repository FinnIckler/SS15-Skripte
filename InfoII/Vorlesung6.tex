\underline{Zusammengesetze Daten}\\
Ein Charakter \underline{besteht} aus drei \underline{Komponenten}\\
\begin{tabular}{clcl}
- & Name des Charakters &(name)\rdelim\}{3}{0mm}
[Datendefinition für zusammengesetzte Daten]\\
- & Handelt es sich um einen Jedi? &(jedi?)&\\
- & Stärke der Macht \hspace*{2.3cm} &(force)&
\end{tabular}\\
Konkrete Charakter:\hspace*{5pt}
\begin{tabular}{|c|c|}
\hline
name & \glqq Luke Skywalker \grqq\\
\hline
jedi? & \#f \\
\hline
force & 25 \\
\hline
\end{tabular}\\
\begin{lstlisting}[frame=listing]
; Ein Charakter (character) besteht aus
; - Name (name)
; - Jedi-Status (jedi?)
; - Stärke der Macht (force)
(: make-character (string boolean real -> character))
(: character? (any -> boolean))
(: character-name (character -> string))
(: character-jedi? (character -> boolean))
(: character-force (character -> real))
(define-record-procedures character
  make-character
  character?
  (character-name
   character-jedi?
   character-force))

; Definiere verschiedene Charaktere des Star Wars Universums
(define luke
  (make-character "Luke Skywalker" #f 25))
(define r2d2
  (make-character "R2D2" #f 0))
(define dooku
  (make-character "Count Dooku" #f 80))
(define yoda
  (make-character "Yoda" #t 85))

\end{lstlisting}
Zusammengesetzte Daten = \underline{Records} in Scheme Record-Definition legt fest:\\
\begin{tabular}{cll}
- & Record-Signatur\\
- & \underline{Konstruktor} & (baut aus Komponenten einen Record)\\
- & Prädikat & (liegt ein Record vor?)\\
- & Liste von \underline{Selektoren}& (lesen jeweils eine Komponente des Records)
\end{tabular}\\
\begin{lstlisting}
(define-record-procedure <t>
	make-<t>
	<t>?
	(<t>-<comp1> ... <t>-<comp2>))
	;Liste der n Selektoren
\end{lstlisting}
Verträge des Konstruktors\/ der Selektoren für Record- Signatur\\
\argt{} mit Komponenten namens \argcomp{1} $\ldots$ \argcomp{n}\\
\begin{lstlisting}
(: make-<t> (<t1>...<t2>) -> <t>)
(: <t>-<comp1> (<t> -> <t1>))
(: <t>-<compn> (<t> -> <tn>))
\end{lstlisting}
Es gilt für alle Strings n, Booleans j und Integer f:
\begin{lstlisting}
(character-name (make-character n j f) n)
(character-jedi? (make-character n j f) j)
(character-force (make-character n j f) f )
\end{lstlisting}
Spezialform check-property:\\
\begin{lstlisting}
(check-property
	(for-all ((<id1> <sig1>) ... 
			  (<idn> <sign>))
	<e>))
	@\latexcode{$\downarrow$}@
;Bezieht sich auf <id1> ... <idn>	
\end{lstlisting}
Test erfolgreich, falls \arge{} für beliebig gewählte Bedeutungen für \argid{1} $\ldots$ \argid{n} immer \#t ergibt\\
Interaktion von Selektoren und Konstruktor:
\begin{lstlisting}[frame=listing]
(check-property 
 (for-all ((n string)
           (j boolean)
           (f real))
   (expect (character-name (make-character n j f)) n)))

(check-property 
 (for-all ((n string)
           (j boolean)
           (f real))
   (expect (character-jedi? (make-character n j f)) j)))

(check-property 
 (for-all ((n string)
           (j boolean)
           (f real))
   (expect-within (character-force (make-character n j f)) f 0.001)))
\end{lstlisting}
\underline{Beispiel:} Die Summe von zwei natürlichen Zahlen ist mindestens so gro\ss \ wie jeder dieser Zahlen: $ \forall x_1 \in \mathbb{N}, x_2 \in \mathbb{N} : x_1 + x_2 \geq \max\{x_1,x_2\}$
\begin{lstlisting}[frame=single]
; Für alle natürlichen Zahlen x1,x2 gilt: x1 + x2 @$\textcolor{kommentar}{\geq}$@ max(x1,x2)
(check-property
 (for-all ((x1 natural)
           (x2 natural))
   (>= (+ x1 x2) (max x1 x2))))
\end{lstlisting}
Konstruktion von Funktionen, die bestimmte gesetzte Daten \underline{konsumiert}.\\
\begin{enumerate}[-]
\i Welche Record-Componenten sind relevant für Funktionen?
\begin{enumerate}[$\rightarrow$]
\i Schablone:
\begin{lstlisting}
(: sith? (character -> boolean))
(define sith?
  (lambda (c)
    ... (character-jedi? c))
    ... (character-force c) )...))
\end{lstlisting}
\end{enumerate}
Konstruktion von Funktionen, die zusammengesetzte Daten \underline{konstruieren}
\i Der konstruktor \underline{muss} aufgerufen werden
\begin{enumerate}[$\rightarrow$]
\i Schablone:
\begin{lstlisting}
(define
	lambda(...)
		... (make-<t>)...)
\end{lstlisting}
\end{enumerate}
\i Konkrete Beispiele:
\begin{lstlisting}[frame=single]
; Könnte Charakter c ein Sith sein?
(: sith? (character -> boolean))
(check-expect (sith? yoda) #f)
(check-expect (sith? r2d2) #f)
(define sith?
  (lambda (c)
    (and (not (character-jedi? c))
         (> (character-force c) 0))))


; Bilde den Charakter c zum Jedi aus (sofern c überhaupt Macht besitzt)
(: train-jedi (character -> character))

(check-expect (train-jedi luke) (make-character "Luke Skywalker" #t 50))
(check-expect (train-jedi r2d2) r2d2)

(define train-jedi
  (lambda (c)
    (make-character (character-name c) 
                    (> (character-force c) 0)
                    (* 2 (character-force c)))))
\end{lstlisting}
\end{enumerate}
