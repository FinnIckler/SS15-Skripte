\emph{Induktive Definition}\\
Konstante Definition der natürlichen Zahlen $\mathbb{N}$.\\
Definition: (Peamo Axiome)
\begin{align}
	0 \in& \mathbb{N} \tag{P1} \\
	\forall n \in& \mathbb{N}: succ(n) \in \mathbb{N} \tag{P2}\\
	\forall n \in& \mathbb{N}: succ(n) \ne 0 \tag{P3}\\
	\forall n,m \in& \mathbb{N} : succ(n) = succ(m) \Leftrightarrow n = m \tag{P4}
\end{align}\\
TODO: "Plot" mit punkten und Pfeilen
\begin{multline}
	\t{Für jede Menge }M \subset N \t{ mit } 0 \in M \\
	\t{und } \forall n : (n \in M \Rightarrow succ(n) \in M), \t{gilt } M = \mathbb{N} \tag{P5}
\end{multline}
''$\mathbb{N}$ enthält nicht mehr als die 0 und die durch $succ()$ generierten Elemente\\
''Nicht ist sonst in $\mathbb{N}$,\\
TODO: Plot von zwei kreisen ineinander
Beweisschema der {\em vollständigen Induktion}\\
Sei $P(n)$ eine Eigenschaft einer Zahl $n \in \mathbb{N}$\\
\lstinline|(: P (natural -> boolean))|\\
Ziel : $\forall n \in \mathbb{N} : P(n)$\\
Definiere $M = \{n \in \mathbb{N} \vert P(n) \} \subset \mathbb{N}$\hfill
M enthält die Zahlen n für die $P(n)$ gilt \\
\emph{Induktionsaxiom}\\
Falls \par 0 $\in M$\\
und \par $\forall n : (n \in M \Rightarrow succ(n) \in M)$\\
dann \par $M \in \mathbb{N}$\\
\begin{minipage}[c]{0.235\textwidth}
	Induktionsstart\bigskip\\
	\ \\
	Induktionsschritt
\end{minipage}
\begin{minipage}[c]{0.765\textwidth}
	\begin{mdframed}
		Falls \par $P(0)$\\
		und \par $\forall (P(n) \Rightarrow P(succ(n))$\\
		dann \par $\forall n \in \mathbb{N} P(n)$
	\end{mdframed}
\end{minipage}
{\em Beispiel}:\\
$\begin{array}{ll}
1 & = 1\\
1 + 3 & = 4\\
1 + 3 + 5 & = 9\\
1 + 3 + 5 + 7 & = 16\\
\ldots\\
P(n) & = \underbrace{\sum\limits_{i=0}^{n} (2i + 1)}_{\substack{\t{Summe der}\\
			\t{ersten n}\\
			\t{ungeraden Zahlen}}} \overset{!}{=} (n+1)^2
\end{array}$\\
\emph{Induktionsschluss P(0)}\\
$\sum\limits_{0}^{0} (2i + 1) = 2 \cdot 0 + 1 = (0 + 1)^ 2 \checkmark$\\
Induktionsschritt $\forall n (P(n)) = P (n+1)\\
\begin{array}{ll}
\sum\limits_{i=0}^{n+1} (2i + 1) &\overset{\sum}{=} \sum\limits_{i=0}^{n} (2i + 1) + (2 (n+1) +1)\\
&\overset{iv.}{=} (n+1)^2 + 2n +3\\
&= n^2 + 4n + 4\\
 &= ((n+1)+1)^2 \checkmark
\end{array}$\\
{\em Beispiel}:\\
\begin{lstlisting}
	(define factorial 
		(lambda (k)
			(if 
				(= k 0) 1
				(* k (factorial (- k 1))))))
\end{lstlisting}
$P(x) \equiv \text{\lstinline!(factorial n)!} = \fbox{n!}$ \hfill $\fbox{x}$:(Racket Repräsentation für $x \in \mathbb{N}$)\\
Zeige : $\forall n \in \mathbb{N} : P(n)$\\
\emph{Induktionsbasis $P(0)$}\\
\lstinline[mathescape]!(factorial($\numrak{0}$))!\\
\begin{enumerate}[\eval]
\i[$\overset{\star}{\text{\eval}}$] \lstinline[mathescape]!((lambda (k)...)$\numrak{0}$)!
\i \lstinline[mathescape]|(if (= $\numrak{0}$ 0)1 ...)|
\i \lstinline[]|(if #t|\lstinline| 1 ...)|
\i 1 = \numrak{0}! $\checkmark $
\end{enumerate}
Induktionsschritt: $\forall n : (P(n) \rightarrow P(n+1))$\\
\lstinline[mathescape]|(factorial $\numrak{n+1}$)|
\begin{enumerate}[\eval]
\i[$\overset{\star}{\text{\eval}}$] \lstinline[mathescape]|((lambda (n)...)$\numrak{n+1}$)|\\
\i \lstinline[mathescape]|(if (= $\numrak{n+1}$ 0)1 ... (...))|
\i \lstinline[]|(if #f|\lstinline| 1 ... (...))|
\i \ \marginpar{\flushleft \textcolor{red}{Unter der Annahme, dass tatsächlich Subtraktion implementiert ist}} \lstinline[mathescape]|(* $\numrak{n+1}$ (factorial (- $\numrak{n+1}$ 1)))|
\i \lstinline[mathescape]|(* $\numrak{n+1}$ (factorial $\numrak{n}$))|
\i[$\overset{iv}{=}$] \lstinline[mathescape]|(* $\numrak{n + 1}$ n!)|
\i[=]$(n+1)!\ \checkmark$
\end{enumerate}
\emph{Beispiel}:\\
Jede durch die Konstruktionsanleitung für Funktionen über natürliche Zahlen konstruierte Funktion liefert ein Ergebnis ({\em terminiert immer})\\
\begin{lstlisting}
(define f
	(lambda (n)
		(if
			(= n 0) base
			(step (f (n-1)) n))))
\end{lstlisting}
\lstinline|(: base natural)|\\
\lstinline|(: step (natural natural -> natural))|
Bsp: step $\rightarrow$ \lstinline|(lambda (x y) (* x y))|\\
Dann gilt $P(n)$ = \lstinline!(f n)! terminiert (Mit Ergebnis der Signatur \lstinline!natural!)\\
Zeige $\forall n \in \mathbb{N} : P(n)$\\
\emph{Induktionsbasis P(0)}:\\
\lstinline[mathescape]|(f $\numrak{0}$)|
\begin{enumerate}[\eval]
\i \lstinline[mathescape]|(if (= $\numrak{0}$ 0) base ...)|
\i \lstinline[]|(if #t base|
\i base $\checkmark $
\end{enumerate}
Induktionsschritt $\forall n : (P(n) \rightarrow P(n+1))$\\
\lstinline[mathescape]|(f $\numrak{n+1}$)|
\begin{enumerate}[\eval]
	\i \lstinline[mathescape]|(if (= $\numrak{n+1}$ 0) base ... (step ...))|
	\i \lstinline[]|(if #f|\lstinline| base ... (step ...))|
	\i \lstinline[mathescape]|(step (f (- $\numrak{n+1}$ 1)) $\numrak{n+1}$)|
	\i \lstinline[mathescape]|(step $\underbrace{(f \numrak{n})}_{\t{terminiert}}$ $\numrak{n+1}$)|
	\i[$\Rightarrow$]\lstinline[mathescape]!(step (f $\numrak{n}$) $\numrak{n+1}$)! terminiert
\end{enumerate}
\emph{Definition}:(Listen.endliche Folge)\\
Die Menge $M^*$ (= Listen mit Elementen aus $M$ + \lstinline|list-of M| ist \emph{induktiv} definiert\\
\begin{align}
\text{\lstinline!empty!} \in& M^* \tag{L1} \marginnote{Nicht leere Liste}\\
\forall x \in& M , xs \in M^* \tag{L2} \marginnote{\flushright \text{\lstinline!(make-pair x xs)!}}\marginnote{\t{$\in M^*$}} \\
\t{Nichts sonst in }& M^* \tag{L3}
\end{align}
Beweisschema \emph{Listeninduktion}\\
So $P(xs)$ eine Eigenschaft von Listen über M.\\
\lstinline|(: P ((list-of M) -> boolean))|\\
\begin{mdframed}
	Falls $P(\text{\lstinline|empty|})$ \marginnote{Induktionsanfang}\\
	und \par$\forall x \in M, xs : P(xs) \Rightarrow (P(xs) \Rightarrow (P($\lstinline|make-pair x xs|))\\
	dann \par $\forall xs \in M^* : P(xs)$ \marginnote{Indukstionsschritt}
\end{mdframed}