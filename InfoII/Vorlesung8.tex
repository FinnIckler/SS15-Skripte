Man kann jedes \code{one-of} durch ein \code{predicate} ersetzen.
\begin{lstlisting}[frame=single]
(: f ((one-of 0 1 2 ) -> natural))
(define f
  (lambda (x)
    x))

; And then the "The Great one-of Extinction" of 2015 occurred @\includegraphics[scale=0.5]{kaboon}@
;.

(: g ((predicate 
       (lambda (x) (or (= x 0) (= x 1) (= x 2)))) -> natural))
(define g
  (lambda (x)
    x))

\end{lstlisting}
 \ \bigskip\\
Geocoding: Übersetze eine Ortsangabe mittels des Google Maps Geocoding API (Application Programm Interface) in eine Position auf der Erdkugel.
\begin{lstlisting}
(: geocoder (string -> (mixed geocode geocode-error)))
\end{lstlisting}
Ein geocode besteht aus \\
\begin{tabular}{crrl}
& \underline{Signatur}\\
- & Adresse & (address) & string\\
- & Ortsangabe & (loc) & location\\
- & Nordostecke & (northeast) & location\\
- & Südwestecke & (southwest) & location\\
- & Typ & (type) & string\\
- & Genauigkeit & (accuracy) & string
\end{tabular}\\
\begin{lstlisting}
(: geocode-adress (geocode -> string))
(: geocode-loc (geocode -> location))
(: geocode-... (geocode -> ...))
\end{lstlisting}
Ein geocode-error besteht aus:\\
\begin{tabular}{crrl}
& \underline{Signatur}\\
- & Fehlerart & (level) & (one-of "TCP'' "HTTP'' "JSON'' ''API'')\\
- & Fehlermeldung & (message) & string\\
\end{tabular}\\
\underline{Gemischte Daten}\\
Die Signatur
\begin{lstlisting}
(mixed @\argt{1}@ ... @\argt{n}@)
\end{lstlisting}
ist gültig für jeden Wert, der mindestens eine der Signaturen \argt{1} $\ldots$ \argt{n} erfüllt.\\
\underline{Beispiel:} Data-Definition\\
Eine Antwort des Geocoders ist \underline{entweder}\\
\begin{enumerate}[-]
\i ein Geocode (geocode) \underline{oder}
\i eine Fehlermeldung (geocode-error)
\end{enumerate}
Beispiel (eingebaute Funktion string-\zu number)
\begin{lstlisting}
(: string->number (string -> (mixed number (one-of #f))))
(string->number "42") @\eval@ 42
(string-> number "foo") @\eval@ #f
\end{lstlisting}
\begin{lstlisting}[frame=single]
(define geocoder-response
  (signature (mixed geocode geocode-error)))

(: sand13 geocoder-response)
(define sand13
  (geocoder "Sand 13, Tübingen"))

(geocode-address sand13)
(geocode-type sand13)
(location-lat (geocode-loc sand13))
(location-lng (geocode-loc sand13))
(geocode-accuracy sand13)
  

(: lady-liberty geocoder-response)
(define lady-liberty
  (geocoder "Statue of Liberty"))

(: alb geocoder-response)
(define alb
  (geocoder "Schwäbische Alb"))

(: A81 geocoder-response)
(define A81
  (geocoder "A81, Germany"))
\end{lstlisting}
Erinnerung:\\
Das Prädikat \argt{}? einer Signatur \argt{} unterscheidet Werte der Signatur \argt{} von allen anderen Werten:
\begin{lstlisting}
(: @\argt{}@? (any -> boolean))
\end{lstlisting}
Auch: Prädikat für eingebaute Signaturen\\
\begin{lstlisting}
number?
complex?
real?
rational?
integer?
natural?
string?
boolean?
\end{lstlisting}
Prozeduren, die gemischte Daten der Signaturen \argt{1} $\ldots$ \argt{n} konsumieren: \\
\underline{Konstruktionsanleitung}:\\
\begin{lstlisting}
(: @\argt{}@ ((mixed @\argt{1}@ ... @\argt{n}@) -> ...))
(define @\argt{}@
	(lambda (x)
		(cond 
		   ((@\argt{1}@? x)...)
		   ...
		   ((@\argt{n}@? x) ...))))
\end{lstlisting}
Mittels \underline{let} lassen sich Werte an \underline{lokale Namen} binden,
\begin{lstlisting}
(let ( 
	(@\argid{1}@ @\arge{1}@)
	(...)
	(@\argid{n}@ @\arge{n}@))
 @\arge{}@
) 
\end{lstlisting}
Die Ausdrücke \arge{1} $\ldots$ \arge{n} werden \underline{parallel} ausgewertet. $\Rightarrow$ \argid{1} $\ldots$ \argid{n}  können in \arge{} (und nur hier) verwendet werden. Der Wert des let Ausdruckes ist der Wert von \arge{}.
\begin{lstlisting}[frame=single]
; Liegt der Geocode r auf der südlichen Erdhalbkugel?
; (Breitengrad < 0@\latexcode{$^{\circ}$}@?)
(: southern-hemisphere? (string -> boolean))

(check-expect (southern-hemisphere? "Cape Town") #t)
(check-expect (southern-hemisphere? "Tübingen") #f)
(check-error  (southern-hemisphere? "Mos Eisley") "Unknown location")

(define southern-hemisphere?
  (lambda (r)
    (let ((gc (geocoder r)))
      (cond ((geocode? gc)  
             (< (location-lat (geocode-loc gc)) 0))
            ((geocode-error? gc) 
             (violation "Unknown location"))))))
\end{lstlisting}
\underline{ACHTUNG:}\\
'let' ist verfügbar auf ab der Sprachebene "Macht der Abstraktion".
\bigskip\\
'let' ist syntaktisches Zucker.
\begin{lstlisting}
(let (			((lambda (@\argid{1}@ ... @\argid{n}@) 
	(@\argid{1}@ @\arge{1}@)		@\arge{}@)
	(...)		@$\equiv$@        @\arge{1}@
	(@\argid{n}@ @\arge{n}@))	       @\arge{2}@ ...
 @\arge{}@			     @\arge{n}@
) 
\end{lstlisting}