Man kann jedes \lstinline!one-of! durch ein \lstinline!predicate! ersetzen.
\lstinputlisting[caption={[Ersetzung one-of druch predicate Siganturen]Das ''gro\ss e One-of Sterben des Jahres 2015''},frame=single]{oneoftopredicate.rkt}
 \ \bigskip\\
Geocoding: Übersetze eine Ortsangabe mittels des Google Maps Geocoding API (Application Programm Interface) in eine Position auf der Erdkugel.\\
\lstinline!(: geocoder (string -> (mixed geocode geocode-error)))!\\
Ein geocode besteht aus: \\
\begin{tabular}{clll}
& \underline{Signatur}\\
- & Adresse & (address) & string\\
- & Ortsangabe & (loc) & location\\
- & Nordostecke & (northeast) & location\\
- & Südwestecke & (southwest) & location\\
- & Typ & (type) & string\\
- & Genauigkeit & (accuracy) & string
\end{tabular}
\begin{figure*}[h!]
\centering
\begin{tabular}{c}
\begin{lstlisting}
(: geocode-adress (geocode -> string))
(: geocode-loc (geocode -> location))
(: geocode-... (geocode -> ...))
\end{lstlisting}\end{tabular}
\end{figure*}
Ein geocode-error besteht aus:\\
\begin{tabular}{clll}
& \underline{Signatur}\\
- & Fehlerart & (level) & (one-of "TCP'' "HTTP'' "JSON'' ''API'')\\
- & Fehlermeldung & (message) & string\\
\end{tabular}\\
\underline{Gemischte Daten}\\
Die Signatur
\begin{lstlisting}
(mixed @\argt{1}@ ... @\argt{n}@)
\end{lstlisting}
ist gültig für jeden Wert, der mindestens eine der Signaturen \argt{1} $\ldots$ \argt{n} erfüllt.\\
\underline{Beispiel:} Data-Definition\\
Eine Antwort des Geocoders ist \underline{entweder}\\
\begin{enumerate}[-]
\i ein Geocode (geocode) \underline{oder}
\i eine Fehlermeldung (geocode-error)
\end{enumerate}
Beispiel (eingebaute Funktion \lstinline|string->number|)
\begin{lstlisting}
(: string->number (string -> (mixed number (one-of #f))))
(string->number "42") @\eval@ 42
(string-> number "foo") @\eval@ #f
\end{lstlisting}
\lstinputlisting[literate={ü}{{\"u}}1 {ä}{{\"a}}1,frame=single,caption={[Geocoding] Die Google Geocode API}]{geocoding.rkt}
Erinnerung:\\
Das Prädikat \argt{}? einer Signatur \argt{} unterscheidet Werte der Signatur \argt{} von allen anderen Werten:\\
\lstinline[mathescape]!(: $\text{\argt{}}$? (any -> boolean))!\\
Auch: Prädikat für eingebaute Signaturen
\begin{lstlisting}
number?
complex?
real?
rational?
integer?
natural?
string?
boolean?
\end{lstlisting}
Prozeduren, die gemischte Daten der Signaturen \argt{1} $\ldots$ \argt{n} konsumieren: \\
\underline{Konstruktionsanleitung}:\\
\begin{lstlisting}
(: @\argt{}@ ((mixed @\argt{1}@ ... @\argt{n}@) -> ...))
(define @\argt{}@
	(lambda (x)
		(cond 
		   ((@\argt{1}@? x)...)
		   ...
		   ((@\argt{n}@? x) ...))))
\end{lstlisting}
Mittels \underline{let} lassen sich Werte an \underline{lokale Namen} binden,
\begin{lstlisting}
(let ( 
	(@\argid{1}@ @\arge{1}@)
	(...)
	(@\argid{n}@ @\arge{n}@))
 @\arge{}@
) 
\end{lstlisting}
Die Ausdrücke \arge{1} $\ldots$ \arge{n} werden \uline{parallel} ausgewertet. $\Rightarrow$ \argid{1} $\ldots$ \argid{n}  können in \arge{} (und nur hier) verwendet werden. Der Wert des let Ausdruckes ist der Wert von \arge{}.
\racket{cond mit gemischten Daten}{Liegt der Geocode r auf der südlichen Erdhalbkugel?}{hemisphere}
\underline{ACHTUNG:}\\
'let' ist verfügbar auf ab der Sprachebene "Macht der Abstraktion".
\bigskip\\
'let' ist syntaktisches Zucker.
\begin{lstlisting}
(let (			((lambda (@\argid{1}@ ... @\argid{n}@) 
	(@\argid{1}@ @\arge{1}@)		@\arge{}@)
	(...)		@$\equiv$@        @\arge{1}@
	(@\argid{n}@ @\arge{n}@))	       @\arge{2}@ ...
 @\arge{}@			     @\arge{n}@
) 
\end{lstlisting}