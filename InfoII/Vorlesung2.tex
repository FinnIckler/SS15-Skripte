
Auswertung \underline{zusammengesetzter Ausdr\"ucke} in mehreren Schritten (Steps), von ``innen nach au\ss en``, bis keine Reduktion mehr m\"oglich ist.\\
\begin{lstlisting}
(+ ( @\uwave{(+ 20 20)}@ (+ 1 1)) @\eval@ (+ 40 @\uwave{(+ 1 1)}@ @\eval@ (+ 40 2) @\eval@  42 
\end{lstlisting}
\rackett{Arithmetik mit Flie\ss kommazahlen}{\textbf{Achtung:} Scheme rundet bei Arithmetik mit Flie\ss kommazahlen (interne Darstellung ist binär}{Grundlagen}{4}{17}
Ein Wert kann an einen \underline{Namen} (auch \underline{Identifier}) gebunden werden, durch \\
\lstinline!(define <id> <e>)! \hspace*{1.5cm} \auf id\zu Identifier \ \auf e\zu Ausdruck\\
Erlaubte konsistente Wiederverwendung, dient der Selbstdokumentation von Programmen\\
\textbf{\underline{Achtung:}} Dies ist eine sogenannte Spezialform und kein Ausdruck. Insbesondere besitzt diese Spezialform \underline{keinen} Wert, sondern einen Effekt Name \auf id\zu \ wird an den \underline{Wert} von \auf e\zu \ gebunden. \\
Namen k\"onnen in Scheme beliebig gewählt werden, solange
\begin{enumerate}
\item[(1)] die Zeichen ( ) $[$ $]$ $\{ \}$ `` , ` ` ; \# $\mid$ \textbackslash nicht vorkommen
\item[(2)] dieser nicht einem numerischen Literal gleicht.
\item[(3)] kein Whitespace (Leerzeichen, Tabulator, Return) enthalten ist.
\end{enumerate}
Beispiel: euro$\rightarrow$US\$\\
\underline{\textbf{Achtung:}} Gro\ss-\textbackslash Kleinschreibung ist irrelevant.
\bigskip\\
\lstinputlisting[frame=single,caption={[Schlüsselwort define]Bindung von Werten an Namen},firstline=19,lastline=24]{Grundlagen.rkt}
Eine \underline{lambda-Abstraktion} (auch Funktion, Prozedur) erlaubt die Formatierung von Ausrdr\"ucken, in denen mittels \underline{Parametern} von konkreten Werten abstrahiert wird.\\
\lstinline[literate=]!(lambda (<p1><p2>...) <e>!\\
\auf e\zu Rumpf: enth\"alt Vorkommen der Parameter \auf $p_n$\zu\\
(lambda($\ldots$)) ist eine Spezialform. Wert der lambda-Abstraktion ist \#\auf procedure\zu\\.
\underline{Anwendung} (auch Application) des lambda-Aufrufs f\"uhrt zur Ersetzung aller Vorkommen der Parameter im Rumpf durch die angegebenen \uline{Argumente}.\\
\lstinputlisting[frame=single,caption={[Lambda Abstraktion] Lambda-Abstraktion},firstline=26,lastline=34]{Grundlagen.rkt}
\lstinline!(lambda (days) (* days (* 155!\lstinline! minutes-in-a-day))) 365)! \eval\\
\lstinline!(* 365 (* 155!\lstinline! minutes-in-a-day))! \eval  \lstinline!81468000!\\
\bigskip\\
In Scheme leitet ein Semikolon einen Kommentar ein, der bis zum Zeilenende reicht und vom System bei der Auswertung ignoriert wird.\\
Prozeduren sollten im Programm ein- bis zweizeilige \underline{Kurzbeschreibungen} direkt vorangestellt werden.