\begin{filecontents*}{CurryUndMengen.rkt}
; Erinnerung: filter
(: filter ((%a -> boolean) (list-of %a) -> (list-of %a)))
(define filter
  (lambda (p? xs)
    (cond ((empty? xs) empty)
          ((pair? xs)  (if (p? (first xs))
                           (make-pair (first xs) (filter p? (rest xs)))
                           (filter p? (rest xs)))))))

; ----------------------------------------------------------------------
; Funktionen, die ihre Argument schrittweise konsumieren

; Konsumiert Argumente x,y in einem Schritt (eine Reduktion von apply_λ)
(: plus (number number -> number))
(define plus
  (lambda (x y)
    (+ x y)))

; Konsumiert Argumente x,y in zwei Schritten (zwei Reduktionen von apply_λ).
; Nach dem ersten Schritt ist nur Argument x festgelegt, Ergebnis ist eine
; Funktion, die das zweite Argument y erwartet.
(: add (number -> (number -> number)))
(define add
  (lambda (x)
    (lambda (y)
      (+ x y))))

(map (add 1)  (list 1 2 3 4 5 6 7 8 9 10))
(map (add 10) (list 1 2 3 4 5 6 7 8 9 10))


; Zweistellige Funktion f in ihre curried Variante überführen
(: curry ((%a %b -> %c) -> (%a -> (%b -> %c))))
(define curry
  (lambda (f)
    (lambda (x)
      (lambda (y)
        (f x y)))))

; Curried Funktion f in Variante überführen, die beide Argumente
; in einem Schritt konsumiert
(: uncurry ((%a -> (%b -> %c)) -> (%a %b -> %c)))
(define uncurry
  (lambda (f)
    (lambda (x y)  
      ((f x) y))))

(map ((curry +) 1)  (list 1 2 3 4 5 6 7 8 9 10))
;@\eval@ (list 2 3 4 5 6 7 8 9 10 11)
(map ((curry +) 10) (list 1 2 3 4 5 6 7 8 9 10))
;@\eval@ (list 11 12 13 14 15 16 17 18 19 20)
(filter ((curry =) 2) (list 1 2 3 4 5 4 3 2 1))
;@\eval@ (list 2 2)


; ----------------------------------------------------------------------

; Differenzenquotienten von f (mit Differenz h)
(: diffquot (real (real -> real) -> (real -> real)))
(define diffquot
  (lambda (h f)
    (lambda (x)
      (/ (- (f (+ x h)) (f x))
         h))))

; Berechne Differenzenquotienten mit Differenz h = 0.00001
; ((derive f) x) @\latexcode{$\equiv$}@ (f' x)
(: derive ((real -> real) -> (real -> real)))
(define derive
  ((curry diffquot) 0.00001))


; Beispielfunktion: f1(x) = x³ + 2x  
(: f1 (real -> real))
(define f1
  (lambda (x) (+ (* x x x)
                 (* 2 x))))

; Ableitung von f1(x)
; f1'(x) = 3x² + 2
(check-property
 (for-all ((x real))
   (expect-within ((derive f1) x)
                  (+ (* 3 x x) 2)
                  0.01)))
; Ableitung von f(x) = atan(x)
; f'(x) = 1 / (1 + x²)
(check-property
 (for-all ((x real))
   (expect-within ((derive atan) x)
                  (/ 1
                     (+ 1 (* x x)))
                  0.01)))

; ----------------------------------------------------------------------

; Charakteristische Funktion (M -> boolean) als Repräsentation 
; für eine Menge S @\latexcode{$\subseteq$}@ M
(define set-of
  (lambda (t)
    (signature (t -> boolean))))

; S42 = { x @\latexcode{$\in$}@ @\latexcode{$\mathbb{Z}$}@ | x > 42 }
(: S42 (set-of integer))
(define S42
  (lambda (x)
    (> x 42))) 

; Leere Menge @\latexcode{\emptyset}@
(: empty-set (set-of %a))
(define empty-set
  (lambda (x)
    #f))

; Ist Element x in der Menge S (x @\latexcode{\in}@ S)?
(: set-member? (%a (set-of %a) -> boolean))
(define set-member?
  (lambda (x S)
    (S x)))

; Element x in Menge S hinzufügen: S @\latexcode{\cup}@ {x}
(: set-insert (number (set-of number) -> (set-of number)))
(define set-insert
  (lambda (x S)
    (lambda (y)
      (or (= y x)
          (S y)))))

; Test: die leere Menge enthält kein Element
(check-property
 (for-all ((x integer))
   (boolean=? (set-member? x empty-set) #f)))

; Test: die Menge @\latexcode{\emptyset}@ @\latexcode{\cup}@ {x} enthält x
(check-property
 (for-all ((x integer))
   (set-member? x (set-insert x empty-set))))


; Konstruiere {1,2,3,4,5} = ((((@\latexcode{\emptyset}@ @\latexcode{\cup}@ {1}) @\latexcode{\cup}@ {2}) @\latexcode{\cup}@ {3}) @\latexcode{\cup}@ {4}) @\latexcode{\cup}@ {5})
(: 1-to-5 (set-of integer))
(define 1-to-5
  (set-insert 
   5
   (set-insert 
    4
    (set-insert 
     3
     (set-insert 
      2
      (set-insert 
       1 empty-set))))))



\end{filecontents*}
\begin{filecontents*}{StreamsUndMengen.rkt}
; Charakteristische Funktion (M -> boolean) als Repräsentation 
; für eine Menge S ⊆ M
(define set-of
  (lambda (t)
    (signature (t -> boolean))))

; S42 = { x ∈ ℤ | x > 42 }
(: S42 (set-of integer))
(define S42
  (lambda (x)
    (> x 42))) 

; Leere Menge ∅
(: empty-set (set-of %a))
(define empty-set
  (lambda (x)
    #f))

; Ist Element x in der Menge S (x ∈ S)?
(: set-member? (%a (set-of %a) -> boolean))
(define set-member?
  (lambda (x S)
    (S x)))

; Element x in Menge S hinzufügen: S ∪ {x}
(: set-insert (number (set-of number) -> (set-of number)))
(define set-insert
  (lambda (x S)
    (lambda (y)
      (or (= y x)
          (S y)))))

; Test: die leere Menge enthält kein Element
(check-property
 (for-all ((x integer))
   (boolean=? (set-member? x empty-set) #f)))

; Test: die Menge ∅ ∪ {x} enthält x
(check-property
 (for-all ((x integer))
   (set-member? x (set-insert x empty-set))))


; Konstruiere {1,2,3,4,5} = ((((∅ ∪ {1}) ∪ {2}) ∪ {3}) ∪ {4}) ∪ {5})
(: 1-to-5 (set-of integer))
(define 1-to-5
  (set-insert 5
              (set-insert 4
                          (set-insert 3
                                      (set-insert 2
                                                  (set-insert 1 empty-set))))))
;-----------------------------------------------------------------------------------------------------
; Konvertiere Liste xs in Menge
(: list->set ((list-of number) -> (set-of number)))
(define list->set
  (lambda (xs)
    (fold empty-set set-insert xs)))

; Beispiel: Konstruiere {1,2,...,10}
(: 1-to-10 (set-of integer))
(define 1-to-10
  (list->set (list 1 2 3 4 5 6 7 8 9 10)))

; Element x aus Menge S löschen
(: set-delete (number (set-of number) -> (set-of number)))
(define set-delete
  (lambda (x S)
    (lambda (y)
      (if (= y x)
          #f
          (S y)))))

; S @\latexcode{$\cup$}@ T
; x @\latexcode{$\in$}@ S @\latexcode{$\cup$}@ T <=> x @\latexcode{$\in$}@ S @\latexcode{$\lor$}@ x @\latexcode{$\in$}@ T
(: set-union ((set-of %a) (set-of %a) -> (set-of %a)))
(define set-union
  (lambda (S T)
    (lambda (x)
      (or (S x) (T x)))))

; S @\latexcode{$\cap$}@ T
; x @\latexcode{$\in$}@ S @\latexcode{$\cap$}@ T @\latexcode{$\Leftrightarrow$}@ x @\latexcode{$\in$}@ S @\latexcode{$\land$}@ x @\latexcode{$\in$}@ T
(: set-intersect ((set-of %a) (set-of %a) -> (set-of %a)))
(define set-intersect
  (lambda (S T)
    (lambda (x)
      (and (S x) (T x)))))

; S @\latexcode{$\setminus$}@ T
; x @\latexcode{$\in$}@ S @\latexcode{$\setminus$}@ T @\latexcode{$\Leftrightarrow$}@ x @\latexcode{$\in$}@ S @\latexcode{$\land$}@ x @\latexcode{$\not\in$}@ T
(: set-difference ((set-of %a) (set-of %a) -> (set-of %a)))
(define set-difference
  (lambda (S T)
    (lambda (x)
      (and (S x) (not (T x))))))

; ----------------------------------------------------------------------

; Promise, ein Wert des Vertrags t zu liefern (0-stellig Prozedur)
(define promise
  (lambda (t)
    (signature (-> t))))

; Verzögerte Auswertung (delay)
;
; Variante 1: 
; (delay e) ≣ (lambda () e)
;
; Variante 2 (nutzt selbstdefinierte Scheme-Syntax-Regel, verfügbar ab
; Sprachebene "DMdA - fortgeschritten"):
;
; (define-syntax delay
;   (syntax-rules ()
;     ((_ e)
;      (lambda () e))))

; Erzwungene Auswertung
(: force ((promise %a) -> %a))
(define force
  (lambda (p)
    (p)))

; Beispiel: 
; Promise (werde 41+1 berechnen, falls gefordert)
(: will-evaluate-to-42 (promise natural))
(define will-evaluate-to-42
  (lambda ()     ; oder äquivalent mit Variante 2: (delay (+ 1 41))
    (+ 41 1)))   

; Verzögerte Ausführung...
will-evaluate-to-42
; ... und erzwungene Ausführung
(force will-evaluate-to-42)

; Polymorphe Paare (isomorph zu `pair')
(: make-cons (%a %b -> (cons-of %a %b)))
(: head ((cons-of %a %b) -> %a))
(: tail ((cons-of %a %b) -> %b))
(define-record-procedures-parametric cons cons-of
  make-cons 
  cons?
  (head
   tail))

; Ein Stream besteht aus
; - einem ersten Element (head)
; - einem Promise, den Rest des Streams generieren zu können (tail)
(define stream-of
  (lambda (t)
    (signature (cons-of t (promise (stream-of t))))))

; Beispiel:
; Stream mit Zahlen ab n erzeugen
(: from (number -> (stream-of number)))
(define from
  (lambda (n)
    (make-cons n (lambda () (from (+ n 1))))))
  
; Beispiel (Stream → Liste):
; Erste n Elemente des Streams str in eine Liste extrahieren
(: stream-take (natural (stream-of %a) -> (list-of %a)))

(check-expect (stream-take 5 (from 1)) (list 1 2 3 4 5))
(check-expect (stream-take 0 (from 1)) empty)

(define stream-take
  (lambda (n str)
    (if (= n 0)
        empty
        (make-pair (head str) 
                   (stream-take (- n 1) (force (tail str)))))))

; Beispiel (Stream → Stream):
; Filtere Stream str bzgl. Prädikat p?
(: stream-filter ((%a -> boolean) (stream-of %a) -> (stream-of %a)))

(check-expect (stream-take 10 
                           (stream-filter (lambda (x) (= (remainder x 2) 0)) 
                                          (from 1)))
              (list 2 4 6 8 10 12 14 16 18 20))

(define stream-filter
  (lambda (p? str)
    (if (p? (head str))
        (make-cons (head str)
                   (lambda () (stream-filter p? (force (tail str)))))
        (stream-filter p? (force (tail str))))))










; Beispiel (Streams → Stream):
; Erzeuge neuen Stream durch die Anwendung von f
; auf die Heads der Streams str1, str2
(: stream-zipWith ((%a %b -> %c) (stream-of %a) (stream-of %b) -> (stream-of %c)))
(define stream-zipWith
  (lambda (f str1 str2)
    (make-cons (f (head str1) (head str2))
               (lambda () (stream-zipWith f 
                                          (force (tail str1))
                                          (force (tail str2)))))))


; Die unendliche Folge der Fibonacci-Zahlen 1, 1, 2, 3, 5, ...
(: fibs (stream-of natural)) 
(check-expect (stream-take 10 fibs) (list 1 1 2 3 5 8 13 21 34 55))
(define fibs
  (make-cons 
   1
   (lambda ()
     (make-cons 
      1 
      (lambda ()
        (stream-zipWith +
                        fibs
                        (force (tail fibs))))))))
 
\end{filecontents*}
\begin{filecontents*}{Animationen-und-HOP-Typ2.rkt}
; Erstellung von Animationen mit Teachpack "universe"
; (1) Zähler

(: scene (natural -> image))
(define scene
  (lambda (t)
    (text (number->string t) 100 "red")))
(big-bang 0
          (on-tick (lambda (t) (+ t 1)))
          (to-draw scene 200 100))




; Erstellung von Animationen mit Teachpack "universe"
; (2) X-Wing Fighter + Scrolling Death Star

(define death-star 
@\includegraphics[width=\textwidth]{death-star}@)
(define x-wing @\includegraphics{xwing}@)


; Erhalte einfachen Scrolling-Effekt durch Herausschneiden von Teilbildern
; aus dem Bild der Todessternoberfläche
; (zu crop und overlay: siehe Dokumentation des Teachpack "image2")
(: scroll-death-star (natural -> image))
(define scroll-death-star
  (lambda (t)
    (overlay x-wing
             (crop (modulo (* 8 t) 200) 0 400 440 death-star))))
(big-bang 0
          (on-tick (lambda (t) (+ t 1)))
          (to-draw scroll-death-star))


; ----------------------------------------------------------------------

; Komposition der Prozeduren f, g (Mathematik: f ∘ g, "f nach g") 
(: compose ((%b -> %c) (%a -> %b) -> (%a -> %c)))
(define compose
  (lambda (f g)
    (lambda (x)
      (f (g x)))))



; Greife auf das zweite Element der Liste xs zu
(: second ((list-of %a) -> %a))
(check-expect (second (list 1 2 3)) 2)
(check-expect (second (string->strings-list "SCF")) "C")
(define second
  (lambda (xs)
    ((compose first rest) xs)))


; Greife auf das dritte Element der Liste xs zu
(: third ((list-of %a) -> %a))
(check-expect (third  (list 1 2 3)) 3)
(check-expect (third (string->strings-list "SCF")) "F")
(define third
  (lambda (xs)
    ((compose first (compose rest rest)) xs)))


; Komponiere f n-mal mit sich selbst (fⁿ, Funktions-Exponentation)
(: repeat (natural (%a -> %a) -> (%a -> %a)))
(define repeat
  (lambda (n f)
    (cond ((= n 0) (lambda (x) x))                      ; ← Identität (id)
          ((> n 0) (compose (repeat (- n 1) f) f))))) 


; Greife auf das n-te Elemente der Liste xs zu
(: nth (natural (list %a) -> %a))
(check-expect (nth 2 (string->strings-list "SCF")) "C")
(check-expect (nth 1 (string->strings-list "SCF")) "S")
(define nth                   
  (lambda (n xs)                 
    ((compose first (repeat (- n 1) rest)) xs)))


; ----------------------------------------------------------------------
; Funktionen, die ihre Argument schrittweise konsumieren

; Konsumiert Argumente x,y in einem Schritt (eine Reduktion von apply_λ)
(: plus (number number -> number))
(define plus
  (lambda (x y)
    (+ x y)))

; Konsumiert Argumente x,y in zwei Schritten (zwei Reduktionen von apply_λ).
; Nach dem ersten Schritt ist nur Argument x festgelegt, Ergebnis ist eine
; Funktion, die das zweite Argument y erwartet.
(: add (number -> (number -> number)))
(define add
  (lambda (x)
    (lambda (y)
      (+ x y))))

(map (add 1)  (list 1 2 3 4 5 6 7 8 9 10))
(map (add 10) (list 1 2 3 4 5 6 7 8 9 10))




\end{filecontents*}
\begin{filecontents*}{./Grundlagen.rkt}
;; Die ersten drei Zeilen dieser Datei wurden von DrRacket eingefügt. Sie enthalten Metadaten
;; über die Sprachebene dieser Datei in einer Form, die DrRacket verarbeiten kann.
#reader(lib "DMdA-beginner-reader.ss" "deinprogramm")((modname Grundlagen) (read-case-sensitive #f) (teachpacks ()) (deinprogramm-settings #(#f write repeating-decimal #f #t none explicit #f ())))
; Achtung: Arithmetik mit Fliesskommazahlen (real)
unterliegt Rundung!
(+ 0.7
(- (/ 1/2 0.25)
(/ 0.6 0.3)))

(- (+ 0.7
(/ 1/2 0.25))
(/ 0.6 0.3))

; Arithmetik mit rationalen Zahlen (rational) ist exakt
(- (+ 7/10
(/ 1/2 1/4))
(/ 6/10 3/10))

(define absoluter-nullpunkt -273.15)
(define pi 3.141592653)
(define Gruendungsjahr-SC-Freiburg 1904)
(define top-level-domain-germany "de")
(define minutes-in-a-day (* 24 60))
(define vorwahl-tuebingen (sqrt 1/2))

; Abstraktion: Ausdruck mit "Loch" @$\odot$@
(lambda (@$\odot$@) (* @$\odot$@ (* 155 minutes-in-a-day)))


; Zuwachs der Weltbevoelkerung innerhalb von days Tagen
(define population-growth-in-days
(lambda (days) (* days (* 155 minutes-in-a-day))))

(population-growth-in-days 7)
\end{filecontents*}
\begin{filecontents*}{Uhr.rkt}
;; Die ersten drei Zeilen dieser Datei wurden von DrRacket eingefügt. Sie enthalten Metadaten
;; über die Sprachebene dieser Datei in einer Form, die DrRacket verarbeiten kann.
#reader(lib "DMdA-beginner-reader.ss" "deinprogramm")((modname Uhr) (read-case-sensitive #f) (teachpacks ()) (deinprogramm-settings #(#f write repeating-decimal #f #t none explicit #f ())))
; Grad, die Minutenzeiger pro Minute zuruecklegt
(define degrees-per-minute 360/60)

; Grad, die Stundenzeiger pro voller Stunde zuruecklegt
(define degrees-per-hour 360/12)

; Zeichne Ziffernblatt zur Stunde h und Minute m
(: draw-clock (natural natural -> image))
(check-expect (draw-clock 4 15) (draw-clock 16 15))
(define draw-clock
(lambda (h m)
(clock-face (position-hour-hand h m)
(position-minute-hand m))))

; Winkel (in Grad), den Minutenzeiger zur Minute m einnimmt
(: position-minute-hand (natural -> rational))
(check-expect (position-minute-hand 15) 90)
(check-expect (position-minute-hand 45) 270)
(define position-minute-hand
(lambda (m)
(* m degrees-per-minute)))

; Winkel (in Grad), den Stundenzeiger zur Stunde h einnimmt
(: position-hour-hand (natural natural -> rational))
(check-expect (position-hour-hand 3 0) 90)
(check-expect (position-hour-hand 18 30) 195)
(define position-hour-hand
(lambda (h m)
(+ (* (modulo h 12) degrees-per-hour)
; h mod 12 in {0,1,...,11}
(* (/ m 60) degrees-per-hour))))

; Zeichne Ziffernblatt mit Minutenzeiger um dm und
; Stundenzeiger um dh Grad gedreht
(: clock-face (rational rational -> image))
(define clock-face
(lambda (dh dm)
(clear-pinhole
(overlay/pinhole
(circle 50 "outline" "black")
(rotate (* -1 dh) (put-pinhole 0 35 (line 0 35 "red")))
(rotate (* -1 dm) (put-pinhole 0 45 (line 0 45 "blue")))))))

\end{filecontents*}
\begin{filecontents*}{myif.rkt}
; Bedingte Auswertung von e1 oder e2 (abhaengig von t1)
(check-expect (my-if (= 42 42) "Yes!" "No!") "Yes!")
(check-expect (my-if (odd? 42) "Yes!" "No!") "No!")
(define my-if
  (lambda (t1 e1 e2)
    (cond (t1 e1)
          (else e2))))

; Sichere Division x/y, auch fuer y = 0
(: safe-/ (real real -> real))
(define safe-/
  (lambda (x y)
    (my-if (= y 0)     ; <-- Funktion my-if wertet ihre Argumente
           x           ;     vor der Applikation aus: (/ x y) wird
           (/ x y))))  ;     in *jedem* Fall reduziert. :-(


(safe-/ 42 0)          ; Fuehrt zu Fehlemeldung "division by zero"
                       ; (Reduktion mit Stepper durchfuehren)
\end{filecontents*}
\begin{filecontents*}{Abs.rkt}
(: my-abs (real -> real))
(check-within (my-abs -4.2) 4.2 0.001)   ; Wichtig:
(check-within (my-abs 4.2) 4.2 0.001)    ; Tesfaelle decken alle Zweige
(check-within (my-abs 0) 0 0.001)        ; der conditional expression an
(define my-abs
  (lambda (x)
    (cond ((< x 0) (- x))
          ((> x 0) x    )
          (else    0    ))))
\end{filecontents*}
\begin{filecontents*}{andor.rkt}
(and #t #f)  ; @\eval@ #f   (Mathematik: Konjunktion)
(or #t #f)   ; @\eval@ #t   (Mathematik: Disjunktion)
; Kennzeichen am/pm fuer Stunde h
(: am/pm (natural -> (one-of "am" "pm" "???")))
(check-expect (am/pm 10) "am")
(check-expect (am/pm 13) "pm")
(check-expect (am/pm 25) "???")
(define am/pm
  (lambda (h)
    (cond ((and (>= h 0) (< h 12))  "am")
          ((and (>= h 12) (< h 24)) "pm")
          (else "???"))))
\end{filecontents*}
\begin{filecontents*}{records.rkt}
; Ein Charakter (character) besteht aus
; - Name (name)
; - Jedi-Status (jedi?)
; - Stärke der Macht (force)
(: make-character (string boolean real -> character))
(: character? (any -> boolean))
(: character-name (character -> string))
(: character-jedi? (character -> boolean))
(: character-force (character -> real))
(define-record-procedures character
  make-character
  character?
  (character-name
   character-jedi?
   character-force))


; Definiere verschiedene Charaktere des Star Wars Universums
(define luke
  (make-character "Luke Skywalker" #f 25))
(define r2d2
  (make-character "R2D2" #f 0))
(define dooku
  (make-character "Count Dooku" #f 80))
(define yoda
  (make-character "Yoda" #t 85))
\end{filecontents*}
\begin{filecontents*}{checkproperty.rkt}
(check-property 
 (for-all ((n string)
           (j boolean)
           (f real))
   (expect (character-name (make-character n j f)) n)))

(check-property 
 (for-all ((n string)
           (j boolean)
           (f real))
   (expect (character-jedi? (make-character n j f)) j)))

(check-property 
 (for-all ((n string)
           (j boolean)
           (f real))
   (expect-within (character-force (make-character n j f)) f 0.001)))
\end{filecontents*}
\begin{filecontents*}{forall.rkt}
; Für alle natürlichen Zahlen x1,x2 gilt: x1 + x2 @$\textcolor{kommentar}{\geq}$@ max(x1,x2)
(check-property
 (for-all ((x1 natural)
           (x2 natural))
   (>= (+ x1 x2) (max x1 x2))))
\end{filecontents*}
\begin{filecontents*}{konstruktor.rkt}
; Könnte Charakter c ein Sith sein?
(: sith? (character -> boolean))
(check-expect (sith? yoda) #f)
(check-expect (sith? r2d2) #f)
(define sith?
  (lambda (c)
    (and (not (character-jedi? c))
         (> (character-force c) 0))))


; Bilde den Charakter c zum Jedi aus (sofern c überhaupt Macht besitzt)
(: train-jedi (character -> character))

(check-expect (train-jedi luke) (make-character "Luke Skywalker" #t 50))
(check-expect (train-jedi r2d2) r2d2)

(define train-jedi
  (lambda (c)
    (make-character (character-name c) 
                    (> (character-force c) 0)
                    (* 2 (character-force c)))))
\end{filecontents*}
\begin{filecontents*}{latandlong.rkt}
; Ist x ein gültiger Breitengrad 
; zwischen Südpol (-90@\latexcode{$^{\circ}$}@) und Nordpol (90@\latexcode{$^{\circ}$}@)?
(: latitude? (real -> boolean))
(check-expect (latitude? 78) #t)
(check-expect (latitude? -92) #f)
(define latitude?
  (lambda (x)
    (within? -90 x 90)))
; Ist x ein gültiger Längengrad westlich (bis -180@\latexcode{$^{\circ}$}@) 
; bzw. östlich (bis 180@\latexcode{$^{\circ}$}@) des Meridians?
(: longitude? (real -> boolean))
(check-expect (longitude? 0) #t)
(check-expect (longitude? 200) #f)
(define longitude?
  (lambda (x)
    (within? -180 x 180)))
; Signaturen für Breiten-/Längengrade basierend auf
; den obigen Prädikaten
(define latitude
  (signature (predicate latitude?)))
(define longitude
  (signature (predicate longitude?)))
\end{filecontents*}
\begin{filecontents*}{oneoftopredicate.rkt}
(: f ((one-of 0 1 2 ) -> natural))
(define f
  (lambda (x)
    x))
; And then the "The Great one-of Extinction" of 2015 occurred @\includegraphics[scale=0.5]{kaboon}@
(: g ((predicate 
       (lambda (x) (or (= x 0) (= x 1) (= x 2)))) -> natural))
(define g
  (lambda (x)
    x))
\end{filecontents*}
\begin{filecontents*}{geocoding.rkt}
(define geocoder-response
  (signature (mixed geocode geocode-error)))

(: sand13 geocoder-response)
(define sand13
  (geocoder "Sand 13, Tübingen"))

(geocode-address sand13)
(geocode-type sand13)
(location-lat (geocode-loc sand13))
(location-lng (geocode-loc sand13))
(geocode-accuracy sand13)
  

(: lady-liberty geocoder-response)
(define lady-liberty
  (geocoder "Statue of Liberty"))

(: alb geocoder-response)
(define alb
  (geocoder "Schwäbische Alb"))

(: A81 geocoder-response)
(define A81
  (geocoder "A81, Germany"))
\end{filecontents*}
\begin{filecontents*}{hemisphere.rkt}
; (Breitengrad < 0@\latexcode{$^{\circ}$}@?)
(: southern-hemisphere? (string -> boolean))

(check-expect (southern-hemisphere? "Cape Town") #t)
(check-expect (southern-hemisphere? "Tübingen") #f)
(check-error  (southern-hemisphere? "Mos Eisley") "Unknown location")

(define southern-hemisphere?
  (lambda (r)
    (let ((gc (geocoder r)))
      (cond ((geocode? gc)  
             (< (location-lat (geocode-loc gc)) 0))
            ((geocode-error? gc) 
             (violation "Unknown location"))))))
\end{filecontents*}
\begin{filecontents*}{distance.rkt}
;Abstand zweier geographischer Positionen l1, l2 auf der Erdkugel in km (lat, lng jeweils in Radian):
;dist(l1,l2) =
;   Erdradius in km *
;   acos(cos(l1.lat) * cos(l1.lng) * cos(l2.lat) * cos(l2.lng) + 
;        cos(l1.lat) * sin(l1.lng) * cos(l2.lat) * sin(l2.lng) +
;        sin(l1.lat) * sin(l2.lat))
; @\latexcode{$\pi$}@
(define pi 3.141592653589793)

; Konvertiere Grad d in Radian (@\latexcode{$\pi$}@ = @\latexcode{$180^{\circ}$}@)
(: radians (real -> real))
(check-within (radians 180) pi 0.001)
(check-within (radians -90) (* -1/2 pi) 0.001)
(define radians
  (lambda (d)
    (* d (/ pi 180))))


; Abstand zweier Orte o1, o2 auf Erdkugel (in km)
; [Wrapper]
(: distance (string string -> real))
(check-within (distance "Tübingen" "Freiburg") (distance "Freiburg" "Tübingen") 0.001)
(define distance
  (lambda (o1 o2)
    (let ((dist (lambda (l1 l2)             ; Abstand zweier Positionen l1, l2 (in km)  [Worker]
                  (let ((earth-radius 6378) ; Erdradius (in km)                  
                        (lat1 (radians (location-lat l1)))
                        (lng1 (radians (location-lng l1)))
                        (lat2 (radians (location-lat l2)))
                        (lng2 (radians (location-lng l2))))
                    (* earth-radius
                       (acos (+ (* (cos lat1) (cos lng1) (cos lat2) (cos lng2))
                                (* (cos lat1) (sin lng1) (cos lat2) (sin lng2))
                                (* (sin lat1) (sin lat2))))))))
          (gc1 (geocoder o1))
          (gc2 (geocoder o2)))
      (if (and (geocode? gc1)
               (geocode? gc2))
          (dist (geocode-loc gc1) (geocode-loc gc2))
          (violation "Unknown location(s)")))))

; ... einmal quer durch die schöne Republik
(distance "Konstanz" "Rostock")
\end{filecontents*}
\begin{filecontents*}{makepair.rkt}
; Ein paar aus natürlichen Zahlen
; FIFA WM 2014
(: deutschland-vs-brasilien (pair-of natural natural))
(define deutschland-vs-brasilien
  (make-pair 7 1)) 

; Ein Paar aus einer reellen Zahl (Messwert) 
; und einer Zeichenkette (Einheit)
(: measurement (pair-of real string))
(define measurement
  (make-pair 36.9 "@$^{\textcolor{string}{\circ}}$@C"))


; "Liste" der Zahlen 1,2,3,4
(define nested
  (make-pair 1
             (make-pair 2
                        (make-pair 3
                                   4))))

; Extrahiere das dritte Element der Liste (hier: 3)
(first (rest (rest nested)))
\end{filecontents*}
\begin{filecontents*}{lists.rkt}
; Noch einmal (jetzt mit Signatur): Liste der natürlichen Zahlen 1,2,3,4
(: one-to-four (list-of natural))
(define one-to-four
  (make-pair 1
             (make-pair 2
                        (make-pair 3
                                   (make-pair 4
                                              empty)))))


; Eine Liste, deren Elemente natürliche Zahlen oder Strings sind
(: abstiegskampf (list-of (mixed number string)))
(define abstiegskampf
  (make-pair "SCF"
             (make-pair 96
                        (make-pair "SCP"
                                   (make-pair "VfB" empty)))))


\end{filecontents*}
\begin{filecontents*}{nestedlist.rkt}
; Geschachtelte Listen
(: jedis-and-siths (list-of (list-of string)))
(define jedis-and-siths
  (MAKE-PAIR (make-pair "Yoda"
                        (make-pair "Obi-Wan" empty))
             (MAKE-PAIR (make-pair "Dooku"
                                   (make-pair "Vader" empty))
                        empty)))

; Navigation in geschachtelten Listen
(check-expect (first (first jedis-and-siths)) "Yoda")
(check-expect (first (rest (first (rest jedis-and-siths)))) "Vader")
\end{filecontents*}
\begin{filecontents*}{recursivelistlength.rkt}
; Länge der Liste xs
(: list-length ((list-of %a) -> natural))

(check-expect (list-length empty) 0)
(check-expect (list-length (list 1 1 3 8)) 4)
(check-expect (list-length jedis-and-siths) 2)    ; nicht 4!

(define list-length
  (lambda (xs)
    (cond ((empty? xs) 0)
          ((pair? xs) (+ 1 
                         (list-length (rest xs)))))))
\end{filecontents*}
\begin{filecontents*}{cat.rkt}
; Füge Listen xs, ys (in dieser Reihenfolge) zusammen
(: cat ((list-of %a) (list-of %a) -> (list-of %a)))

(check-expect (cat (list 1 2) (list 3 4)) (list 1 2 3 4))
(check-expect (cat one-to-four empty) one-to-four)
(check-expect (cat empty one-to-four) one-to-four)

(define cat
  (lambda (xs ys)
    (cond ((empty? xs) 
           ys)
            ((pair? xs)
             (make-pair (first xs) ; <- cat dennoch param. polymorph
                        (cat (rest xs) ys))))))

; Hinweis: Verfügbar als eingebaute Funktion `append'
\end{filecontents*}
\begin{filecontents*}{bluescreen.rkt}
(define yoda @\includegraphics[scale=0.5]{Yoda}@)
(define dagobah @\includegraphics[scale =0.5]{dagobah}@)
;_________________________________________
;Zugriff auf die Liste der Bildpunkte (Pixel) eines Bildes: 
 
;(: image->color-list (image -> (list-of rgb-color)))  
;(: color-list->bitmap ((list-of rgb-color) natural natural -> image))
 
;Breite/Höhe eines Bildes in Pixeln:
 
;(: image-width (image -> natural))
; (: image-height (image -> natural))
 
; Eine Farbe (rgb-color) besteht aus ihrem
; - Rot-Anteil 0..255 (red)
; - Grün-Anteil 0..255 (green)
; - Blau-Anteil 0..255 (blue)

 @\includegraphics[scale=0.5]{pixel}@
 
; (define-record-procedures rgb-color
;   make-color
;   color?
;   (color-red color-green color-blue))
;________________________________________

; Signatur für color-Records nicht in image2.rkt eingebaut.  Roll our own...
(define rgb-color
  (signature (predicate color?)))


; Ist Farbe c bläulich?
(: bluish? (rgb-color -> boolean))
(define bluish?
  (lambda (c)
    (< (/ (+ (color-red c) (color-green c) (color-blue c))
          3)
       (color-blue c))))

; Worker:
; Pixel aus Hintergrund bg scheint durch, wenn der
; entsprechende Pixel im Vordergrund fg bläulich ist.
; Arbeite die Pixellisten von fg und bg synchron ab
; Annahme: fg und bg haben identische Länge!
(: bluescreen ((list-of rgb-color) (list-of rgb-color) -> (list-of rgb-color)))
(define bluescreen
  (lambda (fg bg)
    (cond ((empty? fg)
           empty)
          ((pair? fg)
           (make-pair 
            (if (bluish? (first fg))
                (first bg)
                (first fg))
            (bluescreen (rest fg) (rest bg)))))))

; Wrapper:
; Mische Vordergrund fg und Hintergrund bg nach Bluescreen-Verfahren
(: mix (image image -> image))
(define mix
  (lambda (fg bg)
    (let ((fg-h (image-height fg))
          (fg-w (image-width fg))
          (bg-h (image-height bg))
          (bg-w (image-width bg)))
      (if (and (= fg-h bg-h)
               (= fg-w bg-w))
          (color-list->bitmap
           (bluescreen (image->color-list fg)
                       (image->color-list bg))
           fg-w
           fg-h)
          (violation "Dimensionen von Vorder-/Hintergrund verschieden")))))

; Yoda vor seine Hüte auf Dagobah setzen
(mix yoda dagobah) @\eval \ \includegraphics[scale=0.5]{Yoda_finished}@
\end{filecontents*}
\begin{filecontents*}{notquiteall.rkt}
; Eigenschaft nur auswerten, wenn n > 0 (==>)
(check-property
 (for-all ((n natural))
   (==> (> n 0)   
        (= (succ (pred n)) n))))
\end{filecontents*}
\begin{filecontents*}{Fakultaet.rkt}
; Berechne n!
(: factorial (natural -> natural))
(check-expect (factorial 0) 1)
(check-expect (factorial 3) 6)
(check-expect (factorial 10) 3628800)

(define factorial
  (lambda (n)
    (cond ((= n 0) 1)
          ((> n 0) (* n (factorial (- n 1)))))))
\end{filecontents*}
\begin{filecontents*}{FehlerRekursiv.rkt}
; Fehlerhaft: kein Fortschritt im rekursiven Aufruf
; => potentiell "unendliche" Reduktion
(define unfactorial
  (lambda (n)
    (cond ((= n 0) 1)
          ((> n 0) (* n (unfactorial n))))))

; Fehlerhaft: kein definierter Abbruch der Rekursion
; => Abbruch der Reduktion bei n = 0 ("cond: alle Tests ergaben #f")
(define not-factorial
  (lambda (n)
    (cond ((> n 0) (* n (not-factorial (- n 1)))))))
\end{filecontents*}
\begin{filecontents*}{Endrekursion.rkt}
;; Die ersten drei Zeilen dieser Datei wurden von DrRacket eingefügt. Sie enthalten Metadaten
;; über die Sprachebene dieser Datei in einer Form, die DrRacket verarbeiten kann.
#reader(lib "DMdA-vanilla-reader.ss" "deinprogramm")((modname definitions) (read-case-sensitive #f) (teachpacks ((lib "image2.ss" "teachpack" "deinprogramm"))) (deinprogramm-settings #(#f write repeating-decimal #f #t none explicit #f ((lib "image2.ss" "teachpack" "deinprogramm")))))
; Messung der Reduktionszeit (in Millisekunden) 
; für Ausdruck <e> mittels (time <e>)
(require racket/base)  


; Erinnerung: Generiere Liste von s bis e
(: from-to (natural natural -> (list-of natural)))
(define from-to
  (lambda (s e)
    (cond ((> s e) empty)
          (else (make-pair s (from-to (+ s 1) e)))))) 


; Erinnerung: konkateniere Listen xs und ys
; (Länge von xs bestimmt Anzahl rekursiver Aufrufe)

(: cat ((list-of %a) (list-of %a) -> (list-of %a)))  
(define cat
  (lambda (xs ys)
    (cond ((empty? xs) 
           ys)
          ((pair? xs)
           (make-pair (first xs)
                      (cat (rest xs) ys))))))

; ------------------------------------------------------------------------------

; Berechne n! (nicht endrekursiv)
(: factorial (natural -> natural))
(define factorial
  (lambda (n)
    (cond ((= n 0) 1)
          ((> n 0) (* n (factorial (- n 1)))))))
;                  ^^^^^^^^^^^^^^^^^^^^^^^^
;        nicht-leerer Kontext (* n ▢ ) des rekursiven Aufrufes
           

(factorial 5)


; Berechne n!
; Wrapper
(: fac (natural -> natural))
(check-expect (fac 0) 1)
(check-expect (fac 3) 6)
(define fac
  (lambda (n)
    (fac-worker n 1)))

; Berechne n! (mit Zwischenergebnis/Akkumulator acc), endrekursiv
; Worker
(define fac-worker
  (lambda (n acc)
    (cond ((= n 0) acc)
          ((> n 0) (fac-worker (- n 1) (* n acc))))))
;                  ^^^^^^^^^^^^^^^^^^^^^^^^^^^^^^
;                  endrekursiver Aufruf (tail call)
;                  siehe Syntaxprüfung (rosa Pfeile)

(fac 5)



; Liste xs umdrehen
; Aufwand: 1/2 × n × (n + 1) Aufrufe von make-pair wenn xs die Länge n hat
(: rev ((list-of %a) -> (list-of %a)))

(check-expect (rev empty) empty)
(check-expect (rev (list 1 2 3 4)) (list 4 3 2 1))

(define rev
  (lambda (xs)
    (cond ((empty? xs) empty)
          ((pair? xs) 
           (cat (rev (rest xs)) (list (first xs)))))))


; Liste xs umdrehen (effiziente Variante)
; Wrapper
(: backwards ((list-of %a) -> (list-of %a)))

(check-expect (backwards empty) empty)
(check-expect (backwards (list 1 2 3 4)) (list 4 3 2 1))

(define backwards
  (lambda (xs)
    
    ; Liste xs umdrehen (mit Akkumulator acc, endrekursiv)
    ; Worker
    ; Aufwand: n Aufrufe von make-pair, wenn xs die Länge n hat
    (letrec ((backwards-worker
              (lambda (xs acc)
                (cond ((empty? xs) acc)
                      ((pair? xs) 
                       (backwards-worker (rest xs) (make-pair (first xs) acc)))))))
      
      (backwards-worker xs empty))))

; NB:
; Benutzt letrec, um die rekursive Worker-Definition lediglich
; lokal zu definieren.


; Performance-Test: Umdrehen einer Liste von n = 1000 Elementen
;
; (1) Führt 1/2 × n × (n + 1) = 500500 Aufrufe von make-pair durch:
(time (rev (from-to 1 1000)))

; (2) Führt n = 1000 Aufrufe von make-pair durch:
(time (backwards (from-to 1 1000)))


; NB:
; backwards ist als Funktion `reverse' in DrRacket verfügbar
\end{filecontents*}
\begin{filecontents*}{HigherOrderProcedures.rkt}
; Ist x ein Element in Liste xs?
(: elem? (number (list-of number) -> boolean))
(define elem?
  (lambda (x xs)
    (cond ((empty? xs) #f)
          ((pair? xs)
           (if (= x (first xs))
               #t
               (elem? x (rest xs)))))))
; Gehört der Film f zu den Prequels (Episode I-III, vor "A New Hope")
(: is-prequel? (film -> boolean))
(define is-prequel?
  (lambda (f)
    (let ((a-new-hope (sw-film 1)))
      (< (film-episode f)
         (film-episode a-new-hope)))))

; Erwähnt der Film f den Todesstern in seinem Opening Crawl?
(: mentions-death-star? (film -> boolean))
(define mentions-death-star?
  (lambda (f)
    ; (string-contains s1 s2) liefert #f oder die Position (0..) des Teilstrings s2 in s1
    (natural? (string-contains (film-opening-crawl f) "DEATH STAR"))))
    
; Kommt Obi-Wan Kenobi im Film f vor?
(: features-kenobi? (film -> boolean))
(define features-kenobi?
  (lambda (f)
    (elem? 10 (film-characters f))))



; Filtere Liste xs nach Elementen, die Prädikat p? erfüllen
; (Prozedur höherer Ordnung: Parameter p? ist selbst eine Funktion)
(: filter ((%a -> boolean) (list-of %a) -> (list-of %a)))
(define filter
  (lambda (p? xs)
    (cond ((empty? xs) empty)
          ((pair? xs)  (if (p? (first xs))
                           (make-pair (first xs)
                                      (filter p? (rest xs)))  
                           (filter p? (rest xs)))))))


; Wird filter mit verschiedenen Prädikaten parameterisiert,
; erhalten wir spezifische Filter:

; NB: Auswertung benötigt eine kurze Zeit durch 
;     Zugriff auf das Star Wars Web-API
(define prequels-films
  (filter is-prequel? (sw-films)))

(define death-star-films
  (filter mentions-death-star? (sw-films)))

(define kenobi-films
  (filter features-kenobi? (sw-films)))



; Falte Liste xs bzgl. z und c.  Spine-Transformation:
; - empty     ~~> z
; - make-pair ~~> c
;
;   +--+--+                             
;   |  |  |                             
;   +--+--+                                 c
;   /     \                                / \ 
; x1      +--+--+                        x1   c
;         |  |  |                            / \  
;         +--+--+   ~~(foldr z c xs)~~>    x2   .
;         /     \                                \
;       x2       .                                c
;                 \                              / \  
;                 +--+--+                      xn   z                   |
;                 |  |  |                         
;                 +--+--+                           
;                 /     \             (c x1 (c x2 (... (c xn z)...)))  
;                xn     empty                        
;
(: foldr (%b (%a %b -> %b) (list-of %a) -> %b))
(define foldr
  (lambda (z c xs)
    (cond ((empty? xs) z)
          ((pair? xs)  (c (first xs) (foldr z c (rest xs)))))))
 

; Listenreduktion via foldr: Summe der Liste xs
(: my-sum ((list-of number) -> number))
(define my-sum
  (lambda (xs)
    (foldr 0 + xs)))

; Listenreduktion via foldr: Produkt der Liste xs
(: my-product ((list-of number) -> number))
(define my-product
  (lambda (xs)
    (foldr 1 * xs)))

; Listenreduktion via foldr: Maximum der Liste xs
(: my-maximum ((list-of number) -> number))
(define my-maximum
  (lambda (xs)
    (foldr -inf.0 max xs)))

; Identität (auf Listen), implementiert via foldr
(: my-id ((list-of %a) -> (list-of %a)))
(define my-id
  (lambda (xs)
    (foldr empty make-pair xs)))
    
; Reimplementation von append via foldr
(: my-append ((list-of %a) (list-of %a) -> (list-of %a)))
(define my-append
  (lambda (xs ys)
    (foldr ys make-pair xs)))

; Reimplementation von map via foldr
(: my-map ((%a -> %b) (list-of %a) -> (list-of %b)))
(define my-map
  (lambda (f xs)
    (foldr empty                                
           (lambda (y ys) (make-pair (f y) ys))  
           xs)))

; Reimplementation von reverse via foldr
(: my-reverse ((list-of %a) -> (list-of %a)))
(define my-reverse
  (lambda (xs)
    (foldr empty                                
           (lambda (y ys) (append ys (list y)))
           xs)))

; Listenreduktion via foldr: Länge der Liste xs
(: my-length ((list-of %a) -> natural))
(define my-length
  (lambda (xs)
    (foldr 0 (lambda (x l) (+ 1 l)) xs)))

; Reimplementation von filter mittels foldr
(: my-filter ((%a -> boolean) (list-of %a) -> (list-of %a)))
(define my-filter
  (lambda (p? xs)
    (foldr empty 
           (lambda (y ys) (if (p? y)
                              (make-pair y ys)
                              ys)) 
           xs)))




\end{filecontents*}