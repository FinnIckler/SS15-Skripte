Das \emph{$\lambda$-Kalkül} ist eine Notation für beliebige (berechenbare) Funktionen : Entwickelt in den 1930er Jahren von Alonzo Church (*1903  \dagger1995) als neue Grundlage der Mathematik. Seither verwendet als Theoretischer Unterbau von Programmiersprachen.\\
\emph{Syntax des $\lambda$-Kalküls}\\
Die Menge der Ausdrücke (expressions) E des $\lambda$-Kalküls ist induktiv definiert. Sei V eine unendliche Menge an Variablennamen
\begin{enumerate}[-]
\item $\forall v \in V : v \in E$ \hfill $[$Variablen$]$
\item $\forall e_1,e_2 \in E:$ \lstinline[mathescape]|($\underset{\text{Funktion}}{e_1}\ \underset{\text{Argument}}{e_2})$| $\in E$ \hfill$[$Applikation$]$
\item $\forall v \in V, e_1 \in E :$ \lstinline[mathescape]|($\lambda$ $\underset{\text{Parameter}}{v}$ $\underset{\text{body}}{e_1}$)| $\in E$ \hfill $[$Abstraktion$]$
\end{enumerate}
\emph{Beispiele}:\\
$y \in E$\\
\lstinline[mathescape]|($\lambda$ y y)| $\in E$ Identitätsfunktion\\
\lstinline[mathescape]|($\lambda$ y z)| $\in E$ Funktion ignoriert y,liefert z\\
\lstinline[mathescape]|((f x)y)| $\in E$ Currying\\
\lstinline[mathescape]|($\lambda$ f (f x))| $\in E$ Anwendung von Funktion f auf x (H.O.P)\\
\emph{Abkürzungen}:\\
\lstinline[mathescape]|(..(($e_1$ $e_2$) $e_3$)... $e_n$)| $\equiv$ \lstinline[mathescape]|($e_1$ $e_2$ ... $e_n$|\\
\lstinline|(f x y)| $\equiv$ \lstinline|((f x) y)|
\rackett{Implementation des lambda Kalküls}{Das Lambda Kalkül durch Schemes Quote Technik implementiert}{lambdacalc}{127}{217}
\emph{Freie/Gebundene Variablen}\\
Zur Auswertung von $E_1 \equiv$ \lstinline[mathescape]|(($\lambda$ x (f x y)) z)|
\begin{enumerate}[-]
\item wird der hier nicht bekannte Wert von Variablen f,y,z benötigt, während
\item der Wert von x im Rumpf \lstinline|(f x y)| durch das Argument z festgelegt ist.
\end{enumerate}
in $E_2$ ist
\begin{enumerate}[-]
\item Variable x (durch das \lstinline[mathescape]|$\lambda$ x|) als Parameter \emph{gebunden}, während
\item Variablen f,y,z \emph{frei} sind
\end{enumerate}
Welche Variablen eines Ausdrucks sind frei/gebunden?\\
free(v) = $\{v\}$\\
free(\lstinline[mathescape]|($e_1$ $e_2$)|) = free(\lstinline[mathescape]|$e_1$|) $\cup$ free(\lstinline[mathescape]|$e_2$|)\\
free(\lstinline[mathescape]|($\lambda$ v $e_1$)|) = free(\lstinline[mathescape]|$e_1$|)$\setminus$\{ v \}\\
bound(v) = $\varnothing$\\
bound(\lstinline[mathescape]|($e_1$ $e_2$)|) = bound(\lstinline[mathescape]|$e_1$|) $\cup$ bound(\lstinline[mathescape]|$e_2$|)\\
bound(\lstinline[mathescape]|($\lambda$ v $e_1$)|) = bound(\lstinline[mathescape]|$e_1$|)$\cup$\{ v \}\\
\emph{Beispiel}\\
\reversemarginpar{Klausur}$\begin{cases}
\text{free}(E_1) &= \{fyz\}\\
\text{bound}(E_1)&= \{x\}
\end{cases}$\\
\emph{Achtung:}
Bindung/Freiheit muss für jedes Vorkommen separat entschieden werden.\\
$E_2\equiv$ \lstinline[mathescape]|($x_1$($\lambda$ $x_2$ $x_2$))|\\
free($E_2$) = \{x\}\\
bound($E_2$) = \{x\}
\rackett{Freie und gebundene Variablen im lambda Kalkül}{Prozeduren die herausfinden ob Variablen frei oder gebunden sind}{lambdacalc}{219}{256}