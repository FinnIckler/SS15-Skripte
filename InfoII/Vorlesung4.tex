\underline{Substitutionsmodell}\\
\underline{Reduktionsregeln} f\"ur Scheme (Fallunterscheidung je nach Ausdr\"ucken) wiederhole, bis keine Reduktion mehr m\"oglich\\
\begin{tabular}{lllc}
$-$& literal \lstinline!(1, "abc",#t, ...)!& l \eval&$[\text{eval}_{lit}]$\\
$-$& Identifier id(pi, clock-face,$\ldots$)& id \eval gebundene Wert& $[\text{eval}_{id}]$\\
$-$& lambda Abstraktion &\lstinline!(lambda (...) ...)! \eval \lstinline!(lambda(...) ...)! & $[\text{eval}_{\lambda}]$\\
$-$& Applikationen (f $e_1$ $e_2 \ldots$)\\
\end{tabular}
\begin{equation}
f,e_1,e_2 \text{ reduzieren erhalte:} f`,e_1`,e_2`\\
\end{equation}\\
(2)
$\begin{cases}
\text{Operation }f`\text{ auf }e_1` \text{ und } e_2` \ [\text{apply}_{prim}] &\mbox{falls } f`\text{ primitiv ist}\\
\text{Argumentenwerte in den Rumpf von} f`\text{ einsetzen, dann reduzieren }&\mbox{falls } f`\text{ lambda Abstraktion}
\end{cases}$
\bigskip\\
Beispiel:\\
\lstinline!(+ 40!\lstinline! 2)! $\underset{eval id}{\text{\eval}}$ \lstinline!(#<procedure+> 40!\lstinline! 2)! \eval 42
\bigskip\\
\begin{tabular}{lll}
\lstinline!(position-minute-hand 30)! &$\underset{\t{eval id}}{\t{\eval}}$& \lstinline!((lambda (m) (* degrees-per-minute m)) 30)!\\
&$\underset{\t{eval lambda}}{\t{\eval}}$&\lstinline!(* degrees-per-minute 30)!\\
&$\underset{\t{eval id}}{\t{\eval}}$&\lstinline!(#<procedure *> 360/60!\lstinline! 30)!\\
&$\underset{\t{apply prim}}{\t{\eval}}$&180\\
\end{tabular}\\
Bezeichnen \lstinline!(lambda (x) (* x x))! und \lstinline!lambda (r) (* r r)! die gleiche Prozedur? $\Rightarrow$ JA!\\
Achtung: Das hat Einflu\ss \ auf das Korrekte Einsetzen von Argumenten f\"ur Prozeduren (siehe apply)
\bigskip\\
\section*{Prinzip der Lexikalischen Bindung}
Das \underline{bindende Vorkommen} eines Identifiers id kann im Programmtext systematisch bestimmt werden: Suche strikt von innen nach au\ss en, bis zum ersten
\begin{enumerate}[(1)]
\i \lstinline!(lambda (r) <Rumpf>!
\i \lstinline!(define <e>)!
\end{enumerate}
\"Ubliche Notation in der Mathematik: \uline{Fallunterscheidung}\\
$max(x_1,x_2) =
\begin{cases}
x_1 &\text{ falls } x_1 \geq x_2\\
x_2 &\text{ sonst } 
\end{cases}$\\
\underline{Tests} (auch Pr\"adikate) sind Funktionen, die einen Wert der Signatur boolean liefern. Typische primitive Tests.\\
\lstinline!(: = (number number -> boolean))!\\
\lstinline!(: < (real real -> boolean))!\\
auch \lstinline!>!, \lstinline!<=!, \lstinline!>=!\\
\lstinline!(: String=? (string string -> boolean))!\\
auch \lstinline!string>?!, \lstinline!string<=?!\\
\lstinline!(: zero? (number -> boolean))!\\
auch \lstinline!odd?!, \lstinline!even?!, \lstinline!positive?!, \lstinline!negative?!\\
Bin\"are Fallunterscheidung \underline{if}\\
$\begin{array}{lcl}
if\\
& <e_1>& \t{Mathematik:}\\
& <e_2>& \begin{cases}e_1& \t{falls } t_1\\
					  e_2& \t{sonst}
\end{cases}\\
& <e_2>)
\end{array}$
\rackett{Das Subtitutionsmodell}{Das Subtitutionsmodell}{Prozeduren}{137}{175}