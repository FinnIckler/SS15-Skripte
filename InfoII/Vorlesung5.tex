Die Signatur \underline{one of} lässt genau einen der ausgewählten Werte zu.\\
\lstinline[mathescape]!(one of <$e_1$> <$e_2$> ... <$e_n$>)!\\
\rackett{Die one-of Signatur}{one-of am Beispiel des Fu\ss ballpunktesystems}{Prozeduren}{242}{251}
Reduktion von if:\\
\lstinline[mathescape]!(if $t_1$ <$e_1$> <$e_2$>)!\\
\circled{1} Reduziere $t_1$, erhalte $t_1'$ $\underset{\text{\circled{2}}}{\text{\eval}}
\begin{cases}
\t{\arge{1}} &\t{falls } t_1' = \t{\lstinline!#t!}, \t{\arge{2} niemals ausgewertet}\\
\t{\arge{2}} &\t{falls } t_1' = \t{\lstinline!#f!}, \t{\arge{1} niemals ausgewertet}  
\end{cases}$\\
\rackett{Konstruktion eines eigenen Ifs?}{Koennen wir unser eigenes `if' aus `cond' konstruieren?  (Nein!)}{Prozeduren}{257}{277}
Spezifikation Fallunterscheidung (conditional expression):\\
\begin{tabular}{rlcl}
\lstinline!(cond!& & & Mathematik:\\
&\lstinline[mathescape]!(<$t_1$> <$e_1$>)!&\rdelim\{{5}{0mm}
[] &$e_1$ falls $t_1$ \\
&\lstinline[mathescape]!(<$t_2$> <$e_2$>)!& &$e_2$ falls $t_2$]\\
&\lstinline[mathescape]!$\ldots$!& & $\ldots$\\
&\lstinline[mathescape]!(<$t_n$> <$e_n$>)! & &$e_n$ falls $t_n$\\
&\lstinline[mathescape]!(else <$e_{n+1}$>))! & & $e_{n+1}$ sonst
\end{tabular}\\
Werte die Tests in den Reihenfolge $t_1,t_2.t_3,\ldots,t_n$ aus.\\
Sobald $t_i \#t$ ergibt, werte Zweig $e_i$ aus. $e_i$ ist Ergebnis der Fallunterscheidung. Wenn $t_n \#t$ liefert, dann liefert $\\
\begin{cases}
\t{Fehlermeldung \glqq \textcolor{red}{cond: alle Tests ergaben false}\grqq}& \t{falls kein else Zweig}\\
\text{\lstinline[mathescape]!$\t{\arge{n+1}}$!}& \t{sonst}
\end{cases}$
\newpage
\rackett{Absolutbetrag durch cond}{Absolutwert von x}{Prozeduren}{228}{234}
Reduktion von cond $\lbrack \t{eval}_{\t{cond}}\rbrack $\\
\lstinline[mathescape]!(cond (<$t_1$> <$e_1$>)(<$t_2$> <$e_2$>)$\ldots$(<$t_n$> <$e_n$>))!\\
\circled{1} Reduziere $t_1$ erhalte $t_1'$ $\underset{\t{\circled{2}}}{\t{\eval}} \begin{cases}
\t{\lstinline[mathescape]!<e_1>!} & \t{falls }t_1' = \t{\lstinline!#t!}\\
(\t{\lstinline[mathescape]!cond <t_2> <e_2>)!} & \t{sonst}
\end{cases}$\\
\lstinline[mathescape]!(cond)! \eval \glqq Fehlermeldung : \textcolor{red}{alle Test ergaben false} \grqq\\
\lstinline[mathescape]!(cond (else <$e_{n+1}$>))! \eval $e_{n+1}$\\
\bigskip\\
cond ist syntaktisches Zucker (auch abgeleitete Form) für eine verbundene Anwendung von if \\
\begin{lstlisting}[literate=]
(cond 	(<t1><e1>) 		if (<t1>
	(<t2><e2>)	    	    <e1>
	...				if <t2>
	...				if <e2>
	...				...
	(<tn><en>)       	    	  if <tn>
					     <en>
	(else <en+1>)  	 			<en+1>))..))
\end{lstlisting}
Spezialform 'and' und 'or' \\
\lstinline[mathescape]!(or $\t{\argt{1}}$ $\t{\argt{2}}$ $\ldots$ $\t{\argt{n}}$)! \eval \lstinline[mathescape]!(if $\t{\argt{1}}$ (or $\t{\argt{2}}$ $\ldots$ $\t{\argt{n}}$) #t)!\\
\lstinline[mathescape]!(or)! \eval \lstinline[mathescape]!#f! \\
\lstinline[mathescape]!(and $\t{\argt{1}}$ $\t{\argt{2}}$ $\ldots$ $\t{\argt{n}}$)! \eval \lstinline[mathescape]!(if $\t{\argt{1}}$ (and $\t{\argt{2}}$ $\ldots$ $\t{\argt{n})}$ #f)!\\
\lstinline[mathescape]!(and)! \eval \lstinline[mathescape]!#t!
\newpage
\rackett{Boolsche Ausdrücke mit and und or}{Konstruktion komplexer Prädikate mittels `and' und `or'}{Prozeduren}{281}{296}
