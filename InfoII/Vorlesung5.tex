Die Signatur \underline{one of} lässt genau einen der ausgewählten Werte zu.\\
(one of \auf $e_1$\zu \auf$e_2$\zu \ldots \auf $e_1$\zu)\\
\linie\\
Reduktion von if:\\
(if $t_1$ \auf $e_1$\zu \auf $e_2$\zu)\\
\textcircled{1} Reduziere $t_1$, erhalte $t_1'$ $\t{\eval}_\text{\textcircled{2}} 
\begin{cases}
<e_1\t{\zu} &\t{falls } t_1' = \#t, <e_2\t{\zu niemals ausgewertet}\\
<e_2\t{\zu} &\t{falls } t_1' = \#\t{f}, <e_1\t{\zu niemals ausgewertet}  
\end{cases}$\\
\linie\\
Spezifikation Fallunterscheidung (conditional expression):\\
\begin{tabular}{rlcl}
(cond& & & Mathematik:\\
&(\auf $t_1$\zu \auf $e_1$\zu)&\rdelim\{{5}{0mm}
[] &$e_1$ falls $t_1$ \\
&(\auf $t_2$\zu \auf $e_2$\zu)& &$e_2$ falls $t_2$]\\
&\ldots& & $\ldots$\\
&(\auf $t_n$\zu \auf $e_n$\zu) & &$e_n$ falls $t_n$\\
&(else \auf $e_{n+1}$\zu)) & & $e_{n+1}$ sonst
\end{tabular}\\
Werte die Tests in den Reihenfolge $t_1,t_2.t_3,\ldots,t_n$ aus.\\
Sobald $t_i \#t$ ergibt, werte Zweig $e_i$ aus. $e_i$ ist Ergebnis der Fallunterscheidung. Wenn $t_n \#t$ liefert, dann liefert $\\
\begin{cases}\
\t{Fehlermeldung \glqq cond: alle Tests ergaben false\grqq}& \t{falls kein else Zweig}\\
<e_{n+1}\t{\zu}& \t{sonst}
\end{cases}$\\
Reduktion von cond $\lbrack \t{eval}_{\t{cond}}\rbrack $\\
(cond (\auf $t_1$\zu \auf $e_1$\zu)(\auf $t_2$\zu \auf $e_2$\zu)$\ldots$(\auf $t_n$\zu \auf $e_n$\zu))