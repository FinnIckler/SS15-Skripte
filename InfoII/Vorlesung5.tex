Die Signatur \underline{one of} lässt genau einen der ausgewählten Werte zu.\\
(one of \auf $e_1$\zu \auf$e_2$\zu \ldots \auf $e_1$\zu)\\
\begin{lstlisting}[frame=single]
    


; Punkte der Heimmannschaft bei Ergebnis h:a
(: heim-punkte (natural natural -> (one-of 3 0 1)))
(check-expect (heim-punkte 2 0) 3)
(check-expect (heim-punkte 1 4) 0)
(check-expect (heim-punkte 3 3) 1)
(define heim-punkte
  (lambda (h a)
    (cond ((> h a) 3)
          ((< h a) 0)
          (else    1))))


\end{lstlisting}
Reduktion von if:\\
(if $t_1$ \auf $e_1$\zu \auf $e_2$\zu)\\
\textcircled{1} Reduziere $t_1$, erhalte $t_1'$ $\t{\eval}_\text{\textcircled{2}} 
\begin{cases}
<e_1\t{\zu} &\t{falls } t_1' = \#t, <e_2\t{\zu niemals ausgewertet}\\
<e_2\t{\zu} &\t{falls } t_1' = \#\t{f}, <e_1\t{\zu niemals ausgewertet}  
\end{cases}$\\
\begin{lstlisting}[frame=listing]

; Koennen wir unser eigenes `if' aus `cond' konstruieren?  (Nein!)

; Bedingte Auswertung von e1 oder e2 (abhaengig von t1)
(check-expect (my-if (= 42 42) "Yes!" "No!") "Yes!")
(check-expect (my-if (odd? 42) "Yes!" "No!") "No!")
(define my-if
  (lambda (t1 e1 e2)
    (cond (t1 e1)
          (else e2))))



; Sichere Division x/y, auch fuer y = 0
(: safe-/ (real real -> real))
(define safe-/
  (lambda (x y)
    (my-if (= y 0)     ; <-- Funktion my-if wertet ihre Argumente
           x           ;     vor der Applikation aus: (/ x y) wird
           (/ x y))))  ;     in *jedem* Fall reduziert. :-(


(safe-/ 42 0)          ; Fuehrt zu Fehlemeldung "division by zero"
                       ; (Reduktion mit Stepper durchfuehren)



\end{lstlisting}
Spezifikation Fallunterscheidung (conditional expression):\\
\begin{tabular}{rlcl}
(cond& & & Mathematik:\\
&(\auf $t_1$\zu \auf $e_1$\zu)&\rdelim\{{5}{0mm}
[] &$e_1$ falls $t_1$ \\
&(\auf $t_2$\zu \auf $e_2$\zu)& &$e_2$ falls $t_2$]\\
&\ldots& & $\ldots$\\
&(\auf $t_n$\zu \auf $e_n$\zu) & &$e_n$ falls $t_n$\\
&(else \auf $e_{n+1}$\zu)) & & $e_{n+1}$ sonst
\end{tabular}\\
Werte die Tests in den Reihenfolge $t_1,t_2.t_3,\ldots,t_n$ aus.\\
Sobald $t_i \#t$ ergibt, werte Zweig $e_i$ aus. $e_i$ ist Ergebnis der Fallunterscheidung. Wenn $t_n \#t$ liefert, dann liefert $\\
\begin{cases}
\t{Fehlermeldung \glqq cond: alle Tests ergaben false\grqq}& \t{falls kein else Zweig}\\
<e_{n+1}\t{\zu}& \t{sonst}
\end{cases}$
\bigskip\\
\begin{lstlisting}[frame=listing]
; Absolutwert von x
(: my-abs (real -> real))
(check-within (my-abs -4.2) 4.2 0.001)   ; Wichtig:
(check-within (my-abs 4.2) 4.2 0.001)    ; Tesfaelle decken alle Zweige
(check-within (my-abs 0) 0 0.001)        ; der conditional expression an
(define my-abs
  (lambda (x)
    (cond ((< x 0) (- x))
          ((> x 0) x    )
          (else    0    ))))
\end{lstlisting}
Reduktion von cond $\lbrack \t{eval}_{\t{cond}}\rbrack $\\
(cond (\auf $t_1$\zu \auf $e_1$\zu)(\auf $t_2$\zu \auf $e_2$\zu)$\ldots$(\auf $t_n$\zu \auf $e_n$\zu))\\
\textcircled{1} Reduziere $t_1$ erhalte $t_1'$ $\t{\eval}_{\t{\textcircled{2}}} \begin{cases}
<e_1\t{\zu} & \t{falls }t_1' = \#t\\
(\t{cond }<t_2\t{\zu} <e_2\t{\zu}) & \t{sonst}
\end{cases}$\\
(cond) \eval \glqq Fehlermeldung : alle Test ergaben false \grqq\\
(cond (else \auf $e_{n+1}$)) \eval $e_{n+1}$\\
\bigskip\\
cond ist syntaktisches Zucker (auch abgeleitete Form) für eine verbundene Anwendung von if \\
\begin{lstlisting}
(cond 	(<t1><e1>) 		if (<t1>
	(<t2><e2>)	    	    <e1>
	...				if <t2>
	...				if <e2>
	...				...
	(<tn><en>)       	    	  if <tn>
					     <en>
	(else <en+1>)  	 			<en+1>))..))
\end{lstlisting}
Spezialform 'and' und 'or' \\
(or \argt{1} \argt{2} $\ldots$ \argt{n}) \eval (if \argt{1} (or \argt{2} $\ldots$ \argt{n}) \#t)\\
(or) \eval \#f \\
(and \argt{1} \argt{2} $\ldots$ \argt{n}) \eval (if \argt{1} (and \argt{2} $\ldots$ \argt{n})\#f)\\
(and) \eval \#t
\begin{lstlisting}[frame=single]          
; Konstruktion komplexer Praedikate mittels `and' und `or':

(and #t #f)  ; eval #f   (Mathematik: Konjunktion)
(or #t #f)   ; eval #t   (Mathematik: Disjunktion)


; Kennzeichen am/pm fuer Stunde h
(: am/pm (natural -> (one-of "am" "pm" "???")))
(check-expect (am/pm 10) "am")
(check-expect (am/pm 13) "pm")
(check-expect (am/pm 25) "???")
(define am/pm
  (lambda (h)
    (cond ((and (>= h 0) (< h 12))  "am")
          ((and (>= h 12) (< h 24)) "pm")
          (else "???"))))
\end{lstlisting}