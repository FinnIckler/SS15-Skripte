Erstellung eines Huffman-Tree eines gegebenen Textes txt. Plan:
\begin{enumerate}[(H1)]
\i Stelle Häufigkeit des Vorkommens jedes Zeichen in txt fest. Sortiere in Reihenfolge steigender Häufigkeit (\lstinline|occurences|)
\end{enumerate}
Beispiel: \lstinline|(occurences "erdbeermarmelade")|\eval\par \lstinline[mathescape]|(list $\langle$"l" 1$\rangle$ $\langle$"b" 1$\rangle\ \ldots$ $\langle$"e" 5$\rangle$)|\\
Vorkommen \lstinline[mathescape]|$\langle$i n$\rangle$ (occur)| : Ding i (\lstinline|item|) kommt mit Häufigkeit n (\lstinline|freq|) vor.
\begin{enumerate}[(H2)]
\i Baue Huffman-Tree von Blättern her auf. Initialisiere den Aufbau: Für Vorkommen \lstinline[mathescape]|$\langle$i n$\rangle$| konstruiere \lstinline[mathescape]|$\langle\fbox{\text{i}}$ n$\rangle$|
\i[(H3)] Die beiden Huffman-Trees die die seltensten Zeichen repräsentieren stehen am Anfang der Liste. \lstinline[mathescape]|(list $\langle\fbox{\text{i}}$ n$\rangle$ $\langle\fbox{\text{j}}$ n$\rangle$)|. Konstruktion des Huffman-Tree, die diese \emph{Invariante} bewahren.\\
\vrule\emph{Iteration}: Wiederhole bis Liste Länge 1 hat:\\
\begin{tabular}{|rp{.85\textwidth}}
a)& Fasse Vorkommen \lstinline[mathescape]|$\langle\underset{\text{l}}{\triangle}$ n$\rangle$| und \lstinline[mathescape]|$\langle\underset{\text{r}}{\triangle}$ m$\rangle$| zu einem Vorkommen \lstinline[mathescape]|$\langle$| 
\raisebox{-0.5cm}{\scalebox{0.5}{\begin{tikzpicture}[->,>=stealth',level/.style={sibling distance = 1.25cm/#1,
  level distance = 1.25cm}] 
\node [arn_n] {}
    child{ node [arn_r] {l}                          
    }
    child{ node [arn_r] {r}
		}
; 
\end{tikzpicture}}}
\lstinline[mathescape]|n+m$\rangle$|\\
b)&Sortiere dieses Vorkommen bzgl. Häufigkeit n+m in die Restliste ein.
\end{tabular}\\
\rule{3cm}{1pt}
\item[(H4)] Baum ht in \lstinline[mathescape]|(list $\langle\underset{\text{ht}}{\triangle}$ n)| ist der gesuchte Huffman-Tree
\end{enumerate}
\rackett{Erzeugen eines Huffman-Trees}{Implementation der Erzeugung von Huffman-Trees}{Huffman}{259}{407}
\emph{Neue Sprachebene}: DMdA-fortgeschritten
\begin{enumerate}[-]
\item Neues Ausgabeformat im REPL
\subitem \lstinline|(list x1 ... xn)| $\quad\rightarrow\quad$ \lstinline[]|(x1 .. xn)|
\subitem \lstinline|empty|\qquad\qquad\qquad\ \ $\quad\rightarrow\quad$ \lstinline|()|
\item Neuer (struktureller) Gleichheitstest für Werte aller (auch benutzerdefinierte)\\
Signatur: \lstinline|(: equal? (%a %b -> boolean))|
\end{enumerate}
\emph{Quote}:\\
Sei e ein beliebiger Scheme-Ausdruck. Dann liefert \lstinline|(quote e)| die Repräsentation von e -- e wird \emph{nicht} ausgewertet.\\
Beispiele:\\
\begin{tabular}{ll}
\lstinline|(quote 42)|\eval \lstinline|42|&\rdelim\}{2}{-20mm}[Konstante Literale repräsentieren sich selbst]\\
\lstinline|(quote "UTÜ")|\eval \lstinline|"UTÜ"|&\\
\lstinline|(quote (+ 40 2))|\eval \lstinline|(+ 40 2)|&\rdelim\}{1}{-20mm}[Funktionsapplikation als Liste]
\end{tabular}\\
\emph{Listennotation in Programmen}\\
\lstinline[literate={}]|(list x1 ... xn)|$\quad\equiv\quad$\lstinline|'(x1 ... xn)|\\
\lstinline|empty|\qquad\qquad\qquad\ \ \ \ $\quad\equiv\quad$ \lstinline|'()|\\
\emph{Symbole}:\\
Was ist \lstinline|(first '(x 12))|?\\
Was sind \lstinline|lambda, x, +| in \lstinline|'(lambda (x) (+ x 1))|\\
Neue Signatur \emph{Symbol} zur Repräsentation von Namen in Programmen. Effiziente interne Darstellung/effizient vergleichbar. Kein Zugriff auf die einzelnen Zeichen des Symbols.\\
\emph{Operationen}:
\begin{enumerate}[-]
\item\lstinline|(symbol? (%a -> boolean))|
\item\lstinline|(: symbol->string (symbol -> string))|
\end{enumerate}
Repräsentation und Auswertung arithmetischer Ausdrücke:\\
\emph{Beispiel}: e $\equiv \overbrace{\text{\lstinline[mathescape]|'($\underset{\substack{\uparrow\\\text{Symbol}}}{*}
$(! (+ $\underset{\substack{\uparrow\\\text{Number}}}{1}$ 2))|}}^{\text{list}}$\\
\begin{lstlisting}
(define arith
  (signature (mixed number
                    symbol
                    (list-of airth))))
\end{lstlisting}
Auswertung möglich, wenn Bindungen für Symbole (Variablen \emph{und} Operatoren) an Wert gegeben. \emph{Dictionary} (\emph{Envirement}).\\
\begin{tabular}{lrcl}
$d_1$:\\
&$\lbrace$ x&$\rightarrow$& 3\\
&*&$\rightarrow$& \lstinline|<procedure : *>|\\
&+&$\rightarrow$& \lstinline|<procedure : +>|\\
&!&$\rightarrow$& \lstinline|fac|$\rbrace$\\
\end{tabular}\\
e\eval  18\\
\begin{tabular}{lrcl}
$d_2$:\\
&$\lbrace$ x&$\rightarrow$& 1\\
&*&$\rightarrow$& \lstinline|<procedure : *>|\\
&+&$\rightarrow$& \lstinline|<procedure : +>|\\
&!&$\rightarrow$& \lstinline|(lambda (x) (- x))|$\rbrace$\\
\end{tabular}\\
e\eval -3\\
\rackett{Ausdruck Arithmetischer Ausdrucke mittels Quote}{Eigene Programmiersprache die arithmetische Ausdrücke auswerten kann}{Arithmetic}{43}{126}
Auswertung eines arithmetischen Ausdrucks e (unter Dictionary d)
\[ \underbrace{\text{\lstinline|((eval d) |}}_{\substack{\text{Konfigurierter}\\\text{Ausdruck}}}\text{\lstinline|e)|} \]
\begin{enumerate}[(E1)]
\i \lstinline|((eval d) c)| = c \hfill Konstante
\i \lstinline[mathescape]|((eval $\{ x_1\rightarrow v_1\ldots x_n\rightarrow v_n \})x_i$)| = $v_i$ \hfill $x_i$ Variable
\i \ldots
\end{enumerate}