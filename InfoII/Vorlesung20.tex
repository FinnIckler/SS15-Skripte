\rackett{Berechnung der Grö\ss e eines Baumes}{Die Grö\ss e eines Baumes}{Baeume}{118}{129}
Einschub: Pretty-Printing von Bäumen\\
Prozedur \lstinline|(pp t)| erzeugt formatierten String für Binärbaum t.
\begin{enumerate}
\item[] \lstinline[mathescape]|(pp $\Box$)| = \lstinline[mathescape]|"$\textcolor{string}{\Box}$"|
\item[] \lstinline|(pp |\raisebox{-0.9cm}{\begin{tikzpicture}[->,>=stealth',level/.style={sibling distance = 1.25cm/#1,
  level distance = 1.25cm}] 
\node [arn_n] {x}
    child{ node [arn_r] {l}                          
    }
    child{ node [arn_r] {r}
		}
; 
\end{tikzpicture}}) = \begin{minipage}{0.5\textwidth}
\begin{tikzpicture}

\end{tikzpicture}
\end{minipage}
\end{enumerate}
Idee : Repräsentiere formatierten String als \emph{Liste von Zeilen} (Strings).
\begin{enumerate}[(1)]
\item[$\Rightarrow$(1)] Nutze \lstinline|(string-append)| um Zeilen-String zu definieren (horizontale Konkatenation).
\item[(2)] Nutze \lstinline|(append)| um die einzelnen Zeilen zu einer Liste von Zeilen zusammenzusetzen (vertikale Konkatenation)
\end{enumerate}
Erst direkt vor der Ausgabe werden die Zeilen-Strings zu einem auszugebenden String zusammengesetzt \lstinline|(strings-list->string)|\\
\rackett{Entwicklung einer Pretty Print Methode für Bäume}{Pretty Print eines Baumes}{Baeume}{136}{174}
\emph{Induktion über Binärbäume}\\
Sei $P(t)$ eine Eigenschaft von Binärbäumen $t \in T(M)$, also \lstinline|(: P((btree-of M) -> boolean))|.
\begin{mdframed}
Falls \lstinline|(empty-tree)| \marginnote{Induktionsbasis}
und
\par $\forall x \in M,\,r,l\in T(M):\:P(l) \land P(r) \Rightarrow P$\lstinline|(make-node l x r)|\\
\marginnote{Induktionsschritt} dann
\par $\forall t \in T(M) : P(t)$
\end{mdframed}
Beispiel:\\
Zusammenhang zwischen Grö\ss e (\lstinline|btree-size|) und Tiefe (\lstinline|btree-depth|) eines Binärbaums t. (\glqq Ein Baum der Tiefe n enthält mindestens n und höchstens $2^n-1$ Konten \grqq).\\
$P(t) \equiv $\lstinline|(btree-depth t)|$\le$\lstinline|(btree-size t)|$\le 2^{\text{\lstinline|(btree depth t)|}}-1$\\
\emph{Induktionsbasis P(\lstinline|(empty-tree)|)}\\
\lstinline|(size empty-tree)|
\begin{enumerate}
\item[$\overset{*}{\text{\eval}}$]0
\item[$\underset{[\text{depth}]}{=}$] $2^0 -1 \checkmark$
\end{enumerate}
\emph{Induktionsschritt $(P(l) \land P(r) \Rightarrow P(make-node l x r)$}\\
\lstinline|(size (make-node l x r)))|
\begin{enumerate}
\item[$\underset{[size]}{\text{\eval}}$] \lstinline|(size l)| + 1 + \lstinline|(size r)|
\item[$\underset{[i.v]}{=}$] $2^{\text{\lstinline|(depth l)|}} - 1 + 1 + 2^{\text{\lstinline|(depth r)|}} - 1$
\item[=] $2^{\text{\lstinline|(depth l)|}} + 2^{\text{\lstinline|(depth r)|}} - 1$
\item[$\le$] $2 \cdot \max(2^{\text{\lstinline|(depth l)|}},2^{\text{\lstinline|(depth r)|}})-1$
\item[=] $2 \cdot 2^{\max(\text{\lstinline|(depth rl)|},\text{\lstinline|(depth r)|})}-1$
\item[$\underset{[depth]}{\text{\reval}}$] $2^{\text{\lstinline|(depth (make-node l x r))|}}-1 \checkmark$
\end{enumerate}
Wie müsste sich \lstinline|btree-fold| eine fold-Operation für \emph{Binärbäume} verhalten? Tree Transformer für Baum t:
TODO: Bild\\
\begin{lstlisting}
(: btree-fold (%a (%a %b %a -> %a) (tree-of %b )-> %b))
(define btreefold
  (lambda (z f t)
    (cond ((empty? t)z)
	  ((node? t)(f (btree-fold z f(node-left-branch t)))
		    (node-label t)
		    (btree-fold z f (node-right-branch t))))))
\end{lstlisting}