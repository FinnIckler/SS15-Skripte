\emph{Zeichenkodierungen} bilden Zeichen auf Sequenzen von Bits ab. Derzeit sind Codes fester Länge sehr beliebt.
\begin{enumerate}[]
\item[ASCII] (Code 0-127, 7Bit, American Standard Code for Information Interchange)
\item[ISO8859-1] (Code 0-255, 8Bit, besteht aus lateinischen und Steuerzeichen)
\item[Unicode] (20Bit,codiert Zeichen aus 129 aktuellen und historischen Sprachen, inkl. Klingon)
\end{enumerate}
Beispiel Zeichen '\EUR' : 0000 0010 0000 1010 1100
\bigskip\\
\emph{Huffman-Codes} nutzen Bitsequenzen variabler Länge.\\
Idee: Zeicehn mit hoher Frequenz werden mit weniger Bits codiert, als seltene Zeichen.
$\Rightarrow$ Datenkompression\\
Huffman-Codes sind Binärbäume mit markierten Blättern:\\
Beispiel Huffmann-Codes für: \lstinline|"erdbeermarmelade"|\\
\begin{figure}[h!]
\centering
\begin{tikzpicture}[->,>=stealth',level/.style={sibling distance = 4cm/#1,
  level distance = 1.5cm}] 
\node [arn_r] {}
    child{ node [arn_r] {}
		child{ node [arn_x] {r}}
		child{ node [arn_r] {}
			child{ node [arn_x] {m}}
			child{ node [arn_x] {a}}}
            }
	child{ node [arn_r] {}
	    child{ node [arn_r] {}
			child{ node [arn_r] {}
				child{ node [arn_x] {b}}
				child{ node [arn_x] {l}}}
			child{ node [arn_x] {d}}}
		child{ node [arn_x] {e}}
    }
; 
\end{tikzpicture}
\end{figure}
Code für Zeichen x: Pfad von Wurzel bis Blatt mit Markierung x
\begin{enumerate}[-]
\item Abstieg in linken Teilbaum : Bit 0
\item Abstieg in Rechten Teilbaum: Bit 1
\end{enumerate}
\begin{table}[h!]
\centering
\begin{tabular}{ccl}
Zeichen&Frequenz&Code\\\hline
e & 5 & 11\\
r & 3 & 00\\
m & 2 & 010\\
a & 2 & 101\\
b & 1 & 1000\\
l & 1 & 1001
\end{tabular}
\end{table}
Huffman-Codes sind präfix frei, die Bits eines Zeichens sind niemals Präfix eines anderen Zeichens $\rightarrow$ eindeutige Codierung.\\
$\vert11\vert00\vert101\vert1000\vert11\vert11\ldots \Rightarrow$ erdbee\ldots\\
(Länge 42 Bit, Unicode = 320 Bit)\\
Einsetzung in JPEG, MP3, ZIP\\
Prozeduren zur Huffman-Codierung:
\begin{enumerate}[(1)]
\item Decodierung einer Bitsequenz \par\lstinline|(:huff-decode ((huff-tree-of string) (list-of bit) -> string))|
\item Codierung eines Strungs\par\lstinline|(huff-tree-of string) string -> (list-of bit)|\\
$\forall$ string : \lstinline|(huff-decode (huff-encode s))| = \lstinline|s|
\item Huffman-Tree für gegebenen Text erstellen\par\lstinline|(:huffman-code (string -> (huff-tree-of string)))|
\end{enumerate}
Decodieren eines huffman-codierten string (= eine Liste aus Bits)\\
Plan: Baue \\\lstinline|(: decode((huff-tree-of %a)(huff-tree-of %a)(list-of bit)->(list-of %a)))|
\begin{enumerate}[(1)]
\item \lstinline[mathescape]|(decode $\underset{ht}\triangle\ \fbox{x}\ [\ldots Bits\ldots]$)| = \lstinline[mathescape]|(make-pair x (decode $\underset{ht}\triangle\ \underset{ht}\triangle\ [\ldots Bits \ldots]$))|
\item \lstinline[mathescape]|(decode $\underset{ht}\triangle\ \triangle\ [ \ ]$)| = \lstinline[mathescape]|empty|
\item[(3a)] \lstinline[mathescape]|(decode $\underset{ht}\triangle$ |\raisebox{-0.5cm}{\scalebox{0.5}{\begin{tikzpicture}[->,>=stealth',level/.style={sibling distance = 1.25cm/#1,
  level distance = 1.25cm}] 
\node [arn_n] {}
    child{ node [arn_r] {l}                          
    }
    child{ node [arn_r] {r}
		}
; 
\end{tikzpicture}}}\ \lstinline[mathescape]|$[0\ldots Bits\ldots]$| = \lstinline[mathescape]|(decode $\underset{ht}\triangle\ \underset{l}\triangle\ [\ldots Bits\ldots] $)|
\item[(3b)] \lstinline[mathescape]|(decode $\underset{ht}\triangle$ |\raisebox{-0.5cm}{\scalebox{0.5}{\begin{tikzpicture}[->,>=stealth',level/.style={sibling distance = 1.25cm/#1,
  level distance = 1.25cm}] 
\node [arn_n] {}
    child{ node [arn_r] {l}                          
    }
    child{ node [arn_r] {r}
		}
; 
\end{tikzpicture}}}\ \lstinline[mathescape]|$[1\ldots Bits\ldots]$| = \lstinline[mathescape]|(decode $\underset{ht}\triangle\ \underset{r}\triangle\ [\ldots Bits\ldots] $)|
\end{enumerate}
Zu (1): Neueinstieg an Wurzel des Huffman-Tree $\Rightarrow$ Wurzel des Huffman-Tree als Parameter durchführen.
TODO: Grusts Code\\
Huffman-Codierung eines Strings als Liste von Bits\\
Plan:
\begin{enumerate}[a)]
\item
Codierung eines Zeichens c. Suche mittels einer Tiefensuche von der Wurzel des Huffman-Trees aus. Protokolliere den Pfad beim Absteig als Liste von Bits.\\
Frage : Wie soll ich reagieren, wenn die Tiefensuche zu einem Blatt mit $x \ne c$ führt?
Idee : \raisebox{-0.77cm}{\parbox{0.89\textwidth}{Verfolge an inneren Knoten \lstinline|(make-huff-node l r)| jeweils linken und rechten Teilbaum. Suche nach c schlägt fehl in entweder \lstinline|l| oder \lstinline|r|. Bei Fehlschlag liefern wir die leere Bitliste. Beachte \lstinline|(append empty xs)| = xs und \lstinline|(append xs empty)| = xs}}\end{enumerate}
\lstinline[mathescape]|(: encode ((huff-tree-of %a) $\underset{akku}{\text{(list-of bit)}}\ \underset{Zeichen}{\text{\%a}}$ -> (list-of bit)))|
\begin{enumerate}[(1)]
\item \lstinline[mathescape]|(encode $\fbox{x}\ [\ldots Bits\ldots] $ c)| = \lstinline|empty|
\item \lstinline[mathescape]|(encode $\fbox{c}\ [\ldots Bits\ldots] $ c)| = \lstinline[mathescape]|(reverse $[\ldots Bits \ldots]$)|
\item \lstinline[mathescape]|(encode| \raisebox{-0.5cm}{\scalebox{0.5}{\begin{tikzpicture}[->,>=stealth',level/.style={sibling distance = 1.25cm/#1,
  level distance = 1.25cm}] 
\node [arn_n] {}
    child{ node [arn_r] {l}                          
    }
    child{ node [arn_r] {r}
		}
; 
\end{tikzpicture}}} \lstinline[mathescape]|$[\ldots Bits\ldots]$ c)| = \lstinline[mathescape]|(append|\raisebox{-0.4cm}{\parbox{0.1\textwidth}{\lstinline[mathescape]|(encode $\underset{l}\triangle\ [0\ldots Bits \ldots]\ $ c)|\\ \lstinline[mathescape]|(encode $\underset{r}\triangle\ [1\ldots Bits\ldots]\ $ c))|}}
\end{enumerate}
\begin{enumerate}
\i[b)] Codiere Zeichen des Strings s mit \lstinline|encode| (mittels map), verbinde einzelne Bitlisten mit \lstinline|flatten| (bzw. \lstinline|concat|)
\end{enumerate}