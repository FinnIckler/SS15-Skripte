\documentclass[a4paper,12pt]{scrartcl}
\usepackage[ngerman]{babel}
\usepackage{graphicx} %Bilder einbinden
\usepackage{amsfonts,amsmath,amssymb,amsthm} %erweiterte Mathe-Zeichen
%\usepackage{minted}
\usepackage{mathtools}
\usepackage[utf8]{inputenc} %Umlaute & Co
\usepackage{fancyhdr}
\newcommand{\code}[1]{\textcolor{pseudo}{#1}}
\usepackage{forloop}
\usepackage{ifthen}
\usepackage{courier}
\usepackage{blkarray}
\usepackage{listings}
\usepackage{tikz}
\usepackage{color}
\usepackage{cancel}
\usepackage{enumerate}
\usepackage{multirow,booktabs,bigdelim}
\pagestyle{fancy}
\usepackage{mathtools}
\usepackage{marvosym}
\usepackage[]{algorithm2e}
\usepackage{pgfplots}
\usepackage{mdframed}
\usetikzlibrary{shapes}
\definecolor{string}{rgb}{0.133,0.545,0.133}
\definecolor{kommentar}{rgb}{0.75,0.49,0.07}
\definecolor{pseudo}{rgb}{0,.3,.7}
\newcommand{\latexcode}[1]{\textcolor{kommentar}{#1}}
\usepackage[normalem]{ulem} % [normalem] to set \emph{} to default behaviour (italic instead of underlining)
\usepackage{mdsymbol}
\DeclarePairedDelimiter{\ceil}{\lceil}{\rceil}
\DeclarePairedDelimiter{\floor}{\lfloor}{\rfloor}
\fancyhf{}
\rhead{Informatik II Thorsten Grust}
\lhead{Skript SS15 Finn Ickler}
\newcommand{\linie}{\noindent\makebox[\linewidth]{\rule{\paperwidth}{0.4pt}}}
\newcommand{\auf}{\textless}
\newcommand{\zu}{\textgreater}
\newcommand{\eval}{$\longrightsquigarrow$}
\renewcommand{\i}{\item}
\renewcommand{\t}{\text}
\newcommand{\arge}[1]{\auf $\t{e}_{#1}$\zu}
\newcommand{\argt}[1]{\auf $\t{t}_{#1}$\zu}
\newcommand{\argcomp}[1]{\auf $\t{comp}_{#1}$\zu}
\newcommand{\argid}[1]{\auf $\t{id}_{#1}$\zu}
\lstset{
  language=Macht der Abstraktion
}
\author{Finn Ickler}
\title{Informatik II Skript Sommersemester 2015}
\begin{document}
\maketitle
\tableofcontents
\newpage
\section{14.4.2015}
\section*{Scheme}
\uline{A}usdr\"ucke , \uline{A}uswertung und \uline{A}bstraktion\\
\subsection*{Dr Racket}
\fbox{
\includegraphics[scale=0.15]{drracket}}\\
Die Anwendung von Funktionen wird in Scheme ausschlie\ss lich in Pr\"afixnotation durchgef\"uhrt
\bigskip\\
\begin{center}
\begin{tabular}{c|c}
Mathematik & Scheme \\
\hline
$44-2$ & \lstinline!(- 44!\lstinline! 2)!\\
$f(x,y)$ & \lstinline!(f x y)!\\
$\sqrt{81}$ & \lstinline!(sqrt 81)!\\
$9^2$ & \lstinline|(! 3)|\\
\end{tabular}\\
Allgemein: \lstinline[literate=]!(<funktion><argument1><argument2> ...)!
\end{center}
\lstinline!(+ 40!\lstinline! 2)! und \lstinline!(odd? 42)! sind Beispiele f\"ur \underline{Ausdr\"ucke}, die bei \underline{Auswertung} einen Wert liefern.\\
(Notation: \eval)\\
\lstinline!(+ 40!\lstinline! 2)! $\underbrace{\longrightsquigarrow}_{Reduktion}$ 42\\
\lstinline!(odd? 42)! \eval \ \lstinline!#f!
\bigskip\\
Interaktionsfenster:\hspace*{2.5cm} $\underbrace{Read \rightarrow Eval \rightarrow Print \rightarrow Loop}_{REPL}$\\
\bigskip\\
\underline{Literale} sethen f\"ur einen konstanten Wert (auch: \underline{Konstante}) und sind nicht weiter reduzierbar.\\
\begin{center}
\begin{tabular}{ccc}
Literal &  & Sorte,Typ\\
\hline
\lstinline!#f ,#t! & (true, false, Wahrheitswert) & boolean\\
\lstinline!"x"! & (Zeichenketten) & String\\
\lstinline!0  1904  42  -2 !& (ganze Zahl) & Integer\\
\lstinline!0.42  3.14159 !& (Flie\ss kommazahl) & real\\
\lstinline!1/2, 3/4, -1/10! & (rationale Zahlen) & rational\\
\includegraphics[scale=0.2]{Darth_Vader} & (Bilder) & image

\end{tabular}
\end{center}



\section{16.4.2015}
Auswertung \underline{zusammengesetzter Ausdr\"ucke} in mehreren Schritten (Steps), von ``innen nach au\ss en``, bis keine Reduktion mehr m\"oglich ist.\\
(+ ( \uwave{(+ 20 20)} (+ 1 1)) \eval (+ 40 \uwave{(+ 1 1)} \eval (+ 40 2) \eval  42\\
\textbf{Achtung:} Scheme rudnet bei Arithmetik mit Flie\ss kommazahlen (interne Darstellung ist bin\"ar).\\
Beispiel: Auswertung eines zusammengesetzten Ausdrucks 
\begin{lstlisting}[frame=single]
; Achtung: Arithmetik mit Fliesskommazahlen (real)\\
 unterliegt Rundung!
(+ 0.7
   (- (/ 1/2 0.25)
      (/ 0.6 0.3)))

(- (+ 0.7
      (/ 1/2 0.25))
   (/ 0.6 0.3))

; Arithmetik mit rationalen Zahlen (rational) ist exakt
(- (+ 7/10
      (/ 1/2 1/4))
   (/ 6/10 3/10))
\end{lstlisting}
Ein Wert kann an einen \underline{Namen} (auch \underline{Identifier}) gebunden werden, durch \\
(define \auf id\zu \auf e\zu) \hspace*{1.5cm} \auf id\zu Identifier \auf e\zu Ausdruck\\
Erlaubte konsistente Wiederverwendung, dient der Selbstdokumentation von Programmen\\
\textbf{\underline{Achtung:}} Dies ist eine sogenannte Spezialform und kein Ausdruck. Insbesondere besitzt diese Spezialform \underline{keinen} Wert, sondern einen Effekt Name \auf id\zu \ wird an den \underline{Wert} von \auf e\zu \ gebunden. \\
Namen k\"onnen in Scheme beliebig gewählt werden, solange
\begin{enumerate}
\item[(1)] die Zeichen ( ) $[$ $]$ $\{ \}$ `` , ` ` ; \# $\mid$ \textbackslash nicht vorkommen
\item[(2)] dieser nicht einem numerischen Literal gleicht.
\item[(3)] kein Whitespace (Leerzeichen, Tabulator, Return) enthalten ist.
\end{enumerate}
Beispiel: euro$\rightarrow$US\$\\
\underline{\textbf{Achtung:}} Gro\ss-\textbackslash Kleinschreibung ist irrelevant.
\bigskip\\
\begin{lstlisting}[frame=single]
; Bindung von Werten an Namen
(define absoluter-nullpunkt -273.15)
(define pi 3.141592653)
(define Gruendungsjahr-SC-Freiburg 1904)
(define top-level-domain-germany "de")
(define minutes-in-a-day (* 24 60))
(define vorwahl-tuebingen (sqrt 1/2))
\end{lstlisting}
Eine \underline{lambda-Abstraktion} (auch Funktion, Prozedur) erlaubt die Formatierung von Ausrdr\"ucken, in denen mittels \underline{Parametern} von konkreten Werten abstrahiert wird.\\
(lambda (\auf p1\zu \auf p2 \zu $\ldots$) \auf e\zu\\
\auf e\zu Rumpf: enth\"alt Vorkommen der Parameter \auf $p_n$\zu\\
(lambda($\ldots$)) ist eine Spezialform. Wert der lambda-Abstraktion ist \#\auf procedure\zu\\.
\underline{Anwendung} (auch Application) des lambda-Aufrufs f\"uhrt zur Ersetzung aller Vorkommen der Parameter im Rumpf durch die angegebenen \underline{Argumente}.\\
\begin{lstlisting}[frame = single]
; Abstraktion: Ausdruck mit "Loch" @$\odot$@
(lambda (@$\odot$@) (* @$\odot$@ (* 155 minutes-in-a-day)))


; Zuwachs der Weltbevoelkerung innerhalb von days Tagen
(define population-growth-in-days
    (lambda (days) (* days (* 155 minutes-in-a-day))))

(population-growth-in-days 7)
\end{lstlisting}
(lambda (days) (* days (* 155 minutes-in-a-day))) 365) \eval\\
 (* 365 (* 155 minutes-in-a-day)) \eval 81468000\\
\bigskip\\
In Scheme leitet ein Semikolon einen Kommentar ein, der bis zum Zeilenende reicht und vom System bei der Auswertung ignoriert wird.\\
Prozeduren sollten im Programm ein- bis zweizeilige \underline{Kurzbeschreibungen} direkt vorangestellt werden.



\section{21.4.2015}
Eine Signatur pr\"uft, ob ein Name an einen Wert einer angegebenen Sorte (Typ) gebunden wird. Signaturverletzungen werden protokolliert.\\
(: \auf id\zu \auf signatur\zu)\\
Bereits eingebaute Sinaturen\\
\begin{tabular}{rc|r}
natural&$\mathbb{N}$& boolean\\
integer&$\mathbb{Z}$& string\\
rational&$\mathbb{Q}$& image\\
real&$\mathbb{R}$&$\ldots$\\
numver&$\mathbb{C}$
\end{tabular}\\
(: $\ldots$) ist eine Spezialform und hat keinen Wert, aber einen Effekt: Signaturpr\"ufung\\
\underline{Prozedur Signatur} spezifizieren sowohl Signaturen f\"ur die Parameter $\text{P}_1,\text{P}_2,\ldots\text{P}_n$ als auch den Ergebniswert der Prozedur,\\
(: \auf Signatur $\text{P}_1$\zu $\ldots$ \auf Signatur $\text{P}_n$\zu $->$ \auf Signatur Ergebnis\zu)\\
Prozedur Signaturen werden \underline{bei jeder Anwendung} einer Prozedur auf Verletzung gepr\"uft. \underline{Testf\"alle} dokumentieren das erwartete Ergebnis einer Prozedur f\"ur ausgew\"ahlte Argumente:
\begin{center}
(check-expect \auf $\text{e}_1$\zu \auf
$\text{e}_2$\zu)\end{center}
Werte Ausdruck \auf $\text{e}_1$\zu \ aus und teste, ob der erhaltene Wert der Erwartung \auf $\text{e}_2$\zu \ entspricht (= der Wert von \auf $\text{e}_2$\zu) \
Einer Prozedur sollte Testf\"alle direkt vorangestellt werden.\\
Spezialform: kein Wert, sondern Effekt: Testverletzung protokollieren
\bigskip\\
\underline{Konstruktionsanleitung f\"ur Prozeduren:}
\begin{enumerate}
\item[(1)]Kurzbeschreibung (ein- bis zweizeiliger Kommentar mit Bezug auf Parametername)
\item[(2)]Signaturen
\item[(3)]Testf\"alle
\item[(4)]Prozedurrumpf
\end{enumerate}
\underline{Top-Down-Entwurf} (Programmieren durch ``Wunschdenken``)\\
Beispiel: Zeichne Ziffernblatt (Stunden- und Minutenzeiger) zu Uhrzeit h:m auf einer analogen 24h-Uhr\\
\begin{tikzpicture}
    \begin{axis}[
    legend style={draw none},
    axis equal,ymin = -2,xmin = -2,ymax = 2,
    xmax = 2,xtick ={},
    hide axis,
    xticklabels={},
    ytick ={},
    yticklabels={},
    extra x ticks={0},
    extra x tick label={},
    extra y ticks={0},
    extra y tick labels={},
    disabledatascaling,
    extra tick style = {grid = major}]
    \draw (axis cs:0,0) circle[radius=1];
    \draw[->](axis cs:0,0)--(axis cs:0.64,0.5);
    \draw[->](axis cs:0,0)--(axis cs:0.4,0);
    \end{axis}   
  \end{tikzpicture}\\
  Minutenzeiger legt $\frac{360^{\circ}}{60}$ Grad pro Minute zur\"uck (also $\frac{360}{60} \cdot m$)\\
  Studentenzeiger legt $\frac{360}{12}$ pro Stunde zur\"uck
 ($\frac{360}{12} \cdot h +\frac{360}{12} \cdot \frac{m}{60}$)
\begin{lstlisting}[frame=single]
; Grad, die Minutenzeiger pro Minute zuruecklegt
(define degrees-per-minute 360/60)

; Grad, die Stundenzeiger pro voller Stunde zuruecklegt
(define degrees-per-hour 360/12)

; Zeichne Ziffernblatt zur Stunde h und Minute m
(: draw-clock (natural natural -> image))
(check-expect (draw-clock 4 15) (draw-clock 16 15))
(define draw-clock
  (lambda (h m)
    (clock-face (position-hour-hand h m)  
    	(position-minute-hand m))))

; Winkel (in Grad), den Minutenzeiger zur Minute m einnimmt
(: position-minute-hand (natural -> rational))
(check-expect (position-minute-hand 15) 90)
(check-expect (position-minute-hand 45) 270)
(define position-minute-hand
  (lambda (m)
    (* m degrees-per-minute)))

; Winkel (in Grad), den Stundenzeiger zur Stunde h einnimmt
(: position-hour-hand (natural natural -> rational))
(check-expect (position-hour-hand 3 0) 90)
(check-expect (position-hour-hand 18 30) 195)
(define position-hour-hand
  (lambda (h m)
    (+ (* (modulo h 12) degrees-per-hour)
    ; h mod 12 in {0,1,...,11}
       (* (/ m 60) degrees-per-hour))))

; Zeichne Ziffernblatt mit Minutenzeiger um dm und
; Stundenzeiger um dh Grad gedreht
(: clock-face (rational rational -> image))
(define clock-face
  (lambda (dh dm)
    (clear-pinhole
     (overlay/pinhole
      (circle 50 "outline" "black")
      (rotate (* -1 dh) (put-pinhole 0 35 (line 0 35 "red")))
      (rotate (* -1 dm) (put-pinhole 0 45 (line 0 45 "blue")))))))
\end{lstlisting}

\section{23.4.2015}
\underline{Substitutionsmodell}\\
\underline{Reduktionsregeln} f\"ur Scheme (Fallunterscheidung je nach Ausdr\"ucken) wiederhole, bis keine Reduktion mehr m\"oglich\\
\begin{tabular}{lllc}
$-$& literal \lstinline!(1, "abc",#t, ...)!& l \eval&$[\text{eval}_{lit}]$\\
$-$& Identifier id(pi, clock-face,$\ldots$)& id \eval gebundene Wert& $[\text{eval}_{id}]$\\
$-$& lambda Abstraktion &\lstinline!(lambda (...) ...)! \eval \lstinline!(lambda(...) ...)! & $[\text{eval}_{\lambda}]$\\
$-$& Applikationen (f $e_1$ $e_2 \ldots$)\\
\end{tabular}
\begin{equation}
f,e_1,e_2 \text{ reduzieren erhalte:} f`,e_1`,e_2`\\
\end{equation}\\
(2)
$\begin{cases}
\text{Operation }f`\text{ auf }e_1` \text{ und } e_2` \ [\text{apply}_{prim}] &\mbox{falls } f`\text{ primitiv ist}\\
\text{Argumentenwerte in den Rumpf von} f`\text{ einsetzen, dann reduzieren }&\mbox{falls } f`\text{ lambda Abstraktion}
\end{cases}$
\bigskip\\
Beispiel:\\
\lstinline!(+ 40!\lstinline! 2)! $\underset{eval id}{\text{\eval}}$ \lstinline!(#<procedure+> 40!\lstinline! 2)! \eval 42
\bigskip\\
\begin{tabular}{lll}
\lstinline!(position-minute-hand 30)! &$\underset{\t{eval id}}{\t{\eval}}$& \lstinline!((lambda (m) (* degrees-per-minute m)) 30)!\\
&$\underset{\t{eval lambda}}{\t{\eval}}$&\lstinline!(* degrees-per-minute 30)!\\
&$\underset{\t{eval id}}{\t{\eval}}$&\lstinline!(#<procedure *> 360/60!\lstinline! 30)!\\
&$\underset{\t{apply prim}}{\t{\eval}}$&180\\
\end{tabular}\\
Bezeichnen \lstinline!(lambda (x) (* x x))! und \lstinline!lambda (r) (* r r)! die gleiche Prozedur? $\Rightarrow$ JA!\\
Achtung: Das hat Einflu\ss \ auf das Korrekte Einsetzen von Argumenten f\"ur Prozeduren (siehe apply)
\bigskip\\
\section*{Prinzip der Lexikalischen Bindung}
Das \underline{bindene Vorkommen} eines Identifiers id kann im Programmtext systematisch bestimmt werden: Suche strikt von innen nach au\ss en, bis zum ersten
\begin{enumerate}[(1)]
\i \lstinline!(lambda (r) <Rumpf>!
\i \lstinline!(define <e>)!
\end{enumerate}
\"Ubliche Notation in der Mathematik: \uline{Fallunterscheidung}\\
$max(x_1,x_2) =
\begin{cases}
x_1 &\text{ falls } x_1 \geq x_2\\
x_2 &\text{ sonst } 
\end{cases}$\\
\underline{Tests} (auch Pr\"adikate) sind Funktionen, die einen Wert der Signatur boolean liefern. Typische primitive Tests.\\
\lstinline!(: = (number number -> boolean))!\\
\lstinline!(: < (real real -> boolean))!\\
auch \lstinline!>!, \lstinline!<=!, \lstinline!>=!\\
\lstinline!(: String=? (string string -> boolean))!\\
auch \lstinline!string>?!, \lstinline!string<=?!\\
\lstinline!(: zero? (number -> boolean))!\\
auch \lstinline!odd?!, \lstinline!even?!, \lstinline!positive?!, \lstinline!negative?!\\
Bin\"are Fallunterscheidung \underline{if}\\
$\begin{array}{lcl}
if\\
& <e_1>& \t{Mathematik:}\\
& <e_2>& \begin{cases}e_1& \t{falls } t_1\\
					  e_2& \t{sonst}
\end{cases}\\
& <e_2>)
\end{array}$



\section{28.4.2015}
Die Signatur \underline{one of} lässt genau einen der ausgewählten Werte zu.\\
(one of \auf $e_1$\zu \auf$e_2$\zu \ldots \auf $e_1$\zu)\\
\begin{lstlisting}[frame=single]
    


; Punkte der Heimmannschaft bei Ergebnis h:a
(: heim-punkte (natural natural -> (one-of 3 0 1)))
(check-expect (heim-punkte 2 0) 3)
(check-expect (heim-punkte 1 4) 0)
(check-expect (heim-punkte 3 3) 1)
(define heim-punkte
  (lambda (h a)
    (cond ((> h a) 3)
          ((< h a) 0)
          (else    1))))


\end{lstlisting}
\linie\\
Reduktion von if:\\
(if $t_1$ \auf $e_1$\zu \auf $e_2$\zu)\\
\textcircled{1} Reduziere $t_1$, erhalte $t_1'$ $\t{\eval}_\text{\textcircled{2}} 
\begin{cases}
<e_1\t{\zu} &\t{falls } t_1' = \#t, <e_2\t{\zu niemals ausgewertet}\\
<e_2\t{\zu} &\t{falls } t_1' = \#\t{f}, <e_1\t{\zu niemals ausgewertet}  
\end{cases}$\\
\begin{lstlisting}[frame=listing]

; Koennen wir unser eigenes `if' aus `cond' konstruieren?  (Nein!)

; Bedingte Auswertung von e1 oder e2 (abhaengig von t1)
(check-expect (my-if (= 42 42) "Yes!" "No!") "Yes!")
(check-expect (my-if (odd? 42) "Yes!" "No!") "No!")
(define my-if
  (lambda (t1 e1 e2)
    (cond (t1 e1)
          (else e2))))



; Sichere Division x/y, auch fuer y = 0
(: safe-/ (real real -> real))
(define safe-/
  (lambda (x y)
    (my-if (= y 0)     ; <-- Funktion my-if wertet ihre Argumente
           x           ;     vor der Applikation aus: (/ x y) wird
           (/ x y))))  ;     in *jedem* Fall reduziert. :-(


(safe-/ 42 0)          ; Fuehrt zu Fehlemeldung "division by zero"
                       ; (Reduktion mit Stepper durchfuehren)



\end{lstlisting}
\linie\\
Spezifikation Fallunterscheidung (conditional expression):\\
\begin{tabular}{rlcl}
(cond& & & Mathematik:\\
&(\auf $t_1$\zu \auf $e_1$\zu)&\rdelim\{{5}{0mm}
[] &$e_1$ falls $t_1$ \\
&(\auf $t_2$\zu \auf $e_2$\zu)& &$e_2$ falls $t_2$]\\
&\ldots& & $\ldots$\\
&(\auf $t_n$\zu \auf $e_n$\zu) & &$e_n$ falls $t_n$\\
&(else \auf $e_{n+1}$\zu)) & & $e_{n+1}$ sonst
\end{tabular}\\
Werte die Tests in den Reihenfolge $t_1,t_2.t_3,\ldots,t_n$ aus.\\
Sobald $t_i \#t$ ergibt, werte Zweig $e_i$ aus. $e_i$ ist Ergebnis der Fallunterscheidung. Wenn $t_n \#t$ liefert, dann liefert $\\
\begin{cases}
\t{Fehlermeldung \glqq cond: alle Tests ergaben false\grqq}& \t{falls kein else Zweig}\\
<e_{n+1}\t{\zu}& \t{sonst}
\end{cases}$\\
\begin{lstlisting}[frame=listing]
; Absolutwert von x
(: my-abs (real -> real))
(check-within (my-abs -4.2) 4.2 0.001)   ; Wichtig:
(check-within (my-abs 4.2) 4.2 0.001)    ; Tesfaelle decken alle Zweige
(check-within (my-abs 0) 0 0.001)        ; der conditional expression an
(define my-abs
  (lambda (x)
    (cond ((< x 0) (- x))
          ((> x 0) x    )
          (else    0    ))))
\end{lstlisting}
Reduktion von cond $\lbrack \t{eval}_{\t{cond}}\rbrack $\\
(cond (\auf $t_1$\zu \auf $e_1$\zu)(\auf $t_2$\zu \auf $e_2$\zu)$\ldots$(\auf $t_n$\zu \auf $e_n$\zu))\\
\textcircled{1} Reduziere $t_1$ erhalte $t_1'$ $\t{\eval}_{\t{\textcircled{2}}} \begin{cases}
<e_1\t{\zu} & \t{falls }t_1' = \#t\\
(\t{cond }<t_2\t{\zu} <e_2\t{\zu}) & \t{sonst}
\end{cases}$\\
(cond) \eval \glqq Fehlermeldung : alle Test ergaben false \grqq\\
(cond (else \auf $e_{n+1}$)) \eval $e_{n+1}$\\
\linie\\
cond ist syntaktisches Zucker (auch abgeleitete Form) für eine verbundene Anwendung von if \\
\begin{tabular}{rcrrrl}
(cond & (\argt{1}\arge{1}) & if &\argt{1}\\
& (\argt{2}\arge{2})& & \arge{1}\\
&$\dots$            & & if &\argt{2}\\
&$\dots$            & & & \arge{2}\\
&$\dots$            & & & $\dots$\\
&(\argt{n}\arge{n}) & & & if &\argt{n}\\
&                   & & &    &\arge{n}\\
&(else \arge{n+1}   & & &    &\arge{n+1}))$\ldots$))
\end{tabular}\\
Spezialform 'and' und 'or' \\
(or \argt{1} \argt{2} $\ldots$ \argt{n}) \eval (if \argt{1} (or \argt{2} $\ldots$ \argt{n}) \#t)\\
(or) \eval \#f \\
(and \argt{1} \argt{2} $\ldots$ \argt{n}) \eval (if \argt{1} (and \argt{2} $\ldots$ \argt{n})\#f)\\
(and) \eval \#t
\begin{lstlisting}[frame=single]          
; Konstruktion komplexer Praedikate mittels `and' und `or':

(and #t #f)  ; eval #f   (Mathematik: Konjunktion)
(or #t #f)   ; eval #t   (Mathematik: Disjunktion)


; Kennzeichen am/pm fuer Stunde h
(: am/pm (natural -> (one-of "am" "pm" "???")))
(check-expect (am/pm 10) "am")
(check-expect (am/pm 13) "pm")
(check-expect (am/pm 25) "???")
(define am/pm
  (lambda (h)
    (cond ((and (>= h 0) (< h 12))  "am")
          ((and (>= h 12) (< h 24)) "pm")
          (else "???"))))
\end{lstlisting}
\section{30.4.2015}
\underline{Zusammengesetze Daten}\\
Ein Charakter \underline{besteht} aus drei \underline{Komponenten}\\
\begin{tabular}{clcl}
- & Name des Charakters &(name)\rdelim\}{3}{0mm}
[Datendefinition für zusammengesetzte Daten]\\
- & Handelt es sich um einen Jedi? &(jedi?)&\\
- & Stärke der Macht \hspace*{2.3cm} &(force)&
\end{tabular}\\
Konkrete Charakter:\hspace*{5pt}
\begin{tabular}{|c|c|}
\hline
name & \glqq Luke Skywalker \grqq\\
\hline
jedi? & \#f \\
\hline
force & 25 \\
\hline
\end{tabular}\\
\racket{Record Definitionen}{Starwars Charakter als Racket Records}{records}
Zusammengesetzte Daten = \underline{Records} in Scheme Record-Definition legt fest:\\
\begin{tabular}{cll}
- & Record-Signatur\\
- & \underline{Konstruktor} & (baut aus Komponenten einen Record)\\
- & Prädikat & (liegt ein Record vor?)\\
- & Liste von \underline{Selektoren}& (lesen jeweils eine Komponente des Records)
\end{tabular}\\
\begin{lstlisting}
(define-record-procedure <t>
	make-<t>
	<t>?
	(<t>-<comp1> ... <t>-<comp2>))
	;Liste der n Selektoren
\end{lstlisting}
Verträge des Konstruktors\/ der Selektoren für Record- Signatur\\
\argt{} mit Komponenten namens \argcomp{1} $\ldots$ \argcomp{n}\\
\begin{lstlisting}
(: make-<t> (<t1>...<t2>) -> <t>)
(: <t>-<comp1> (<t> -> <t1>))
(: <t>-<compn> (<t> -> <tn>))
\end{lstlisting}
Es gilt für alle Strings n, Booleans j und Integer f:
\begin{lstlisting}
(character-name (make-character n j f) n)
(character-jedi? (make-character n j f) j)
(character-force (make-character n j f) f )
\end{lstlisting}
Spezialform check-property:\\
\begin{lstlisting}
(check-property
	(for-all ((<id1> <sig1>) ... 
			  (<idn> <sign>))
	<e>))
	@\latexcode{$\downarrow$}@
;Bezieht sich auf <id1> ... <idn>	
\end{lstlisting}
Test erfolgreich, falls \arge{} für beliebig gewählte Bedeutungen für \argid{1} $\ldots$ \argid{n} immer \#t ergibt\\
\racket{Check-property}{Interaktion von Selektoren und Konstruktor:}{checkproperty}
\underline{Beispiel:} Die Summe von zwei natürlichen Zahlen ist mindestens so gro\ss \ wie jeder dieser Zahlen: $ \forall x_1 \in \mathbb{N}, x_2 \in \mathbb{N} : x_1 + x_2 \geq \max\{x_1,x_2\}$
\lstinputlisting[frame=single,literate={ü}{{\"u}}1,caption={[Übersetzung mathematischer Aussagen in check-property]Mathematische $\forall$-Aussage in Racket}]{forall.rkt}
Konstruktion von Funktionen, die bestimmte gesetzte Daten \underline{konsumiert}.\\
\begin{enumerate}[-]
\i Welche Record-Componenten sind relevant für Funktionen?
\begin{enumerate}[$\rightarrow$]
\i Schablone:
\begin{lstlisting}
(: sith? (character -> boolean))
(define sith?
  (lambda (c)
    ... (character-jedi? c))
    ... (character-force c) )...))
\end{lstlisting}
\end{enumerate}
Konstruktion von Funktionen, die zusammengesetzte Daten \underline{konstruieren}
\i Der konstruktor \underline{muss} aufgerufen werden
\begin{enumerate}[$\rightarrow$]
\i Schablone:
\begin{lstlisting}
(define
	lambda(...)
		... (make-<t>)...)
\end{lstlisting}
\end{enumerate}
\i Konkrete Beispiele:
\racket{Konstruktoren und Selektoren}{Abfragen der Eigenschaften von character Records}{konstruktor}
\end{enumerate}

\section{5.5.2015}
\begin{tikzpicture}[
    grow=right,
    level 1/.style={sibling distance=3.5cm,level distance=5.2cm},
    level 2/.style={sibling distance=3.5cm, level distance=6.7cm},
    edge from parent/.style={very thick,draw=blue!40!black!60,
        shorten >=5pt, shorten <=5pt},
    edge from parent path={(\tikzparentnode.east) -- (\tikzchildnode.west)},
    kant/.style={text width=2cm, text centered, sloped},
    every node/.style={text ragged, inner sep=2mm},
    punkt/.style={rectangle, rounded corners, shade, top color=white,
    bottom color=blue!50!black!20, draw=blue!40!black!60, very
    thick }
    ]

\node[punkt, text width=5.5em] {lego-character}
    %Lower part lv1
    child {
        node[punkt] [rectangle split, rectangle split, rectangle split parts=4,
         text ragged] {
            \textbf{character}
                  \nodepart{second}
            name
                  \nodepart{third}
            jedi?
            	  \nodepart{fourth}
            force	  
        }
        edge from parent
            node[kant, below, pos=.6] {}
    }
    %Upper part, lv1
    child {
        node[punkt, text width=6em] {Figure}
        %child 2
            edge from parent{
                node[kant, above] {}}
    };
\end{tikzpicture}\\
\includegraphics[scale=0.5]{Latitude}
\includegraphics[scale=0.5]{Longitude}\\
Position Nord/Südwest vom Äquator Position west/östlich vom Nullmeridian\\
Sei \lstinline!<p>!ein Prädikat mit Signatur \lstinline!(<t> -> boolean)!.\\
Eine Signatur der Form \lstinline!(predicate <p>! gilt für jeden Wert der Signatur \argt{} sofern (\auf p\zu) \eval \lstinline!#t!\\
Signaturen des Typs \lstinline!predicate <p>)! sind damit \uline{spezifischer} (restriktiver) als die Signatur \argt{} selbst.\\
\lstinline!(define <newt> (signature <t>!\\
\uline{Beispiele:}
\begin{lstlisting}
(define farbe
	(signature (one-of "Blatt" "Herz" "Blatt" "Eichel" "Schell")))
\end{lstlisting}
\pagebreak
\racket{predicate Signaturen am Beispiel von Längen- und Breitengrade}{Restriktive Signaturen mit predicate}{latandlong}\newpage
\section{7.5.2015}
Man kann jedes \code{one-of} durch ein \code{predicate} ersetzen.
\begin{lstlisting}[frame=single]
(: f ((one-of 0 1 2 ) -> natural))
(define f
  (lambda (x)
    x))

; And then the "The Great one-of Extinction" of 2015 occurred @\includegraphics[scale=0.5]{kaboon}@
;.

(: g ((predicate 
       (lambda (x) (or (= x 0) (= x 1) (= x 2)))) -> natural))
(define g
  (lambda (x)
    x))

\end{lstlisting}
 \ \bigskip\\
Geocoding: Übersetze eine Ortsangabe mittels des Google Maps Geocoding API (Application Programm Interface) in eine Position auf der Erdkugel.
\begin{lstlisting}
(: geocoder (string -> (mixed geocode geocode-error)))
\end{lstlisting}
Ein geocode besteht aus \\
\begin{tabular}{crrl}
& \underline{Signatur}\\
- & Adresse & (address) & string\\
- & Ortsangabe & (loc) & location\\
- & Nordostecke & (northeast) & location\\
- & Südwestecke & (southwest) & location\\
- & Typ & (type) & string\\
- & Genauigkeit & (accuracy) & string
\end{tabular}\\
\begin{lstlisting}
(: geocode-adress (geocode -> string))
(: geocode-loc (geocode -> location))
(: geocode-... (geocode -> ...))
\end{lstlisting}
Ein geocode-error besteht aus:\\
\begin{tabular}{crrl}
& \underline{Signatur}\\
- & Fehlerart & (level) & (one-of "TCP'' "HTTP'' "JSON'' ''API'')\\
- & Fehlermeldung & (message) & string\\
\end{tabular}\\
\underline{Gemischte Daten}\\
Die Signatur
\begin{lstlisting}
(mixed @\argt{1}@ ... @\argt{n}@)
\end{lstlisting}
ist gültig für jeden Wert, der mindestens eine der Signaturen \argt{1} $\ldots$ \argt{n} erfüllt.\\
\underline{Beispiel:} Data-Definition\\
Eine Antwort des Geocoders ist \underline{entweder}\\
\begin{enumerate}[-]
\i ein Geocode (geocode) \underline{oder}
\i eine Fehlermeldung (geocode-error)
\end{enumerate}
Beispiel (eingebaute Funktion string-\zu number)
\begin{lstlisting}
(: string->number (string -> (mixed number (one-of #f))))
(string->number "42") @\eval@ 42
(string-> number "foo") @\eval@ #f
\end{lstlisting}
\begin{lstlisting}[frame=single]
(define geocoder-response
  (signature (mixed geocode geocode-error)))

(: sand13 geocoder-response)
(define sand13
  (geocoder "Sand 13, Tübingen"))

(geocode-address sand13)
(geocode-type sand13)
(location-lat (geocode-loc sand13))
(location-lng (geocode-loc sand13))
(geocode-accuracy sand13)
  

(: lady-liberty geocoder-response)
(define lady-liberty
  (geocoder "Statue of Liberty"))

(: alb geocoder-response)
(define alb
  (geocoder "Schwäbische Alb"))

(: A81 geocoder-response)
(define A81
  (geocoder "A81, Germany"))
\end{lstlisting}
Erinnerung:\\
Das Prädikat \argt{}? einer Signatur \argt{} unterscheidet Werte der Signatur \argt{} von allen anderen Werten:
\begin{lstlisting}
(: @\argt{}@? (any -> boolean))
\end{lstlisting}
Auch: Prädikat für eingebaute Signaturen\\
\begin{lstlisting}
number?
complex?
real?
rational?
integer?
natural?
string?
boolean?
\end{lstlisting}
Prozeduren, die gemischte Daten der Signaturen \argt{1} $\ldots$ \argt{n} konsumieren: \\
\underline{Konstruktionsanleitung}:\\
\begin{lstlisting}
(: @\argt{}@ ((mixed @\argt{1}@ ... @\argt{n}@) -> ...))
(define @\argt{}@
	(lambda (x)
		(cond 
		   ((@\argt{1}@? x)...)
		   ...
		   ((@\argt{n}@? x) ...))))
\end{lstlisting}
Mittels \underline{let} lassen sich Werte an \underline{lokale Namen} binden,
\begin{lstlisting}
(let ( 
	(@\argid{1}@ @\arge{1}@)
	(...)
	(@\argid{n}@ @\arge{n}@))
 @\arge{}@
) 
\end{lstlisting}
Die Ausdrücke \arge{1} $\ldots$ \arge{n} werden \underline{parallel} ausgewertet. $\Rightarrow$ \argid{1} $\ldots$ \argid{n}  können in \arge{} (und nur hier) verwendet werden. Der Wert des let Ausdruckes ist der Wert von \arge{}.
\begin{lstlisting}[frame=single]
; Liegt der Geocode r auf der südlichen Erdhalbkugel?
; (Breitengrad < 0@\latexcode{$^{\circ}$}@?)
(: southern-hemisphere? (string -> boolean))

(check-expect (southern-hemisphere? "Cape Town") #t)
(check-expect (southern-hemisphere? "Tübingen") #f)
(check-error  (southern-hemisphere? "Mos Eisley") "Unknown location")

(define southern-hemisphere?
  (lambda (r)
    (let ((gc (geocoder r)))
      (cond ((geocode? gc)  
             (< (location-lat (geocode-loc gc)) 0))
            ((geocode-error? gc) 
             (violation "Unknown location"))))))
\end{lstlisting}
\underline{ACHTUNG:}\\
'let' ist verfügbar auf ab der Sprachebene "Macht der Abstraktion".
\bigskip\\
'let' ist syntaktisches Zucker.
\begin{lstlisting}
(let (			((lambda (@\argid{1}@ ... @\argid{n}@) 
	(@\argid{1}@ @\arge{1}@)		@\arge{}@)
	(...)		@$\equiv$@        @\arge{1}@
	(@\argid{n}@ @\arge{n}@))	       @\arge{2}@ ...
 @\arge{}@			     @\arge{n}@
) 
\end{lstlisting}
\section{12.5.2015}
Abstand zweier geographischer Positionen $b_1,b_2$ auf der Erdkugel in km (lat, lng jeweils in Radian).\\
\racket{Wrapper und Worker}{Abstand zweier geographischer Positionen}{distance}
\underline{\textsc{Parametrisch Polymorphe Prozeduren}}\\
Beobachtung: Manche Prozeduren arbeiten unabhängig von den Signaturen ihrer Argumente : \underline{parametrisch polymorphe Funktion} (griechisch : vielgestaltig).\\
Nutze \underline{Signaturvariablen} \%a , \%b,...\\
Beispiel:\\
\begin{lstlisting}
; die Identität
(: id (%a -> %a))
(define id
  (lambda (x) x))

; die konstante Funktion
(: const (%a %b -> %a))
(define const
  (lambda (x y) x))

; die Projektion
(: proj ((one-of 1 2) %a %b -> (mixed %a %b)))
(define proj
  (lambda (i x y)
    (cond ((= i 1) x)
          ((= i 2) y))))
\end{lstlisting}
Eine polymorphe Signatur steht für alle Signaturen, in denen die Signaturvariablen durch konkrete Signaturen ersetzt werden.\\
Beispiel: Wenn eine Prozedur \lstinline!(: number %a %b -> %a)! erfúllt, dann auch:\\
\begin{tabular}{lcl}
\lstinline!(: number string boolean!&\lstinline! ->!& \lstinline!string)! \\
\lstinline!(: number boolean natural!&\lstinline! ->!& \lstinline!boolean)! \\
\lstinline!(: number number number!&\lstinline! ->!& \lstinline!number)! \\
\end{tabular}
\vspace*{1cm}\\
\fbox{''x''}\fbox{23}\hspace*{1cm} \fbox{2}\fbox{\#f}\\
\begin{lstlisting}
; Ein polymorphes Paar (pair-of %a %b) besteht aus
; - einer ersten Komponente (first)
; - einer zweiten Komponente (rest)
(: make-pair (%a %b -> (pair-of %a %b)))
(: pair? (any -> boolean))
(: first ((pair-of %a %b) -> %a))
(: rest  ((pair-of %a %b) -> %b))
(define-record-procedures-parametric pair pair-of
  make-pair
  pair?
  (first
   rest))
\end{lstlisting}
\lstinline!(pair-of <t1> <t2>)! ist eine Signatur für Paare deren erster bzw. zweiter Komponente die Signaturen \argt{1} bzw. \argt{2} erfüllen.\\
\begin{lstlisting}
;@\latexcode{$\rightarrow$}@ pair-of Signatur mit (zwei) Parametern
(: make-pair (%a %b -> (pair-of % a %b)))
(: pair? (any -> boolean))
(: first ((pair-of %a %b ) -> %a))
(: rest ((pair-of %a %b ) -> %b))
\end{lstlisting}
\racket{make-pair, ein polymorpher Datentyp}{Paare aus verschiedenen Datentypen}{makepair}
Eine \underline{Liste} von Werten der Signatur \argt{t} ist entweder
\begin{enumerate}[-]
\i leer (Signatur \lstinline!empty-list!) oder:
\i ein Paar (Signatur \lstinline!pair-of!) aus einem Wert der Signatur \argt{} und einer Liste von Werten der Signatur \argt{}. \\
\end{enumerate}
\begin{lstlisting}
(define list-of
  (lambda (t)
    (signature (mixed empty-list
                      (pair-of t (list-of t))))))

\end{lstlisting}
Signatur \lstinline!empty-list! bereits in Racket vordefiniert.\\
Ebenfalls vordefiniert:\\
\lstinline!(:empty empty-list)!\\
\lstinline!(: empty? (any -\zu boolean))!\\
\underline{Operatoren auf Listen}\\
\begin{enumerate}
\i[Konstruktoren]
\begin{tabular}{ll}
\lstinline!(: empty-list)!& leere liste\\
\lstinline!(: make-pair (% a (list-of % a))! & Konstruiert Liste aus Kopf und Rest
\end{tabular}
\i[Predikate:]
\begin{tabular}{ll}
\lstinline!(: empty (any -> boolean)!& liegt leere Liste vor?\\
\lstinline!(: pair? (any -> boolean))! & Nicht leere Liste?
\end{tabular}
\i[Selektoren:]
\begin{tabular}{ll}
\lstinline!(: first (list-of %a) -> %a)!& Kopf-Element\\
\lstinline!(: rest (list-of %a) -> (list-of %a))! & Rest Liste
\end{tabular}
\end{enumerate}
\racket{Listen mit Signatur list-of}{Listen aus einem oder verschiedenen Datentypen}{lists}
\end{document}
