\lstinputlisting[firstline=105,lastline=108]{Animationen-und-HOP-Typ2.rkt}
Teachpack 'universe' nutzt H.O.P Animationen (Sequenzen von Bildern/Szenen) zu definieren.
\begin{lstlisting}
(big bang
  (@\auf init\zu@)
  (ontick @\auf tock\zu@)
  (todraw @\auf render\zu \auf w\zu\auf h\zu@))
\end{lstlisting}
\begin{enumerate}[-]
\item \lstinline[mathescape]|($\az{init}$ %a)| Startzustand
\item \lstinline[mathescape]|(: $\az{tock}$ (%a -> %a))|
Funktion, die einen neuen Zustand aus alten Zustand berechnet
\item \lstinline[mathescape]|(: $\az{render}$ (%a -> image))| Funktio, die aus dem aktuellen eine Szene berechnet (wird in Fenster mit Dimension $\az{w} \cdot \az{h}$ Pixel angezeigt)
\item Beim Schlie\ss en der Animation wird der letzte Zustand zurückgegeben
\end{enumerate}
\rackett{Animation 1: Ein Zähler}{Ein animierter Zähler}{Animationen-und-HOP-Typ2}{4}{13}
\rackett{Animation 2: Ein Raumschiff}{Ein animiertes Raumschiff}{Animationen-und-HOP-Typ2}{18}{35}
Ausgabe der römischen Episoden nummern für Film f : \lstinline|(roman (film-episode f))|\\
Gesuchte Funktion ist \emph{Komposition} von zwei existierenden Funktionen:
\begin{enumerate}[(1)]
\item Erst \lstinline[mathescape]|film-episode| anwenden, \emph{dann}
\item Wende \lstinline[mathescape]|roman| auf das Ergebnis von (1) an
\end{enumerate}
Komposition von Prozeduren allgemein:\\
\lstinline[mathescape]|($\underbrace{\text{(compose f g)}}_{\substack{\text{neue Prozedur realisiert}\\ \text{Komposition von f und g}}}$x)| $\equiv$ \lstinline|(f (g x))|\\
$\lbrack$ Mathematisch \lstinline|(compose f g)| $\equiv f \circ g$ $\rbrack$\\
\begin{lstlisting}
(: compose (%b -> %c) ($a -> %b) -> (%a -> %c))
(define compose
  (lambda (f g)
    (lambda (x)
      (f (g x)))))
\end{lstlisting}
\rackett{Anwendungen von Combined}{Zweites und Drittes Element durch Combined}{Animationen-und-HOP-Typ2}{50}{65}
\lstinline|repeat|: n-fache Komposition von f auf sich selbst\\
(n-fache Anwendung von f, Exponentation)\\
\begin{align*}
f^0 &= \text{\lstinline|id|} &(\text{\lstinline|id|} \equiv \text{\lstinline|(lambda(x)x)|)}\\
f^n &= f \circ f^{n-1}& 
\end{align*}
\begin{lstlisting}
(: repeat (natural (%a -> %a) -> (%a -> %a)))
(define repeat
  (lambda (n f)
    (cond
      ((= n 0)(lambda (x) x)
      ((> n 0)(compose f (repeat (- n 1) f)))))))
;Greife auf das n-te Element der Liste xs zu
(: nth (natural (list-of %a) -> %a))
(define nth
  (lambda (n xs)
    ((compose first (repeat (- n 1) rest))xs)))
\end{lstlisting}

\rackett{+ als Higher Order Funktion}{Gibt die Funktion + zurück}{Animationen-und-HOP-Typ2}{86}{105}

Reduktion:
\lstinline|((add 1) 41)|
\begin{enumerate}
\item[$\underset{\t{eval}_{id}}{\text{\eval}}$] \lstinline|((lambda (x) (lambda (y) (+ x y))1)41)|
\item[$\underset{\substack{\t{apply}_{\lambda}\\ [ \text{lambda(x)} ]}}{\text{\eval}}$] \lstinline[mathescape]|($\underbrace{\text{(lambda (y) (+ 1 y)}}_{\substack{\text{Funktion die 1 auf}\\ \text{ihr Argument anwenden} }}$41)|
\item[$\underset{\substack{\t{apply}_{\lambda}\\ [ \text{lambda(y)} ]}}{\text{\eval}}$] \lstinline[mathescape]|(+ 1 $ $ 41)|
\end{enumerate}